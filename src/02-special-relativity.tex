\chapter{Special Relativity}
\begin{table}
    \centering
    \begin{tabulars}{lll}
        \toprule
        &Newton&Einstein (SR)\\
        \midrule
        
        & Law of inertia
        & Law of inertia \\
        Newton's Laws
        & $\vec{F}=m\ddot{\vec{x}}=m\od{\vec{p}}{t}$
        &$\vec{F}=m\od{\vec{p}}{\tau}$\\
        & $\vec{F}_{12}=-\vec{F}_{21}$
        & Momentum conservations (postulate)\\
        Force
        &$\vec{F}=\frac{G\textsubscript{N}m_1m_2}{r^2}$
        &$F^\alpha= \frac{q}{m} U_\beta
        \tensor{F}{^\alpha^\beta}$,\,  $F
        =\gamma\left(\begin{smallmatrix}
        \beta
        f_0\\
        \vec{f}
        \end{smallmatrix}\right)$\\
 		&$\Delta \Phi =4\pi\rho$ &Maxwell equations
        \\
        \bottomrule
    \end{tabulars}
    \caption{Comparison of electrostatics and Newtonian gravity}
\end{table}
We want to introduce a Lagrangian for special relativity. We start by a
variation principle. We consider massive particles, i.e. timelike paths. We
postulate that the action $S$ is proportional to the spacetime distance 
\begin{equation}
S= \alpha \int_a^b\dif s=- \alpha \int_a^b\dif \tau = \alpha
\int_{t_1}^{t^2}\sqrt{1-v^2}\dif t =: \int_{t_1}^{t^2}L\dif t\, .
\end{equation}
Where the Lagrangian $L$ is given by
\begin{equation}
L=-\alpha\sqrt{1-v^2}\simeq -\alpha +\alpha\frac{1}{2}v^2+\ldots\, .
\end{equation}
We demand that we recover the classical theory in the limit $v\to 0$. The
dynamical term contribution, i.e. the lowest order term containing $v$ is 
\begin{equation}
L_0=\alpha\frac{1}{2}v^2\, .
\end{equation}
Since we know that $L_0=\frac{1}{2}mv^2$ we find $\alpha=m$. 
If we restore units of $c$ for a moment we get that the constant term in $L$ is
equal to $mc^2$. This is Einsteins famous $E=mc^2$. 

If we substitute $\alpha$ we recover the Lagrangian of special relativity
\begin{equation}
L\textsubscript{SR}=-m\sqrt{1-v^2}\, .
\end{equation}
We can proceed calculate the generalisized momenta
\begin{equation}
p_i=\dpd{L}{v_i}=\frac{mv_i}{\sqrt{1-v^2}}=\gamma m v_i\, .
\end{equation}
The energy can be calculated via the Hamiltonian $H$
\begin{equation}
E=H=\vec{p}\cdot\vec{v}-L=\gamma m v^2 + m\gamma^{-1} =\gamma m
\end{equation}
Expanded in $v$ the energy reads as
\begin{equation}
E=m+\frac{1}{2}mv^2+\ldots\, .
\end{equation}
We can further relate energy and momentum to each other. Therefore consider 
\begin{equation}
p^2=\frac{mv^2}{1-v^2}\, .
\end{equation}
Solving for $v$ yields
\begin{equation}
v^2=\frac{p^2}{p^2+m^2}\, .
\end{equation}
Which we can insert in the expression for the energy
\begin{equation}
E^2=\frac{m^2}{1-v^2}=p^2+m^2 \label{eq:onshell}\, .
\end{equation}
The equation \eqref{eq:onshell} is called the \emph{on shell condition}.
\begin{sidenote}
There is nothing in the theory that predicst that the photon is massless. It
could be possible that its mass is only really small and that in fact that the
photon does not travel at the speed of light. (This would make it
convinient to rename the constant $c$) Even though the mass should be so small
that the de Broglie wavelength is of the order of the size of the universe,
massive photons would have a huge impact.
\end{sidenote}
