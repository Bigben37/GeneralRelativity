\chapter{Special Relativity}
\section{Postulates and Definitions}
\begin{enumerate}
\item Principle of Relativity: 
The laws of physics acquire the same form in all inertial systems.
\item Constancy of the speed of light:
The speed of light in vacuum is constant ${c\approx\unitfrac[3\cdot
10^8]{m}{s}}$
\end{enumerate}
\begin{definition}[System of reference]
A system of reference is a system of three spacial coordinates to indicate the
position and one time coordinate to indicate the time
\end{definition}
\begin{definition}[Inertial system]
An inertial system $I$ belongs to a particular subspace of reference systems in
which a freely moving body, i.e. a body that is not subject to any external
force, moves with constant velocity on an straight line.
\end{definition}
\begin{definition}[Event]
An Event or world point, is a point in spacetime. In any system of reference it
can be described by four coordinates 
\begin{equation}
x^\mu: (x^0,x^1,x^2,x^3)\, .
\end{equation}
For example in cartesian coordinates we have 
\begin{equation}
(x^0,x^1,x^2,x^3) = (ct,x,y,z)\, .
\end{equation}
\end{definition}
\begin{definition}[Worldline]
A worldline is a parametrised curve in spacetime. To each parameter, there
corresponds an event that lies on the worldline.
\end{definition}
\section{Propagation of light waves and the line element}
(Pictures)
Inertial System $I$:
$P:$ Event of emission $(t_1,x_1,y_1,z_1)$\\
$Q:$ Event of absorption $(t_2,x_2,y_2,z_2)$
The spatially distance between the events is
\begin{equation}
r=\left[(x_2-x_1)^2+(y_2-y_1)^2+(z_2-z_1)^2\right]^{\frac{1}{2}}\, .
\end{equation}
Since we are tracing a light ray and the events are absoption and emission, we
furter have
\begin{equation}
r=c(t_2-t_1)\, .
\end{equation}
We construct the quantity
\begin{equation}
{\Delta s}^2=-c^2(t_2-t_1)^2+(x_2-x_1)^2+(y_2-y_1)^2+(z_2-z_1)^2\, ,
\end{equation}
so that ${\Delta s}^2=0$ for light.\\
Inertial System $I'$:
$P:$ Event of emission $(t_1^\prime,x_1^\prime,y_1^\prime,z_1^\prime)$\\
$Q:$ Event of absorption $(t_2^\prime,x_2^\prime,y_2^\prime,z_2^\prime)$
\begin{equation}
{\Delta
s^\prime}^2=-c^2\left(t^\prime_2-t^\prime_1\right)^2
+\left(x^\prime_2-x^\prime_1\right)^2
+\left(y^\prime_2-y^\prime_1\right)^2
+\left(z^\prime_2-z^\prime_1\right)^2\,.
\end{equation}
Again ${\Delta s^\prime}^2=0$ for light.\\
We define an infinitisimal interval, or \emph{line element}:
\begin{equation}
\dif s^2=-c^2\dif t^2+\dif x^2+\dif y^2+\dif z^2\, .
\end{equation}
If the intervall is zero in one IS it should be zero in all IS. To relate the
linelements of different systems to each other, we make the following thought:
Suppose we have three interial Systems $I,I_1,I_2$.
The Systems $I_1$ and $I_2$ move at constant velocity $\vec{v}_1$ and $\vec{v}_2$ relative to $I$. Further
$I_2$ moves with velocity $\vec{v}_{12}$ relative to $I_1$. The demand implies
that there exists a function $\alpha$ so that
\begin{equation}
\dif s'^2=\alpha(\vec{v})\dif s^2
\end{equation}
We have
\begin{equation}
\dif s^2=\alpha(\vec{v}_1)\dif s_1^2\, ,\quad\dif s^2=\alpha(\vec{v}_2)\dif
s_2^2\, ,\quad\dif s_1^2=\alpha(\vec{v}_{12})\dif s_2^2\, .
\end{equation}
Together they imply
\begin{equation}
\alpha(\vec{v}_{12})=\frac{\alpha(\vec{v}_{2})}{\alpha(\vec{v}_{1})}\, ,
\end{equation}
but since the velocities where arbitary $\alpha$ has to be constant and we are
free to chose $\alpha\equiv 1$. Therefore the lineelement has to stay invariant.
\section{Invariant distances Metric and Signature}
In flat space we can calculate the distance $\Delta l$ between two points
$(x_1,y_1)$ and $(x_2,y_2)$ by the Pythagorean theorem
\begin{equation}
\Delta l^2=\Delta x^2+\Delta y^2\, ,
\quad \Delta x=x_2-x_1\,,\quad \Delta
y=y_2-y_1\, .
\end{equation}
For an infinitisimal distance $\dif l$ we recover
\begin{equation}
\dif l^2=\dif x^2+\dif y^2=\delta_{ij}\dif x^i\dif x^j\, .
\end{equation}
If we introduce
\begin{equation}
(\dif x^i)=\begin{pmatrix}
\dif x\\
\dif y
\end{pmatrix}\,\quad ,(\delta_{ij})=\begin{pmatrix}
1 & 0\\
0 & 1
\end{pmatrix}\, .
\end{equation}
We can expand the formula for the infinitisimal element and obtain
\begin{equation}
\dif l^2=\begin{pmatrix}
\dif x & 
\dif y
\end{pmatrix}
\begin{pmatrix}
1 & 0\\
0 & 1
\end{pmatrix}
\begin{pmatrix}
\dif x\\
\dif y
\end{pmatrix}
\end{equation}
Which we do to point out the similarity to the line element $\dif s^2$
\begin{equation}
\dif s^2=\begin{pmatrix}
c\dif t & 
\dif x & 
\dif y &
\dif z 
\end{pmatrix}
\begin{pmatrix}
-1 & 0 & 0 & 0\\
0  & 1 & 0 & 0\\
0  & 0 & 1 & 0\\
0  & 0 & 0 & 1\\
\end{pmatrix}
\begin{pmatrix}
c\dif t\\
\dif x\\
\dif y\\
\dif z\\
\end{pmatrix}=:\eta_{\mu\nu}\dif x^\mu\dif x^\nu\, .
\end{equation}
We call the matrix $\eta_{\mu\nu}=\mathrm{diag}(-1,1,1,1)$ the
\emph{Mikovski-metric}. It has some obvious properties:
\begin{itemize}
  \item constancy: $\pd{\tensor{\eta}{_\mu_\nu}}{{x^\rho}}=0$
  \item symmetry: $\tensor{\eta}{_\mu_\nu}=\tensor{\eta}{_\nu_\nu}$
  \item self inverse:
  $\tensor{\eta}{^\mu^\nu}:=\tensor{{\eta^{-1}}}{_\mu_\nu}=\tensor{\eta}{_\mu_\nu}$
\end{itemize}
We can use the metric to raise and lower indices:
\begin{equation}
x_\nu=\tensor{\eta}{_\mu_\nu}x^\mu\, ,\quad x^\mu=\tensor{\eta}{^\mu^\nu}x_\nu\,
.
\end{equation}
The trace of the metric is given by
\begin{equation}
\tensor{\eta}{^\mu_\mu}=\tensor{\eta}{^\mu^\nu}\tensor{\eta}{_\nu_\mu}
=\tensor{\delta}{^\mu_\mu}=4\, .
\end{equation}
We say $\tensor{\eta}{_\mu_\nu}$ has signature $(-,+,+,+)$
\subsection{Einstein summation convention}
Wir brauchen ein evtl. kapitel mit Notationen 
We write short $x^ix_i$ for $\sum_i x^ix_i$..
\subsection{Poincaré-transformation}
\begin{equation}
\dif {s^\prime}^2 = \eta_{\mu\nu}\dif {x^\prime}^\mu\dif
{x^\prime}^\nu=\eta_{\mu\nu}\dif x^\mu\dif
x^\nu = \dif s^2
\end{equation}
We consider afine coordinate transformations 
\begin{equation}
{x^\prime}^\mu= f^\mu(x^\nu)=\tensor{L}{^\mu_\nu}x^\nu+a^\mu\, .
\end{equation}
Where $\tensor{L}{^\mu_\nu}$ is a a lorentz transformation and $a^\mu$ is a
constant shift.\\
We can now inspect the transformation properties of an allowed transformation.
The invariance of the line element implies
\begin{equation}
\dif {s^\prime}^2 = \eta_{\mu\nu}\dif {x^\prime}^\mu\dif
{x^\prime}^\nu=\eta_{\rho\sigma}\tensor{L}{^\rho_\nu}\dif
x^\mu\tensor{L}{^\sigma_\nu}\dif x^\nu = \dif s^2
\end{equation}
This implies that
\begin{equation}
\eta_{\mu\nu}=\tensor{L}{^\rho_\nu}\eta_{\rho\sigma}\tensor{L}{^\sigma_\nu}\label{eq:invariance}
\end{equation}
We can take a look at infinitimal transformations i.e.
\begin{equation}
\tensor{L}{^\mu_\nu}=\tensor{\delta}{^\mu_\nu}
+\varepsilon\tensor{\omega}{^\mu_\nu}+\mathcal{O}\left(\varepsilon^2\right)\, .
\end{equation}
Plugging into \eqref{eq:invariance} results in
\begin{equation}
\begin{split}
\eta_{\mu\nu}&=\eta_{\rho\sigma}(\tensor{\delta}{^\rho_\mu}
+\varepsilon\tensor{\omega}{^\rho_\mu})(\tensor{\delta}{^\sigma_\nu}
+\varepsilon\tensor{\omega}{^\sigma_\nu})\\
&=
\eta_{\mu\nu}+\varepsilon(\tensor{\omega}{_\mu_\nu}+\tensor{\omega}{_\nu_\mu})
\end{split}
\end{equation}
Since this must hold true for arbitary $\varepsilon$ the generators
$\tensor{\omega}{_\nu_\mu}$ must satisfie
$\tensor{\omega}{_\nu_\mu}=-\tensor{\omega}{_\mu_\nu}$, i.e. be antisymmetric.
In general we have $\frac{n(n-1)}{2}$ antisymetric linear maps. Since we are
considering 4d spacetime, there are $6$ possible maps. We can classify the maps
in two groups
\begin{enumerate}
  \item Rotations. E.g. Rotation around $z$-Axis:
  \begin{align*}
  \tensor{\omega}{^\mu_\nu}
  =\begin{pmatrix}
  0&0 & 0&0\\
  0&0 &1&0\\
  0&-1&0 &0\\
  0&0 &0 &0
  \end{pmatrix}\implies
  \begin{pmatrix}
  ct\\
  x\\
  y\\
  z\\
  \end{pmatrix}=
  \begin{pmatrix}
  1&0 & 0&0\\
  0&\cos\alpha &\sin\alpha&0\\
  0&-\sin\alpha&\cos\alpha &0\\
  0&0 &0 &1
  \end{pmatrix}
  \begin{pmatrix}
  ct^\prime\\
  x^\prime\\
  y^\prime\\
  z^\prime
  \end{pmatrix}
  \end{align*}
  \item Boosts, i.e. maps mixing time and spacial coordinates. Boost in $x$
  direction:
  \begin{equation}
  -(ct)^2+x^2=(ct^\prime)^2+(x^\prime)^2
  \end{equation}
  This hyperbolic equation can be parametrised as
  \begin{equation}
  ct = ct^\prime\cosh\psi+x^\prime\sinh\psi\,, \quad x =
  ct^\prime\sinh\psi+x^\prime\cosh\psi\, .
  \end{equation}
  Where $\psi$ is called the \emph{rapidity}. In matrix form the transformation
  reads
  \begin{align*}
  \tensor{\omega}{^\mu_\nu}
  =\begin{pmatrix}
  0&1 & 0&0\\
  1&0 &0&0\\
  0&0&0 &0\\
  0&0 &0 &0
  \end{pmatrix}\implies
  \begin{pmatrix}
  ct\\
  x\\
  y\\
  z\\
  \end{pmatrix}=
  \begin{pmatrix}
  \cosh\psi&\sinh\psi & 0&0\\
  \sinh\psi&\cosh\psi &0&0\\
  0&0&1 &0\\
  0&0 &0 &1
  \end{pmatrix}
  \begin{pmatrix}
  ct^\prime\\
  x^\prime\\
  y^\prime\\
  z^\prime
  \end{pmatrix}\, .
  \end{align*}
  To which velocity $v$ does this boost corespond? Set $x^\prime= 0$. Then 
  \begin{equation}
  ct=ct^\prime\cosh\psi\, ,\quad
  x=x^\prime\sinh\psi\, .
  \end{equation}
  Then we have 
  \begin{equation}
  v=\frac{x}{t}=\tanh\psi
  \end{equation}
  It is convinient to introduce the parameters $\beta$ and $\gamma$ defined as
  follows
  \begin{equation}
  \beta:=\frac{v}{c}\, ,\quad\gamma:=\frac{1}{\sqrt{1-\beta}}\, .
  \end{equation}
  In terms of these we find
  \begin{equation}
  \sinh\psi = \beta\gamma\, , \quad \cosh\psi=\gamma
  \end{equation}
  Which gives 
  \begin{align}
  t&=\gamma\left(t^\prime+\frac{v}{c^2}x^\prime\right)\, ,\\
  x&=\gamma\left(x^\prime+vt^\prime\right)\, .
  \end{align}
\end{enumerate}
All together we have 10 independent (affine) transformations that leave the line
element invariant. Namely:
\begin{itemize}
  \item 4 shifts by $a^\mu$
  \item 3 rotations in space
  \item 3 boosts with constant velocity $\vec{v}$
\end{itemize}
The poincare transformations form a 10 parameter group.
\begin{table}
    \centering
    \begin{tabulars}{lll}
        \toprule
        &Newton&Einstein (SR)\\
        \midrule
        & Law of inertia
        & Law of inertia \\
        Newton's Laws
        & $\vec{F}=m\ddot{\vec{x}}=m\od{\vec{p}}{t}$
        &$\vec{F}=m\od{\vec{p}}{\tau}$\\
        & $\vec{F}_{12}=-\vec{F}_{21}$
        & Momentum conservations (postulate)\\
        Transformations
        & boosts
        & Galilei Transformation\\
        
        & absolute structure
        & absolute time\\
                
        & $c=\mathrm{const.}$
        & $t^\prime=t$\\
        Force
        &$\vec{F}=\frac{G\textsubscript{N}m_1m_2}{r^2}$
        &$F^\alpha= \frac{q}{m} U_\beta
        \tensor{F}{^\alpha^\beta}$,\,  $F
        =\gamma\left(\begin{smallmatrix}
        \beta
        f_0\\
        \vec{f}
        \end{smallmatrix}\right)$\\
 		&$\Delta \Phi =4\pi\rho$ &Maxwell equations
        \\
        \bottomrule
    \end{tabulars}
    \caption{Comparison of Newtonian theory and special relativity}
\end{table}

\section{Lagrangian Principle}
We want to introduce a Lagrangian for special relativity. We start by a
variation principle. We consider massive particles, i.e. timelike paths. We
postulate that the action $S$ is proportional to the spacetime distance 
\begin{equation}
S= \alpha \int_a^b\dif s=- \alpha \int_a^b\dif \tau = \alpha
\int_{t_1}^{t^2}\sqrt{1-v^2}\dif t =: \int_{t_1}^{t^2}L\dif t\, ,
\end{equation}
whith $\alpha$ a constant that has to be determined. The Lagrangian $L$ is
given by
\begin{equation}
L=-\alpha\sqrt{1-v^2}\simeq -\alpha +\alpha\frac{1}{2}v^2+\ldots\, .
\end{equation}
We demand that we recover the classical theory in the limit $v\to 0$. The
the lowest order kinetic term is 
\begin{equation}
L_0=\alpha\frac{1}{2}v^2\, .
\end{equation}
Comparing with the classical kinetic energy $L_0=\frac{1}{2}mv^2$ yields
$\alpha=m$. If we substitute $\alpha$ we recover the Lagrangian of special
relativity
\begin{equation}
L\textsubscript{SR}=-m\sqrt{1-v^2}\, .
\end{equation}
We can proceed calculate the generalized momenta
\begin{equation}
p_i=\dpd{L}{v_i}=\frac{mv_i}{\sqrt{1-v^2}}=\gamma m v_i\, .
\end{equation}
The energy can be calculated via the Hamiltonian $H$
\begin{equation}
E=H=\vec{p}\cdot\vec{v}-L=\gamma m \vec{v}^2 + m\gamma^{-1} =\gamma m
\end{equation}
Expanded in $\vec{v}^2$ the energy reads as
\begin{equation}
E=m+\frac{1}{2}m\vec{v}^2+\ldots\, .
\end{equation}
If we restore units of $c$ for a moment we get that the constant term is
equal to $mc^2$. This is Einsteins famous $E=mc^2$. 
We can further relate energy and momentum to each other. Therefore consider the
square of the momentum $\vec{p}$
\begin{equation}
\vec{p}^2=\frac{m\vec{v}^2}{1-\vec{v}^2}\, .
\end{equation}
Solving for $\vec{v}^2$ yields
\begin{equation}
\vec{v}^2=\frac{\vec{p}^2}{\vec{p}^2+m^2}\, .
\end{equation}
Which we can insert in the expression for the energy
\begin{equation}
E^2=\frac{m^2}{1-\vec{v}^2}=\vec{p}^2+m^2 \label{eq:onshell}\, .
\end{equation}
The equation \eqref{eq:onshell} is called the \emph{on shell condition}.
\begin{sidenote}
There is nothing in the theory that predicst that the photon is massless. It
could be possible that its mass is only really small and that in fact that the
photon does not travel at the speed of light. (This would make it
convinient to rename the constant $c$) Even though the mass should be so small
that the de Broglie wavelength is of the order of the size of the universe,
massive photons would have a huge impact.
\end{sidenote}
