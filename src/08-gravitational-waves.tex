\chapter{Gravitational Waves}
\begin{itemize}
    \item vacuum solution of linearized Einstein equations
    \begin{equation}
        \Box \overline{h}_{\mu\nu} = 0 \qquad \text{in de Donder gauge}
    \end{equation}
    \item describes \emph{weak} gravitational waves \emph{only} $\rightarrow$ linearized treatment justified
    \item description breaks down for strong gravitational fields as the theory becomes essentially \emph{non-linear} (e.g.\ two black holes merge)
    \item analysis similar to electrodynamics, but here $h_{\mu\nu}$: spin-2 field, $A_\mu$: spin-1 field
    \item vacuum equations
    \begin{equation}
        \Box \overline{h}_{\mu\nu} = 0 \qquad \overline{h}_{\mu\nu}=h_{\mu\nu} - \frac{1}{2} \eta_{\mu\nu} h \qquad \partial^\mu \overline{h}_{\mu\nu} = 0
    \end{equation}
    \item Gauge freedom not yet completely exhausted by de Donder gauge.
    \begin{equation}
        \partial^\mu \overline{h'}_{\mu\nu} = \underbrace{\partial^\mu \overline{h'}_{\mu\nu}}_{=0} + \Box \xi_\nu = 0
    \end{equation}
    $\implies$ All gauge transformations generated by $\xi_\nu$ that satisfy $\Box \xi_\nu = 0$ do not lead out of the de Donder gauge. \\
    Compare with electromagnetic field:
    \begin{align}
        & A_\mu \to {A'}^\mu = A^\mu + \partial_\mu \lambda(x) \\
        & \partial_\mu A^\mu = 0 \qquad \text{Lorentz gauge} \\
        & \partial_\mu {A'}^\mu = \underbrace{ \partial_\mu A^\mu}_{=0} + \partial_\mu \partial^\mu \lambda(x) = 0 \implies \Box \lambda = 0
    \end{align}
    Exploit this remaining gauge freedom to make perturbations $h_{\mu\nu}$ \emph{transverse} and \emph{traceless}
    \item transversality
    \begin{equation}
        \partial^\mu h'_{\mu\nu} = \partial^\mu h_{\mu\nu} + \Box \xi_\nu + \partial_\nu \partial^\mu \xi_\mu \overset{!}{=} 0
    \end{equation}
    Since only gauge transformations that satisfy $\Box \xi_\nu$ are allowed (do not lead out of de Donder gauge), the equation that shoud
    be solved for $\xi_\nu$ is
    \begin{equation}
        \partial_\nu \partial^\mu \xi_\mu = - \partial^\mu h_{\mu\nu}
    \end{equation}
    \item In order for the perturbations to be traceless, we need to find a solution to
    \begin{equation}
        h' = h + \partial_\mu \xi^\mu = 0 \implies \partial_\mu \xi^\mu =
        -\frac{1}{2}h
    \end{equation}
\end{itemize}
\begin{figure}[hbtp!]
\centering
 \includegraphics{gwave1.pdf}
  \includegraphics{gwave2.pdf}
  \includegraphics{gwave3.pdf}
\caption{Polarisations of gravitational waves. From top to bottom:
$+$,$\times$ and mixed (circular) polarised waves.}
%TODO Caption
\end{figure}

% \begin{figure}
% \centering
% \begin{tikzpicture}[decoration={markings,
%   mark=between positions 0 and 1 step 22.3pt
%   with { \draw [fill] (0,0) circle [radius=2pt];}}]
% \draw[->] (0,1) -- (3,1);
% \draw[->] (0,1) -- (0,4); 
% \node[text width=0.25cm] at (3,1.25){$x$};
% \node[text width=0.25cm] at (0.25,4){$y$};
% 
% \node[text width=2cm, align=center] (1) at (2,5){$\omega t= 0$};
% \node[text width=2cm, align=center] (1) at (5,5){$\omega t= \frac{\pi}{2}$};
% \node[text width=2cm, align=center] (1) at (8,5){$\omega t= \pi$};
% \node[text width=2cm, align=center] (1) at (11,5){$\omega t= \frac{3\pi}{2}$};
% \node[text width=2cm, align=center] (1) at (14,5){$\omega t= 2\pi$};
% \begin{scope}[]
% \draw[dots along my path]  (2,3) ellipse (1 and 1);
% \end{scope}
% \begin{scope}[shift={(3,-0.5)},transform canvas={yscale=1.2}]
% \draw[dots along my path]  (2,3) ellipse (1 and 1);
% \end{scope}
% \begin{scope}[shift={(6,0)}]
% \draw[dots along my path]  (2,3) ellipse (1 and 1);
% \end{scope}
% \begin{scope}[shift={(7.2,0)},transform canvas={xscale=1.2}]
% \draw[dots along my path]  (2,3) ellipse (1 and 1);
% \end{scope}
% \begin{scope}[shift={(12,0)}]
% \draw[dots along my path]  (2,3) ellipse (1 and 1);
% \end{scope}
% \end{tikzpicture}
% \caption{Gravitational wave.}
% \end{figure}
% \begin{figure}
% \centering
% \begin{tikzpicture}[decoration={markings,
%   mark=between positions 0 and 1 step 22.3pt
%   with { \draw [fill] (0,0) circle [radius=2pt];}}]
% \draw[->] (0,1) -- (3,1);
% \draw[->] (0,1) -- (0,4); 
% \node[text width=0.25cm] at (3,1.25){$x$};
% \node[text width=0.25cm] at (0.25,4){$y$};
% 
% \node[text width=2cm, align=center] (1) at (2,5){$\omega t= 0$};
% \node[text width=2cm, align=center] (1) at (5,5){$\omega t= \frac{\pi}{2}$};
% \node[text width=2cm, align=center] (1) at (8,5){$\omega t= \pi$};
% \node[text width=2cm, align=center] (1) at (11,5){$\omega t= \frac{3\pi}{2}$};
% \node[text width=2cm, align=center] (1) at (14,5){$\omega t= 2\pi$};
% \begin{scope}[]
% \draw[dots along my path]  (2,3) ellipse (1 and 1);
% \end{scope}
% \begin{scope}[shift={(3,-0.5)},transform canvas={yscale=1.2}]
% \draw[dots along my path]  (2,3) ellipse (1 and 1);
% \end{scope}
% \begin{scope}[shift={(6,0)}]
% \draw[dots along my path]  (2,3) ellipse (1 and 1);
% \end{scope}
% \begin{scope}[shift={(7.2,0)},transform canvas={xscale=1.2},rotate around
% ={45,(9.2,3)}] \draw[dots along my path]  (2,3) ellipse (1 and 1);
% \end{scope}
% \begin{scope}[shift={(12,0)}]
% \draw[dots along my path]  (2,3) ellipse (1 and 1);
% \end{scope}
% \end{tikzpicture}
% \caption{Gravitational wave.}
% \end{figure}


\section{Degrees of Freedom (DoF) and Scalar Vector Tensor (SVT) Decomposition in Space/Time}

Parametrize the line element as
\begin{equation}
	\dif s^2 = - \left(1+2\Phi\right)\dif t^2 + \tensor{v}{_i} \left(\dif t \dif \tensor{x}{^i} + \dif t \dif \tensor{x}{^i}\right) + \left(\tensor{\delta}{_{ij}} + \tensor{h}{_{ij}} \right)\dif \tensor{x}{^i}\dif \tensor{x}{^j}\,.
\end{equation}
Block matrix
\begin{equation}
	\tensor{h}{_{\mu\nu}} =
	\begin{bmatrix}
		\tensor{h}{_{00}} & \tensor{h}{_{0j}} \\
		\tensor{h}{_{i0}} & \tensor{h}{_{ij}}
	\end{bmatrix}
	=
	\begin{bmatrix}
		-2\Phi & \tensor{v}{_{j}} \\
		\tensor{v}{_{i}} & \tensor{h}{_{ij}}
	\end{bmatrix}
	\, , \quad \abs{\Phi},\abs{\tensor{v}{_i}},\abs{\tensor{h}{_{ij}}} \ll 1\,.
\end{equation}
In general algebraic decomposition: symmetric/antisymmetric
\begin{equation}
	\tensor{T}{_{\mu\nu}} = \tensor*{T}{_{\mu\nu}^{\text{s}}} + \tensor*{T}{_{\mu\nu}^{\text{as}}} = \tensor{T}{_{(\mu\nu)}} + \tensor{T}{_{[\mu\nu]}} = \frac{1}{2}\left(\tensor{T}{_{\mu\nu}}+\tensor{T}{_{\nu\mu}}\right) + \frac{1}{2}\left(\tensor{T}{_{\mu\nu}}-\tensor{T}{_{\nu\mu}}\right)\,.
\end{equation}
Can we decompose $\tensor*{T}{_{\mu\nu}^{\text{s}}}$ further?\newline
Wake the trace
\begin{equation}
	\tensor{T}{^{\text{s}}} = \tensor{g}{^{\mu\nu}}\tensor*{T}{_{\mu\nu}^{\text{s}}}\, ,
\end{equation}
so we can write $\tensor*{T}{_{\mu\nu}^{\text{s}}}$ as follows:
\begin{equation}
	\tensor*{T}{_{\mu\nu}^{\text{s}}} = \tensor*{T}{_{\mu\nu}^{\text{tf}}} + \frac{1}{d}\,\tensor{g}{_{\mu\nu}}\tensor{T}{^{\text{s}}}\,,
\end{equation}
where $d$ denotes the dimension of spacetime and $\tensor*{T}{_{\mu\nu}^{\text{tf}}}$ is a tracefree, symmetric tensor.
Can we decompose $\tensor*{T}{_{\mu\nu}^{\text{tf}}}$ any further? Possible for tensor fields $\tensor{T}{_{\mu\nu}}(x)$ as this involves derivatives.
Decompose metric perturbations:
\begin{equation}
	\tensor{h}{_{00}} = -2\Phi
\end{equation}
\begin{equation}
	\tensor{h}{_{0i}} = \tensor{v}{_i}
\end{equation}
\begin{equation}
	\tensor{h}{_{ij}} = 2\tensor{s}{_{ij}} - 2\Psi\tensor{s}{_{ij}}\,\quad
	\begin{cases}
		\Psi := -\frac{1}{6}\tensor{\delta}{^{ij}}\tensor{h}{_{ij}}\,\quad
		&\text{``trace''} \\
		\tensor{s}{_{ij}} := \frac{1}{2} \left(\tensor{h}{_{ij}} - \frac{1}{3}\tensor{\delta}{^{kl}}\tensor{h}{_{kl}}\tensor{\delta}{_{ij}}\right)\,\quad &\text{``strain''}
	\end{cases}
\end{equation}
\begin{equation}
	\dif s^2 = -\left(1+2\Phi\right)\dif t^2 + \tensor{v}{_i} \left(\dif t \dif \tensor{x}{^i} + \dif t \dif \tensor{x}{^i}\right) + \left[\left(1-2\Psi\right)\tensor{\delta}{_{ij}} + 2 \tensor{S}{_{ij}} \right]\dif \tensor{x}{^i}\dif \tensor{x}{^j}
\end{equation}
$\tensor{\partial}{_i} = \tensor{k}{_i}$ (momentum space)
Decompose vector $\tensor{\omega}{_i}$ into transverse and longitudinal components.
\begin{align}
	\tensor{v}{_i} = &\tensor{S}{_i} + \tensor{\partial}{_i}B\\
	&3\,\,+\,\,\,\,\,1\,\,\,\,\underbrace{-1}_{\tensor{\partial}{_i}\tensor{S}{^i}=0}\,\quad \text{DoF}
\end{align}
\begin{equation}
	\tensor{S}{_{ij}} = \tensor{\partial}{_{(i}}\tensor{F}{_{j)}} + \left( \tensor{\partial}{_i}\tensor{\partial}{_j} - \frac{1}{3}\,\tensor{\delta}{_{ij}}\Delta\right) E + \tensor*{h}{_{ij}^{\text{\tiny TT}}}
\end{equation}
\begin{center}
\begin{tabular}{c l}
$4$ & 4 scalars \\
$+$ & \\
$2\,(3-1)$ & 2 transverse vectors \\
$+$ & \\
$6-1-3$ & 1 symmetric transverse traceless tensor \\
\midrule
$10$ & independent components of $\tensor{h}{_{\mu\nu}}$
\end{tabular}
\end{center}
valid for (reduce free components):
\begin{equation}
\tensor{\partial}{_i}\tensor{F}{^i}=0\,,\quad\tensor*{h}{_{ij}^{\text{\tiny TT}}}=\tensor*{h}{_{ji}^{\text{\tiny TT}}}\,,\quad\tensor*{h}{_{ij}^{\text{\tiny TT}}}\tensor{\delta}{^{ij}}=0\,,\quad\tensor{\partial}{^i}\tensor*{h}{_{ij}^{\text{\tiny TT}}}=0
\end{equation}
\begin{figure}[hbtp!]
\centering
 \includegraphics{antisymmetricMatrix.pdf}
\caption{}
%TODO Caption
\end{figure}
\begin{figure}[hbtp!]
\centering
 \includegraphics{Coorddist.pdf}
\caption{}
%TODO Caption
\end{figure}

\begin{figure}[hbtp!]
\centering
 \includegraphics[scale=.75]{ellipse1.pdf}
 \includegraphics[scale=.75]{ellipse2.pdf}
\caption{}
%TODO Caption
\end{figure}

