\chapter{The Energy Momentum Tensor}
In special relativity we have seen that the energy momentum tensor
$\tensor{T}{^\mu^\nu}$ is conserved or divergencefree respectively, i.e.
\begin{equation}
\tensor{\partial}{_\mu}\tensor{T}{^\mu^\nu}=0\,.\label{eq:EMcons}
\end{equation}
The problem we are faced in general relativity is the very definition of local
energy. Because gravity shurely contributes to the energy, a problem arises
as we can always transform to local flat space.
We start by revisiting the example of dust
\begin{example}[Dust]
\begin{equation}
\tensor{T}{^\mu^\nu}=\rho_0\tensor{u}{^\mu}\tensor{u}{^\nu}\,.
\end{equation}
In special relativity:
${\tensor{u}{^\mu}=\od{\tensor{x}{^\mu}}{\tau}=\gamma(1,\vec{v})\transpose}$,
${\tensor{T}{^0^0}=\rho_0\left(\od{t}{\tau}\right)^2}=\gamma^2\rho_0:=\rho$.
Where $\rho$ is the density with respect to an observer at rest. For the Volume
we have $V=\gamma^{-1}V_0$. For the Energy $E=\gamma\omega_0$. Then the density
is given by $\rho=\frac{E}{V}=\gamma^2\rho_0$. The conservation of
$\tensor{T}{^0^\nu}$ implies
\begin{equation}
\begin{split}
0&=\tensor{T}{^0^\nu_{,\nu}}\\
&=\tensor{T}{^0^0_{,0}}+\tensor{T}{^0^i_{,i}}\\
&=\tensor{\partial}{_t}\left(\rho_0\gamma^2\right)
+\tensor{\partial}{_i}\left(\rho_0\gamma^2\tensor{v}{^i}\right)\\
&=\tensor{\partial}{_t}\rho
+\tensor{\partial}{_i}\left(\rho \tensor{v}{^i}\right)\\
&=\dot{\rho}
+\boldsymbol{\nabla}\left(\rho \vec{v}\right)\,,
\end{split}
\end{equation}
the \emph{continuity equation}. For the remaining spatial components we get
\begin{equation}
\begin{split}
0&=\tensor{T}{^i^\nu_{,\nu}}\\
&=\tensor{T}{^i^0_{,0}}+\tensor{T}{^j^i_{,i}}\\
&=\dot{\rho}\tensor{v}{^i}
+\rho\tensor{\dot{v}}{^i}
+\tensor{v}{^i}\tensor{\partial}{_j}\left(\rho\tensor{v}{^j}\right)
+\tensor{\dot{v}}{^j}\rho\tensor{\partial}{_j}\tensor{v}{^i}\\
&=\tensor{v}{^i}\left[\dot{\rho}+\tensor{\partial}{_j}\left(\rho\tensor{\dot{v}}{^j}\right)\right]
+\rho\left(\tensor{\dot{v}}{^i}+\tensor{\dot{v}}{^j}\tensor{\partial}{_j}\tensor{v}{^i}\right)\\
&=\rho\left(\tensor{\dot{v}}{^i}+\tensor{\dot{v}}{^j}\tensor{\partial}{_j}\tensor{v}{^i}\right)\,,
\end{split}
\end{equation}
the \emph{Euler equation} for vanishing pressure (which was the key assumption
for dust). It is natural to generalize equation \eqref{eq:EMcons} to curved
space
\begin{equation}
\tensor{T}{^\mu^\nu_{;\nu}}=0\,.
\end{equation}
In expanded form
\begin{equation}
\begin{split}
0
&=\tensor{\nabla}{_\nu}\left(\rho_0\tensor{u}{^\mu}\tensor{u}{^\nu}\right)\\
&=\tensor{{\rho_0}}{_{;\nu}}\tensor{u}{^\mu}\tensor{u}{^\nu}
+\rho_0\tensor{u}{^\mu_{;\nu}}\tensor{u}{^\nu}
+\rho_0\tensor{u}{^\mu}\tensor{u}{^\nu_{;\nu}}\\
&=\tensor{u}{^\mu}\left(\rho_0\tensor{u}{^\nu_{;\nu}}+\tensor{{\rho_0}}{_{;\nu}}\tensor{u}{^\nu}\right)
+\rho_0\tensor{u}{^\mu_{;\nu}}\tensor{u}{^\nu} \label{eq:DustEnCons}
\end{split}
\end{equation}
We multiply both sides with $\tensor{u}{_\mu}$
\begin{equation}
\begin{split}
0
&=-\left(\rho_0\tensor{u}{^\nu_{;\nu}}+\tensor{{\rho_0}}{_{;\nu}}\tensor{u}{^\nu}\right)
+\rho_0\tensor{u}{^\nu}\tensor{u}{_\mu}\tensor{u}{^\mu_{;\nu}}\\
&=-\left(\rho_0\tensor{u}{^\nu_{;\nu}}
+\tensor{{\rho_0}}{_{;\nu}}\tensor{u}{^\nu}\right)
\end{split}
\end{equation}
If we plugg this back into equation \eqref{eq:DustEnCons} we get
\begin{equation}
\begin{split}
0&=\tensor{u}{^\nu}\tensor{\nabla}{_\nu}\tensor{u}{^\mu}\\
&=\tensor{u}{^\nu}\tensor{\partial}{_\nu}\tensor{u}{^\mu}
+\tensor{u}{^\nu}\cSym{\mu}{\nu}{\rho}\tensor{u}{^\rho}\\
&=\dod{\tensor{x}{^\nu}}{\tau}\dpd{\tensor{u}{^\mu}}{\tensor{x}{^\nu}}
+\cSym{\mu}{\nu}{\rho}\dod{\tensor{x}{^\nu}}{\tau}\dod{\tensor{x}{^\nu}}{\tau}\\
&=\dpd[2]{\tensor{x}{^\mu}}{\tau}
+\cSym{\mu}{\nu}{\rho}\dod{\tensor{x}{^\nu}}{\nu}\dod{\tensor{x}{^\nu}}{\tau}\\
\end{split}
\end{equation}
The geodesic equation \eqref{eq:geodeq}
\end{example}
\begin{remark}
This is a diference between electrodynamics and general relativity; dust moves
on geodesics, i.e. the path is determined by the field equations alone. In contrast
in electrodynamics an additional Force (Lorentz force) has to be
\emph{postulated} to describe the motion of test particles. The case is not
settled however e.g. it is unclear wheater the paths of spin particles is also
determined by the field equations.
\end{remark}