\chapter{Linearized theory and Newtonian limit}
\section{Linearized theory}
Consider a weak gravitational field. Then we can split the full spacetime metric $\tensor{g}{_\mu_\nu}(x)$ into two parts.
\begin{definition}[Linearization of the metric field.]
    \begin{equation}
        \tensor{g}{_\mu_\nu}(x) = \tensor{\eta}{_\mu_\nu} +
        \tensor{h}{_\mu_\nu}(x) + \landauO(h^2) \, .
    \end{equation}
\end{definition}
Thereby $\tensor{h}{_\mu_\nu}$ is the flat, constant ``background'' metric of Minkowski
space, i.e.\ there is no gravitational field present.
The field $h_{\mu\nu}(x)$ can be interpreted as a perturbation on the fixed background $\eta_{\mu\nu}$.
One can identify a spin-2 particle, the so-called \emph{graviton}, with the excitations (quantized fluctuations) of this field.
Because only the linear order of $h$ is considered, the nonlinearity of Einstein's equations is lost.
We can raise and lower indices with $\tensor{h}{_\mu_\nu}$ and
$\tensor{h}{^\mu^\nu}$.

\begin{remark}
This works only for a weak gravitational field, since a strong gravitational field produces a strong back reaction of ``matter''
on the geometry, which follows from the nonlinearity of Einstein's equations.
Exactly this back reaction is neglected in the linearized theory.
\end{remark}

\subsection{Derivation of the linearized Einstein's equations}
In the following we neglect all terms with $\landauO(h^2)$.
Our goal is to express Einstein's field equations in the linearized approximation.
For this we need to calculate the Christoffel symbols, the Riemann tensor, the Ricci tensor and the Ricci scalar.

\subsubsection*{Christoffel symbols}
\begin{equation}
    \csym{\mu}{\nu}{\varrho} = \frac{1}{2} \left( \tensor{h}{_\mu_\varrho_,_\nu} + \tensor{h}{_\nu_\varrho_,_\mu}
    - \tensor{h}{_\mu_\nu_,_\varrho} \right) + \landauO(h^2) \, .
\end{equation}
Christoffel symbols of the second kind:
\begin{equation}
    \cSym{\varrho}{\mu}{\nu} = g^{\varrho\sigma} \csym{\mu}{\nu}{\sigma} = \eta^{\varrho\sigma} \csym{\mu}{\nu}{\sigma} + \landauO(h^2)
    = \frac{1}{2} \left( \tensor{h}{_\mu^\varrho_,_\nu} + \tensor{h}{_\nu^\varrho_,_\mu} - \tensor{h}{_\mu_\nu^,^\varrho} \right) + \landauO(h^2) \, .
\end{equation}
\subsubsection*{Riemann tensor}
The Riemann tensor can be calculated to
\begin{equation}
    \begin{split}
        \tensor{R}{^\varrho_\sigma_\mu_\nu}
        &= \partial_\mu \cSym{\varrho}{\nu}{\sigma} - \partial_\nu \cSym{\varrho}{\mu}{\sigma}
        + \underbrace{\cSym{\varrho}{\mu}{\lambda} \cSym{\lambda}{\nu}{\sigma} - \cSym{\varrho}{\nu}{\lambda} \cSym{\lambda}{\mu}{\sigma}}_{\landauO(h^2)} \\
        &= \frac{1}{2} \left( \tensor{h}{_\nu^\varrho_,_\sigma_\mu} +
        {\tensor{h}{_\sigma^\varrho_,_\nu_\mu}} - \tensor{h}{_\nu_\sigma^,^\varrho_\mu} - \tensor{h}{^\varrho_\mu_,_\sigma_\nu} -
        {\tensor{h}{_\sigma^\varrho_,_\mu_\nu}} +
        \tensor{h}{_\mu_\sigma^,^\varrho_\nu} \right) + \landauO(h^2) \\
        &= \frac{1}{2} \left( \tensor{h}{_\nu^\varrho_,_\sigma_\mu} - \tensor{h}{_\nu_\sigma^,^\varrho_\mu}
        - \tensor{h}{^\varrho_\mu_,_\sigma_\nu} +
        \tensor{h}{_\mu_\sigma^,^\varrho_\nu} \right) + \landauO(h^2) \, .
    \end{split}
\end{equation}
By contracting we get the Ricci tensor
\begin{equation}
    \tensor{R}{_\sigma_\nu} = \tensor{R}{^\varrho_\sigma_\varrho_\nu}
    = \frac{1}{2} \left( \tensor{h}{_\nu^\varrho_,_\sigma_\varrho} - \tensor{h}{_\nu_\sigma^,^\varrho_\varrho}
    - \tensor{h}{_,_\sigma_\nu} + \tensor{h}{_\varrho_\sigma^,^\varrho_\nu} \right) + \landauO(h^2)\, ,
\end{equation}
where for convenience the trace of $h$ is denoted with $h\coloneqq
h_{\mu\nu}\eta^{\mu\nu}$. Lastly the Ricci scalar is given by
\begin{equation}
    R = g^{\sigma\nu}R_{\sigma\nu} = \eta^{\sigma\nu}R_{\sigma\nu} + \landauO(h^2)
    = \tensor{h}{^\sigma^\nu_,_\sigma_\nu} - \tensor{h}{_,_\sigma^\sigma} + \landauO(h^2)
\end{equation}
We define
\begin{equation}
    \overline{h}_{\mu\nu} \coloneqq h_{\mu\nu} - \frac{1}{2} \eta_{\mu\nu}h\,.
\end{equation}
The trace is given by
\begin{equation}
    \overline{h} \coloneqq  \overline{h}_{\mu\nu}\eta^{\mu\nu} = h -   
    \frac{h}{2} = \frac{h}{2}\,.
\end{equation}
If we repeat the procedure we arrive at the initial metric:
\begin{equation}
    \overline{\overline{h}}_{\mu\nu} = \overline{h}_{\mu\nu} - \frac{1}{2}
    \eta_{\mu\nu}\overline{h} = \overline{h}_{\mu\nu} + \frac{1}{2}
    \eta_{\mu\nu}h= h_{\mu\nu}\,.
\end{equation}
%TODO stimmt die Korrektur?
% \begin{align}
%     \overline{h} \coloneqq &\ \overline{h}_{\mu\nu}\eta^{\mu\nu} = h - 2h = -h \\
%     h_{\mu_\nu} =&\ \overline{h}_{\mu\nu} + \frac{1}{2} \eta_{\mu\nu}h = \overline{h}_{\mu\nu} - \frac{1}{2} \eta_{\mu\nu}\overline{h}
% \end{align}
\subsubsection{Linearized Einstein tensor \texorpdfstring{$G_{\mu\nu}$}{Gmunu} in terms of \texorpdfstring{$\overline{h}_{\mu\nu}$}{hbarmunu}}
The linearized Einstein Tensor is given as
\begin{equation}
    \begin{split}
        G_{\mu\nu}^{\text{(L)}} =\ & R_{\mu\nu}^{\text{(L)}} - \frac{1}{2} \eta_{\mu\nu} R^{\text{(L)}} \\
        =\ & \frac{1}{2} \partial_\mu \partial_\varrho \tensor{h}{_\nu^\varrho} + \frac{1}{2} \partial_\nu \partial_\varrho \tensor{h}{_\mu^\varrho}
        - \frac{1}{2} \Box h_{\mu\nu} - \frac{1}{2}
        \partial_{\mu}\partial_\nu h-\frac{1}{2}
        \eta_{\mu\nu}\partial_\varrho\partial_\sigma h^{\varrho\sigma} + \frac{1}{2} \eta_{\mu\nu}\Box h \\
        =\ & \frac{1}{2} \partial_\mu\partial_\varrho \tensor{\overline{h}}{_\nu^\varrho}
        - {\frac{1}{4}\partial_\mu\partial_\nu\overline{h}}
        + \frac{1}{2} \partial_\nu\partial_\varrho\tensor{\overline{h}}{_\mu^\varrho}
        - {\frac{1}{4}\partial_\nu\partial_\mu\overline{h}} - \frac{1}{2}\Box\overline{h}_{\mu\nu} \\
        & + {\frac{1}{2}\eta_{\mu\nu}\Box\overline{h}} + {\frac{1}{2}\partial_\mu\partial_\nu\overline{h}}
        - \frac{1}{2}\eta_{\mu\nu}\partial_\varrho\partial_\sigma\overline{h}^{\varrho\sigma}
        + {\frac{1}{4}\eta_{\mu\nu}\Box\overline{h}} - {\frac{1}{2}\eta_{\mu\nu}\Box\overline{h}} \\
        =\ & -\frac{1}{2} \Box \overline{h}_{\mu\nu} + \partial_\varrho \tensor{\partial}{_(_\mu}\tensor{\overline{h}}{_\nu_)^\varrho}
        - \frac{1}{2} \eta_{\mu\nu}\partial_\varrho\partial_\sigma
        \overline{h}^{\varrho\sigma}\,,
    \end{split}
\end{equation} 
% \begin{equation}
%     \begin{split}
%         G_{\mu\nu}^{\text{(L)}} =\ & R_{\mu\nu}^{\text{(L)}} - \frac{1}{2} \eta_{\mu\nu} R^{\text{(L)}} \\
%         =\ & \frac{1}{2} \partial_\mu \partial_\varrho \tensor{h}{_\nu^\varrho} + \frac{1}{2} \partial_\nu \partial_\varrho \tensor{h}{_\mu^\varrho}
%         - \frac{1}{2} \Box h_{\mu\nu} - \frac{1}{2}
%         \partial_{\mu}\partial_\nu h-\frac{1}{2}
%         \eta_{\mu\nu}\partial_\varrho\partial_\sigma h^{\varrho\sigma} + \frac{1}{2} \eta_{\mu\nu}\Box h \\
%         =\ & \frac{1}{2} \partial_\mu\partial_\varrho \tensor{\overline{h}}{_\nu^\varrho}
%         - \mathunderline{blue}{\frac{1}{4}\partial_\mu\partial_\nu\overline{h}}
%         + \frac{1}{2} \partial_\nu\partial_\varrho\tensor{\overline{h}}{_\mu^\varrho}
%         - \mathunderline{blue}{\frac{1}{4}\partial_\nu\partial_\mu\overline{h}} - \frac{1}{2}\Box\overline{h}_{\mu\nu} \\
%         & + \mathunderline{green}{\frac{1}{2}\eta_{\mu\nu}\Box\overline{h}} + \mathunderline{blue}{\frac{1}{2}\partial_\mu\partial_\nu\overline{h}}
%         - \frac{1}{2}\eta_{\mu\nu}\partial_\varrho\partial_\sigma\overline{h}^{\varrho\sigma}
%         + \mathunderline{green}{\frac{1}{4}\eta_{\mu\nu}\Box\overline{h}} - \mathunderline{green}{\frac{1}{2}\eta_{\mu\nu}\Box\overline{h}} \\
%         =\ & -\frac{1}{2} \Box \overline{h}_{\mu\nu} + \partial_\varrho \tensor{\partial}{_(_\mu}\tensor{\overline{h}}{_\nu_)^\varrho}
%         - \frac{1}{2} \eta_{\mu\nu}\partial_\varrho\partial_\sigma \overline{h}^{\varrho\sigma} \\
%         \overset{!}{=}\ & \kappa T_{\mu\nu}
%     \end{split}
% \end{equation}
%TODO introduce symmetration brackets somewhere
%TODO Vorzeichen der Transformation inkosistent!!!
with the (linearized) d'Alembert operator
\begin{equation}
    \Box^{\text{(L)}} = \Box = \partial_\mu\partial_\nu
    \eta^{\mu\nu}=\partial_\mu\partial^\mu\,.
\end{equation}
\begin{definition}[Linearized Einstein equations]
    \begin{equation}
        \label{eq:lineinsteineqs}
        -\frac{1}{2} \Box \overline{h}_{\mu\nu} + \partial_\varrho \tensor{\partial}{_(_\mu}\tensor{\overline{h}}{_\nu_)^\varrho}
        - \frac{1}{2} \eta_{\mu\nu}\partial_\varrho\partial_\sigma \overline{h}^{\varrho\sigma} = \kappa T_{\mu\nu}
    \end{equation}
\end{definition}

\subsubsection{Gauge transformations}
Usually field equations are in the form of
\begin{equation}
    \Box \text{``field''} = \text{``source''}\,.
\end{equation}
Equation~\eqref{eq:lineinsteineqs} can be written in this form:
\begin{equation}
    \underbrace{\Box \overline{h}_{\mu\nu}}_{\Box\text{``field''}}
    \underbrace{- 2 \partial_\varrho \tensor{\partial}{_(_\mu} \tensor{\overline{h}}{_\nu_)^\varrho}
    + \eta_{\mu\nu}\partial_\varrho \overline{h}^{\varrho\sigma}}_{\text{ensures gauge invariance of equation}}
    = \underbrace{-2\kappa T_{\mu\nu}}_{\text{``source''}}
\end{equation}

%TODO Michi: mathematical background \\%TODO mathematical background @Michi
We are now considering infinitesimal diffeomorphisms, which are given by affine
transformations
\begin{equation}
    x^\mu = x'^\mu + \xi^\mu(x'^\mu), \qquad \xi^\mu \ll 1
\end{equation}
In the following we neglect terms with $\landauO(\xi^2)$, $\landauO(\xi h)$, and
$\landauO(h^2)$ and higher order terms, which we denote by $\landauO$.
The transformed metric reads
\begin{equation}
    \begin{split}
        \eta_{\mu\nu} + h'_{\mu\nu}(x') &= g'_{\mu\nu} \\
        &= \frac{\partial x^\varrho}{\partial x'^\mu} \frac{\partial x^\sigma}{\partial x'^\nu} g_{\varrho\sigma}(x) \\
        &= \frac{\partial \left( x'^\varrho + \xi^\varrho \right)}{\partial x'^\mu}
        \frac{\partial \left( x'^\sigma + \xi^\sigma \right)}{\partial x'^\nu}
        \left( \eta_{\varrho\sigma} + h_{\varrho\sigma}(x) \right) + \landauO \\
        &= \left( \tensor{\delta}{_\mu^\varrho} + \tensor{\xi}{^\varrho_,_\mu} \right)
        \left( \tensor{\delta}{_\nu^\sigma} + \tensor{\xi}{^\sigma_,_\nu} \right)
        \left( \eta_{\varrho\sigma} + h_{\varrho\sigma}(x) \right) + \landauO \\
        &= \left( \tensor{\delta}{_\mu^\varrho} + \tensor{\xi}{^\varrho_,_\mu} \right)
        \left( \eta_{\varrho\nu} + h_{\varrho\nu} + \tensor{\xi}{_\varrho_,_\nu}
        \right) + \landauO \\
        &= \eta_{\mu\nu} + h_{\mu\nu} + \tensor{\xi}{_\mu_,_\nu} +
        \tensor{\xi}{_\nu_,_\mu} + \landauO\,.
    \end{split}
\end{equation}
The perturbation $h_{\mu\nu}$ therefore transforms  under infinitesimal
diffeomorphisms in the following way
\begin{equation}
    \begin{split}
        h'_{\mu\nu}(x) &= h_{\mu\nu}(x) + \tensor{\xi}{_\mu_,_\nu} + \tensor{\xi}{_\nu_,_\mu} \\
        &= h_{\mu\nu}(x) + \left(\liedif{\xi}{\eta} \right)_{\mu\nu}\,.
    \end{split}
\end{equation}
\begin{definition}[Lie derivative]
    The Lie derivative off a tensor field $T$ with $k$ contravariant and $l$
    covariant indices along the vector $\xi$ is defined as
    \begin{equation}
        \begin{split}
            \left( \liedif{\xi}{T} \right)^{\alpha_1\ldots\alpha_k}_{\beta_1\ldots\beta_l}
            \coloneqq \xi^\mu \partial_\mu T^{\alpha_1\ldots\alpha_k}_{\beta_1\ldots\beta_l}
            & - \left( \partial_\mu \xi^{\alpha_1} \right) T^{\mu\alpha_2\ldots\alpha_k}_{\beta_1\ldots\beta_l} - \ldots
            - \left( \partial_\mu \xi^{\alpha_k} \right) T^{\alpha_1\ldots\alpha_{k-1}\mu}_{\beta_1\ldots\beta_l} \\
            & + \left( \partial_{\beta_1} \xi^\mu \right) T^{\alpha_1\ldots\alpha_k}_{\mu\beta_2\ldots\beta_l} + \ldots
            +  \left( \partial_{\beta_l} \xi^\mu \right)
            T^{\alpha_1\ldots\alpha_k}_{\beta_1\ldots\beta_{l-1}\mu}\,.
        \end{split}
    \end{equation}
\end{definition}
Therefore
\begin{equation}
    \left( \liedif{\xi}{\eta} \right)_{\mu\nu} = \underbrace{\xi^\varrho\partial_\varrho\eta_{\mu\nu}}_{=0}
    + \tensor{\xi}{_\mu_,_\nu} + \tensor{\xi}{_\nu_,_\mu} =
    \tensor{\xi}{_\mu_,_\nu} + \tensor{\xi}{_\nu_,_\mu}\,.
\end{equation}
If the derivative of a metric vanishes for a given $\xi^\mu$, then one obtains
the killing equations 
\begin{equation}
    \tensor{\xi}{_\mu_,_\nu} + \tensor{\xi}{_\nu_,_\mu}=0  %TODO order of indices in second term?
\end{equation}
for $\xi^\mu$ and the solutions are referred to as \emph{killing vector fields}.
In Minkowski-space the ten infinitesimal killing vectors correspond to the Poincaré-generators.
\begin{sidenote}
We can use the Lie-derivative on metric to detect symmetries of the Manifold.
\end{sidenote}
\subsubsection{Invariance of the linearized field equations under infinitesimal
diffeomorphisms} 
We now check that linearized field equations are invariant under infinitesimal
diffeomorphism
\begin{equation}
    h'_{\mu\nu} = h_{\mu\nu} + \tensor{\xi}{_\mu_,_\nu} +
    \tensor{\xi}{_\nu_,_\mu}\,.
\end{equation}
The barred metric transforms as
\begin{equation}
    \begin{split}
        \overline{h'}_{\mu\nu} &= h'_{\mu\nu} - \frac{1}{2} \eta_{\mu\nu}h' \\
        &= h_{\mu\nu} + \tensor{\xi}{_\mu_,_\nu} + \tensor{\xi}{_\nu_,_\mu} - \frac{1}{2} \eta_{\mu\nu}h
        -\frac{1}{2}\eta_{\mu\nu}\partial^\varrho\xi_\varrho - \frac{1}{2} \eta_{\mu\nu}\partial^\varrho\xi_\varrho \\
        &= \overline{h}_{\mu\nu} + \tensor{\xi}{_\mu_,_\nu} +
        \tensor{\xi}{_\nu_,_\mu} -
        \eta_{\mu\nu}\tensor{\xi}{^\varrho_,_\varrho}\,.
    \end{split}
\end{equation}
We proceed by plugging this into Einstein's equations, the relevant terms are
\begin{align}
    -\frac{1}{2}\Box \overline{h'}_{\mu\nu} &= -\frac{1}{2}\Box\overline{h}_{\mu\nu} - \frac{1}{2}\Box\tensor{\xi}{_\mu_,_\nu}
    -\frac{1}{2}\Box\tensor{\xi}{_\nu_,_\mu} + \frac{1}{2}
    \eta_{\mu\nu}\Box\tensor{\xi}{^\varrho_,_\varrho}\,, \\
    -\frac{1}{2} \eta_{\mu\nu}\partial_\varrho\partial_\sigma \overline{h'}^{\varrho\sigma} &=
    -\frac{1}{2} \eta_{\mu\nu}\partial_\varrho\partial_\sigma \left( \tensor{\xi}{^\varrho^,^\sigma} + \tensor{\xi}{^\sigma^,^\varrho}
    - \eta^{\varrho\sigma} \tensor{\xi}{^\alpha_,_\alpha} \right) - \frac{1}{2}
    \eta_{\mu\nu}\partial_\varrho\partial_\sigma \overline{h}^{\varrho\sigma}\,,
    \\
    \partial^\varrho \tensor{\partial}{_(_\mu} \tensor{\overline{h'}}{_\nu_)_\varrho} &=
    \partial^\varrho \tensor{\partial}{_(_\mu} \tensor{\overline{h}}{_\nu_)_\varrho} + \frac{1}{2}\Box\tensor{\xi}{_\nu_,_\mu}
    + \frac{1}{2}\Box\eta_{\mu\nu}\,.
\end{align}
Therefore
\begin{equation}
    \begin{split}
        & -\frac{1}{2} \Box \overline{h'}_{\mu\nu} + \partial_\varrho \tensor{\partial}{_(_\mu}\tensor{\overline{h'}}{_\nu_)^\varrho}
        - \frac{1}{2} \eta_{\mu\nu}\partial_\varrho\partial_\sigma \overline{h'}^{\varrho\sigma} \\
        =\ & -\frac{1}{2}\Box\overline{h}_{\mu\nu} - {\frac{1}{2}\Box\tensor{\xi}{_\mu_,_\nu}}
        -{\frac{1}{2}\Box\tensor{\xi}{_\nu_,_\mu}}
        + {\frac{1}{2} \eta_{\mu\nu}\Box\tensor{\xi}{^\varrho_,_\varrho}}
        + \partial^\varrho \tensor{\partial}{_(_\mu}\tensor{\overline{h}}{_\nu_)_\varrho} \\
        & + {\Box\tensor{\xi}{_(_\mu_,_\nu_)}}
        - {\frac{1}{2}\eta_{\mu\nu}\Box\tensor{\xi}{^\varrho_,_\varrho}}
        - \frac{1}{2} \eta_{\mu\nu}\partial_\varrho\partial_\sigma \overline{h}^{\varrho\sigma} \\
        =\ & -\frac{1}{2} \Box \overline{h}_{\mu\nu} + \partial_\varrho \tensor{\partial}{_(_\mu}\tensor{\overline{h}}{_\nu_)^\varrho}
        - \frac{1}{2} \eta_{\mu\nu}\partial_\varrho\partial_\sigma \overline{h}^{\varrho\sigma}
    \end{split}
\end{equation}
This shows that the Einstein equations are invariant under an infinitesimal diffeomorphisms.
Therefore $\overline{h}_{\mu\nu}$ and $\overline{h'}_{\mu\nu}$ are the same \emph{physical} field.

\subsubsection{Harmonic gauge in linearized gravity}
As described above we want to bring the field equation in the form
$\Box\text{``field''}=\text{``source''}$, i.e. a wave equation.
This can ge done with the gauge condition
\begin{equation}
    \chi_\nu \left[ \overline{h} \right] \coloneqq \partial^\mu \overline{h}_{\mu\nu} = 0.
\end{equation}
In therms of the original field this condition reads
\begin{definition}[de Donder gauge, harmonic gauge]
    \begin{equation}
        \chi_\nu \left[ h \right] = \partial^\mu h_{\mu\nu} - \frac{1}{2} h_{\mu\nu} \partial^\mu h = 0
    \end{equation}
\end{definition}
Proof: %TODO of what?
\begin{equation}
    \begin{split}
        \partial^\mu \overline{h'}_{\mu\nu} &= \partial^\mu \overline{h}_{\mu\nu} + \Box \xi_\nu + \partial_\nu \partial^\mu \xi_\mu -
        \eta_{\mu\nu} \partial^\mu\partial_\varrho\xi^\varrho \\
        &= \partial^\mu \overline{h}_{\mu\nu} + \Box \xi_\nu = 0
    \end{split}
\end{equation}
Solve for $\Box\xi_\nu$
\begin{equation}
    \implies \Box \overline{h'}_{\mu\nu} = -2\kappa T_{\mu\nu}
\end{equation}
Since $\overline{h}_{\mu\nu}$ and $\overline{h'}_{\mu\nu}$ correspond to the same physical field configuration, we can drop the prime.
\begin{definition}{Linearized field equations in de Donder gauge.}
    \begin{equation}
        \Box \overline{h}_{\mu\nu} = - 2 \kappa T_{\mu\nu}
\end{equation}
\end{definition}
\afterpage{
\clearpage
\thispagestyle{empty}
\begin{landscape}
    \begin{table}[h]
        \caption{Comparison between linearized gravity and electrodynamics.}
        \centering
        \begin{tabulars}{lll}
            \toprule
            & linearized gravity & electrodynamics \\
            \midrule

            basic field
            & $\overline{h}_{\mu\nu}$ ($h_{\mu\nu}$), spin-2, \emph{graviton}
            & $A_\mu$, spin-1, \emph{gauge boson}, \emph{gauge potential}
            \\%Photon????


            field equations
            & $ \underbrace{\Box \overline{h}_{\mu\nu}}_{\Box\text{``field''}} - \underbrace{2 \partial_\rho \tensor{\partial}{_(_\mu}\tensor{\overline{h}}{_\nu_)^\rho} - \eta_{\mu\nu}\partial_\rho\partial_\sigma \overline{h}^{\rho\sigma}}_{\text{ensures gauge inv.}} = -\underbrace{2 \kappa T_{\mu\nu}}_{\text{``source''}}$
            & $\underbrace{\Box A_\mu}_{\Box\text{``field''}} - \underbrace{\partial_\mu \left( \partial_\nu A^\nu \right)}_{\text{ensures gauge inv.}} = - \underbrace{4 \pi j_\mu}_{\text{source}}$ \\

            transf. under inf. gauge trafos
            & $\overline{h'}_{\mu\nu} = h_{\mu\nu}+\tensor{\xi}{_\mu_,_\nu} + \tensor{\xi}{_\nu_,_\mu} - \eta_{\mu\nu} \tensor{\xi}{^\rho_,_\rho}$
            & $A'_\mu = A_\mu + \partial_\mu\lambda(x)$\\

            & inf. coordinate transformation
            & internal symmetry \\

            inv. of field eqs
            & yes
            & yes \\

            specific gauges
            & de Donder gauge, $\partial_\mu \overline{h}^{\mu\nu}=0$
            & Lorentz gauge, $\partial_\mu A^\mu = 0$ \\

            field eqs. in specific gauges
            & $\Box \overline{h}_{\mu\nu} = - 2 \kappa T_{\mu\nu}$
            & $\Box A_\mu = - 4 \pi j_\mu$ \\

            inv. tensors under gauge trafo
            & $\tensor{{R^{(L)\prime}}}{^\rho_\mu_\nu_\rho} =
             \tensor{{R^{(L)}}}{^\rho_\mu_\nu_\rho}$ 
%TODO Ricci statt Riemann?
            & $F'_{\mu\nu} =F_{\mu\nu}$\\
            \bottomrule
        \end{tabulars}
    \end{table}
\end{landscape}
}

\begin{remark}[Fierz-Pauli action, 1939]
\begin{equation}
    \lagrangian_{\text{FP}} =
    \frac{1}{2} \left( \partial_\mu h^{\mu\nu} \right) \left( \partial_\nu h \right)
    - \partial_\mu h^{\rho \sigma} \partial_\rho \tensor{h}{^\mu_\sigma}
    + \frac{1}{2} \eta^{\mu\nu} \left( \partial_\mu h^{\rho\sigma} \right) \left( \partial_\nu h_{\rho \sigma} \right)
    - \frac{1}{2} \eta^{\mu\nu} \left( \partial_\mu h \right) \left( \partial_\nu h \right)
\end{equation}
For vacuum this is the Lagrangian of a massless spin-2 field $h_{\mu\nu}(x)$ (``the graviton'') in flat spacetime $h^{\mu\nu}$. \\
Problem: non-linearity (in electrodynamics: linear coupling)
\begin{equation}
    T_{\mu\nu}h^{\mu\nu} \rightarrow h_{\mu\nu}^{(2)} \propto \left( h_{\mu\nu}^{(1)}  \right)^2
\end{equation}
$\rightarrow$ Deser 1970: Iterative procedure \\
$\hookrightarrow$ including gravitational self energy and resuming one recovers the full nonlinear Einstein equations.
\end{remark}

\newpage

\section{Newtonian Limit}
Empirically we know
\begin{enumerate}
    \item Newtonian gravity describes the dynamics in our solar system to a high accuracy
    \item On earth, we can measure the gravitation constant $G_\text{N}$ e.g.\ by Cavendish-type  experiments
\end{enumerate}
If General Relativity is a more fundamental gravitational theory than Newton's theory it should
\begin{enumerate}
    \item recover \name{Newton}'s theory in appropriate limit, i.e.\ in the domain where Newtonian Gravity is a good description
    \item be more accurate than Newton's theory, i.e.\ it should predict small corrections to Newtonian Gravity.
\end{enumerate}
Conditions for the Newtonian limit:
\begin{enumerate}
    \item $v \ll c$ (sources move slowly) \\
    slowly changing geometry $\approx$ static: no $\dif{x}^i\dif(t)$ terms in $\dif{s}^2$ (would violate $t\rightarrow -t$ invariance)
    \item $g_{\mu\nu} = \eta_{\mu\nu} + h_{\mu\nu}$ with $\abs{h_{\mu\nu}} \ll 1$ (weak gravitational field)
    \item $p \ll \rho$ (sources have low internal pressure)
\end{enumerate}
\begin{enumerate}[{ad} 1.]
    \item $v\ll c$ is required as (special) relativistic effects must be small
    \item Consider the solar system as a closed system: Then a particle in the outer region with $v\ll c$ initially,
    will fall into the inner region (center of mass) and it will be accelerated by gravity. It will be then have a kinetic energy
    $E_\text{kin} = \frac{1}{2} m v^2 \sim \abs{m \Phi}$, where $\Phi < 0$ is the gravitational Newtonian potential with boundary condition
    $\Phi(\infty) = 0$. Small velocities of the sources imply weak gravitational fields.
    \item Speed of sound
    \begin{align}
        & c_s \coloneqq \abs{\frac{T_{ij}}{T_{00}}} \quad \text{with} \quad T_{\mu\nu} = \diag \left( \rho, \frac{p}{c^2}, \frac{p}{c^2}, \frac{p}{c^2} \right) \quad \text{(perfect fluid)} \\
        & c_s \sim \left( \frac{p}{\rho} \right)^{1/2}
    \end{align}
    The internal pressure of the sources must be small, otherwise they would also create (fast) motion of sound waves. \\
    $\implies p \ll \rho$ \\
    $\implies$ energy-momentum tensor of dust
    \begin{align}
        T^{\mu\nu} &= \rho_0 t^\mu t^\nu \\
        t^\mu &= \delta^\mu_0 = \left( \frac{\partial}{\partial x^0} \right)^\mu
    \end{align}
    $t^\mu$ is the ``direction'' of an internal coordinate system of time
    \begin{equation}
        \Box \overline{h}_{\mu\nu} \approx \Delta \overline{h}_{\mu\nu}
    \end{equation}
    alternatively $\Box = - \frac{1}{c^2} \frac{\partial^2}{\partial t^2} + \Delta \approx \Delta$ as 
    $\frac{1}{c} \frac{\partial}{\partial t} = \frac{1}{c} \frac{\partial}{\partial x} \frac{\partial x}{\partial t} 
    \sim \frac{v}{c} \frac{\partial}{\partial x} \ll \frac{\partial}{\partial x}$
\end{enumerate}

For the Newtonian limit, we must look for solutions to $\Box \overline{h}_{\mu\nu} = - 2 \kappa T_{\mu\nu}$, where time-derivatives are 
negligible and where the energy-momentum tensor is the one of dust. 
\begin{equation}
    \Delta \overline{h}_{\mu\nu} = 
    \begin{cases}
    	-2 \kappa \rho_0 & \mu=\nu=0 \\
    	0 & \text{else}
    \end{cases}
\end{equation}
Consider first the \name{Poisson} equation with vanishing sources. 
The unique solution is $\overline{h}_{\mu\nu} = \const$. The $\const$ can be always adjusted to zero by a residual gauge transformation:
\begin{equation}
    \overline{h}_{\mu\nu} = 0, \quad \mu \neq \nu = 0
\end{equation}
Residual gauge transformation
\begin{equation}
    \partial^\mu \overline{h'}_{_\mu\nu} = \underbrace{\partial^\mu \overline{h}_{\mu\nu}}_{=0} + \Box \xi_\nu = 0
\end{equation}
This means that all gauge transformations $\xi_\mu$ with $\Box \xi_\mu = 0$ are compatible with the de Donder gauge 
(i.e.\ that doesn't lead out of the de Donder gauge). \\
As far we know:
\begin{align}
    \overline{h}_{\mu\nu} &= 0 \qquad \text{for the 0-0 component } \\
    \Delta \overline{h}_{00} &= - 2\kappa \rho_0 \qquad \mu \neq \nu = 0
\end{align}
We identify the gravitational potential as
\begin{equation}
    \Phi \coloneqq -\frac{1}{4} \overline{h}_{00}
\end{equation}
We obtain the \name{Poisson} equation
\begin{equation}
    \Delta \Phi = \frac{\kappa}{2} \rho_0 = 4 \pi G_\text{N} \rho_0
\end{equation}
Solution in terms of the original field $h_{\mu\nu}$:
\begin{equation}
    h_{\mu\nu} = \overline{h}_{\mu\nu} - \frac{1}{2} \eta_{\mu\nu} \overline{h} 
    = \overline{h}_{00} \left( \tensor{\delta}{_\mu^0} \tensor{\delta}{_\nu^0} + \frac{1}{2} \eta_{\mu\nu} \right) 
    = - 4\Phi \left( \tensor{\delta}{_\mu^0} \tensor{\delta}{_\nu^0} + \frac{1}{2} \eta_{\mu\nu} \right)
\end{equation}
We have used $\overline{h}=\eta^{\rho\sigma}\overline{h}_{\rho\sigma} = -\overline{h}_{00} = 4 \Phi$
\begin{align}
    & \implies h_{00}=-2\Phi \qquad h_{ij}=-2\Phi\delta_{ij} \qquad h_{0\mu}=0 \\
    & \implies \dif{s}^2 = g_{\mu\nu} \dif{x}^\mu \dif{x}^\nu = - (1+2\Phi)\dif{t}^2 + (1-2\Phi)\delta_{ij} \dif{x}^i \dif{x}^j
\end{align}
This is the Newtonian geometry.
%TODO ref to first appearance newt. geo.
\subsection{Motion of test particles in Newtonian Geometry}