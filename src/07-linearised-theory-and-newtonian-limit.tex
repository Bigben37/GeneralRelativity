\chapter{Linearized theory and Newtonian limit}
\section{Linearized theory}
Consider a weak gravitational field. Then we can split the full spacetime metric $g_{\mu\nu}(x)$ into two parts.
\begin{definition}[Linearization of the metric field.]
    \begin{equation}
        g_{\mu\nu}(x) = \eta_{\mu\nu} + h_{\mu\nu}(x) + \landauO(h^2) \, .
    \end{equation}
\end{definition}
$\eta_{\mu\nu}$ is the flat, constant ``background'' metric of Minkowski space, i.e.\ there is no gravitational field present.
The field $h_{\mu\nu}(x)$ can be interpreted as a perturbation on the fixed background $\eta_{\mu\nu}$.
One can identify a spin-2 particle, the so-called \emph{graviton}, with the excitations (quantized fluctuations) of this field.
Because only the linear order of $h$ is considered, the nonlinearity of Einstein's equations is lost.
We can raise and lower indices with $\eta_{\mu\nu}$ and $\eta^{\mu\nu}$.

\begin{remark}
This works only for a weak gravitational field, since a strong gravitational field produces a strong back reaction of ``matter''
on the geometry, which follows from the nonlinearity of Einstein's equations.
Exactly this back reaction is neglected in the linearized theory.
\end{remark}

\subsection{Derivation of the linearized Einstein's equations}
In the following we neglect all terms with $\landauO(h^2)$.
Our goal is to express Einstein's field equations in the linearized approximation.
For this we need to calculate the Christoffel symbols, the Riemann tensor, the Ricci tensor and the Ricci scalar.

Christoffel symbols of the first kind:
\begin{equation}
    \csym{\mu}{\nu}{\varrho} = \frac{1}{2} \left( \tensor{h}{_\mu_\varrho_,_\nu} + \tensor{h}{_\nu_\varrho_,_\mu}
    - \tensor{h}{_\mu_\nu_,_\varrho} \right) + \landauO(h^2) \, .
\end{equation}
Christoffel symbols of the second kind:
\begin{equation}
    \cSym{\varrho}{\mu}{\nu} = g^{\varrho\sigma} \csym{\mu}{\nu}{\sigma} = \eta^{\varrho\sigma} \csym{\mu}{\nu}{\sigma} + \landauO(h^2)
    = \frac{1}{2} \left( \tensor{h}{_\mu^\varrho_,_\nu} + \tensor{h}{_\nu^\varrho_,_\mu} - \tensor{h}{_\mu_\nu^,^\varrho} \right) + \landauO(h^2) \, .
\end{equation}
Riemann tensor:
\begin{equation}
    \begin{split}
        \tensor{R}{^\varrho_\sigma_\mu_\nu}
        &= \partial_\mu \cSym{\varrho}{\nu}{\sigma} - \partial_\nu \cSym{\varrho}{\mu}{\sigma}
        + \underbrace{\cSym{\varrho}{\mu}{\lambda} \cSym{\lambda}{\nu}{\sigma} - \cSym{\varrho}{\nu}{\lambda} \cSym{\lambda}{\mu}{\sigma}}_{\landauO(h^2)} \\
        &= \frac{1}{2} \left( \tensor{h}{_\nu^\varrho_,_\sigma_\mu} + \mathunderline{blue}{\tensor{h}{^\varrho_\sigma_,_\nu_\mu}} - \tensor{h}{_\nu_\sigma^,^\varrho_\mu}
        - \mathunderline{blue}{\tensor{h}{^\varrho_\mu_,_\sigma_\nu}} + \tensor{h}{^\varrho_\sigma_,_\mu_\nu} + \tensor{h}{_\mu_\sigma^,^\varrho_\nu} \right) + \landauO(h^2) \\
        &= \frac{1}{2} \left( \tensor{h}{_\nu^\varrho_,_\sigma_\mu} - \tensor{h}{_\nu_\sigma^,^\varrho_\mu}
        - \tensor{h}{^\varrho_\sigma_,_\mu_\nu} + \tensor{h}{_\mu_\sigma^,^\varrho_\nu} \right) + \landauO(h^2) \, .
    \end{split}
\end{equation}
%TODO color does not fit
Ricci tensor
\begin{equation}
    \tensor{R}{_\sigma_\nu} = \tensor{R}{^\varrho_\sigma_\varrho_\nu}
    = \frac{1}{2} \left( \tensor{h}{_\nu^\varrho_,_\sigma_\varrho} - \tensor{h}{_\nu_\sigma^,^\varrho_\varrho}
    - \tensor{h}{_,_\sigma_\nu} + \tensor{h}{_\varrho_\sigma^,^\varrho_\nu} \right) + \landauO(h^2)\, ,
\end{equation}
with $h\coloneqq h_{\mu\nu}\eta^{\mu\nu}$

Ricci scalar
\begin{equation}
    R = g^{\sigma\nu}R_{\sigma\nu} = \eta^{\sigma\nu}R_{\sigma\nu} + \landauO(h^2)
    = \tensor{h}{^\sigma^\nu_,_\sigma_\nu} - \tensor{h}{_,_\sigma^\sigma} + \landauO(h^2)
\end{equation}
We define
\begin{equation}
    \overline{h}_{\mu\nu} \coloneqq h_{\mu\nu} - \frac{1}{2} \eta_{\mu_\nu}h\,.
\end{equation}
With the following two lines we can show that $\overline{\overline{h}} = h$:
\begin{align}
    \overline{h} \coloneqq &\ \overline{h}_{\mu\nu}\eta^{\mu\nu} = h - 2h = -h \\
    h_{\mu_\nu} =&\ \overline{h}_{\mu\nu} + \frac{1}{2} \eta_{\mu\nu}h = \overline{h}_{\mu\nu} - \frac{1}{2} \eta_{\mu\nu}\overline{h}
\end{align}

\subsubsection{Linearized Einstein tensor \texorpdfstring{$G_{\mu\nu}$}{Gmunu} in terms of \texorpdfstring{$\overline{h}_{\mu\nu}$}{hbarmunu}}

\begin{equation}
    \begin{split}
        G_{\mu\nu}^{\text{(L)}} =\ & R_{\mu\nu}^{\text{(L)}} - \frac{1}{2} \eta_{\mu\nu} R^{\text{(L)}} \\
        =\ & \frac{1}{2} \partial_\mu \partial_\varrho \tensor{h}{_\nu^\varrho} + \frac{1}{2} \partial_\nu \partial_\varrho \tensor{h}{_\mu^\varrho}
        - \frac{1}{2} \Box h_{\mu\nu} - \frac{1}{2} \partial_{\mu\nu}h-\frac{1}{2} \eta_{\mu\nu}\partial_\varrho\partial_\sigma h^{\varrho\sigma}
        + \frac{1}{2} \eta_{\mu_\nu}\Box h \\
        =\ & \frac{1}{2} \partial_\mu\partial_\varrho \tensor{\overline{h}}{_\nu^\varrho}
        - \mathunderline{blue}{\frac{1}{4}\partial_\mu\partial_\nu\overline{h}}
        + \frac{1}{2} \partial_\nu\partial_\varrho\tensor{\overline{h}}{_\mu^\varrho}
        - \mathunderline{blue}{\frac{1}{4}\partial_\nu\partial_\mu\overline{h}} - \frac{1}{2}\Box\overline{h}_{\mu\nu} \\
        & + \mathunderline{green}{\frac{1}{2}\eta_{\mu\nu}\Box\overline{h}} + \mathunderline{blue}{\frac{1}{2}\partial_\mu\partial_\nu\overline{h}}
        - \frac{1}{2}\eta_{\mu\nu}\partial_\varrho\partial_\sigma\overline{h}^{\varrho\sigma}
        + \mathunderline{green}{\frac{1}{4}\eta_{\mu\nu}\Box\overline{h}} - \mathunderline{green}{\frac{1}{2}\eta_{\mu\nu}\Box\overline{h}} \\
        =\ & -\frac{1}{2} \Box \overline{h}_{\mu\nu} + \partial_\varrho \tensor{\partial}{_(_\mu}\tensor{\overline{h}}{_\nu_)^\varrho}
        - \frac{1}{2} \eta_{\mu\nu}\partial_\varrho\partial_\sigma \overline{h}^{\varrho\sigma} \\
        \overset{!}{=}\ & \kappa T_{\mu\nu}
    \end{split}
\end{equation} %TODO introduce symmetration brackets somewhere
with the (linearized) d'Alembert operator
\begin{equation}
    \Box^{\text{(L)}} = \Box = \partial_\mu\partial_\nu \eta^{\mu\nu}=\partial_\mu\partial^\mu
\end{equation}
\begin{definition}[Linearized Einstein equations]
    \begin{equation}
        \label{eq:lineinsteineqs}
        -\frac{1}{2} \Box \overline{h}_{\mu\nu} + \partial_\varrho \tensor{\partial}{_(_\mu}\tensor{\overline{h}}{_\nu_)^\varrho}
        - \frac{1}{2} \eta_{\mu\nu}\partial_\varrho\partial_\sigma \overline{h}^{\varrho\sigma} = \kappa T_{\mu\nu}
    \end{equation}
\end{definition}

\subsubsection{Gauge transformations}
Usually field equations are in the form of
\begin{equation}
    \Box \text{``field''} = \text{``source''}
\end{equation}
Equation~\eqref{eq:lineinsteineqs} can be written in this form:
\begin{equation}
    \underbrace{\Box \overline{h}_{\mu\nu}}_{\Box\text{``field''}}
    \underbrace{- 2 \partial_\varrho \tensor{\partial}{_(_\mu} \tensor{\overline{h}}{_\nu_)^\varrho}
    + \eta_{\mu\nu}\partial_\varrho \overline{h}^{\varrho\sigma}}_{\text{ensures gauge invariance of equation}}
    = \underbrace{-2\kappa T_{\mu\nu}}_{\text{``source''}}
\end{equation}

%TODO Michi: mathematical background \\%TODO mathematical background @Michi
Infinitesimal diffeomorphisms = affine transformations
\begin{definition}
    \begin{equation}
        x^\mu = x'^\mu + \xi^\mu(x'^\mu), \qquad \xi^\mu \ll 1
    \end{equation}
\end{definition}
In the following we neglect terms with $\landauO(\xi^2)$, $\landauO(\xi h)$, and $\landauO(h^2)$ and higher order terms.
The transformed metric reads
\begin{equation}
    \begin{split}
        \eta_{\mu\nu} + h'_{\mu\nu}(x') &= g'_{\mu\nu} \\
        &= \frac{\partial x^\varrho}{\partial x'^\mu} \frac{\partial x^\sigma}{\partial x'^\nu} g_{\varrho\sigma}(x) \\
        &= \frac{\partial \left( x'^\varrho + \xi^\varrho \right)}{\partial x'^\mu}
        \frac{\partial \left( x'^\sigma + \xi^\sigma \right)}{\partial x'^\nu}
        \left( \eta_{\varrho\sigma} + h_{\varrho\sigma}(x) \right) + \landauO(\xi^2) \\
        &= \left( \tensor{\delta}{_\mu^\varrho} + \tensor{\xi}{^\varrho_,_\mu} \right)
        \left( \tensor{\delta}{_\nu^\sigma} + \tensor{\xi}{^\sigma_,_\nu} \right)
        \left( \eta_{\varrho\sigma} + h_{\varrho\sigma}(x) \right) + \landauO(\xi^2) \\
        &= \left( \tensor{\delta}{_\mu^\varrho} + \tensor{\xi}{^\varrho_,_\mu} \right)
        \left( \eta_{\varrho\nu} + h_{\varrho\nu} + \tensor{\xi}{_\varrho_,_\nu} \right) + \landauO(\xi^2)  \\
        &= \eta_{\mu\nu} + h_{\mu\nu} + \tensor{\xi}{_\mu_,_\nu} + \tensor{\xi}{_\nu_,_\mu} + \landauO(\xi^2,h^2,\xi h)
    \end{split}
\end{equation}
$\implies$ The perturbation $h_{\mu\nu}$ transforms under infinitesimal diffeomorphisms in the following way
\begin{equation}
    \begin{split}
        h'_{\mu\nu}(x) &= h_{\mu\nu}(x) + \tensor{\xi}{_\mu_,_\nu} + \tensor{\xi}{_\nu_,_\mu} \\
        &= h_{\mu\nu}(x) + \left(\liedif{\xi}{\eta} \right)_{\mu\nu}
    \end{split}
\end{equation}
\begin{definition}[Lie derivative]
    The Lie derivative off a tensor field $T$ with $k$ contravariant and $l$ covariant indices along the vector $\xi$ is given by
    \begin{equation}
        \begin{split}
            \left( \liedif{\xi}{T} \right)^{\alpha_1\ldots\alpha_k}_{\beta_1\ldots\beta_l}
            \coloneqq \xi^\mu \partial_\mu T^{\alpha_1\ldots\alpha_k}_{\beta_1\ldots\beta_l}
            & - \left( \partial_\mu \xi^{\alpha_1} \right) T^{\mu\alpha_2\ldots\alpha_k}_{\beta_1\ldots\beta_l} - \ldots
            - \left( \partial_\mu \xi^{\alpha_k} \right) T^{\alpha_1\ldots\alpha_{k-1}\mu}_{\beta_1\ldots\beta_l} \\
            & + \left( \partial_{\beta_1} \xi^\mu \right) T^{\alpha_1\ldots\alpha_k}_{\mu\beta_2\ldots\beta_l} + \ldots
            + + \left( \partial_{\beta_l} \xi^\mu \right) T^{\alpha_1\ldots\alpha_k}_{\beta_1\ldots\beta_{l-1}\mu}
        \end{split}
    \end{equation}
\end{definition}
Therefore
\begin{equation}
    \left( \liedif{\xi}{\eta} \right)_{\mu_\nu} = \underbrace{\xi^\varrho\partial_\varrho\eta_{\mu\nu}}_{=0}
    + \tensor{\xi}{_\mu_,_\nu} + \tensor{\xi}{_\nu_,_\mu} = \tensor{\xi}{_\mu_,_\nu} + \tensor{\xi}{_\nu_,_\mu}
\end{equation}
If the derivative of a metric vanishes for a given $\xi^\mu$, then one obtains the killing equations for $\xi^\mu$ and the solution of the
killing equations are denoted killing vector field.
\begin{equation}
    \tensor{\xi}{_\mu_,_\nu} + \tensor{\xi}{_\nu_,_\mu}=0  %TODO order of indices in second term?
\end{equation}
In Minkowski-space the ten infinitesimal killing vectors correspond to the Poincaré-generators.
\begin{sidenote}
Lie-derivative on metric $\implies$ detect symmetries of Manifold.
\end{sidenote}
\subsubsection{Check that linearized field equations are invariant under infinitesimal diffeomorphisms}
\begin{equation}
    h'_{\mu\nu} = h_{\mu\nu} + \tensor{\xi}{_\mu_,_\nu} + \tensor{\xi}{_\nu_,_\mu}
\end{equation}
\begin{equation}
    \begin{split}
        \overline{h'}_{\mu\nu} &= h'_{\mu\nu} - \frac{1}{2} \eta_{\mu\nu}h' \\
        &= h_{\mu\nu} + \tensor{\xi}{_\mu_,_\nu} + \tensor{\xi}{_\nu_,_\mu} - \frac{1}{2} \eta_{\mu\nu}h
        -\frac{1}{2}\eta_{\mu\nu}\partial^\varrho\xi_\varrho - \frac{1}{2} \eta_{\mu\nu}\partial^\varrho\xi_\varrho \\
        &= \overline{h}_{\mu\nu} + \tensor{\xi}{_\mu_,_\nu} + \tensor{\xi}{_\nu_,_\mu} - \eta_{\mu\nu}\tensor{\xi}{^\varrho_,_\varrho}
    \end{split}
\end{equation}
Plug this into Einstein's equations (term by term)
\begin{align}
    -\frac{1}{2}\Box \overline{h'}_{\mu\nu} &= -\frac{1}{2}\Box\overline{h}_{\mu\nu} - \frac{1}{2}\Box\tensor{\xi}{_\mu_,_\nu}
    -\frac{1}{2}\Box\tensor{\xi}{_\nu_,_\mu} + \frac{1}{2} \eta_{\mu\nu}\Box\tensor{\xi}{^\varrho_,_\varrho} \\
    -\frac{1}{2} \eta_{\mu\nu}\partial_\varrho\partial_\sigma \overline{h'}^{\varrho\sigma} &=
    -\frac{1}{2} \eta_{\mu\nu}\partial_\varrho\partial_\sigma \left( \tensor{\xi}{^\varrho^,^\sigma} + \tensor{\xi}{^\sigma^,^\varrho}
    - \eta^{\varrho\sigma} \tensor{\xi}{^\alpha_,_\alpha} \right) - \frac{1}{2} \eta_{\mu\nu}\partial_\varrho\partial_\sigma \overline{h}^{\varrho\sigma} \\
    \partial^\varrho \tensor{\partial}{_(_\mu} \tensor{\overline{h'}}{_\nu_)_\varrho} &=
    \partial^\varrho \tensor{\partial}{_(_\mu} \tensor{\overline{h}}{_\nu_)_\varrho} + \frac{1}{2}\Box\tensor{\xi}{_\nu_,_\mu}
    + \frac{1}{2}\Box\eta_{\mu\nu}
\end{align}
Therefore
\begin{equation}
    \begin{split}
        & -\frac{1}{2} \Box \overline{h'}_{\mu\nu} + \partial_\varrho \tensor{\partial}{_(_\mu}\tensor{\overline{h'}}{_\nu_)^\varrho}
        - \frac{1}{2} \eta_{\mu\nu}\partial_\varrho\partial_\sigma \overline{h'}^{\varrho\sigma} \\
        =\ & -\frac{1}{2}\Box\overline{h}_{\mu\nu} - \mathunderline{blue}{\frac{1}{2}\Box\tensor{\xi}{_\mu_,_\nu}}
        -\mathunderline{blue}{\frac{1}{2}\Box\tensor{\xi}{_\nu_,_\mu}}
        + \mathunderline{green}{\frac{1}{2} \eta_{\mu\nu}\Box\tensor{\xi}{^\varrho_,_\varrho}}
        + \partial^\varrho \tensor{\partial}{_(_\mu}\tensor{\overline{h}}{_\nu_)_\varrho} \\
        & + \mathunderline{blue}{\Box\tensor{\xi}{_(_\mu_,_\nu_)}}
        - \mathunderline{green}{\frac{1}{2}\eta_{\mu\nu}\Box\tensor{\xi}{^\varrho_,_\varrho}}
        - \frac{1}{2} \eta_{\mu\nu}\partial_\varrho\partial_\sigma \overline{h}^{\varrho\sigma} \\
        =\ & -\frac{1}{2} \Box \overline{h}_{\mu\nu} + \partial_\varrho \tensor{\partial}{_(_\mu}\tensor{\overline{h}}{_\nu_)^\varrho}
        - \frac{1}{2} \eta_{\mu\nu}\partial_\varrho\partial_\sigma \overline{h}^{\varrho\sigma}
    \end{split}
\end{equation}
This shows that the Einstein equations are invariant under an infinitesimal diffeomorphisms.
Therefore $\overline{h}_{\mu\nu}$ and $\overline{h'}_{\mu\nu}$ are the same \emph{physical} field.

\subsubsection{Harmonic gauge in linearized gravity}
We would like to have the linearized field equations in the form $\Box\text{``field''}=\text{``source''}$ (wave equation).
This can ge done with the gauge condition
\begin{equation}
    \chi_\nu \left[ \overline{h} \right] \coloneqq \partial^\mu \overline{h}_{\mu\nu} = 0.
\end{equation}
In therms of the original field this condition reads
\begin{definition}[de Donder gauge, harmonic gauge]
    \begin{equation}
        \chi_\nu \left[ h \right] = \partial^\mu h_{\mu\nu} - \frac{1}{2} h_{\mu\nu} \partial^\mu h = 0
    \end{equation}
\end{definition}
Proof:
\begin{equation}
    \begin{split}
        \partial^\mu \overline{h'}_{\mu\nu} &= \partial^\mu \overline{h}_{\mu\nu} + \Box \xi_\nu + \partial_\nu \partial^\mu \xi_\mu -
        \eta_{\mu\nu} \partial^\mu\partial_\varrho\xi^\varrho \\
        &= \partial^\mu \overline{h}_{\mu\nu} + \Box \xi_\nu = 0
    \end{split}
\end{equation}
Solve for $\Box\xi_\nu$
\begin{equation}
    \implies \Box \overline{h'}_{\mu\nu} = -2\kappa T_{\mu\nu}
\end{equation}
Since $\overline{h}_{\mu\nu}$ and $\overline{h'}_{\mu\nu}$ correspond to the same physical field configuration, we can drop the prime.
\begin{definition}{Linearized field equations in de Donder gauge.}
    \begin{equation}
        \Box \overline{h}_{\mu\nu} = - 2 \kappa T_{\mu\nu}
\end{equation}
\end{definition}
\afterpage{
\clearpage
\thispagestyle{empty}
\begin{landscape}
    \begin{table}[h]
        \caption{Comparison between linearized gravity and electrodynamics.}
        \centering
        \begin{tabulars}{lcc}
            \toprule
            & linearized gravity & electrodynamics \\
            \midrule

            basic field
            & $\overline{h}_{\mu\nu}$ ($h_{\mu\nu}$), spin 2, \emph{graviton}
            & $A_\mu$, spin 1, \emph{gauge boson}, \emph{gauge potential} \\


            field equations
            & $ \underbrace{\Box \overline{h}_{\mu\nu}}_{\Box\text{``field''}} - \underbrace{2 \partial_\rho \tensor{\partial}{_(_\mu}\tensor{\overline{h}}{_\nu_)^\rho} - \eta_{\mu\nu}\partial_\rho\partial_\sigma \overline{h}^{\rho\sigma}}_{\text{ensures gauge inv.}} = -\underbrace{2 \kappa T_{\mu\nu}}_{\text{``source''}}$
            & $\underbrace{\Box A_\mu}_{\Box\text{``field''}} - \underbrace{\partial_\mu \left( \partial_\nu A^\nu \right)}_{\text{ensures gauge inv.}} = - \underbrace{4 \pi j_\mu}_{\text{source}}$ \\

            transf. under inf. gauge trafos
            & $\overline{h'}_{\mu\nu} = h_{\mu\nu}+\tensor{\xi}{_\mu_,_\nu} + \tensor{\xi}{_\nu_,_\mu} - \eta_{\mu\nu} \tensor{\xi}{^\rho_,_\rho}$
            & $A'_\mu = A_\mu + \partial_\mu\lambda(x)$\\

            & inf. coordinate transformation
            & internal symmetry \\

            inv. of field eqs
            & yes
            & yes \\

            specific gauges
            & de Donder gauge, $\partial_\mu \overline{h}^{\mu\nu}=0$
            & Lorentz gauge, $\partial_\mu A^\mu = 0$ \\

            field eqs. in specific gauges
            & $\Box \overline{h}_{\mu\nu} = - 2 \kappa T_{\mu\nu}$
            & $\Box A_\mu = - 4 \pi j_\mu$ \\

            inv. tensors under gauge trafo
            & $\tensor{{R^{(L)\prime}}}{^\rho_\mu_\nu_\rho} =
             \tensor{{R^{(L)}}}{^\rho_\mu_\nu_\rho}$ 
            & $F'_{\mu\nu} =F_{\mu\nu}$\\
            \bottomrule
        \end{tabulars}
    \end{table}
\end{landscape}
}

