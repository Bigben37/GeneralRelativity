\documentclass[
a4paper,                                               % Papierformat
oneside,                                               % Einseitig
%twoside,                                              % Zweiseitig
12pt,                                                  % Schriftgröße
headsepline,                                           % Linie unter der Kopfzeile
%draft=true                                            % Markiert zu lange und zu kurze Zeilen
xcolor=dvipsnames
]{scrreprt}
% \usepackage{geometry}
\usepackage{tabularx}
\usepackage{tikz}
\usepackage{listings}
\usepackage{tensor}
\usetikzlibrary{trees}
\usetikzlibrary{decorations.pathmorphing}
\usetikzlibrary{positioning,arrows,snakes}
\usetikzlibrary{decorations.markings}
\usepackage[utf8]{inputenc}
\usepackage[OT1]{fontenc}                                                     	% Verwendung der Zeichentabelle T1, für deutschsprachige Dokumente sinnvoll                                                          % LaTeX Default-Font für geglättete Schrift
\usepackage[english,ngerman]{babel}                                           	% Silbentrennung nach neuer deutscher und englischer Rechtschreibung
\usepackage{amsmath}
\usepackage{amsthm}
%\usepackage{libertine}  	%
%\usepackage{cmbright}
%\usepackage{mathdesign}
\usepackage{pdfpages}
\usepackage{xifthen}                                                          	% Wird benötigt um \ifthenelse zu benutzen
% Zum flexiblen Einbinden von Grafiken, pdftex ist optional
\usepackage{units}                                                            	% Ermöglicht die Nutzung von \unit[Zahl]{Einheit}
\usepackage{setspace}                                                         	% Einfaches wechseln zwischen unterschiedlichen Zeilenabständen
\usepackage{bpchem}
\usepackage{xcolor}

\definecolor{darkblue}{RGB}{26,52,132}
\definecolor{darkblue2}{rgb}{0.0, 0.0, 0.4}

\usepackage[font=small,labelfont=bf,labelsep=endash,format=plain]{caption}    	% Darstellung für Caption s.u.
\usepackage{subfig}                                                           	% Bilder nebeneinander
\usepackage{wrapfig}                                                          	% Fließtext um Figure-Umgebung
\usepackage{cite}                                                             	% Zusatzfunktionen zum zitieren
\usepackage{scrpage2}                                                         	% Wird für Kopf- und Fußzeile benötigt
\usepackage{array,dcolumn}                                                    	% Beide Pakete werden für die Ausrichtung der Tabellenspalten benötigt
\usepackage{braket}
%\usepackage[paper=a4paper,left=25mm,right=25mm,top=25mm,bottom=25mm]{geometry}
\usepackage{mathtools}
\usepackage{amsfonts}
\usepackage{tabularx}
\usepackage{amssymb}
\usepackage{multirow}
\usepackage{booktabs,tabularx}
\usepackage[section]{placeins}
%\usepackage{morefloats}
\usepackage{tikz}
\usepackage{nameref}
\usepackage{dsfont}
\usepackage{commath}
\usepackage{tensor}
\usetikzlibrary{shapes,arrows}
\usetikzlibrary{positioning}
\usepackage[
colorlinks=true,
urlcolor=blue,
linkcolor=black,
citecolor=black
]{hyperref}
\usepackage{glossaries}
\usepackage[a4paper,pdftex]{geometry}										% A4paper margins
\setlength{\oddsidemargin}{5mm}												% Remove 'twosided' indentation
\setlength{\evensidemargin}{5mm}

\usepackage[protrusion=true,expansion=true]{microtype}
\usepackage{graphicx}
%Diese Befehle sortieren die Einträge in den
%einzelnen Listen:
%makeindex -s datei.ist -t datei.alg -o datei.acr datei.acn
%makeindex -s datei.ist -t datei.glg -o datei.gls datei.glo
%makeindex -s datei.ist -t datei.slg -o datei.syi datei.syg


\newcommand\grad{\ensuremath{^\circ}}
\newcommand{\imI}{\ensuremath{\mathrm{i}}}
\newcommand{\Cn}{\ensuremath{\mathbb{C}^n}}
\newcommand{\Sn}{\ensuremath{\mathbb{S}^n}}
\newcommand{\Rn}{\ensuremath{\mathbb{R}^n}}
\DeclareMathOperator{\tr}{Tr}
\DeclareMathOperator{\re}{Re}
\DeclareMathOperator{\im}{Im}
\DeclareMathOperator{\id}{id}
\let\originalleft\left  %fix bracket spacing when using \left( \right)
\let\originalright\right
\renewcommand{\left}{\mathopen{}\mathclose\bgroup\originalleft}
\renewcommand{\right}{\aftergroup\egroup\originalright}
\renewcommand{\vec}{\mathbf}

\newtagform{brackets}{[}{]}
\usetagform{brackets}

\tikzstyle{block} = [draw, fill=blue!20, rectangle,
minimum height=3em, minimum width=6em]


% The block diagram code is probably more verbose than necessary
\tikzstyle{line} = [draw, -latex']


% #1 ist dabei ein Parameter, den man \cImage übergeben muss. In 10_Titelseite.tex wird dieser Befehl verwendet. Der Parameter ist dort Bilder/titelseite.jpg.
% Benötigt man keine Parameter, dann lässt man [1] weg. Werden zusätzliche Parameter benötigt, dann kann man die Zahl auf maximal 9 erhöhen.


% #########################
% # Beginn des Dokumentes #
% #########################




\automark[section]{chapter} % ???


\begin{document}
\selectlanguage{english}
\numberwithin{equation}{section}                           % Nummerierung der Formeln entsprechend der Section (z.B. 1.1)
\addtokomafont{caption}{\small\linespread{1}\selectfont}   % Ändert Schriftgröße und Zeilenabstand bei captions
\renewcommand*{\headfont}{\normalfont}					   % nicht kursive Kopfzeile

% Römische Ziffern als Seitenzahlen für Titelseite bis einschließlich dem Inhaltsverzeichnis
\setcounter{page}{1}
\pagenumbering{roman}

% #######################################
% # Kopf- und Fußzeile konfigurieren    #
% #######################################

%\ihead{Loschmidt echo and non-Markovian dynamics in the quantum Ising model }                      % Innenseite der Kopfzeile
%	\chead{}                                                   % Mitte der Kopfzeile
%\ohead{Michael Ruf}                                 % Außenseite der Kopfzeile
%	\ifoot{}                                                   % Innnenseite der Fußzeile
\cfoot{- \textit{\pagemark} -}                             % Mitte der Fußzeile
%	\ofoot{}
%    	\lhead{\leftmark}                                               % Aussenseite der Fußzeile

% ###################################
% # Ausrichtung der Tabellenspalten #
% ###################################

\newcolumntype{,}[1]{D{,}{,}{#1}}                          % , in Tabellen untereinander stellen
\newcolumntype{p}{D{p}{\pm}{-1}}                           % +- in Tabellen untereinander stellen

% ########################
% # Titelseite einbinden #
% ########################
\numberwithin{equation}{chapter}
\thispagestyle{empty}
\vfill
\restoregeometry
\onehalfspacing

% ################################
% # Inhaltsverzeichnis einbinden #
% ################################
\setcounter{page}{1}
\tableofcontents
\newpage

% Zurücksetzen der Seitenzahlen auf arabische Ziffern
\setcounter{page}{1}
\pagenumbering{arabic}
\pagestyle{scrheadings}

\chapter{Newtonian Gravity}
In \name{Newton}ian physics we assume that we have absolute space and time
that can be described by the a set of numbers $x^1,x^2,x^3,t$.
We express the coordinates as functions of time.
\section{Forces}
The force $\vec{F}_{AB}$, which a massive body $A$ with mass $m_A$ exerts on another massive body $B$ with mass $m_B$, is given by
\begin{equation}
    \vec{F}_{AB}=-m_B\frac{G\textsubscript{N}m_A}{r^2}\vec{e}_r\, ,
\end{equation}
%TODO remove subscript of G
where $G\textsubscript{N}$ denotes \emph{\name{Newton}'s constant},
numerically equal to $G\textsubscript{N}\approx \unitfrac[6.673\cdot 10^{-11}]{m^3}{kg\,s}$.
Although there is no need for $G\textsubscript{N}$ to be constant over time,
there is evidence that the relative variation is less than $10^{-12}$ per year.
The force can be expressed in terms of \emph{gravitational potential} $\Phi$:
\begin{equation}
    \vec{F}_{AB}=-m_B\nabla\left(-\frac{G\textsubscript{N}m_A}{r}\right)=:
    -m_B\nabla\Phi(\vec{r}_B)\, .
\end{equation}
Given $N$ particles labeled by $n$, the total force $B$ experiences is
\begin{equation}
    \vec{F}_{B}=-\sum_n \vec{F}_{nB}\, .
\end{equation}
The potential at $\vec{r}$ is then easily found to be
\begin{equation}
    \Phi(\vec{r})=-G\textsubscript{N}\sum_n\frac{m_n}{|\vec{r}-\vec{r}_n|}\, .
\end{equation}
In general, we assume a mass distribution $\varrho(\vec{r})$ and the sum is
replaced by an integral:
\begin{equation}
    \Phi(\vec{r}) = -G\textsubscript{N}\int_{\Reals^3}\dif{\vec{r}^{\prime}}
    \frac{\varrho(\vec{r}^{\prime})}{|\vec{r}-\vec{r}^{\prime}|}\,.
\end{equation}
\section{Comparison with electrostatics}
The classical theory of gravity bears a striking similarity to electrostatics.
To make this clearer, we introduce the gravitational field
$\vec{g}(\vec{r}):= -\nabla\Phi(\vec{r})$.
\begin{table}[htb]
    \caption{Comparison of electrostatics and \name{Newton}ian gravity.}
    \begin{center}
        \begin{tabulars}{lll}
            \toprule
            &\name{Newton}ian Gravity&Electrostatics\\
            \midrule
            Force&$\displaystyle\vec{F}=q\frac{kQ}{r^2}\vec{e}_r$&$\vec{F}=m\frac{G\textsubscript{N}M}{r^2}\vec{e}_r$\\
            Potential
            &$\Phi\textsubscript{el}(\vec{r})=q\frac{kQ}{r}$
            &$\Phi\textsubscript{g}(\vec{r})=m\frac{G\textsubscript{N}M}{r}$\\
            Field
            &$\vec{E}(\vec{r})=-\nabla\Phi\textsubscript{el}(\vec{r})$
            &$\vec{g}(\vec{r})=-\nabla\Phi\textsubscript{g}(\vec{r})$\\
            \name{Laplace} equation
            &$\Delta\Phi\textsubscript{el}=-4\pi k\varrho\textsubscript{el}(\vec{r})$&
            $\Delta\Phi\textsubscript{g}=4\pi
            G\textsubscript{N}\varrho\textsubscript{g}(\vec{r})$
            \\
            \bottomrule
        \end{tabulars}
    \end{center}
\end{table}
\begin{example}[Field of a spherical mass distribution]
Assume we have a spherical mass distribution, i.e.\ $\varrho(\vec{r})=\varrho(r)$.
By symmetry considerations it follows, that the gravitational field can be
expressed by
\begin{equation}
    \vec{g}(\vec{r})=g(r)\vec{e}_r\, .
\end{equation}
We integrate the divergence of the field over a ball $B$ of radius $r$
\begin{equation}
    \int_B\dif{\vec{r}}\,\nabla\vec{g}=-\int_B\dif{\vec{r}}\, \Delta\Phi
    = -4\pi G\textsubscript{N}\int_B\dif{\vec{r}}\,\varrho(r)= -4\pi
    G\textsubscript{N} M\, ,
\end{equation}
where $M$ is the mass enclosed in $B$. On the other hand we can use Gauss's theorem to deduce
\begin{equation}
    \int_B\dif{\vec{r}}\,\nabla\vec{g}=\oint_{\partial B}\dif{\vec{A}}\cdot
    \vec{g} = \oint_{\Omega}\dif{\Omega}\, g(r)r^2=4\pi r^2g(r)\, .
\end{equation}
Together the gravitational field is given by
\begin{equation}
    \vec{g}(r)=-\frac{G\textsubscript{N}M}{r^2}\vec{e}_{r}\, .
\end{equation}
\end{example}
% \subsection{Inertial systems}
% \begin{definition}
% An inertial system is a system in which force-free particles move with constant uniform velocity on straight lines.
% \end{definition}
\subsection*{Weak Equivalence Principle (WEP)}
\name{Newton}'s first law reads
\begin{equation}
    \vec{F}=m\textsubscript{I} \vec{\ddot{x}} \, ,
\end{equation}
where $m\textsubscript{I}$ is the inertial mass that works against the acceleration of the body.
The force which a body with ``active'' mass $m\textsubscript{g,a}$ exerts on
another body with mass $m\textsubscript{g,p}$ is given by 
\begin{equation}
    \vec{F}=m\textsubscript{g,p}\frac{G\textsubscript{N}m\textsubscript{g,a}}{r^2}\vec{e}_r \, .
\end{equation}
A priori, there is no reason to assume any relation between this masses.
The first question one might ask is whether the active and the passive mass are equal.
Suppose we have two masses $A$ and $B$. Using \name{Newton}'s first law, we
can explicitly write
\begin{align}
    m^{B}_{\text{I}}\vec{\ddot{x}}&=\vec{F}_{AB}=-
    m^{B}_{\text{g,p}}\frac{G\textsubscript{N}m^{A}_{\text{g,a}}}{r^2}\vec{e}_r
    \, ,\\
    m^{A}_{\text{I}}\vec{\ddot{x}}&=\vec{F}_{BA}=-
    m^{A}_{\text{g,p}}\frac{G\textsubscript{N}m^{B}_{\text{g,a}}}{r^2}\vec{e}_r
    \, .
\end{align}
By the third law $\vec{F}_{AB}=-\vec{F}_{BA}$ we have
\begin{equation}
\frac{m^{B}_{\text{g,p}}}{m^{B}_{\text{g,a}}}=\frac{m^{B}_{\text{g,p}}}{m^{B}_{\text{g,a}}}\,.
\end{equation}
By proper choice of mass units we can set this quotient to one so that 
\begin{equation}
m\textsubscript{g,a}=m\textsubscript{g,p}=:m\textsubscript{g}
\end{equation}
The next question is wheater the inertial mass equivalent to the gravitional
mass.
By \name{Newton}'s first law we have
\begin{equation}
m\textsubscript{I}\vec{\ddot{x}}
=-m_{\text{g}}\frac{G\textsubscript{N}M_{\text{g}}}{r^2}\vec{e}_r 
=-m_{\text{g}}\vec{g}\,.
\end{equation}
As a experimental result that has been measured up to a high accuracy (compare
tabular~\ref{tab:WEPExp}) all bodys recive the same acceleration due to gravity
$\ddot{\vec{x}}\sim \vec{g}$. By a 'proper' choice of units of the flight-time $t=\sqrt{\frac{m\textsubscript{I}}{m\textsubscript{g}}}\sqrt{\frac{2h}{g}}$,
we get
\begin{equation}
m\textsubscript{I}=m\textsubscript{g}=:m\,.
\end{equation}
\begin{table}
\centering
\begin{tabulars}{rllr}
\toprule
Year&Experimenter&Experiment&Accuracy\\
\midrule
1636&\name{Galilei}&inclined planes&$10^{-2}$\\
1689&\name{Newton}&pendulum&$10^{-3}$\\
1832&\name{Bessel}&pendulum&$10^{-5}$\\
1922&\name{Eötvös}&pendulum&$10^{-9}$\\
1922&\name{Shapro} et al.&pendulum&$10^{-12}$\\
1999&\name{Baesler}&torsion balance&$10^{-14}$\\
\bottomrule
\end{tabulars}
\caption{Experiments measuring the ratio
$\frac{m\textsubscript{I}}{m\textsubscript{g}}$.\label{tab:WEPExp}}
\end{table}
\subsection{Tidal Forces}
Assume we have a body of finite extention in a gravitationational
potential $\Phi$, an example beeing the earth in the potential of the moon.
On the center of the body we have 
%TODO masses???
\begin{equation}
\dod[2]{\tensor{x}{^i}}{t}=-\dpd{\Phi}{\tensor{x}{_i}}\,.
\end{equation}
If we consider a point shifted by $\tensor{\chi}{^i}$ from the center then the
acceleration is given as
\begin{equation}
\begin{split}
\dod[2]{}{t}\left(\tensor{x}{^i}+\tensor{\chi}{^i}\right)
&=-\dpd{\Phi\left(\tensor{x}{^i}+\tensor{\chi}{^i}\right)}{\tensor{x}{_i}}\\
&\simeq-\dpd{\Phi\left(\tensor{x}{^i}\right)}{\tensor{x}{_i}}
-\dmd{\Phi\left(\tensor{x}{^i}\right)}{2}{\tensor{x}{_i}}{}{\tensor{x}{_j}}{}\tensor{\chi}{^j}\,.
\end{split}
\end{equation}
Subtracting the previous equations yields the tidal force
\begin{equation}
\dod[2]{\tensor{\chi}{^i}}{t}=-\dmd{\Phi\left(\tensor{x}{^i}\right)}{2}{\tensor{x}{_i}}{}{\tensor{x}{_j}}{}\tensor{\chi}{^j}\,.
\end{equation}
The tidal force tensor $\dmd{\Phi}{2}{\tensor{x}{_i}}{}{\tensor{x}{_j}}{}$ is of
the form
\begin{equation}
\dmd{\Phi}{2}{\tensor{x}{_i}}{}{\tensor{x}{_j}}{}
=\frac{G\textsubscript{N}M}{r^3}\left(\tensor{\delta}{_i_j}-3\frac{\tensor{x}{_i}\tensor{x}{_j}}{r^2}\right)\,.
\end{equation}
%TODO newtons laws?
%TODO picture
%TODO finish chapter 1
\chapter{Special Relativity}
\section{Postulates and Definitions}
Special relativity is based on two main postulates, namely
\begin{enumerate}
    \item Principle of relativity: \\
    The laws of physics acquire the same form in all inertial systems.
    \item Constancy of the speed of light:\\
    The speed of light in vacuum is constant ${c\approx\unitfrac[3\cdot
    10^8]{m}{s}}$.
\end{enumerate}
We further define a series of objects that will come handy when describing
relativity:
\begin{definition}[System of reference]
    A system of reference $K$ is a system of three spacial coordinates to indicate
    the position and one time coordinate to indicate the time.
\end{definition}
\begin{definition}[Inertial system]
    An inertial system $I$ belongs to a particular subspace of reference systems in
    which a freely moving bodies, i.e. that are not subject to any external
    force, move with constant velocity on an straight lines.
\end{definition}
\begin{definition}[Event]
    An event or world point $x$ is a point in spacetime. In any system of reference
    it can be described by four coordinates
    \begin{equation}
        x^\mu: (x^0,x^1,x^2,x^3)\, .
    \end{equation}
    For example in cartesian coordinates, we have
    \begin{equation}
        (x^0,x^1,x^2,x^3) = (ct,x,y,z)\, .
    \end{equation}
\end{definition}
\begin{definition}[Worldline]
    A worldline $\gamma$ is a parametrised curve in spacetime. To each parameter,
    there corresponds an event that lies on the worldline
    \begin{equation}
        \tensor{\gamma}{^\mu}(\lambda)=\tensor{x}{^\mu}(\lambda)\, .
    \end{equation}
\end{definition}
\subsection{Einstein Summation Convention}
Since we will have to deal which many indices, we will find a way to keep
notation as compact as possible. We agree that whenever we have a summation over an
upper and a lower index like $\sum_\mu \tensor{x}{_\mu}\tensor{x}{^\mu}$, we
drop the sumation and assume that terms of type
$\tensor{x}{_\mu}\tensor{x}{^\mu}$ are always summed over.
\section{Propagation of Light Waves and the Line Element}
For vividness, we restore factors of $c$ in this section.
(Pictures)\\
Inertial System $I$:
\begin{itemize}
    \item $P:$ Event of emission of the ray $(ct_1,x_1,y_1,z_1)$
    \item $Q:$ Event of absorption of the ray $(ct_2,x_2,y_2,z_2)$
\end{itemize}
The spatially distance between the events is
\begin{equation}
    r=\left[(x_2-x_1)^2+(y_2-y_1)^2+(z_2-z_1)^2\right]^{\frac{1}{2}}\, .
\end{equation}
Since we are tracking a light ray and the events are absorption and emission, we
further have
\begin{equation}
    r=c(t_2-t_1)\, .
\end{equation}
We construct the quantity
\begin{equation}
    {\Delta s}^2=-c^2(t_2-t_1)^2+(x_2-x_1)^2+(y_2-y_1)^2+(z_2-z_1)^2\, ,
\end{equation}
so that ${\Delta s}^2=0$ for light.

We consider another Inertial System $I'$, in which we the events are given by
\begin{itemize}
    \item $P:$ Event of emission $(ct_1^\prime,x_1^\prime,y_1^\prime,z_1^\prime)$
    \item $Q:$ Event of absorption
    $(ct_2^\prime,x_2^\prime,y_2^\prime,z_2^\prime)$
\end{itemize}
By the same reasoning we have ${\Delta s^\prime}^2=0$ for light.
We define an infinitesimal interval, or \emph{line element}:
\begin{equation}
    \dif s^2=-c^2\dif t^2+\dif x^2+\dif y^2+\dif z^2\, .
\end{equation}
(PICTURE)
Our preliminary considerations lead to the following statement: if the interval
is zero in one inertial system, it should be zero in all.
To relate the linelements of different systems to each other, we make the following thought:
Suppose we have three inertial Systems $I,I_1,I_2$.
The Systems $I_1$ and $I_2$ move at constant velocity $\vec{v}_1$ and $\vec{v}_2$ relative to $I$. Further
$I_2$ moves with velocity $\vec{v}_{12}$ relative to $I_1$. The demand implies
that there exists a function $\alpha$, only dependent on the velocity $\vec{v}$,
so that
\begin{equation}
    \dif s'^2=\alpha(\vec{v})\dif s^2\, .
\end{equation}
We have
\begin{equation}
    \dif s^2=\alpha(\vec{v}_1)\dif s_1^2\, ,\quad\dif s^2=\alpha(\vec{v}_2)\dif
    s_2^2\, ,\quad\dif s_1^2=\alpha(\vec{v}_{12})\dif s_2^2\, .
\end{equation}
Together they imply
\begin{equation}
    \alpha(\vec{v}_{12})=\frac{\alpha(\vec{v}_{2})}{\alpha(\vec{v}_{1})}\, ,
\end{equation}
but since the velocities where arbitary $\alpha$ has to be constant and we are
free to chose $\alpha\equiv 1$. Therefore the line element has to stay
invariant.
\section{Invariant Distances, Metric and Signature}
In flat space we can calculate the distance $\Delta l$ between two points
$(x_1,y_1)$ and $(x_2,y_2)$ by the Pythagorean theorem
\begin{equation}
    \Delta l^2=\Delta x^2+\Delta y^2\, ,
    \quad \Delta x=x_2-x_1\,,\quad \Delta
    y=y_2-y_1\, .
\end{equation}
For an infinitesimal distance $\dif l$ we recover
\begin{equation}
    \dif l^2=\dif x^2+\dif y^2=\tensor{\delta}{_i_j}\dif x^i\dif x^j\, .
\end{equation}
If we introduce
\begin{equation}
    (\dif x^i)=\begin{bmatrix}
\dif x\\
\dif y
\end{bmatrix}\,,\quad (\delta_{ij})
=\begin{bmatrix}
1 & 0\\
0 & 1
\end{bmatrix}\, .
\end{equation}
We can expand the formula for the infinitesimal element and obtain
\begin{equation}
    \dif l^2=
    \begin{bmatrix}
        \dif x &
        \dif y
    \end{bmatrix}
    \begin{bmatrix}
        1 & 0\\
        0 & 1
    \end{bmatrix}
    \begin{bmatrix}
        \dif x\\
        \dif y
    \end{bmatrix}\, .
\end{equation}
Which we do to point out the similarity to the line element $\dif s^2$
\begin{equation}
    \dif s^2=
    \begin{bmatrix}
        \dif t &
        \dif x &
        \dif y &
        \dif z
    \end{bmatrix}
    \begin{bmatrix}
        -1 & 0 & 0 & 0\\
        0  & 1 & 0 & 0\\
        0  & 0 & 1 & 0\\
        0  & 0 & 0 & 1\\
    \end{bmatrix}
    \begin{bmatrix}
        \dif t\\
        \dif x\\
        \dif y\\
        \dif z\\
    \end{bmatrix}=:\eta_{\mu\nu}\dif x^\mu\dif x^\nu\, .
\end{equation}
We call the matrix $\eta_{\mu\nu}=\mathrm{diag}(-1,1,1,1)$ the
\emph{Minkowski-metric}. It has some obvious properties:
\begin{itemize}
    \item constancy: $\tensor{\eta}{_\mu_\nu_,_\varrho}=0$
    \item symmetry: $\tensor{\eta}{_\mu_\nu}=\tensor{\eta}{_\nu_\mu}$
    \item self inverse:
    $\tensor{\eta}{^\mu^\nu}:=\tensor{{\eta^{-1}}}{_\mu_\nu}=\tensor{\eta}{_\mu_\nu}$
\end{itemize}
We can use the metric to raise and lower indices:
\begin{equation}
    x_\nu=\tensor{\eta}{_\mu_\nu}x^\mu\, ,\quad x^\mu=\tensor{\eta}{^\mu^\nu}x_\nu\, .
\end{equation}
One has to pay attention with the matrix form, for example, the \emph{trace} of
the metric is given by
\begin{equation}
    \tensor{\eta}{^\mu_\mu}=\tensor{\eta}{^\mu^\nu}\tensor{\eta}{_\nu_\mu}
    =\tensor{\delta}{^\mu_\mu}=4\, .
\end{equation}
We say $\tensor{\eta}{_\mu_\nu}$ has signature $(-,+,+,+)$ or $(1,3)$, because
it has one negative and three positive eigenvalues.
\subsection{Poincaré-transformation}
To be consistent with a constant speed of light a transformation between two
coordinate systems must leave the line element invariant, i.e.
\begin{equation}
    \dif {s^\prime}^2 = \eta_{\mu\nu}\dif {x^\prime}^\mu\dif
    {x^\prime}^\nu=\eta_{\mu\nu}\dif x^\mu\dif
    x^\nu = \dif s^2\,. \label{eq:invarline}
\end{equation}
We consider afine coordinate transformations
\begin{equation}
    {x^\prime}^\mu= f^\mu(x^\nu)=\tensor{L}{^\mu_\nu}x^\nu+a^\mu\, .
\end{equation}
Where $\tensor{L}{^\mu_\nu}$ is a a Lorentz transformation and $a^\mu$ is a
constant shift.\\
We can now inspect the transformation properties of an allowed transformation.
The invariance of the line element implies
\begin{equation}
    \dif {s^\prime}^2 = \eta_{\mu\nu}\dif {x^\prime}^\mu\dif
    {x^\prime}^\nu=\eta_{\varrho\sigma}\tensor{L}{^\varrho_\mu}\tensor{L}{^\sigma_\nu}\dif
    x^\mu\dif x^\nu \stackrel{!}{=} \dif s^2\,,
\end{equation}
independent of the shift $a^\mu$. We will therefore focus our attention towards
the linerar transformation $\tensor{L}{^\mu_\nu}$. Equation
\eqref{eq:invarline} implies that
\begin{equation}
    \eta_{\mu\nu}=\tensor{L}{^\varrho_\nu}\eta_{\varrho\sigma}\tensor{L}{^\sigma_\nu}\,.\label{eq:invariance}
\end{equation}
We can take a look at an infinitimal transformation, that is up to an order of
$\varepsilon^2$
\begin{equation}
    \tensor{L}{^\mu_\nu}=\tensor{\delta}{^\mu_\nu}
    +\varepsilon\tensor{\omega}{^\mu_\nu}\, .
\end{equation}
The $\tensor{\omega}{^\mu_\nu}$ are called the \emph{generators} of the
transformation. Plugging into \eqref{eq:invariance}, ignoring higher powers in
$\varepsilon^2$ results in
\begin{equation}
    \begin{split}
        \eta_{\mu\nu}&=\eta_{\varrho\sigma}\left(\tensor{\delta}{^\varrho_\mu}
        +\varepsilon\tensor{\omega}{^\varrho_\mu}\right)\left(\tensor{\delta}{^\sigma_\nu}
        +\varepsilon\tensor{\omega}{^\sigma_\nu}\right)\\
        &=
        \eta_{\mu\nu}+\varepsilon\left(\tensor{\omega}{_\mu_\nu}+\tensor{\omega}{_\nu_\mu}\right)\,
        .
    \end{split}
\end{equation}
Since this must hold true for arbitary $\varepsilon$, the generators
$\tensor{\omega}{_\nu_\mu}$ must satisfy
$\tensor{\omega}{_\nu_\mu}=-\tensor{\omega}{_\mu_\nu}$, i.e. be antisymmetric.
In general we have $\frac{n(n-1)}{2}$ of these objects, where $n$ is the
dimension of the underlying space.
Since we are considering fourdimensional spacetime, there are six generators.
The finite transformations by an generalized angle $\alpha\in\Complex$ can be
obtained from the generators via
\begin{equation}
    \tensor{L}{^\mu_\nu}=\exp\left(\imI\alpha\tensor{\omega}{^\mu_\nu}\right)\,.
\end{equation}
We can
classify the resulting transformations into two groups:
\subsubsection{Rotations}
Rotations mix the spatial coordinates, but leave the time fixed. Consider for
example an rotation around $z$-axis. Then we have
\begin{equation}
    x^\prime=x\cos\alpha+y\sin\alpha \, ,\quad y^\prime=-x\sin\alpha+y\cos\alpha \, .
\end{equation}
The infinitisimal generator is
\begin{equation}
    \tensor{\omega}{^\mu_\nu}
    =
    \begin{bmatrix}
        0&0 & 0&0\\
        0&0 &1&0\\
        0&-1&0 &0\\
        0&0 &0 &0
    \end{bmatrix}\,
\end{equation}
and the transformation can be expressed in matrix form
\begin{equation}
    \begin{bmatrix}
        t\\
        x\\
        y\\
        z\\
    \end{bmatrix}=
    \begin{bmatrix}
        1&0 & 0&0\\
        0&\cos\alpha &\sin\alpha&0\\
        0&-\sin\alpha&\cos\alpha &0\\
        0&0 &0 &1
    \end{bmatrix}
    \begin{bmatrix}
        t^\prime\\
        x^\prime\\
        y^\prime\\
        z^\prime
    \end{bmatrix}
\end{equation}
\subsubsection*{Boosts}
Boosts can be thought of as rotations between time and spacial coordinates. We
consider a boost in $x$-direction. Since the line element must be invariant and
a linear transformation keeps the origin fixed we have
\begin{equation}
    -t^2+x^2=-{t^\prime}^2+{x^\prime}^2\, .
\end{equation}
This is an hyperbolic equation and can be parametrised by
\begin{equation}
    t = t^\prime\cosh\psi+x^\prime\sinh\psi\,, \quad x =
    t^\prime\sinh\psi+x^\prime\cosh\psi\, .
\end{equation}
Where $\psi$ is called the \emph{rapidity}.
The generator is
\begin{equation}
    \tensor{\omega}{^\mu_\nu}
    =
    \begin{bmatrix}
        0&1 & 0&0\\
        1&0 &0&0\\
        0&0&0 &0\\
        0&0 &0 &0
    \end{bmatrix}\,.
\end{equation}
The matrix form of the transformation is
\begin{equation}
    \begin{bmatrix}
        t\\
        x\\
        y\\
        z\\
    \end{bmatrix}=
    \begin{bmatrix}
        \cosh\psi&\sinh\psi & 0&0\\
        \sinh\psi&\cosh\psi &0&0\\
        0&0&1 &0\\
        0&0 &0 &1
    \end{bmatrix}
    \begin{bmatrix}
        t^\prime\\
        x^\prime\\
        y^\prime\\
        z^\prime
    \end{bmatrix}\, .
\end{equation}
The rapidity $\psi$ is not related to a physical quantity yet. We may ask to
which velocity $v$ does this boost correspond.
Therefore we consider the line $x^\prime= 0$.
Then
\begin{equation}
    t=t^\prime\cosh\psi\, ,\quad
    x=t^\prime\sinh\psi\, .
\end{equation}
We can express the velocity $v$, as seen by an observer in the primed system
\begin{equation}
    v=\frac{x}{t}=\tanh\psi\,.
\end{equation}
It is convenient to introduce the parameters $\beta$ and $\gamma$ defined as
follows
\begin{equation}
    \beta:=v\, ,\quad\gamma:=\frac{1}{\sqrt{1-\beta^2}}\, .
\end{equation}
In terms of these we find
\begin{equation}
    \sinh\psi = \beta\gamma\, , \quad \cosh\psi=\gamma\,.
\end{equation}
Therefore an alternative form is given by
\begin{equation}
    \tensor{L}{^\mu_\nu}=\begin{bmatrix}
\gamma&\beta\gamma & 0&0\\
\beta\gamma&\gamma &0&0\\
0&0&1 &0\\
0&0 &0 &1
\end{bmatrix}\, .
\end{equation}
And the coordinates are therfore related by
\begin{equation}
    t=\gamma\left(t^\prime+\beta x^\prime\right)\,,\quad
    x=\gamma\left(x^\prime+\beta t^\prime\right)\, .
\end{equation}
All together we have ten independent (affine) transformations that leave the
line element invariant. Namely:
\begin{itemize}
    \item four shifts by $a^\mu=(a^0,a^1,a^2,a^3)\transpose$
    \item three rotations in space corresponding to three real angles
    $\boldsymbol{\theta}=(\theta_1,\theta_2,\theta_3)\transpose$ (Euler angles)
    \item three boosts with constant velocity $\vec{v}=(v_1,v_2,v_3)\transpose$
\end{itemize}
The Poincaré transformations form a ten parameter group.
\begin{table}
    \centering
    \begin{tabulars}{lll}
        \toprule
        &Newton&Einstein (SR)\\
        \midrule
        & Law of inertia
        & Law of inertia \\
        Newton's Laws
        & $\vec{F}=m\ddot{\vec{x}}=m\od{\vec{p}}{t}$
        &$\vec{F}=m\od{\vec{p}}{\tau}$\\
        & $\vec{F}_{12}=-\vec{F}_{21}$
        & Momentum conservations (postulate)\\
        Transformations
        & boosts
        & Galilei Transformation\\
        
        & absolute structure
        & absolute time\\
        
        & $c=\mathrm{const.}$
        & $t^\prime=t$\\
        Force
        &$\vec{F}=\frac{G\textsubscript{N}m_1m_2}{r^2}$
        &$F^\alpha= \frac{q}{m} U_\beta
        \tensor{F}{^\alpha^\beta}$,\,  $F
        =\gamma\left(\begin{smallmatrix}
\beta
f_0\\
\vec{f}
\end{smallmatrix}\right)$\\
&$\Delta \Phi =4\pi\varrho$ &Maxwell equations
\\
\bottomrule
    \end{tabulars}
    \caption{Comparison of Newtonian theory and special relativity}
\end{table}
\subsection{Fourvectors}
There are three types of vectors\footnote{We could also put up a fourth class,
containing solely the zero vector.}:
\begin{enumerate}
    \item \emph{Spacalike} $\Delta\tensor{x}{^\mu}\Delta\tensor{x}{_\mu}>0$
    \item \emph{Lightlike} $\Delta\tensor{x}{^\mu}\Delta\tensor{x}{_\mu}=0$
    \item \emph{Timelike} $\Delta\tensor{x}{^\mu}\Delta\tensor{x}{_\mu}<0$
\end{enumerate}
All classes are transfomed into themself by Lorentz transformation. There is a
set of transformations that is forbidden by physical considerations: namely the
Parity change $\vec{x}\to -\vec{x}$ and time reversal $t\to -t$ because this are
no real symmetries of nature.
We will mainly deal with timelike and lightlike vectors for which the time order
does not change. For spacelike intervals the order can change. To remain causal
all particles have to move at velocity $v\leq c$. This is in contrast to
Newtonian theory where everything can affect everything, because the velocities
are unbounded. A direct consequence of this is for example that even the two
body problem has no exakt solution in special relativity.
Tabular~\ref{tab:fourvecs} shows a list of the dynamic fourvectors in SR,
notice that in contrast to the classical, theory the fouraccelaration
$\tensor{a}{_\mu}$ has no importance in SR.
\begin{table}
    \centering
    \begin{tabulars}{lll}
      	\toprule
		Fourvector&Definition&Normalisation\\
		\midrule
		Velocity
		&$\tensor{u}{^\mu}=\od{\tensor{x}{^\mu}}{\tau}=\gamma(1,\vec{v})^{\transpose}$
		&$\tensor{u}{_\mu}\tensor{u}{^\mu}=-1$\\
		Momentum
		&$\tensor{p}{^\mu}:=m\tensor{u}{^\mu}$
		&$\tensor{p}{_\mu}\tensor{p}{^\mu}=-m^2$\\
		Acceleration&$\tensor{a}{^\mu}:=\od{\tensor{u}{^\mu}}{\tau}$&$\tensor{a}{_\mu}\tensor{a}{^\mu}=0$\\
		\bottomrule
    \end{tabulars}
    \caption{Examples of fourvectors and normalisation.\label{tab:fourvecs}}
\end{table}
\section{Lagrangian Formalism}
We want to formulate special relativity in terms of a variation principle.
Therefore we consider massive particles, i.e. timelike paths. As a postulate we
take that the action $S$ is proportional to the proper time (as a generalized
``distance'') $\dif \tau:=\sqrt{-\dif s^2}$
\begin{equation}
    S= -\alpha \int_a^b\dif \tau = -\alpha
    \int_{t_1}^{t^2}\sqrt{1-\vec{v}^2}\dif t =: \int_{t_1}^{t^2}L\dif t\, ,
\end{equation}
whith $\alpha$ a constant, that has to be determined. The Lagrangian $L$ is
given by
\begin{equation}
    L=-\alpha\sqrt{1-\vec{v}^2}\simeq
    -\alpha\left(1-\frac{1}{2}\vec{v}^2+\ldots\right) \, .
\end{equation}
We demand that we recover the classical theory in the limit $\vec{v}\to 0$. The
the lowest order kinetic term is
\begin{equation}
    T=\frac{1}{2}\alpha \vec{v}^2\, .
\end{equation}
Comparing with the classical kinetic energy $T=\frac{1}{2}m\vec{v}^2$ yields
$\alpha=m$. If we substitute $\alpha$ we recover the Lagrangian of special
relativity
\begin{equation}
    L\textsubscript{SR}=-m\sqrt{1-\vec{v}^2}=-m\gamma^{-1}\, .
\end{equation}
We can proceed calculate the generalized momenta
\begin{equation}
    p_i=\dpd{L}{v_i}=\frac{mv_i}{\sqrt{1-\vec{v}^2}}=\gamma m v_i\, .
\end{equation}
The energy can be calculated via the Hamiltonian $H$
\begin{equation}
    E=H=\vec{p}\cdot\vec{v}-L=\gamma m \vec{v}^2 + m\gamma^{-1} =\gamma m\, .
\end{equation}
Expanded in $\vec{v}^2$ the energy reads as
\begin{equation}
    E=m+\frac{1}{2}m\vec{v}^2+\ldots\, .
\end{equation}
If we restore units of $c$ for a moment we get that the constant term is
equal to $mc^2$. This is Einsteins famous $E=mc^2$.
We can further relate energy and momentum to each other. Therefore consider the
square of the momentum $\vec{p}$
\begin{equation}
    \vec{p}^2=\frac{m\vec{v}^2}{1-\vec{v}^2}\, .
\end{equation}
Solving for $\vec{v}^2$ yields
\begin{equation}
    \vec{v}^2=\frac{\vec{p}^2}{\vec{p}^2+m^2}\, .
\end{equation}
Which we can insert in the expression for the energy
\begin{equation}
    E^2=\frac{m^2}{1-\vec{v}^2}=\vec{p}^2+m^2 \label{eq:onshell}\, .
\end{equation}
The equation \eqref{eq:onshell} is called the \emph{on shell condition}.
\begin{sidenote}[On massive photons]
There is nothing in the theory that predicts that the photon is massless. It
could be possible that its mass is only really small and that in fact that the
photon does not travel at the speed of light. (This would make it
convinient to rename the constant $c$) Even though the mass should be so small
that the de Broglie wavelength is of the order of the size of the universe,
massive photons would have a huge impact.
\end{sidenote}
\begin{sidenote}[Complex Time]
We could in principle make a substitution $t\to \imI t$. Which would free us
from the need to distingish between co- and contravariant vectors, because the
inner product would be given as
\begin{equation}
    g(x,y)=\left(\imI x^0\right)\left(\imI
    y^0\right)+\vec{x}\cdot\vec{y}=-x^0y^0+\vec{x}\cdot\vec{y}
\end{equation}
and hence $\tensor{g}{_i_j}=\tensor{\delta}{_i_j}$, which basically means we can
raise and lower indices at will. This is practical whenever you do calculations,
e.g. in computer simulations.
\end{sidenote}
\section{Vectors and Tensors in SRT}
At this point we will just list some objects, that we call vectors and tensors
and delay the definition of those to a later chapter.
An example of a vector is the four current $J^\mu$:
\begin{equation}
    J^\mu=\gamma(\varrho,\vec{j})\, ,
\end{equation}
with $\varrho$ the charge density and $\vec{j}$ the 3-current density. A bit sloppy
we refer to a tensor as a 'vector with more indices' that transforms
accordingly, i.e.
\begin{equation}
    \tensor{T}{^\mu^\nu}\to
    \tensor{L}{^\mu_\alpha}\tensor{L}{^\nu_\beta}\tensor{F}{^\alpha^\beta}\,
\end{equation}
and similar for an arbitary number of indices. A typical example is the field strength tensor
$\tensor{F}{^\mu^\nu}$ of electrodynamics given in terms of the
vector potential $\tensor{A}{_\mu}$:
\begin{equation}
    \tensor{F}{_\mu_\nu}=\partial_\mu\tensor{A}{_\nu}-\partial_\nu\tensor{A}{_\mu}\,
    .
\end{equation}
The matrix form is
\begin{equation}
    \tensor{F}{^\mu^\nu}=
    \begin{bmatrix}
        0  &   -E_1 &  -E_2 &  -E_3 \\
        E_1 &   0  &  -B_3 & B_2 \\
        E_2 & B_3 &   0  &  -B_1 \\
        E_3 &  -B_2 & B_1 &   0  \\
    \end{bmatrix}\, .
\end{equation}
Therefore the tensor $\tensor{F}{^\mu^\nu}$ takes the role of the fields
$\vec{E},\vec{B}$. In tensor form \textsc{Maxwell} equations take the particular
simple form, namely
\begin{equation}
    \partial_\nu\tensor{F}{^\mu^\nu}=4\pi\tensor{J}{^\mu}\,, \quad
    \partial_\alpha\tensor{F}{_\mu_\nu}
    +\partial_\nu\tensor{F}{_\alpha_\mu}
    +\partial_\mu\tensor{F}{_\nu_\alpha}=0\,.
\end{equation}
The Lorentz force is given by
\begin{equation}
    \tensor{F}{^\mu}=q\tensor{F}{^\mu^\nu}\tensor{u}{_\nu}
\end{equation}
Where $\tensor{u}{^\mu}=(1,\vec{v})\transpose$ is the four velocity. As
annother example we can take a look at plane waves. The vector potential of a plane wave is
\begin{equation}
    \tensor{A}{^\mu}=\tensor{\hat{A}}{^\mu}
    \exp\left[\imI(\vec{k}\cdot\vec{r}-\omega
    t)\right]
    =\tensor{\hat{A}}{^\mu}\exp\left(\imI\tensor{K}{^\mu} \tensor{X}{_\mu}\right)\,
    .
\end{equation}
Where $\tensor{K}{^\mu}=\left(\omega,\vec{k}\right)\transpose$ must be a four
vector, because otherwise we could use plane waves to distinguish between
inertial systems. The dispersion relation reads as
\begin{equation}
    \tensor{K}{^\mu}\tensor{K}{_\mu}=-\omega^2+\vec{k}^2=0\, .
\end{equation}
It is equivalent to the on shell condition \eqref{eq:onshell}, when we identify
the photons energy $E=\omega$ momentum $\vec{p}=\vec{k}$ and mass $m=0$.
\subsection{The Energy Momentum Tensor}
Evtl. aus anderer Quelle, arges Durcheinander mit komplexer Notation\ldots kommt
später auch nochmal
\begin{sidenote}
Above a certain limit a system cannot be stabilized by pressure, because the
gravity cuples to the pressure and therefore increasing the pressure would lead
to a stronger attraction due to gravity and hence inevetiable collapse.
\end{sidenote}
The energy momentum tensor is symmetric
\begin{equation}
    \tensor{T}{^\mu^\nu}=\tensor{T}{^\nu^\mu}\, .
\end{equation}
Its trace is given by
\begin{equation}
    \tensor{T}{^\mu_\mu}=(\varrho-p)+4p=-\varrho+3p\, .
\end{equation}
It is traceless for photons. If an external field is present the divergence is
given by
\begin{equation}
    \tensor{T}{^\mu^\nu_,_\nu}=\tensor{D}{^\mu}\,  .
\end{equation}
Where the external momentum is itselve given as the divergence
\begin{equation}
    \tensor{D}{^\mu} =-\tensor{S}{^\mu^\nu_,_\nu}\, .
\end{equation}
Putting both terms together gives
\begin{equation}
    \partial_\mu(\tensor{T}{^\mu^\nu}+\tensor{S}{^\mu^\nu})=0\,,
\end{equation}
Which can be interpretet as the total energy conservation.
\subsection{Noether's Theorem}
We introduce a tensorial generalisation of the angular momentum (density???)
\begin{equation}
    \tensor{T}{^\lambda^\mu^\nu}:=
    \tensor{x}{^\lambda}\tensor{T}{^\mu^\nu}-\tensor{x}{^\mu}\tensor{T}{^\lambda^\nu}\,
    .
\end{equation}
The angular momentum is given by (etwas viele Komponenten?)
\begin{equation}
    \tensor{L}{^i^j} = \int\dif{}^3 x \tensor{M}{^i^j^0}\, .
\end{equation}
It can be shown that it is a conserved quantity
\begin{equation}
    \tensor{\partial}{_\nu}\left(\tensor{x}{^\lambda}\tensor{T}{^\mu^\nu}-\tensor{x}{^\nu}\tensor{T}{^\lambda^\nu}\right)
    =\tensor{T}{^\mu^\lambda}-\tensor{T}{^\lambda^\mu}+\tensor{x}{^\lambda}
    \tensor{T}{^\mu^\nu_,_\nu}-\tensor{x}{^\nu}\tensor{T}{^\lambda^\nu_,_\nu}=0\, .
\end{equation}
\begin{remark}
The existence of a conserved symmetric tensor $\tensor{T}{^\mu^\nu}$ is
nessecary in order to build a relativistic theory
\end{remark}
\begin{example}[Electrodynamics]
We start by inspecting the Lorentz Force $\tensor{F}{^\mu}$ which is given by
\begin{equation}
    \tensor{F}{^\mu}=e\tensor{F}{^\mu^\nu}\tensor{u}{_\nu}=\tensor{F}{^\mu^\nu}\tensor{J}{_\nu}=\tensor{D}{^\mu}\,
    .
\end{equation}
We now ask whether ther is a potential $\tensor{S}{^\mu^\nu}$ so that
$\tensor{D}{^\mu}=-\tensor{S}{^\mu^\nu_{,\nu}}$. Indeed a potential is given by
\begin{equation}
    \tensor{S}{^\mu^\nu}=\tensor{F}{^\mu_\alpha}\tensor{F}{^\nu^\alpha}-\frac{1}{4}\tensor{\eta}{^\mu^\nu}\tensor{F}{_\alpha_\beta}\tensor{F}{^\alpha^\beta}\,
    .
\end{equation}
We can check
\begin{equation}
    \tensor{S}{^\mu^\nu_{,\nu}}=\tensor{F}{^\mu_\alpha_{,\nu}}\tensor{F}{^\nu^\alpha}+
    \tensor{F}{^\mu_\alpha}\tensor{F}{^\nu^\alpha_{,\nu}}
    -\frac{1}{2}\tensor{\eta}{^\mu^\nu}\tensor{F}{_\alpha_\beta_{,\nu}}\tensor{F}{^\alpha^\beta}\,
    .
\end{equation}
\begin{align}
    \tensor{F}{^\mu_\alpha_{,\nu}}\tensor{F}{^\nu^\alpha}
    &=\tensor{F}{^\mu_\nu_{,\alpha}}\tensor{F}{^\alpha^\nu}
    =\tensor{F}{^\nu_\mu_{,\alpha}}\tensor{F}{^\nu^\alpha}
    \\
    %\tensor{F}{^\mu_\alpha}\tensor{F}{^\nu^\alpha_{,\nu}}&=\\
    \tensor{\eta}{^\mu^\nu}\tensor{F}{_\alpha_\beta_{,\nu}}\tensor{F}{^\alpha^\beta}
    &=\tensor{F}{_\alpha_\nu^{,\mu}}\tensor{F}{^\alpha^\nu}
    = -\tensor{F}{_\alpha_\nu^{,\mu}}\tensor{F}{^\nu^\alpha}
    \,
    .
\end{align}
\begin{equation}
\tensor{S}{^\mu^\nu_{,\nu}}=\tensor{F}{^\nu_\mu_{,\alpha}}\tensor{F}{^\nu^\alpha}
+\frac{1}{2}\tensor{F}{_\alpha_\nu^{,\mu}}\tensor{F}{^\nu^\alpha}
\end{equation}
The Lagrangian of electrodynamics is given by
\begin{equation}
    \mathcal{L}=-\frac{1}{2}\tensor{F}{_\mu_\nu}\tensor{F}{^\mu^\nu}+\tensor{J}{_\mu}\tensor{A}{^\mu}-\frac{1}{2}m^2\tensor{A}{_\mu}\tensor{A}{^\mu}\,
    .
\end{equation}
 The tensor $S$ is known as \emph{electromagnetic stress–energy tensor}.
\begin{equation}
\tensor{S}{^\mu^\nu}=
\begin{pmatrix}
u&\vec{S}\transpose\\
\vec{S}& -\tensor{\sigma}{_i_j}
\end{pmatrix}
\end{equation}
Where
$\tensor{\sigma}{_i_j}=\tensor{E}{_i}\tensor{E}{_j}+\tensor{E}{_i}\tensor{E}{_j}-\frac{1}{2}\tensor{\delta}{_i_j}\left(\vec{E}^2+\vec{B}^2\right)$
is the \emph{Maxwell stress tensor}, $\vec{S}=\vec{E}\times\vec{B}$
\emph{Poynting vector} and energy density $u
=\frac{1}{2}\left(\vec{E}^2+\vec{B}^2\right)$.
\end{example}

\chapter{Gravity and Geometry}
The observable universe is stable. There are two obvious configurations in which this is possible:
\begin{enumerate}
    \item Static universe, masses are arranged in a grid, all nett forces cancel.
    However small fluctuations cause the system to collapse therefore this is
    no possible description for the universe. 
%TODO picture
    \item Expanding universe, all masses move away from each other, overcoming the gravitational attraction.
    Theoretically such a system can be described by using Newtonian Physics introducing additional energy contributions.
    This turns out to be inconsistent.
\end{enumerate}
Since in the second description all particles are accelerated relative to each other, there are no inertial systems.
A theory in which all observers are equal must therefore be local and thus be described by means of differential geometry.
We claim that the laws of physics are the same in every system.
If we assume that the \name{Maxwell}'s equations are right, the
Newtonian theory of gravity must be wrong.
Implications:
All free falling systems are equivalent (i.e. indistinguishable by the observer).
Light must bend, otherwise a beam could be used to deduce whether your system is inertial.
The following example illustrates that Euclidean geometry is no
adequate description of space-time.
\begin{example}[Rotating Sphere]
see Introduction to tensor calculus
%TODO copy or reference page
\end{example}
\section{Coordinate Systems}
We will start by studying coordinate systems in the flat space $\Reals^2$, which
should be familiar.
\subsection*{Cartesian Coordinates}
Cartesian coordinates are described by two coordinates $x,y$ that are measured
in two orthogonal directions from the origin. The distance $s$ between two
arbitrary points $(x_1,y_1)$ and $(x_2,y_2)$ can be calculated using
\name{Pythagoras}' theorem
\begin{equation}
    s^2=(x_1-x_2)^2+(y_1-y_2)^2\,.
\end{equation}
An infinitesimal distance is likewise given by
\begin{equation}
    \dif s^2=\dif x^2+\dif y^2\,.  \label{eq:cartline}
\end{equation}
\subsection*{Polar Coordinates}
If we describe a point in flat space by an angle $\varphi$ and an distance $r$
from the origin, we get polar coordinates. The conversion between the systems
reads
\begin{equation}
    x= r\cos\varphi\quad y= r\sin\varphi\,.
\end{equation}
A infinitival change in the polar coordinates therefore results in 
\begin{align}
    \dif x&= \dpd{x}{r}\dif r+\dpd{x}{\varphi}\dif \varphi = \cos\varphi\dif
    r-r\sin\varphi\dif \varphi\,,\\
    \dif y&= \dpd{x}{r}\dif r+\dpd{y}{\varphi}\dif \varphi = \sin\varphi\dif
    r+r\cos\varphi\dif \varphi\,.
\end{align}
Plugging this into \eqref{eq:cartline} gives the line element in polar coordinates
\begin{equation}
    \dif s^2=\dif r^2+r^2\dif \varphi^2
\end{equation}
\begin{figure}[hbtp!]
\centering
 \includegraphics{cartcoord.pdf}
 \includegraphics{polarcoord.pdf}
\caption{}
%TODO Caption
\end{figure}
\begin{figure}[hbtp!]
\centering
 \includegraphics[scale=0.75]{CoordinateGridCartesian.pdf}\quad
 \includegraphics[scale=0.75]{CoordinateGridPolar.pdf}
\caption{Coordinate grids.}
%TODO Caption
\end{figure}

In matrix form
\begin{equation}
\dif s^2=
\begin{bmatrix}
\dif r& \dif \varphi
\end{bmatrix}
\begin{bmatrix}
1& 0\\
0& r^2\\
\end{bmatrix}
\begin{bmatrix}
\dif r\\ \dif \varphi
\end{bmatrix}\, .
\end{equation}
The matrix
\begin{equation}
g(\vec{r})=
\begin{bmatrix}
1& 0\\
0& r^2\\
\end{bmatrix}\, ,
\end{equation}
is called the \emph{metric}.
In general we have
\begin{equation}
    \dif s^2 = g_{ij}\dif x^i\dif x^j\, .
\end{equation}
The idea is to keep the law of inertia, i.e. particles still move on straight
line. However, we need to generalize the concept of a 'straight' line, in a
curved space.
\section{Variation Principle}
\label{sec:varprinc}
We know that straight lines are curves minimizing the distance between two
points. We generalize this concept to curved space by an variation principle.
Again we take a look at flat space, but with curved coordinates.
The length $S$ of a curve $\gamma$ with $\gamma^i(\lambda) = x^i(\lambda)$ is
given by the integral
\begin{equation}
    S=\int_{\gamma}\sqrt{\dif s^2} =
    \int_{\gamma}\sqrt{\tensor{g}{_i_j}\dif \tensor{x}{^i}\dif
    \tensor{x}{^j}}=\int_{a}^{b}\sqrt{\tensor{g}{_i_j}\dod{\tensor{x}{^i}}{\lambda}
    \dod{\tensor{x}{^j}}{\lambda}}\dif \lambda\, .
\end{equation}
As stated above generalised straight lines satisfy $\delta S = 0$. If we define
$L:=\left(\tensor{g}{_i_j}\od{\tensor{x}{^i}}{\lambda}
\od{\tensor{x}{^j}}{\lambda}\right)^{\nicefrac{1}{2}}$, $S$ takes a form
familiar from classical mechanics:
\begin{equation}
    S=\int_a^b L\dif \lambda\, .
\end{equation}
The extremal condition implies the Euler Lagrange equations
\begin{equation}
    \dod{}{\lambda}\pd{L}{\left(\pd{\tensor{x}{^i}}{\lambda}\right)}
    -\pd{L}{\tensor{x}{^i}}
    =0\, .		\end{equation}
We can calculate the relevant terms to 
\begin{align}
\dpd{L}{\tensor{x}{^i}}&=\frac{1}{2\sqrt{g_{ij}\od{x^i}{\lambda}
\od{x^j}{\lambda}}}\tensor{g}{_j_k_{,i}}\dod{x^j}{\lambda}
\dod{x^k}{\lambda}\,,\\
\dpd{L}{\left(\pd{\tensor{x}{^i}}{\lambda}\right)}
&=\frac{1}{\sqrt{g_{ij}\od{x^i}{\lambda}
\od{x^j}{\lambda}}}\tensor{g}{_j_i}\dod{x^j}{\lambda}\, .
\end{align}
If we choose the parameter $\lambda$ so that we are parametrised by the arc
length\footnote{this is impossible for null i.e. lightlike geodesics, it can be
shown however, that the resulting equation also holds true for null geodesics.}
i.e.
\begin{equation}
\od{}{\lambda}\left(\sqrt{g_{ij}\od{x^i}{\lambda}\od{x^j}{\lambda}}\right)=0\,,
\end{equation}
the Euler Lagrange equations simplify to
\begin{equation}
0=\frac{1}{\sqrt{\tensor{g}{_i_j}\od{\tensor{x}{^j}}{\lambda}
\od{\tensor{x}{^j}}{\lambda}}}\dod{}{\lambda}
\left(\tensor{g}{_j_i}\dod{\tensor{x}{^j}}{\lambda}\right)
-\frac{1}{2\sqrt{\tensor{g}{_i_j}\od{x^i}{\lambda}
\od{\tensor{x}{^j}}{\lambda}}}\tensor{g}{_j_k_{,i}}\dod{\tensor{x}{^j}}{\lambda}
\dod{\tensor{x}{^k}}{\lambda}\,,
\end{equation}
or equivalently
\begin{equation}
\begin{split}
0
&=\dod{}{\lambda}\left(\tensor{g}{_j_i}\dod{\tensor{x}{^j}}{\lambda}\right)
-\frac{1}{2}\tensor{g}{_j_k_{,i}}\dod{x^a}{\lambda}
\dod{x^k}{\lambda}\\
&=\tensor{g}{_j_i_{,k}}\dod{\tensor{x}{^j}}{\lambda}\dod{\tensor{x}{^k}}{\lambda}
+\tensor{g}{_j_i}\dod[2]{\tensor{x}{^j}}{\lambda}
-\frac{1}{2}\tensor{g}{_j_k_{,i}}\dod{\tensor{x}{^j}}{\lambda}\\
&=\tensor{g}{_j_i}\dod[2]{\tensor{x}{^j}}{\lambda}
+\frac{1}{2}\left(\tensor{g}{_j_i_{,k}}+\tensor{g}{_i_j_{,k}}
-\tensor{g}{_j_k_{,i}}\right)\dod{\tensor{x}{^j}}{\lambda}
\dod{\tensor{x}{^k}}{\lambda}\label{eq:PreGeo}\,.
\end{split}
\end{equation}
The term invoking derivatives of the metric defines the \emph{Christoffel
symbols of the first kind}
\begin{equation}
    \csym{j}{k}{i}:=\frac{1}{2}
    \left(\tensor{g}{_j_i_{,k}}+\tensor{g}{_i_j_{,k}} -\tensor{g}{_j_k_{,i}}\right)\, .
\end{equation}
It is convenient to multiply \eqref{eq:PreGeo} by the inverse metric $g^{li}$ so
that we obtain the \emph{geodesic equation}
\begin{equation}
    0 =
    \od[2]{\tensor{x}{^l}}{\lambda}
    +\cSym{l}{j}{k}\od{\tensor{x}{^j}}{\lambda}\od{\tensor{x}{^k}}{\lambda}\,
    .\label{eq:geodeq}
\end{equation}
Where $\cSym{l}{j}{k}$ are the \emph{Christoffel symbols of the second kind}
\begin{equation}
    \cSym{l}{j}{k}:=g^{li}\csym{j}{k}{i}=\frac{1}{2}g^{li}
    \left(\tensor{g}{_j_i_{,k}}+\tensor{g}{_i_j_{,k}} -\tensor{g}{_j_k_{,i}}\right)\, .
\end{equation}
% remark is obsolete as long as bracket notation for christoffel symbols is used
%\begin{remark}
%Although the notation looks as the Christoffel symbols form a tensor, however
% they do not.
%\end{remark}
In flat space we have $\tensor{g}{_i_j}=\tensor{\eta}{_i_j}$ and can easily
check that all Christoffel symbols vanish. We therefore recover the ordinary equation of motion for a free particle
\begin{equation}
    0 = \od[2]{\tensor{x}{^i}}{\lambda}\, .
\end{equation}
\begin{figure}[hbtp!]
\centering
 \includegraphics{sphere_geodesics1.pdf}
\caption{Great circles are geodesics, i.e. shortest connections of points, on
a sphere.}
%TODO Caption
\end{figure}
\begin{figure}[hbtp!]
\centering
 \includegraphics{WorldlineLightcones.pdf}
\caption{}
%TODO Caption
\end{figure}


% \begin{example}
% Suppose a observer follows a free falling body in a homogeneous field.
% Therefore a transformation between the system of the earth and the one of the body are given by
% (for simplicity we only consider the coordinate along it is falling)
% \begin{equation}
%     (t,x)\to\left(t,x-\frac{1}{2}gt^2\right)
% \end{equation}
% Analogous to the Riemannian case discussed before, the line element takes the form
% \begin{equation}
%     \begin{split}
% \dif s^2&=-\dif t^2 +\dif x^2\\
% &=-\dif t'^2+(\dif x'- gt\dif t')(\dif x'- gt\dif t')\\
% &=(g^2t'^2-1)\dif t'^2-2gt\dif x'\dif t'+\dif x'^2
% \end{split}
% \end{equation}
% \end{example}
% \section{Newtonian Limit}

\chapter{Differential Geometry}
As we have noted before general relativity is a inherent local theory. It is convenient to formulate it in terms of differential geometry.
We introduce the notion of a manifold.

\begin{definition}
A $n$ dimensional manifold $M$ is a Hausdorff space with countable basis, that
is locally homeomorphic to $\mathbb{R}^n$.
\end{definition}
\begin{remark}
The requirements Hausdorff and countable basis are of a more technical nature and are satisfied for most of the objects one can imagine 
except some pathological examples (we won't go into the details on this).

Locally homeomorphic to $\mathbb{R}^n$ means there exists a set of \emph{charts} 
$(\varphi,U^\varphi)$ called an \emph{atlas} $\mathcal{A}$ with $\cup_{\varphi\in\mathcal{A}} U^\varphi =M$, 
i.e. the charts cover the whole manifold. The maps $\varphi:U^\varphi\to \varphi(U^\varphi)\subset\mathbb{R}^n $ are homoemorphisms, 
meaning that $U^\varphi$ is open, $\varphi$ is bijective and both $\varphi$ and $\varphi^{-1}$ are continuous.
Further for any two $\varphi,\psi\in \mathcal{A}$, the coordinate changes 
$\varphi\circ\psi^{-1}:\psi(U^\psi\cup U^\varphi)\to \phi(U^\psi\cup U^\varphi)$ be infinitely often differentiable.
(BIlder)
\end{remark}

We can now reduce differentiation on the manifold to the ordinary differentiation in $\mathbb{R}^n$. 
Since physical laws are described in terms of differential equations, we can formulate them on $M$. 
The fact that the coordinate changes are $C^\infty$ ensures that differentiability is well defined (and thus the physical laws are).

\begin{sidenote}[Differentiability depends on the Differential Structure]
There can be different \emph{differential structures} on a manifold, 
which means there are multiple (maximal)alases, which
could not be merged because the coordinate changes would not be $C^\infty$. Those differentiable structures therefore imply different notions of differentiability. 
Remarkably this may even play a role in some physical theories. 
As an example an 11d-supergravity can be described as a product
$\Reals^{3+1}\times \Sphere^7$.
Where $\Sphere^7$ is the 7-sphere and $\Reals^{3+1}$ Mikovski space.
This means on every point in the $\mathbb{R}^{3+1}$ there is a (small) $\Sphere^7$  located that contains additional spatial dimensions. 
The $\Sphere^7$ has 28 different differential structures, so the choice of
such a structure affects the theory for the above reasons.
\end{sidenote}
All simple examples we come of can be embedded in a higher space. It holds true
that every real $n$-dimensional Manifold can be embedded to $\Reals^{2n}$
(This is however not true for complex, i.e. analytic manifolds).
For example the $\Sphere^2$ can be interpreted as submanifold of the $\Reals^3$.
However manifolds are objects that exists independent of such embeddings. 
For example a torus can be thought of as a square with the opposite sides identified (leaving to the left results in re-entering in the left).
\begin{sidenote}[Topology of the Universe]
In addition to the local structure, we may question the global, i.e. the
topological structure of the universe.
On may for example imagine that we live on the surface of a 3-sphere (finite but boundless universe). 
However this might be observable in crosscorelation in the cosmic microwave background from photons reaching us 
from different directions but coming from the same event. There is no evidence of such phenomena so far. 
Most models can be excluded to some certainty. A cylindrical
universe is still possible (finite in one, infinite in the
other directions).
\end{sidenote}
\section{Vectors}
Vectors are important objects describing physics. The naive view as an "arrow pointing frow one point to another" is flawed. 
For example on a sphere an arrow connecting two points does not make much sense.
We want to find a description of vectors as objects that are naturally related to the structure of the manifold independent of the embedding.
There are three equivalent definitions for a vector:
\begin{enumerate}
    \item algebraic (mathematical, suitable for proofs)
    \item physically
    \item geometrically (ugly, but plastic)
\end{enumerate}
\subsection*{Definitions}
\begin{definition}[Vector, algebraic]
A vector is a derivation at the germ of a function at $p$.
\end{definition}
The germ is the set of all functions that are locally equal,i.e. vectors are local objects.
\begin{definition}[Derivation]
A derivation $D$ satisfies the following rules for all $f,g\in C^\infty(\Reals)$ and $\lambda \in \Reals$:
\begin{align}
    D(af+bg) &=Df+Dg\\
    D(\lambda f)&=\lambda f\\
    D(fg)&= (Df)g+f(Dg)
\end{align}
\end{definition}
Given two vectors we can construct a new one, the \emph{Lie Braket}
\begin{equation}
    [X,Y]f:=X(Yf)-Y(Xf)\, .
\end{equation}
The only property that has to be checked is that it satisfies the Leibniz rule.
\begin{equation}
    XY(fg)=X((Yf)g+f(Yg))=(XYf)g+(Yf)(Xg)+(Yg)(Xf)+(XYg)f\\
\end{equation}
Subtracting $YX(fg)$ proves that $[X,Y]$ is indeed a vector. The fact that we
have a natural vectorspace structure on the set of vectors at $p$ motivates the
following
\begin{definition}
The Tangentspace $T_pM$ is the space of all vectors in a point $p\in M$.
\end{definition}
A basis of $T_pM$ is given by the derivation along the
coordinates $\partial_i$, therefore its dimension is equal to that of the manifold $M$.
Proof sketch:
\begin{enumerate}
    \item Show $f(x^i)=f(0)+x^i\tilde{f}(x^i)$
    \item Write $X=a^i\partial_i$
    \item Show $Xf=0\quad \forall f \iff X=0$
\end{enumerate}
Every vector $A$ can be written as $A=A^i\pd{}{{x^i}}$, where $A^i$ are the
components of the vector. We can now look how the components of the vector
transform under a change of coordinates (the vector itself is invariant!). 
We usually denote the elements of the transformed systems with a bar.
By the chain rule we have
\begin{equation}
    A= a^k\pd{}{{x^k}}= a^k\pd{\overline{x}^i}{{x^k}}\pd{}{{\overline{x}^k}}\, .
\end{equation}
We can also express $A$ directly in the new basis
\begin{equation}
    A= \overline{a}^i\pd{}{{\overline{x}^i}}\, .
\end{equation}
Comparing the coefficients gives the vector transformation law
\begin{equation}
    \overline{a}^i=a^k\pd{\overline{x}^i}{{x^k}}\label{eq:coefftrafo}\, .
\end{equation}
Sometimes a vector is defined as a object that transforms according to \ref{eq:coefftrafo} under a change of coordinates, 
this is the physical definition. It is a priori not clear that a vector also corresponds to a geometrical object. 
Consider a curve on $M$, i.e. a map $\gamma:\mathbb{R}\to M$, with
$\gamma(0)=p$, $\dot{\gamma}(0)=X$. Then $D_X f=\od{}{t}(f\circ\gamma)(0)$ is
a derivative, namely the directional derivative along $X$.
For the special curves $\gamma_i(t)=p+te_i$
$D_{\dot{\gamma}_i} f=\partial_if$, so we can identify the derivatives with
the geometrical tangent space.
Since we have a basis we can work in local coordinates, 
e.g. let $A=A^i\pd{}{{x^i}}$, $B=B^i\pd{}{{x^i}}$, then the Lie bracket reads
\begin{equation}
    [A,B]^j=A^i\partial_iB^j-B^i\partial_iA^i\, .
\end{equation}
Since the tangent space is a vector space, we can define its dual space
\begin{equation}
    T_pM^*=\{L:T_pM\to \mathbb{R}\, |\, L \text{ linear}\}\, ,
\end{equation}
which is again a vector space of the same dimension. Its elements are called dual or covariant vectors.
We can define a basis on $	T_pM^*$, which we denote by $\dif x^i$ and  which acts on $T_pM$ via
\begin{equation}
    \dif x^i(\partial_j)=\delta^i_j\, . \label{eq:orthdual}
\end{equation}
It can easily deduced by \eqref{eq:orthdual} that the components of a dual vector transform as
\begin{equation}
    \overline{a}_i=\pd{x^k}{{\overline{x}^i}}a_k\, .
\end{equation}
\begin{remark}[Dual vectors in flat space]
If $\vec{a},\vec{b}\in\mathbb{R}^n$ contain the component of a vector and a dual vector respectively, 
then the transformation can be written in matrix form
\begin{align}
    \vec{a}&\to V\vec{a}\, ,\\
    \vec{b}&\to\left(V^\intercal\right)^{-1}\vec{b}\, ,
\end{align}
with $V_{ij}=\pd{\overline{x}^i}{{x^j}}$. 
In normal calculus we restrict ourself to orthogonal transformations (i.e.
mapping orthonormal bases onto each other) for which $(O^\intercal)^{-1}=O$.
Which is the reason why we do not bother to distinguish between vectors and dual vectors because they transform identically. 
In special relativity we have e.g. $(\Lambda^\intercal)^{-1}\neq\Lambda$ for a
boost, the difference becomes even more important in general relativity where the relation can become arbitrarily complicated.
\end{remark}
\section{Tensors}
From vectors $A$ ,$B$ we can construct new objects with multiple indices that posses well defined transformation behaviour. 
For example consider
\begin{equation}
    \overline{T}^{ij}=a^ib^j\, ,
\end{equation}
which transforms as
\begin{equation}
    T^{ij}=\pd{\overline{x}^i}{{x^k}}\pd{\overline{x}^j}{{x^l}}a^kb^l=\pd{\overline{x}^i}{{x^k}}\pd{\overline{x}^j}{{x^l}}T^{kl}\,
   .\label{eq:tensortrafo}
\end{equation}
We call an object that transforms in this way a \emph{tensor}. 
As with vectors, it is possible to define tensors in a coordinate independent
way.
At this point we will make things easier and only consider the physical
definition, i.e. classify tensors by a transformation according to \eqref{eq:tensortrafo}.

A tensor is said to be symmetric in two indices if it stays invariant when exchanging those indices, e.g.
\begin{equation}
    T_{ab}=T_{ba}\, .
\end{equation}
\begin{remark}
We have not yet established a relation between upper and lower indices, i.e. we have no metric. Expressions of the form
\begin{equation}
    \tensor{T}{^a_b}=\tensor{T}{^b_a}
\end{equation}
make no sense, since they can not be true in every system.
\end{remark}
\section{The Metric}
\begin{definition}[Metric]
The metric $g$ is a non-degenerate ($\det(g)\neq 0$), symmetric covariant two
tensor.
\end{definition}
We have already seen examples of metrics for the flat space, e.g. in spherical coordinates $g$ was given as
\begin{equation}
    g=
    \begin{pmatrix}
        1 & 0\\
        0 & r^2\\
    \end{pmatrix}
\end{equation}
Given a metric we relate vectors and dual vectors to each other by
\begin{equation}
    a_i=g_{ij}a^j
\end{equation}
\section{Parallel Transport}
Idea: Generalize parralel transport from flat space.
(PICTURES) 
If we express a vector in non-cartesian coordinates and shift it it's
coordinates do not change.
We take a look on two operations:
\begin{enumerate}
\item the change of the vector itself
\item the change of its coordinates
\end{enumerate}
Let $A_i$ be the coordinates of a vector in a system $x^i$ and $B_i$ the in a
system associated with coordinates $y^i$ respectively.
\begin{equation}
A_i\pd{{y^j}}{{x^i}}B_j\, ,\quad B_i\pd{{x^j}}{{y^i}}A_j\, .
\end{equation}
We look at vectors whose coodinates in the system $y^i$ do not change i.e.
$\delta B_i=0$ The variation of $A_i$ is given by
\begin{equation}
\delta
A_i=\delta\left(\pd{{y^j}}{{x^i}}\right)B_j
=\md{{y^j}}{2}{{x^i}}{}{{x^k}}{}\delta
x^k B_j\, .
\end{equation}
Expressing $B_i$ in terms of $A_i$ yields
\begin{equation}
\delta A_i = \md{{y^j}}{2}{{x^i}}{}{{x^k}}{}\pd{{x^l}}{{y^j}}A_l\delta x^k
=:\affin{l}{i}{k}A_l\delta x^k
\end{equation}
$\affin{l}{i}{k}$ is called \emph{affine connection} or short affinity.
\begin{remark}
We can always find a coordinate system in wich $\affin{l}{i}{k}\equiv 0$.  
\end{remark}
We notice that if 
\begin{equation}
\left(\pd{{A_i}}{{x^j}}-\affin{l}{i}{k}A_l\right)\delta x^k = 0\, ,
\end{equation}
The vector $A$ does not change its cordinates. We define a \emph{covariant
derivative} 
\begin{equation}
\tensor{A}{_i_;_j}:=\pd{{A_i}}{{x^j}}-\affin{l}{i}{k}A_l\, .
\end{equation}
It can easyly be seen that the covariant derivative of a tensor transforms as a
tensor. 
\begin{remark}[The
Covariant Derivative in Electrodynamics] Example from Electrodynamics concerning the covariant derivative. 
The theory is invariant under transformations $\phi\to e^{\imI \alpha}\phi$, 
because $\phi^*\phi$ and $\phi^*\nabla\phi-\phi\nabla\phi^*$ do not change.
Vervollständigen\ldots
\end{remark}
Since we have now established a relation between vectors and dual vectors, we
can also determine the covariant derivative of a dual vector. Therefore we
consider the scalar $A_iB^i$. Since the covariant derivative satisfies the
Leibniz rule we get
\begin{align}
(A_iB^i);j = \tensor{A}{_i_;_j}\tensor{B}{^i}+\tensor{A}{_i}\tensor{B}{^i_;_j}
\end{align}
But for scalars the covariant derivative is identical to the normal derivative
so that 
\begin{align}
(A_iB^i)_{;j} =(A_iB^i)_{,j}=
\tensor{A}{_i_,_j}\tensor{B}{^i}+\tensor{A}{_i}\tensor{B}{^i_,_j}
\end{align}
If we put in the covariant derivative of a we get 
\begin{equation}
\tensor{A}{_i}\tensor{B}{^i_;_j}=\tensor{A}{_i}\left(\pd{}{{x^j}}B^i+\Gamma^i_{kj}\tensor{B}{^k}\right)
\end{equation}
Since $A$ was arbitary, we can deduce that
\begin{equation}
\tensor{B}{^i_;_j}=\left(\pd{}{{x^j}}B^i+\Gamma^i_{kj}\tensor{B}{^k}\right)
\end{equation}
for a (1,1)-tensor we get:
\begin{equation}
\tensor{A}{^i_j_;_k}=\pd{}{{x^k}}\tensor{A}{^i_j}-\Gamma^a_{jk}\tensor{A}{^i_a}+\Gamma^i_{ak}\tensor{A}{^a_j}\,
.\end{equation}
Similar expressions hold for tensors of arbitary rank where each index gives an
aditional term containing a contraction with the affinity $\Gamma^i_{jk}$. 
We now want to consider curved spaces. This can not immediatly be determined by
the metric, for example the polar coordinates do not look flat even though they
describe the ordinary $\Reals^2$. (PICTURES)

\chapter{Einstein's Field Equations}
We will derive Einsteins Equations by physical considerations. The poisson
equation reads as:
\begin{equation}
\Delta\Phi=4\pi\rho
\end{equation}
\begin{equation}
S\textsubscript{EH}=\int\dif x^4 \sqrt{-g}(R-2\Lambda)
\end{equation}
In SR, the energy impuls tensor $\tensor{T}{_\mu_\nu}$ is conserved i.e.
$\tensor{T}{_\mu_\nu^{,\nu}}=0$.As a natural extension we demand that the energy
impuls tensor of general relativity is \emph{covariantly} conserved
\begin{equation}
\tensor{\nabla}{^\nu}\tensor{T}{_\mu_\nu}=\tensor{T}{_\mu_\nu^{;\nu}}=0
\end{equation}
We now ask for the most general tensor satisfying this equation.
\begin{theorem}[Lovelock]
For a fourdimensional space the most general divergence free tensor
$\tensor{A}{_\mu_\nu}$ is given by
\begin{equation}
\tensor{A}{_\mu_\nu}= c_1\tensor{G}{_\mu_\nu}+c_2\tensor{g}{_\mu_\nu}\, .
\end{equation}
Where $\tensor{G}{_\mu_\nu}$ is the \emph{Einstein tensor}
$\tensor{G}{_\mu_\nu}:=\tensor{R}{_\mu_\nu}-\frac{1}{2}\tensor{g}{_\mu_\nu}R$.
\end{theorem}
The Theorem imediatly implies that
\begin{equation}
\tensor{R}{_\mu_\nu}-\frac{1}{2}R\tensor{g}{_\mu_\nu}-\Lambda\tensor{g}{_\mu_\nu}=\kappa\tensor{T}{_\mu_\nu}
\end{equation}
for some constants $\kappa$, $\Lambda$.
Of course we identify $\kappa=\frac{8\pi}{c^2}$ and $\Lambda$ is the
cosmological constant. We can rewrite equation (??) as
\begin{equation}
\tensor{R}{_\mu_\nu}-\frac{1}{2}R\tensor{g}{_\mu_\nu}
=\kappa\left(\tensor{T}{_\mu_\nu}-\frac{\Lambda}{\kappa}\tensor{g}{_\mu_\nu}\right)
\end{equation}
so that the left hand side represents the geometrical part and the right hand
side the matter content.
Wheeler: ``Geometry tells matter how to move, matter tells geometry how to
curve.''
\begin{sidenote}
In the time of the inflation the cosmological constant must have been large.
Since it is small today it has do decay with time
\end{sidenote}
We can identfy $\Lambda$ with an vacuum energy so that. Can we get the Einstein
equations from a variation principle?
\begin{equation}
S\textsubscript{g}=\int\dif x^4 \sqrt{-g}\tilde{\mathcal{L}}
\end{equation}
$\tilde{\mathcal{L}}$ must transform a (scalar) density, therefore we define a
scalar $\sqrt{-g}\mathcal{L}=\mathcal{L}$
\begin{equation}
S\textsubscript{g}=\int\dif x^4 \sqrt{-g}\mathcal{L}
\end{equation}
One can think of various contributions to $\mathcal{L}$
\begin{equation*}
R,\, \square
R,\,\tensor{\nabla}{^\mu}\tensor{\nabla}{^\mu}\tensor{R}{_\mu_\nu},\,
\tensor{R}{_\mu_\nu}\tensor{R}{^\mu^\nu}
,\,\tensor{R}{_\mu_\nu_\sigma_\rho}\tensor{R}{^\mu^\nu^\sigma^\rho}\dots
\end{equation*}
Which have to be contracted so that the resulting quantity becomes a scalar.
We have no contributions of the metric because
$\tensor{g}{_\mu_\nu_{;\sigma}}=0$. From Yang-Mills theory one would expect a
structure
\begin{equation}
\mathcal{L}\sim\tensor{F}{_\mu_\nu}\tensor{F}{^\mu^\nu}
\end{equation}
but $\Gamma$ is not the fundamental field but $g$ is. If we demand that we only
have up to second derivatives of $g$ the only allowed term in the Lagrangian is
$R$.
\begin{sidenote}[On higher derivatives]
If we include higher order derivatives of $g$ in the right way we can make the
resulting theory renormalisable. However we violate unitarity and introduce so
called ghost fields which are associated with the additional degrees of freedom
we get.
\end{sidenote}
\begin{remark}[Dimensions]
In natural units the line element $\dif s^2$ has dimension ${[\dif
s^2]=\textrm{M}^{-2}}$, Since further
${\left[\tensor{x}{^\mu}\right]=\textrm{M}^{-1}}$ the Lagrange density must have Dimension ${\left[\tilde{\mathcal{L}}\right]=\textrm{M}^{4}}$
\end{remark}
This constraint leads to the \emph{Einstein-Hilbert-action}
\begin{equation}
S\textsubscript{EH}=\frac{1}{2\kappa}\int\dif x^4 \sqrt{-g}(R-2\Lambda)
\end{equation}
We now check that its variation indeed reproduces Einsteins equations. To do
so we introduce the formalism of \emph{functional derivative}. Let therefore
$\Phi=\{\varphi,\tensor{A}{^\mu},\Psi,\dots\}$ be a collection of fields.
$F[\Phi]$ a functional.
We define the variation of $F$ as
\begin{equation}
\delta F:=\int \dif x \frac{\delta F}{\delta\Phi^i}\delta\Phi^i\,.
\end{equation}
Typically the functionals are givenn in the form
\begin{equation}
S[\Phi]=\int \dif x L(x,\Phi)\, ,
\end{equation}
where $L$ is some local function.
\begin{equation}
\frac{\delta\tensor{g}{_\rho_\sigma}(x)}{\delta\delta\tensor{g}{_\mu_\nu}(x')}=\tensor*{\delta}{*^\mu*^\nu*_\rho*_\sigma}\delta(x,x')\
\end{equation}
Where
$\tensor*{\delta}{*^\mu*^\nu*_\rho*_\sigma}=\frac{1}{2}\left(\tensor*{\delta}{^\nu_\rho}\tensor*{\delta}{^\mu_\sigma}+\tensor*{\delta}{^\mu_\rho}\tensor*{\delta}{^\nu_\sigma}\right)$
is the unity of the space of symmetric rank two tensors
\begin{remark}
In general $\delta(x,x')\neq \delta(x-x')$
\end{remark}
\section{Introduction of Matter}
In the gravitational kontext we mean be \emph{matter} any non gravitational
fields this include scalar fields $\varphi$, spinor fields $\Psi$, gauge fields
$\tensor{A}{^\mu}$,\dots. We collect all of them in a multivariable $\Phi$
A local action can be written as
\begin{equation}
S\textsubscript{m}[\Phi,g]=\int \dif x^4\sqrt{-g}
L\textsubscript{m}\left(\Phi,\tensor{\nabla}{_\mu}\Phi,g\right)
\end{equation}
$\tensor{g}{^\mu^\nu}$ appears in $L\textsubscript{m}$ because the derivatives
$\tensor{\nabla}{_\mu}, \tensor{\partial}{_\mu}$ must be contracted.
Additionally it enters via $\sqrt{-g}$
\begin{example}[free scalar field in Minkowski space]
\begin{equation}
S\textsubscript{m}=\int \dif x^4 \left(-\frac{1}{2}\tensor{\eta}{^\mu^\nu}
\tensor{\partial}{_\mu}\varphi\tensor{\partial}{_\nu}\varphi-\frac{1}{2}m^2\varphi^2\right)
\end{equation}
The minus sign in front of the partial derivative should come as no surprise
since we have $\tensor{\eta}{^0^0}=-1\, \dot{\varphi}^2>0$. In a non inertial
frame we have to make the usual replacements
\begin{equation}
\tensor{\eta}{_\mu_\nu}\to \tensor{g}{_\mu_\nu}\, , \quad
\tensor{\partial}{_\mu}\to
\tensor{\nabla}{_\mu}\, \quad \dif x^4\to \dif x^4\sqrt{-g}\, .
\end{equation}
\end{example}
\emph{minimal cuppling description} (Kontext???) The action reads as
\begin{equation}
S\textsubscript{m}=\int \dif x^4 \sqrt{-g}\left(-\frac{1}{2}\tensor{g}{^\mu^\nu}
\tensor{\nabla}{_\mu}\varphi\tensor{\nabla}{_\nu}\varphi-\frac{1}{2}m^2\varphi^2\right)
\end{equation}
which is the action for a scalar $\varphi$ in the presence of gravity, i.e. a
dynamical $\tensor{g}{_\mu_\nu}(x)$. The combined action of scalar field and
gravity is given as
\begin{equation}
S[g,\varphi]=S\textsubscript{g}[g]+S\textsubscript{m}[g,\varphi]
\end{equation}
The Euler Lagange Equations in terms of the scalar field read as
\begin{equation}
\frac{\delta S[g,\varphi]}{\delta
\varphi\left(x'\right)}=\frac{\delta S\textsubscript{m}[g,\varphi]}{\delta \varphi\left(x'\right)}
\end{equation}

\begin{equation}
\frac{\delta S\textsubscript{m}[g,\varphi]}{\delta \varphi\left(x'\right)}=\int
\dif x^4\sqrt{-g}\left[-\tensor{g}{^\mu^\nu}
\tensor{\nabla}{_\mu}\varphi\tensor{\nabla}{_\nu}\left(\frac{\delta
\varphi(x)}{\delta \varphi\left(x'\right)}\right)-m^2\varphi\frac{\delta
\varphi(x)}{\delta \varphi\left(x'\right)}\right]
\end{equation}
where we used that the $\delta$ and $\tensor{\nabla}{_\mu}$ commute. Partial
integration yields
\begin{equation}
\begin{split}
\frac{\delta S\textsubscript{m}[g,\varphi]}{\delta \varphi\left(x'\right)}&=\int
\dif
x^4\sqrt{-g}\left(\square_{g}-m^2\right)\varphi\delta(x,x')\\
&=\sqrt{-g}\left(\square_{g}-m^2\right)\varphi
\end{split}
\end{equation}
where $\square_{g}:=\tensor{g}{^\mu^\nu}
\tensor{\nabla}{_\mu}\tensor{\nabla}{_\nu} $ is the
\emph{Laplace–Beltrami operator}, a generalisation of the ordinary laplacian to
curved space.
Demanding that the variation with respect to $\varphi$ vanishes implies the
\emph{Klein-Gordon equation}
\begin{equation}
\left(\square_g-m^2\right)\varphi=0
\end{equation}
Variing with respect to the metric $\tensor{g}{_\mu_\nu}$
\begin{equation}
\frac{\delta S[g,\varphi]}{\delta
\tensor{g}{_\mu_\nu}\left(x'\right)}=
\frac{\delta S\textsubscript{g}[g]}{\delta
\tensor{g}{_\mu_\nu}\left(x'\right)}
+\frac{\delta S\textsubscript{m}[g,\varphi]}{\delta
\tensor{g}{_\mu_\nu}\left(x'\right)}
=\frac{\sqrt{-g}}{2\kappa}\left(\tensor{G}{^\mu^\nu}+\Lambda\tensor{g}{^\mu^\nu}
\right)+\frac{\delta S\textsubscript{m}[g,\varphi]}{\delta
\tensor{g}{_\mu_\nu}\left(x'\right)}
\end{equation}
This makes it convenient to define the energy-momentum tensor (Vorzeichen????)
\begin{equation}
\tensor{T}{^\mu^\nu}:=\frac{2}{\sqrt{-g}}\frac{\delta
S\textsubscript{m}[g,\varphi]}{\delta \tensor{g}{_\mu_\nu}\left(x'\right)}\,.
\end{equation}
\begin{remark}
Attention $\delta_g
\tensor{g}{^\mu^\nu}=-\tensor{g}{^\rho^{(\mu}}\tensor{g}{^{\nu)}^\sigma}\delta_g
\tensor{g}{_\mu_\nu}$
\end{remark}
We can now proceed in calculating the quantity we have ust introduced for a
scalar field
\begin{equation}
\begin{split}
\frac{\delta
S\textsubscript{m}[g,\varphi]}{\delta \tensor{g}{_\mu_\nu}\left(x'\right)}
&=\int \dif x^4 \frac{\delta\sqrt{-g}}{\delta
\tensor{g}{_\mu_\nu}}\left(-\frac{1}{2}\tensor{g}{^\mu^\nu}
\tensor{\nabla}{_\mu}\varphi\tensor{\nabla}{_\nu}\varphi-\frac{1}{2}m^2\varphi^2\right)\\
&\phantom{=}+
\sqrt{-g}\left(-\frac{1}{2}\tensor{g}{^\alpha^\rho}\tensor{g}{^\beta^\sigma} \tensor{\nabla}{_\alpha}\varphi\tensor{\nabla}{_\beta}\varphi\frac{\delta
\tensor{g}{_\rho_\sigma}}{\delta \tensor{g}{_\mu_\nu}}\right)\\
&=\frac{1}{2}\int \dif x^4
\sqrt{-g}\left(-\frac{1}{2}\tensor{g}{^\mu^\nu}\tensor{\nabla}{_\rho}\varphi\tensor{\nabla}{^\rho}\varphi-\frac{1}{2}\tensor{g}{^\mu^\nu}m^2\varphi^2
+\tensor{\nabla}{^\mu}\varphi\tensor{\nabla}{^\nu}\varphi\right)\delta(x,x')\\
&=\frac{1}{2}\sqrt{-g}\left(-\frac{1}{2}\tensor{g}{^\mu^\nu}\tensor{\nabla}{_\rho}\varphi\tensor{\nabla}{^\rho}\varphi-\frac{1}{2}\tensor{g}{^\mu^\nu}m^2\varphi^2
+\tensor{\nabla}{^\mu}\varphi\tensor{\nabla}{^\nu}\varphi\right)
\end{split}
\end{equation}
So that
\begin{equation}
\tensor{T}{^\mu^\nu}(\varphi)
=-\frac{1}{2}\tensor{g}{^\mu^\nu}\tensor{\nabla}{_\rho}\varphi\tensor{\nabla}{^\rho}\varphi
+\tensor{\nabla}{^\mu}\varphi\tensor{\nabla}{^\nu}\varphi
-\frac{1}{2}\tensor{g}{^\mu^\nu}m^2\varphi^2
\end{equation}
As we have noticed, the Einstein Tensor is covariantly conserved (contracted
Bianci identities). The Einstein equation then implies that also
$\tensor{T}{^\mu^\nu_{;\nu}}=0$ this can be checked for the given Tensor
\begin{equation}
\begin{split}
\tensor{\nabla}{_\mu}\tensor{T}{^\mu^\nu}
&=\tensor{g}{^\mu^\nu}\tensor{\nabla}{_\mu}\tensor{\nabla}{_\rho}\varphi\tensor{\nabla}{^\rho}\varphi+\square\varphi\tensor{\nabla}{^\nu}\varphi+\tensor{\nabla}{^\mu}\varphi\tensor{\nabla}{_\mu}\tensor{\nabla}{^\nu}\varphi
-\tensor{g}{^\mu^\nu}m^2\varphi\tensor{\nabla}{_\mu}\varphi\\
&=\tensor{\nabla}{^\nu}\varphi\left(\square-m^2\right)\varphi\\
&=0
\end{split}
\end{equation}
Where the last equality holds because $\varphi$ satisfies the Klein-Gordon
equation. The Einstein equations are 10 quasi linear, i.e. the highest order
derivative apears only linear, differential equations for the metric field
$\tensor{g}{_\mu_\nu}$. Strictly speaking the Einstein equations are
\emph{nonlinear}.
\begin{sidenote}
If you substract the constrains imposed by the Bianci identities you end with 2
DOFs, which are associated with the polarisation states of the graviton.
\end{sidenote}
How do we find a solution to this equations?
\begin{enumerate}
  \item Prescribe $\tensor{T}{_\mu_\nu}$. This is only possible for high
  symmetry problems, e.g. the Schwazschild solution and the cosmological
  solutions (Friedmans equations)
  \item Assume $\tensor{g}{_\mu_\nu}$, then compute $\tensor{T}{_\mu_\nu}$ and
  (try!) to interprete this.
\end{enumerate}
Intrinsic vs extrinsic curvature ????
\subsection{ADM-Decomposition}
(Bild)
Erklärung einschieben was passiert!!!
Assume we know (??)
\begin{itemize}
  \item $\tensor{g}{_\mu_\nu}$ on $\Sigma_{t_0}$
  \item $\tensor{g}{_\mu_\nu_{;j}}$, $\tensor{g}{_\mu_\nu_{;0}}$ on
  $\Sigma_{t_0}$ also allowed
\end{itemize}
In vakuum the field equation are
\begin{equation}
0=G=R-2R\implies\tensor{R}{_\mu_\nu}=0
\end{equation}
The resulting set of equations is (prüfen!!!)
\begin{align}
0&=\tensor{R}{_0_0}=-\frac{1}{2}\tensor{g}{^i^j}\tensor{g}{_i_j_{,00}}+\tensor{M}{_0_0}\\
0&=\tensor{R}{_0_i}=-\frac{1}{2}\tensor{g}{^0^j}\tensor{g}{_i_j_{,00}}+\tensor{M}{_0_i}\\
0&=\tensor{R}{_i_j}=-\frac{1}{2}\tensor{g}{^0^0}\tensor{g}{_i_j_{,00}}+\tensor{M}{_i_j}
\end{align}
Where $\tensor{M}{_\mu_\nu}$ is a rest term containing lower order derivaties.
This shows that ther are no second order time derivatives of
$\tensor{g}{_0_\mu}$. We have 10 euations and 6 undetetrmined functions. The
DOFs can be used for a coordinate transformation, so that
$\tensor{g}{_0_\mu_{,00}}=0$ on $\Sigma_{t_0}$. This is allways possible but we
will not proof this. It can be further shown, by means of the contracted Bianci
identities, that this implies $\tensor{g}{_0_\mu_{,00}}=0$ on \emph{all}
hypersurfaces $\Sigma_{t}$.
\begin{equation}
\tensor{\partial}{_0}\tensor{G}{^0^\nu}=
\tensor{\partial}{_i}\tensor{G}{^i^\nu}
-\cSym{\nu}{0}{\lambda}\tensor{G}{^\lambda^\nu}
-\cSym{0}{\nu}{\lambda}\tensor{G}{^\mu^\lambda}
\end{equation}
Hier muss man nochmal schauen das macht noch nicht viel sinn\ldots\ldots.
We have the freedom to choose four coodinates
\begin{equation}
\tensor{x}{^{\mu^\prime}}=\tensor{f}{^{\mu^\prime}}\left(\tensor{x}{^\mu}\right)
\end{equation}
One typical choice is the \emph{harmonic\footnote{a function $f$ satisfying
$\square f = 0$ is called harmonic.} gauge}
\begin{equation}
\square\tensor{x}{^\mu}=0\,.
\end{equation}
\begin{equation}
\begin{split}
\square\tensor{x}{^\mu}&=g^{-\nicefrac{1}{2}}\tensor{\partial}{_\rho}\left(g^{\nicefrac{1}{2}}\tensor{g}{^\rho^\sigma}\tensor{\partial}{_\sigma}\tensor{x}{^\mu}\right)\\
&=g^{-\nicefrac{1}{2}}\tensor{\partial}{_\rho}\left(g^{\nicefrac{1}{2}}\tensor{g}{^\rho^\sigma}\tensor{\delta}{_\sigma^\mu}\right)\\
&=g^{-\nicefrac{1}{2}}\tensor{\partial}{_\rho}\left(g^{\nicefrac{1}{2}}\tensor{g}{^\rho^\mu}\right)\\
\end{split}
\end{equation}
(Formel für $\square$ die hier genutzt wurde referenzieren)
The harmonic gauge is therefore equivalent to
\begin{equation}
\tensor{\partial}{_\rho}\left(g^{\nicefrac{1}{2}}\tensor{g}{^\rho^\mu}\right)=0\,
.\\
\end{equation}
The equation can be sorted in spatial and time components and derive by the
zero component, so that
\begin{equation}
\tensor*{\partial}{*_0^2}\left(g^{\nicefrac{1}{2}}\tensor{g}{^0^\mu}\right)
=
-\tensor{\partial}{_i}\left[\tensor{\partial}{_0}\left(g^{\nicefrac{1}{2}}\tensor{g}{^0^\mu}\right)\right]\,
,\end{equation}
which fixes the second order time derivatives of the relevant components
$\tensor{g}{^0^\mu}$. Therefore now the time evolution can solved. (Fehlt hier
was???)
\subsubsection{Degrees of freedom}
\begin{itemize}
  \item[\color{section_color}\textsf{\textbf{10}}]
  componnents for every spacetime point from the symmetric $\tensor{g}{_\mu_\nu}(x)$
  \item[\color{section_color}\textsf{\textbf{-4}}] from the
  constraint equation $\tensor{G}{_\mu_\nu^{;\nu}}=0$
  \begin{itemize}
    \item
    $\tensor{G}{^0^0}=\kappa \tensor{T}{^0^0}$ ensures that the
    evolution is independent of the choice of spatial coordinates on
    $\Sigma_{t_0}$.
    \item
    $\tensor{G}{^i^0}=\kappa \tensor{T}{^i^0}$ ensures that the time
    evolution is independent of the way we foliated sacetime into spacial
    hypersurfaces $\Sigma_{t}.    $
  \end{itemize}
  \item[\color{section_color}\textsf{\textbf{-4}}] due to the freedom
  to choose coordinates (i.e. a gauge).
\end{itemize}
We are left with 2 physical degrees of freedom which may be interpreted as the
polarisation states of the graviton field.
\subsubsection{Comparison with electrodynamics in flat spacetime}
In electrodynamics instead of the einstein equations we have the field equations
for the four potential $\tensor{A}{_\mu}$:
\begin{equation}
\square\tensor{A}{_\mu}-\partial_\mu\left(\partial_\nu\tensor{A}{^\nu}
\right)=0\,.
\end{equation}
As we did for the gravitational field we take a look
at the 0 component. We find
\begin{equation}
\begin{split}
0&=-\partial_0^2\tensor{A}{_0}+\partial_i\partial^i\tensor{A}{_0}
-\partial_0\left(-\partial_0\tensor{A}{_0}+\partial_iA^i\right)\\
&= \partial_i\partial^i\tensor{A}{_0}-\partial_0\partial_iA^i
\end{split}
\end{equation}
This equation is equivalent to $\nabla\vec{E}=0$ and the bianci identities.
So once again $\tensor{A}{_0}$ is \emph{not} determined by the dynamical
evolution equation because there is no second order time derivative analogous to
$\tensor{g}{_0_0}$. Since $\tensor{A}{_0}$ is not determinend and cannot be
specified on initial time slice. This reflects some internal redundancy namely
gauge invariance of the theory. For any scalar function $\Lambda$
\begin{equation}
\tensor{A}{_\mu}\to\tensor*{A}{*_\mu^\prime}=
\tensor{A}{_\mu}+\partial_\mu\Lambda
\end{equation}
leaves the physics invariant. It is trivial to check that
the field strength tensor $\tensor{F}{_\mu_\nu}=\partial_\mu
\tensor{A}{_\nu}-\partial_\nu\tensor{A}{_\mu}$ stays invariant. Perhaps more
interesting the field equation is also gauge invariant:
\begin{equation}
\begin{split}
\square\tensor*{A}{*_\mu^\prime}-\partial_\mu\left(\partial_\nu\tensor*{A}{*^\nu^\prime}
\right)
&=
\square\tensor{A}{_\mu}+\square\partial_\mu\Lambda-\partial_\mu\left(\partial_\nu\tensor{A}{^\nu}
\right)-\partial_\mu\square\Lambda\\
&=
\square\tensor{A}{_\mu}-\partial_\mu\left(\partial_\nu\tensor{A}{^\nu}
\right)\,.
\end{split}
\end{equation}
Thus if $\tensor{A}{_\mu}$ is a solution to the field
equation $\tensor*{A}{*_\mu^\prime}$ is and therefore both are physically
undistinguishable. We can also fix a gauge for example the \emph{Lorentz gauge}:
\begin{equation}
\partial_\mu\tensor{A}{^\mu}=0
\end{equation}
If we derive this by the 0 component we get
\begin{equation}
\partial_{0}^2\tensor{A}{^0}=-\partial_i\partial_0\tensor{A}{^i}\, ,
\end{equation}
so as with $\tensor{g}{_0_0}$ the evolution of the 0 component is now related to
the other components. There is still one residual gauge condition, namely we can
still transform
\begin{equation}
\tensor{A}{_\mu}\to\tensor*{A}{*_\mu^\prime}=
\tensor{A}{_\mu}+\partial_\mu\Gamma\, ,
\end{equation}
but to keep the gauge we have also to demand that $\square\Gamma=0$.
Again we count the DOFs:
\begin{itemize}
  \item[\color{section_color}\textsf{\textbf{4}}] components of the potential
  $\tensor{A}{_\mu}$.
  \item[\color{section_color}\textsf{\textbf{-1}}] from constraint
  $\nabla\vec{E}=0$.
  \item[\color{section_color}\textsf{\textbf{-1}}] from gauge freedom
  $\Lambda$.
\end{itemize}
This leaves two physical degrees of freedom, the polarisation states of a
photon.
\begin{remark}
As we have have seen there is a direct correspondence between the gauge freedom
in electrodynamics and the freedom of choice of coordinates of coordinates in
GR.
\end{remark}


\chapter{The Energy Momentum Tensor}
\chapter{Linearised Theory and Newtonian Limit}
\chapter{Gravitational Waves}
\chapter{The Schwarzschild Solution}
\chapter{Experimental Tests in the Solar System}
\chapter{Black Holes}
\chapter{Cosmology}
\end{document}