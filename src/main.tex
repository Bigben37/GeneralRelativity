% !TEX program = pdflatex
% !TEX encoding = UTF-8

\documentclass[a4paper,oneside,12pt]{scrreprt}

%%% PACKAGES + MODIFICATIONS%%%
 
%language and encoding
\usepackage[utf8]{inputenc}
\usepackage[T1]{fontenc}
%\usepackage{lmodern}
\usepackage[english,ngerman]{babel}

%graphics
\usepackage{graphicx}
\usepackage{xcolor}

%math
\usepackage{amsmath}
\usepackage{amssymb}
\usepackage{amsfonts}
\usepackage{amsthm}
\usepackage{thmtools}
\usepackage{mathtools}
\usepackage{tensor}
\usepackage{commath}
\usepackage{dsfont}		% double stroke characters
\usepackage{braket}
\usepackage{mdframed}	% framed environments that can split at page bound­aries

%science
\usepackage{units}
\usepackage{bpchem}	% type­set chem­i­cal names, for­mu­lae, etc

%layout + style
\definecolor{section_color}{rgb}{0.35,0.0,0}								% colors
\definecolor{MyGray}{rgb}{0.96,0.97,0.98}

\usepackage[bottom]{footmisc}
\usepackage[automark,headsepline]{scrlayer-scrpage} 						% headings
\renewcommand*{\headfont}{\normalfont}										% nicht kursive Kopfzeile
\usepackage{setspace}														% set space be­tween lines
\usepackage[font=small,labelfont=bf,labelsep=endash,format=plain]{caption}	% change style of captions
\usepackage[section]{placeins}												% de­fines a \FloatBar­rier com­mand, be­yond which floats may not pass.
\usepackage[protrusion=true,expansion=true]{microtype}						% sublim­i­nal re­fine­ments to­wards ty­po­graph­i­cal per­fec­tion
\addtokomafont{caption}{\small\linespread{1}\selectfont}					% Ändert Schriftgröße und Zeilenabstand bei captions
\usepackage{enumerate} % better way to config enumerates
\usepackage{pdflscape} % landscape mode
\usepackage{afterpage}
% styles for math environments
\declaretheoremstyle[														
  spaceabove = 6pt,
  spacebelow = 6pt,
  headfont = \color{section_color}\sffamily\bfseries,
  notefont = \mdseries,
  notebraces = {(}{)},
  bodyfont = \normalfont,
  postheadspace = 1em,
  qed = ,
]{mythmstyle} %theorems
\declaretheoremstyle[														
  spaceabove = 6pt,
  spacebelow = 6pt,
  headfont = \color{section_color}\sffamily\bfseries,
  notefont = \mdseries,
  notebraces = {(}{)},
  bodyfont = \normalfont,
  postheadspace = 1em,
  qed = ,
]{myrmstyle} %remarks

%TODO outdated package, produces warnings, use package 'titlesec' to control sectioning format -- Ben 7.12

% TODO I disabled the sectsty package for the moment, this fixes the messed up spacing between chapter number and description -- Ben 7.12
% \usepackage{sectsty}	 % con­trol sec­tional head­ers
% \chapterfont{\color{section_color}}
% \sectionfont{\color{section_color}}
% \subsectionfont{\color{section_color}}
% \subsubsectionfont{\color{section_color}}
% \paragraphfont{\color{section_color}}

\renewcommand{\labelitemi}{\color{section_color}\scriptsize$\blacksquare$} 
\renewcommand{\labelenumi}{\color{section_color}\textsf{\textbf{\arabic{enumi}}}.}

%tikz
\usepackage{tikz}
\usetikzlibrary{decorations.markings,arrows.meta}

\tikzset{
	->-/.style={decoration={
			markings,
			mark=at position .85 with {\arrow{Latex}}},postaction={decorate}},
	-<-/.style={decoration={
			markings,
			mark=at position .15 with {\arrow{Latex[reversed]}}},postaction={decorate}},
}
%tables
\usepackage{tabularx}	% tab­u­lars with ad­justable-width columns
\usepackage{multirow}	% create tab­u­lar cells span­ning mul­ti­ple rows
\usepackage{booktabs}	% publi­ca­tion qual­ity ta­bles in LaTeX
\usepackage{array}
\usepackage{dcolumn}	% align on the dec­i­mal point of num­bers in tab­u­lar columns

%hyperref
\usepackage[pdftex]{hyperref}
\hypersetup{
	pdftitle={General Relativity},
	pdfsubject={General Relativity},
	pdfkeywords={general,relativity,space,time,minkowksi},
	pdfauthor={Michael Ruf and Benjamin Rottler},
	pdfcreator={Michael Ruf and Benjamin Rottler},
	pdfproducer={Michael Ruf and Benjamin Rottler},
	bookmarksnumbered=true, % bookmarks are numbered
	bookmarksopen=true,     % show bookmarks at start of pdf viewer
	bookmarksopenlevel=2,   % level to which the bookmarks are opened
	bookmarksdepth=3,       % depth of bookmarks 
    unicode=true,           % non-Latin characters in pdf viewer's bookmarks
    pdftoolbar=false,       % show pdf viewer's toolbar?
    pdfmenubar=true,        % show pdf viewer's menu?
    pdffitwindow=false,     % window fit to page when opened
    pdfstartview={FitH},    % fits the width of the page to the window
    pdfnewwindow=true,      % links in new window
    pdfborder={0 0 1},		% no border for links
    colorlinks=false,		% false: black links; true: colored links
    linkcolor=section_color,         % color of internal links (change box color with
    % linkbordercolor)
    citecolor=green,        % color of links to bibliography
    filecolor=magenta,      % color of file links
    urlcolor=blue			% color of external links
}
\usepackage{nameref}

%general stuff
\usepackage{cite}
\usepackage{glossaries}	% create glos­saries and lists of acronyms


%%% NEW COMMANDS %%%

\newcommand\grad{\ensuremath{^\circ}}
\newcommand{\imI}{\ensuremath{\mathrm{i}}}
\newcommand{\Reals}{\ensuremath{\mathbb{R}}}
\newcommand{\Complex}{\ensuremath{\mathbb{C}}}
\newcommand{\Sphere}{\ensuremath{\mathbb{S}}}
\newcommand{\transpose}{^\top}
\newcommand{\cSym}[3]{\ensuremath{\begin{Bmatrix} #1 \\ #2 #3 \end{Bmatrix}}}
\newcommand{\csym}[3]{\ensuremath{[#1 #2,\, #3]}}
\newcommand{\affin}[3]{\ensuremath{\Gamma^{#1}_{#2 #3}}}
\newcommand{\landauO}{\mathcal{O}}
\newcommand{\name}[1]{\textsc{#1}}
\newcommand{\fourint}{\int\dif{}^4 x\,}
\DeclareMathOperator{\tr}{Tr}
\DeclareMathOperator{\re}{Re}
\DeclareMathOperator{\im}{Im}
\DeclareMathOperator{\id}{id}
\DeclareMathOperator{\Div}{div}
\let\originalleft\left  %fix bracket spacing when using \left( \right)
\let\originalright\right
\renewcommand{\left}{\mathopen{}\mathclose\bgroup\originalleft}
\renewcommand{\right}{\aftergroup\egroup\originalright}
\renewcommand{\vec}{\mathbf}
\def\mathunderline#1#2{\color{#1}\underline{{\color{black}#2}}\color{black}}
\newcommand{\liedif}[2]{\ensuremath{\mathcal{L}_{#1}#2}}
\newcommand{\lagrangian}{\mathcal{L}}
\DeclareMathOperator{\diag}{diag}
\newcommand{\const}{\text{const.}}
\DeclareMathOperator{\difD}{D}

%new environments
\newenvironment{thm}[1][]{%
  \definecolor{shadethmcolor}{rgb}{.9,.9,.95}%
  \definecolor{shaderulecolor}{rgb}{0.0,0.0,0.4}%
  %\setlength{\shadeboxrule}{1.5pt}%
  \begin{thms}[#1]\hspace*{1mm}%
}{\end{thms}}
\theoremstyle{definition}
\declaretheorem[
  style = mythmstyle,
  name = Theorem,
  shaded = {
    bgcolor = MyGray,
    padding = 2mm,
    textwidth = 0.98\textwidth
}
]{theorem}
\declaretheorem[
  style = mythmstyle,
  name = Definition,
  numberlike=theorem,
  shaded = {
    bgcolor = MyGray,
    padding = 2mm,
    textwidth = 0.98\textwidth
}
]{definition}
\declaretheorem[
  style = myrmstyle,
  name = Remark,
  numberlike=theorem,
  shaded = {
    bgcolor = white,
    padding = 2mm,
    textwidth = 0.98\textwidth
}
]{remark}
\declaretheorem[
style = myrmstyle,
name = Aside,
numberlike=theorem,
shaded = {
bgcolor = white,
padding = 2mm,
textwidth = 0.98\textwidth
}
]{sidenote}
\declaretheorem[
  style = myrmstyle,
  name = Example,
  numberlike=theorem,
  shaded = {
    bgcolor = white,
    padding = 2mm,
    textwidth = 0.98\textwidth
}
]{example}

\makeatletter
\newtoks\FTN@ftn
\def\pushftn{%
 \let\@footnotetext\FTN@ftntext\let\@xfootnotenext\FTN@xftntext
  \let\@xfootnote\FTN@xfootnote}
\def\popftn{%
 \global\FTN@ftn\expandafter{\expandafter}\the\FTN@ftn}
\long\def\FTN@ftntext#1{%
  \edef\@tempa{\the\FTN@ftn\noexpand\footnotetext
                    [\the\csname c@\@mpfn\endcsname]}%
  \global\FTN@ftn\expandafter{\@tempa{#1}}}%
\long\def\FTN@xftntext[#1]#2{%
  \global\FTN@ftn\expandafter{\the\FTN@ftn\footnotetext[#1]{#2}}}
\def\FTN@xfootnote[#1]{%
   \begingroup
     \csname c@\@mpfn\endcsname #1\relax
     \unrestored@protected@xdef\@thefnmark{\thempfn}%
   \endgroup
   \@footnotemark\FTN@xftntext[#1]}

\makeatother
\renewcommand{\thefootnote}{\fnsymbol{footnote}}



%\newtheorem*{sidenote}{Sidenote}
%\newtheorem*{example}{Example}
\newenvironment{tabulars}[1]{\renewcommand*{\arraystretch}{2}\tabular{#1}}{\endtabular}		% stretched table

%%% DOCUMENT %%%

\begin{document}
\selectlanguage{english}
\onehalfspacing


\hypersetup{pageanchor=false} %stop page numbering (hyperref) to prevent for double page numers

\newcommand{\HRule}{\rule{\linewidth}{0.5mm}}
\begin{titlepage}
\begin{center}
  \HRule \\[0.4cm]
  { \huge \bfseries General Relativity}\\
  \HRule \\[0.5cm]
  \large Winter term 2015/2016 \\[0.5cm]  
  Lecture of Prof. Dr. J. J. van der Bij and Dr. Christian Steinwachs\\
  Transcript of Michael Ruf and Benjamin Rottler \\[1.5cm]
  \today
  \vfill
  \normalsize
  \textsc{Physikalisches Institut} \\
  \textsc{Albert-Ludwigs-Universität} \\
  \textsc{Freiburg im Breisgau}
\end{center}
\end{titlepage}

\thispagestyle{empty}
\newpage

\pagenumbering{roman}
\setcounter{page}{1}
\cfoot[- \textit{\pagemark} -]{- \textit{\pagemark} -}

\tableofcontents
\newpage

\pagenumbering{arabic} 
\hypersetup{pageanchor=true} %start page numbering again
\setcounter{page}{1}
\cfoot[\pagemark]{\pagemark}

\chapter{Newtonian Gravity}
In \name{Newton}ian physics we assume that we have absolute space and time
that can be described by the a set of numbers $x^1,x^2,x^3,t$.
We express the coordinates as functions of time.
\section{Forces}
The force $\vec{F}_{AB}$, which a massive body $A$ with mass $m_A$ exerts on another massive body $B$ with mass $m_B$, is given by
\begin{equation}
    \vec{F}_{AB}=-m_B\frac{G\textsubscript{N}m_A}{r^2}\vec{e}_r\, ,
\end{equation}
%TODO remove subscript of G
where $G\textsubscript{N}$ denotes \emph{\name{Newton}'s constant},
numerically equal to $G\textsubscript{N}\approx \unitfrac[6.673\cdot 10^{-11}]{m^3}{kg\,s}$.
Although there is no need for $G\textsubscript{N}$ to be constant over time,
there is evidence that the relative variation is less than $10^{-12}$ per year.
The force can be expressed in terms of \emph{gravitational potential} $\Phi$:
\begin{equation}
    \vec{F}_{AB}=-m_B\nabla\left(-\frac{G\textsubscript{N}m_A}{r}\right)=:
    -m_B\nabla\Phi(\vec{r}_B)\, .
\end{equation}
Given $N$ particles labeled by $n$, the total force $B$ experiences is
\begin{equation}
    \vec{F}_{B}=-\sum_n \vec{F}_{nB}\, .
\end{equation}
The potential at $\vec{r}$ is then easily found to be
\begin{equation}
    \Phi(\vec{r})=-G\textsubscript{N}\sum_n\frac{m_n}{|\vec{r}-\vec{r}_n|}\, .
\end{equation}
In general, we assume a mass distribution $\varrho(\vec{r})$ and the sum is
replaced by an integral:
\begin{equation}
    \Phi(\vec{r}) = -G\textsubscript{N}\int_{\Reals^3}\dif{\vec{r}^{\prime}}
    \frac{\varrho(\vec{r}^{\prime})}{|\vec{r}-\vec{r}^{\prime}|}\,.
\end{equation}
\section{Comparison with electrostatics}
The classical theory of gravity bears a striking similarity to electrostatics.
To make this clearer, we introduce the gravitational field
$\vec{g}(\vec{r}):= -\nabla\Phi(\vec{r})$.
\begin{table}[htb]
    \caption{Comparison of electrostatics and \name{Newton}ian gravity.}
    \begin{center}
        \begin{tabulars}{lll}
            \toprule
            &\name{Newton}ian Gravity&Electrostatics\\
            \midrule
            Force&$\displaystyle\vec{F}=q\frac{kQ}{r^2}\vec{e}_r$&$\vec{F}=m\frac{G\textsubscript{N}M}{r^2}\vec{e}_r$\\
            Potential
            &$\Phi\textsubscript{el}(\vec{r})=q\frac{kQ}{r}$
            &$\Phi\textsubscript{g}(\vec{r})=m\frac{G\textsubscript{N}M}{r}$\\
            Field
            &$\vec{E}(\vec{r})=-\nabla\Phi\textsubscript{el}(\vec{r})$
            &$\vec{g}(\vec{r})=-\nabla\Phi\textsubscript{g}(\vec{r})$\\
            \name{Laplace} equation
            &$\Delta\Phi\textsubscript{el}=-4\pi k\varrho\textsubscript{el}(\vec{r})$&
            $\Delta\Phi\textsubscript{g}=4\pi
            G\textsubscript{N}\varrho\textsubscript{g}(\vec{r})$
            \\
            \bottomrule
        \end{tabulars}
    \end{center}
\end{table}
\begin{example}[Field of a spherical mass distribution]
Assume we have a spherical mass distribution, i.e.\ $\varrho(\vec{r})=\varrho(r)$.
By symmetry considerations it follows, that the gravitational field can be
expressed by
\begin{equation}
    \vec{g}(\vec{r})=g(r)\vec{e}_r\, .
\end{equation}
We integrate the divergence of the field over a ball $B$ of radius $r$
\begin{equation}
    \int_B\dif{\vec{r}}\,\nabla\vec{g}=-\int_B\dif{\vec{r}}\, \Delta\Phi
    = -4\pi G\textsubscript{N}\int_B\dif{\vec{r}}\,\varrho(r)= -4\pi
    G\textsubscript{N} M\, ,
\end{equation}
where $M$ is the mass enclosed in $B$. On the other hand we can use Gauss's theorem to deduce
\begin{equation}
    \int_B\dif{\vec{r}}\,\nabla\vec{g}=\oint_{\partial B}\dif{\vec{A}}\cdot
    \vec{g} = \oint_{\Omega}\dif{\Omega}\, g(r)r^2=4\pi r^2g(r)\, .
\end{equation}
Together the gravitational field is given by
\begin{equation}
    \vec{g}(r)=-\frac{G\textsubscript{N}M}{r^2}\vec{e}_{r}\, .
\end{equation}
\end{example}
% \subsection{Inertial systems}
% \begin{definition}
% An inertial system is a system in which force-free particles move with constant uniform velocity on straight lines.
% \end{definition}
\subsection*{Weak Equivalence Principle (WEP)}
\name{Newton}'s first law reads
\begin{equation}
    \vec{F}=m\textsubscript{I} \vec{\ddot{x}} \, ,
\end{equation}
where $m\textsubscript{I}$ is the inertial mass that works against the acceleration of the body.
The force which a body with ``active'' mass $m\textsubscript{g,a}$ exerts on
another body with mass $m\textsubscript{g,p}$ is given by 
\begin{equation}
    \vec{F}=m\textsubscript{g,p}\frac{G\textsubscript{N}m\textsubscript{g,a}}{r^2}\vec{e}_r \, .
\end{equation}
A priori, there is no reason to assume any relation between this masses.
The first question one might ask is whether the active and the passive mass are equal.
Suppose we have two masses $A$ and $B$. Using \name{Newton}'s first law, we
can explicitly write
\begin{align}
    m^{B}_{\text{I}}\vec{\ddot{x}}&=\vec{F}_{AB}=-
    m^{B}_{\text{g,p}}\frac{G\textsubscript{N}m^{A}_{\text{g,a}}}{r^2}\vec{e}_r
    \, ,\\
    m^{A}_{\text{I}}\vec{\ddot{x}}&=\vec{F}_{BA}=-
    m^{A}_{\text{g,p}}\frac{G\textsubscript{N}m^{B}_{\text{g,a}}}{r^2}\vec{e}_r
    \, .
\end{align}
By the third law $\vec{F}_{AB}=-\vec{F}_{BA}$ we have
\begin{equation}
\frac{m^{B}_{\text{g,p}}}{m^{B}_{\text{g,a}}}=\frac{m^{B}_{\text{g,p}}}{m^{B}_{\text{g,a}}}\,.
\end{equation}
By proper choice of mass units we can set this quotient to one so that 
\begin{equation}
m\textsubscript{g,a}=m\textsubscript{g,p}=:m\textsubscript{g}
\end{equation}
The next question is wheater the inertial mass equivalent to the gravitional
mass.
By \name{Newton}'s first law we have
\begin{equation}
m\textsubscript{I}\vec{\ddot{x}}
=-m_{\text{g}}\frac{G\textsubscript{N}M_{\text{g}}}{r^2}\vec{e}_r 
=-m_{\text{g}}\vec{g}\,.
\end{equation}
As a experimental result that has been measured up to a high accuracy (compare
tabular~\ref{tab:WEPExp}) all bodys recive the same acceleration due to gravity
$\ddot{\vec{x}}\sim \vec{g}$. By a 'proper' choice of units of the flight-time $t=\sqrt{\frac{m\textsubscript{I}}{m\textsubscript{g}}}\sqrt{\frac{2h}{g}}$,
we get
\begin{equation}
m\textsubscript{I}=m\textsubscript{g}=:m\,.
\end{equation}
\begin{table}
\centering
\begin{tabulars}{rllr}
\toprule
Year&Experimenter&Experiment&Accuracy\\
\midrule
1636&\name{Galilei}&inclined planes&$10^{-2}$\\
1689&\name{Newton}&pendulum&$10^{-3}$\\
1832&\name{Bessel}&pendulum&$10^{-5}$\\
1922&\name{Eötvös}&pendulum&$10^{-9}$\\
1922&\name{Shapro} et al.&pendulum&$10^{-12}$\\
1999&\name{Baesler}&torsion balance&$10^{-14}$\\
\bottomrule
\end{tabulars}
\caption{Experiments measuring the ratio
$\frac{m\textsubscript{I}}{m\textsubscript{g}}$.\label{tab:WEPExp}}
\end{table}
\subsection{Tidal Forces}
Assume we have a body of finite extention in a gravitationational
potential $\Phi$, an example beeing the earth in the potential of the moon.
On the center of the body we have 
%TODO masses???
\begin{equation}
\dod[2]{\tensor{x}{^i}}{t}=-\dpd{\Phi}{\tensor{x}{_i}}\,.
\end{equation}
If we consider a point shifted by $\tensor{\chi}{^i}$ from the center then the
acceleration is given as
\begin{equation}
\begin{split}
\dod[2]{}{t}\left(\tensor{x}{^i}+\tensor{\chi}{^i}\right)
&=-\dpd{\Phi\left(\tensor{x}{^i}+\tensor{\chi}{^i}\right)}{\tensor{x}{_i}}\\
&\simeq-\dpd{\Phi\left(\tensor{x}{^i}\right)}{\tensor{x}{_i}}
-\dmd{\Phi\left(\tensor{x}{^i}\right)}{2}{\tensor{x}{_i}}{}{\tensor{x}{_j}}{}\tensor{\chi}{^j}\,.
\end{split}
\end{equation}
Subtracting the previous equations yields the tidal force
\begin{equation}
\dod[2]{\tensor{\chi}{^i}}{t}=-\dmd{\Phi\left(\tensor{x}{^i}\right)}{2}{\tensor{x}{_i}}{}{\tensor{x}{_j}}{}\tensor{\chi}{^j}\,.
\end{equation}
The tidal force tensor $\dmd{\Phi}{2}{\tensor{x}{_i}}{}{\tensor{x}{_j}}{}$ is of
the form
\begin{equation}
\dmd{\Phi}{2}{\tensor{x}{_i}}{}{\tensor{x}{_j}}{}
=\frac{G\textsubscript{N}M}{r^3}\left(\tensor{\delta}{_i_j}-3\frac{\tensor{x}{_i}\tensor{x}{_j}}{r^2}\right)\,.
\end{equation}
%TODO newtons laws?
%TODO picture
%TODO finish chapter 1
\chapter{Special Relativity}
\section{Postulates and Definitions}
Special relativity is based on two main postulates, namely
\begin{enumerate}
    \item Principle of relativity: \\
    The laws of physics acquire the same form in all inertial systems.
    \item Constancy of the speed of light:\\
    The speed of light in vacuum is constant ${c\approx\unitfrac[3\cdot
    10^8]{m}{s}}$.
\end{enumerate}
We further define a series of objects that will come handy when describing
relativity:
\begin{definition}[System of reference]
    A system of reference $K$ is a system of three spacial coordinates to indicate
    the position and one time coordinate to indicate the time.
\end{definition}
\begin{definition}[Inertial system]
    An inertial system $I$ belongs to a particular subspace of reference systems in
    which a freely moving bodies, i.e. that are not subject to any external
    force, move with constant velocity on an straight lines.
\end{definition}
\begin{definition}[Event]
    An event or world point $x$ is a point in spacetime. In any system of reference
    it can be described by four coordinates
    \begin{equation}
        x^\mu: (x^0,x^1,x^2,x^3)\, .
    \end{equation}
    For example in cartesian coordinates, we have
    \begin{equation}
        (x^0,x^1,x^2,x^3) = (ct,x,y,z)\, .
    \end{equation}
\end{definition}
\begin{definition}[Worldline]
    A worldline $\gamma$ is a parametrised curve in spacetime. To each parameter,
    there corresponds an event that lies on the worldline
    \begin{equation}
        \tensor{\gamma}{^\mu}(\lambda)=\tensor{x}{^\mu}(\lambda)\, .
    \end{equation}
\end{definition}
\subsection{Einstein Summation Convention}
Since we will have to deal which many indices, we will find a way to keep
notation as compact as possible. We agree that whenever we have a summation over an
upper and a lower index like $\sum_\mu \tensor{x}{_\mu}\tensor{x}{^\mu}$, we
drop the sumation and assume that terms of type
$\tensor{x}{_\mu}\tensor{x}{^\mu}$ are always summed over.
\section{Propagation of Light Waves and the Line Element}
For vividness, we restore factors of $c$ in this section.
(Pictures)\\
Inertial System $I$:
\begin{itemize}
    \item $P:$ Event of emission of the ray $(ct_1,x_1,y_1,z_1)$
    \item $Q:$ Event of absorption of the ray $(ct_2,x_2,y_2,z_2)$
\end{itemize}
The spatially distance between the events is
\begin{equation}
    r=\left[(x_2-x_1)^2+(y_2-y_1)^2+(z_2-z_1)^2\right]^{\frac{1}{2}}\, .
\end{equation}
Since we are tracking a light ray and the events are absorption and emission, we
further have
\begin{equation}
    r=c(t_2-t_1)\, .
\end{equation}
We construct the quantity
\begin{equation}
    {\Delta s}^2=-c^2(t_2-t_1)^2+(x_2-x_1)^2+(y_2-y_1)^2+(z_2-z_1)^2\, ,
\end{equation}
so that ${\Delta s}^2=0$ for light.

We consider another Inertial System $I'$, in which we the events are given by
\begin{itemize}
    \item $P:$ Event of emission $(ct_1^\prime,x_1^\prime,y_1^\prime,z_1^\prime)$
    \item $Q:$ Event of absorption
    $(ct_2^\prime,x_2^\prime,y_2^\prime,z_2^\prime)$
\end{itemize}
By the same reasoning we have ${\Delta s^\prime}^2=0$ for light.
We define an infinitesimal interval, or \emph{line element}:
\begin{equation}
    \dif s^2=-c^2\dif t^2+\dif x^2+\dif y^2+\dif z^2\, .
\end{equation}
(PICTURE)
Our preliminary considerations lead to the following statement: if the interval
is zero in one inertial system, it should be zero in all.
To relate the linelements of different systems to each other, we make the following thought:
Suppose we have three inertial Systems $I,I_1,I_2$.
The Systems $I_1$ and $I_2$ move at constant velocity $\vec{v}_1$ and $\vec{v}_2$ relative to $I$. Further
$I_2$ moves with velocity $\vec{v}_{12}$ relative to $I_1$. The demand implies
that there exists a function $\alpha$, only dependent on the velocity $\vec{v}$,
so that
\begin{equation}
    \dif s'^2=\alpha(\vec{v})\dif s^2\, .
\end{equation}
We have
\begin{equation}
    \dif s^2=\alpha(\vec{v}_1)\dif s_1^2\, ,\quad\dif s^2=\alpha(\vec{v}_2)\dif
    s_2^2\, ,\quad\dif s_1^2=\alpha(\vec{v}_{12})\dif s_2^2\, .
\end{equation}
Together they imply
\begin{equation}
    \alpha(\vec{v}_{12})=\frac{\alpha(\vec{v}_{2})}{\alpha(\vec{v}_{1})}\, ,
\end{equation}
but since the velocities where arbitary $\alpha$ has to be constant and we are
free to chose $\alpha\equiv 1$. Therefore the line element has to stay
invariant.
\section{Invariant Distances, Metric and Signature}
In flat space we can calculate the distance $\Delta l$ between two points
$(x_1,y_1)$ and $(x_2,y_2)$ by the Pythagorean theorem
\begin{equation}
    \Delta l^2=\Delta x^2+\Delta y^2\, ,
    \quad \Delta x=x_2-x_1\,,\quad \Delta
    y=y_2-y_1\, .
\end{equation}
For an infinitesimal distance $\dif l$ we recover
\begin{equation}
    \dif l^2=\dif x^2+\dif y^2=\tensor{\delta}{_i_j}\dif x^i\dif x^j\, .
\end{equation}
If we introduce
\begin{equation}
    (\dif x^i)=\begin{bmatrix}
\dif x\\
\dif y
\end{bmatrix}\,,\quad (\delta_{ij})
=\begin{bmatrix}
1 & 0\\
0 & 1
\end{bmatrix}\, .
\end{equation}
We can expand the formula for the infinitesimal element and obtain
\begin{equation}
    \dif l^2=
    \begin{bmatrix}
        \dif x &
        \dif y
    \end{bmatrix}
    \begin{bmatrix}
        1 & 0\\
        0 & 1
    \end{bmatrix}
    \begin{bmatrix}
        \dif x\\
        \dif y
    \end{bmatrix}\, .
\end{equation}
Which we do to point out the similarity to the line element $\dif s^2$
\begin{equation}
    \dif s^2=
    \begin{bmatrix}
        \dif t &
        \dif x &
        \dif y &
        \dif z
    \end{bmatrix}
    \begin{bmatrix}
        -1 & 0 & 0 & 0\\
        0  & 1 & 0 & 0\\
        0  & 0 & 1 & 0\\
        0  & 0 & 0 & 1\\
    \end{bmatrix}
    \begin{bmatrix}
        \dif t\\
        \dif x\\
        \dif y\\
        \dif z\\
    \end{bmatrix}=:\eta_{\mu\nu}\dif x^\mu\dif x^\nu\, .
\end{equation}
We call the matrix $\eta_{\mu\nu}=\mathrm{diag}(-1,1,1,1)$ the
\emph{Minkowski-metric}. It has some obvious properties:
\begin{itemize}
    \item constancy: $\tensor{\eta}{_\mu_\nu_,_\varrho}=0$
    \item symmetry: $\tensor{\eta}{_\mu_\nu}=\tensor{\eta}{_\nu_\mu}$
    \item self inverse:
    $\tensor{\eta}{^\mu^\nu}:=\tensor{{\eta^{-1}}}{_\mu_\nu}=\tensor{\eta}{_\mu_\nu}$
\end{itemize}
We can use the metric to raise and lower indices:
\begin{equation}
    x_\nu=\tensor{\eta}{_\mu_\nu}x^\mu\, ,\quad x^\mu=\tensor{\eta}{^\mu^\nu}x_\nu\, .
\end{equation}
One has to pay attention with the matrix form, for example, the \emph{trace} of
the metric is given by
\begin{equation}
    \tensor{\eta}{^\mu_\mu}=\tensor{\eta}{^\mu^\nu}\tensor{\eta}{_\nu_\mu}
    =\tensor{\delta}{^\mu_\mu}=4\, .
\end{equation}
We say $\tensor{\eta}{_\mu_\nu}$ has signature $(-,+,+,+)$ or $(1,3)$, because
it has one negative and three positive eigenvalues.
\subsection{Poincaré-transformation}
To be consistent with a constant speed of light a transformation between two
coordinate systems must leave the line element invariant, i.e.
\begin{equation}
    \dif {s^\prime}^2 = \eta_{\mu\nu}\dif {x^\prime}^\mu\dif
    {x^\prime}^\nu=\eta_{\mu\nu}\dif x^\mu\dif
    x^\nu = \dif s^2\,. \label{eq:invarline}
\end{equation}
We consider afine coordinate transformations
\begin{equation}
    {x^\prime}^\mu= f^\mu(x^\nu)=\tensor{L}{^\mu_\nu}x^\nu+a^\mu\, .
\end{equation}
Where $\tensor{L}{^\mu_\nu}$ is a a Lorentz transformation and $a^\mu$ is a
constant shift.\\
We can now inspect the transformation properties of an allowed transformation.
The invariance of the line element implies
\begin{equation}
    \dif {s^\prime}^2 = \eta_{\mu\nu}\dif {x^\prime}^\mu\dif
    {x^\prime}^\nu=\eta_{\varrho\sigma}\tensor{L}{^\varrho_\mu}\tensor{L}{^\sigma_\nu}\dif
    x^\mu\dif x^\nu \stackrel{!}{=} \dif s^2\,,
\end{equation}
independent of the shift $a^\mu$. We will therefore focus our attention towards
the linerar transformation $\tensor{L}{^\mu_\nu}$. Equation
\eqref{eq:invarline} implies that
\begin{equation}
    \eta_{\mu\nu}=\tensor{L}{^\varrho_\nu}\eta_{\varrho\sigma}\tensor{L}{^\sigma_\nu}\,.\label{eq:invariance}
\end{equation}
We can take a look at an infinitimal transformation, that is up to an order of
$\varepsilon^2$
\begin{equation}
    \tensor{L}{^\mu_\nu}=\tensor{\delta}{^\mu_\nu}
    +\varepsilon\tensor{\omega}{^\mu_\nu}\, .
\end{equation}
The $\tensor{\omega}{^\mu_\nu}$ are called the \emph{generators} of the
transformation. Plugging into \eqref{eq:invariance}, ignoring higher powers in
$\varepsilon^2$ results in
\begin{equation}
    \begin{split}
        \eta_{\mu\nu}&=\eta_{\varrho\sigma}\left(\tensor{\delta}{^\varrho_\mu}
        +\varepsilon\tensor{\omega}{^\varrho_\mu}\right)\left(\tensor{\delta}{^\sigma_\nu}
        +\varepsilon\tensor{\omega}{^\sigma_\nu}\right)\\
        &=
        \eta_{\mu\nu}+\varepsilon\left(\tensor{\omega}{_\mu_\nu}+\tensor{\omega}{_\nu_\mu}\right)\,
        .
    \end{split}
\end{equation}
Since this must hold true for arbitary $\varepsilon$, the generators
$\tensor{\omega}{_\nu_\mu}$ must satisfy
$\tensor{\omega}{_\nu_\mu}=-\tensor{\omega}{_\mu_\nu}$, i.e. be antisymmetric.
In general we have $\frac{n(n-1)}{2}$ of these objects, where $n$ is the
dimension of the underlying space.
Since we are considering fourdimensional spacetime, there are six generators.
The finite transformations by an generalized angle $\alpha\in\Complex$ can be
obtained from the generators via
\begin{equation}
    \tensor{L}{^\mu_\nu}=\exp\left(\imI\alpha\tensor{\omega}{^\mu_\nu}\right)\,.
\end{equation}
We can
classify the resulting transformations into two groups:
\subsubsection{Rotations}
Rotations mix the spatial coordinates, but leave the time fixed. Consider for
example an rotation around $z$-axis. Then we have
\begin{equation}
    x^\prime=x\cos\alpha+y\sin\alpha \, ,\quad y^\prime=-x\sin\alpha+y\cos\alpha \, .
\end{equation}
The infinitisimal generator is
\begin{equation}
    \tensor{\omega}{^\mu_\nu}
    =
    \begin{bmatrix}
        0&0 & 0&0\\
        0&0 &1&0\\
        0&-1&0 &0\\
        0&0 &0 &0
    \end{bmatrix}\,
\end{equation}
and the transformation can be expressed in matrix form
\begin{equation}
    \begin{bmatrix}
        t\\
        x\\
        y\\
        z\\
    \end{bmatrix}=
    \begin{bmatrix}
        1&0 & 0&0\\
        0&\cos\alpha &\sin\alpha&0\\
        0&-\sin\alpha&\cos\alpha &0\\
        0&0 &0 &1
    \end{bmatrix}
    \begin{bmatrix}
        t^\prime\\
        x^\prime\\
        y^\prime\\
        z^\prime
    \end{bmatrix}
\end{equation}
\subsubsection*{Boosts}
Boosts can be thought of as rotations between time and spacial coordinates. We
consider a boost in $x$-direction. Since the line element must be invariant and
a linear transformation keeps the origin fixed we have
\begin{equation}
    -t^2+x^2=-{t^\prime}^2+{x^\prime}^2\, .
\end{equation}
This is an hyperbolic equation and can be parametrised by
\begin{equation}
    t = t^\prime\cosh\psi+x^\prime\sinh\psi\,, \quad x =
    t^\prime\sinh\psi+x^\prime\cosh\psi\, .
\end{equation}
Where $\psi$ is called the \emph{rapidity}.
The generator is
\begin{equation}
    \tensor{\omega}{^\mu_\nu}
    =
    \begin{bmatrix}
        0&1 & 0&0\\
        1&0 &0&0\\
        0&0&0 &0\\
        0&0 &0 &0
    \end{bmatrix}\,.
\end{equation}
The matrix form of the transformation is
\begin{equation}
    \begin{bmatrix}
        t\\
        x\\
        y\\
        z\\
    \end{bmatrix}=
    \begin{bmatrix}
        \cosh\psi&\sinh\psi & 0&0\\
        \sinh\psi&\cosh\psi &0&0\\
        0&0&1 &0\\
        0&0 &0 &1
    \end{bmatrix}
    \begin{bmatrix}
        t^\prime\\
        x^\prime\\
        y^\prime\\
        z^\prime
    \end{bmatrix}\, .
\end{equation}
The rapidity $\psi$ is not related to a physical quantity yet. We may ask to
which velocity $v$ does this boost correspond.
Therefore we consider the line $x^\prime= 0$.
Then
\begin{equation}
    t=t^\prime\cosh\psi\, ,\quad
    x=t^\prime\sinh\psi\, .
\end{equation}
We can express the velocity $v$, as seen by an observer in the primed system
\begin{equation}
    v=\frac{x}{t}=\tanh\psi\,.
\end{equation}
It is convenient to introduce the parameters $\beta$ and $\gamma$ defined as
follows
\begin{equation}
    \beta:=v\, ,\quad\gamma:=\frac{1}{\sqrt{1-\beta^2}}\, .
\end{equation}
In terms of these we find
\begin{equation}
    \sinh\psi = \beta\gamma\, , \quad \cosh\psi=\gamma\,.
\end{equation}
Therefore an alternative form is given by
\begin{equation}
    \tensor{L}{^\mu_\nu}=\begin{bmatrix}
\gamma&\beta\gamma & 0&0\\
\beta\gamma&\gamma &0&0\\
0&0&1 &0\\
0&0 &0 &1
\end{bmatrix}\, .
\end{equation}
And the coordinates are therfore related by
\begin{equation}
    t=\gamma\left(t^\prime+\beta x^\prime\right)\,,\quad
    x=\gamma\left(x^\prime+\beta t^\prime\right)\, .
\end{equation}
All together we have ten independent (affine) transformations that leave the
line element invariant. Namely:
\begin{itemize}
    \item four shifts by $a^\mu=(a^0,a^1,a^2,a^3)\transpose$
    \item three rotations in space corresponding to three real angles
    $\boldsymbol{\theta}=(\theta_1,\theta_2,\theta_3)\transpose$ (Euler angles)
    \item three boosts with constant velocity $\vec{v}=(v_1,v_2,v_3)\transpose$
\end{itemize}
The Poincaré transformations form a ten parameter group.
\begin{table}
    \centering
    \begin{tabulars}{lll}
        \toprule
        &Newton&Einstein (SR)\\
        \midrule
        & Law of inertia
        & Law of inertia \\
        Newton's Laws
        & $\vec{F}=m\ddot{\vec{x}}=m\od{\vec{p}}{t}$
        &$\vec{F}=m\od{\vec{p}}{\tau}$\\
        & $\vec{F}_{12}=-\vec{F}_{21}$
        & Momentum conservations (postulate)\\
        Transformations
        & boosts
        & Galilei Transformation\\
        
        & absolute structure
        & absolute time\\
        
        & $c=\mathrm{const.}$
        & $t^\prime=t$\\
        Force
        &$\vec{F}=\frac{G\textsubscript{N}m_1m_2}{r^2}$
        &$F^\alpha= \frac{q}{m} U_\beta
        \tensor{F}{^\alpha^\beta}$,\,  $F
        =\gamma\left(\begin{smallmatrix}
\beta
f_0\\
\vec{f}
\end{smallmatrix}\right)$\\
&$\Delta \Phi =4\pi\varrho$ &Maxwell equations
\\
\bottomrule
    \end{tabulars}
    \caption{Comparison of Newtonian theory and special relativity}
\end{table}
\subsection{Fourvectors}
There are three types of vectors\footnote{We could also put up a fourth class,
containing solely the zero vector.}:
\begin{enumerate}
    \item \emph{Spacalike} $\Delta\tensor{x}{^\mu}\Delta\tensor{x}{_\mu}>0$
    \item \emph{Lightlike} $\Delta\tensor{x}{^\mu}\Delta\tensor{x}{_\mu}=0$
    \item \emph{Timelike} $\Delta\tensor{x}{^\mu}\Delta\tensor{x}{_\mu}<0$
\end{enumerate}
All classes are transfomed into themself by Lorentz transformation. There is a
set of transformations that is forbidden by physical considerations: namely the
Parity change $\vec{x}\to -\vec{x}$ and time reversal $t\to -t$ because this are
no real symmetries of nature.
We will mainly deal with timelike and lightlike vectors for which the time order
does not change. For spacelike intervals the order can change. To remain causal
all particles have to move at velocity $v\leq c$. This is in contrast to
Newtonian theory where everything can affect everything, because the velocities
are unbounded. A direct consequence of this is for example that even the two
body problem has no exakt solution in special relativity.
Tabular~\ref{tab:fourvecs} shows a list of the dynamic fourvectors in SR,
notice that in contrast to the classical, theory the fouraccelaration
$\tensor{a}{_\mu}$ has no importance in SR.
\begin{table}
    \centering
    \begin{tabulars}{lll}
      	\toprule
		Fourvector&Definition&Normalisation\\
		\midrule
		Velocity
		&$\tensor{u}{^\mu}=\od{\tensor{x}{^\mu}}{\tau}=\gamma(1,\vec{v})^{\transpose}$
		&$\tensor{u}{_\mu}\tensor{u}{^\mu}=-1$\\
		Momentum
		&$\tensor{p}{^\mu}:=m\tensor{u}{^\mu}$
		&$\tensor{p}{_\mu}\tensor{p}{^\mu}=-m^2$\\
		Acceleration&$\tensor{a}{^\mu}:=\od{\tensor{u}{^\mu}}{\tau}$&$\tensor{a}{_\mu}\tensor{a}{^\mu}=0$\\
		\bottomrule
    \end{tabulars}
    \caption{Examples of fourvectors and normalisation.\label{tab:fourvecs}}
\end{table}
\section{Lagrangian Formalism}
We want to formulate special relativity in terms of a variation principle.
Therefore we consider massive particles, i.e. timelike paths. As a postulate we
take that the action $S$ is proportional to the proper time (as a generalized
``distance'') $\dif \tau:=\sqrt{-\dif s^2}$
\begin{equation}
    S= -\alpha \int_a^b\dif \tau = -\alpha
    \int_{t_1}^{t^2}\sqrt{1-\vec{v}^2}\dif t =: \int_{t_1}^{t^2}L\dif t\, ,
\end{equation}
whith $\alpha$ a constant, that has to be determined. The Lagrangian $L$ is
given by
\begin{equation}
    L=-\alpha\sqrt{1-\vec{v}^2}\simeq
    -\alpha\left(1-\frac{1}{2}\vec{v}^2+\ldots\right) \, .
\end{equation}
We demand that we recover the classical theory in the limit $\vec{v}\to 0$. The
the lowest order kinetic term is
\begin{equation}
    T=\frac{1}{2}\alpha \vec{v}^2\, .
\end{equation}
Comparing with the classical kinetic energy $T=\frac{1}{2}m\vec{v}^2$ yields
$\alpha=m$. If we substitute $\alpha$ we recover the Lagrangian of special
relativity
\begin{equation}
    L\textsubscript{SR}=-m\sqrt{1-\vec{v}^2}=-m\gamma^{-1}\, .
\end{equation}
We can proceed calculate the generalized momenta
\begin{equation}
    p_i=\dpd{L}{v_i}=\frac{mv_i}{\sqrt{1-\vec{v}^2}}=\gamma m v_i\, .
\end{equation}
The energy can be calculated via the Hamiltonian $H$
\begin{equation}
    E=H=\vec{p}\cdot\vec{v}-L=\gamma m \vec{v}^2 + m\gamma^{-1} =\gamma m\, .
\end{equation}
Expanded in $\vec{v}^2$ the energy reads as
\begin{equation}
    E=m+\frac{1}{2}m\vec{v}^2+\ldots\, .
\end{equation}
If we restore units of $c$ for a moment we get that the constant term is
equal to $mc^2$. This is Einsteins famous $E=mc^2$.
We can further relate energy and momentum to each other. Therefore consider the
square of the momentum $\vec{p}$
\begin{equation}
    \vec{p}^2=\frac{m\vec{v}^2}{1-\vec{v}^2}\, .
\end{equation}
Solving for $\vec{v}^2$ yields
\begin{equation}
    \vec{v}^2=\frac{\vec{p}^2}{\vec{p}^2+m^2}\, .
\end{equation}
Which we can insert in the expression for the energy
\begin{equation}
    E^2=\frac{m^2}{1-\vec{v}^2}=\vec{p}^2+m^2 \label{eq:onshell}\, .
\end{equation}
The equation \eqref{eq:onshell} is called the \emph{on shell condition}.
\begin{sidenote}[On massive photons]
There is nothing in the theory that predicts that the photon is massless. It
could be possible that its mass is only really small and that in fact that the
photon does not travel at the speed of light. (This would make it
convinient to rename the constant $c$) Even though the mass should be so small
that the de Broglie wavelength is of the order of the size of the universe,
massive photons would have a huge impact.
\end{sidenote}
\begin{sidenote}[Complex Time]
We could in principle make a substitution $t\to \imI t$. Which would free us
from the need to distingish between co- and contravariant vectors, because the
inner product would be given as
\begin{equation}
    g(x,y)=\left(\imI x^0\right)\left(\imI
    y^0\right)+\vec{x}\cdot\vec{y}=-x^0y^0+\vec{x}\cdot\vec{y}
\end{equation}
and hence $\tensor{g}{_i_j}=\tensor{\delta}{_i_j}$, which basically means we can
raise and lower indices at will. This is practical whenever you do calculations,
e.g. in computer simulations.
\end{sidenote}
\section{Vectors and Tensors in SRT}
At this point we will just list some objects, that we call vectors and tensors
and delay the definition of those to a later chapter.
An example of a vector is the four current $J^\mu$:
\begin{equation}
    J^\mu=\gamma(\varrho,\vec{j})\, ,
\end{equation}
with $\varrho$ the charge density and $\vec{j}$ the 3-current density. A bit sloppy
we refer to a tensor as a 'vector with more indices' that transforms
accordingly, i.e.
\begin{equation}
    \tensor{T}{^\mu^\nu}\to
    \tensor{L}{^\mu_\alpha}\tensor{L}{^\nu_\beta}\tensor{F}{^\alpha^\beta}\,
\end{equation}
and similar for an arbitary number of indices. A typical example is the field strength tensor
$\tensor{F}{^\mu^\nu}$ of electrodynamics given in terms of the
vector potential $\tensor{A}{_\mu}$:
\begin{equation}
    \tensor{F}{_\mu_\nu}=\partial_\mu\tensor{A}{_\nu}-\partial_\nu\tensor{A}{_\mu}\,
    .
\end{equation}
The matrix form is
\begin{equation}
    \tensor{F}{^\mu^\nu}=
    \begin{bmatrix}
        0  &   -E_1 &  -E_2 &  -E_3 \\
        E_1 &   0  &  -B_3 & B_2 \\
        E_2 & B_3 &   0  &  -B_1 \\
        E_3 &  -B_2 & B_1 &   0  \\
    \end{bmatrix}\, .
\end{equation}
Therefore the tensor $\tensor{F}{^\mu^\nu}$ takes the role of the fields
$\vec{E},\vec{B}$. In tensor form \textsc{Maxwell} equations take the particular
simple form, namely
\begin{equation}
    \partial_\nu\tensor{F}{^\mu^\nu}=4\pi\tensor{J}{^\mu}\,, \quad
    \partial_\alpha\tensor{F}{_\mu_\nu}
    +\partial_\nu\tensor{F}{_\alpha_\mu}
    +\partial_\mu\tensor{F}{_\nu_\alpha}=0\,.
\end{equation}
The Lorentz force is given by
\begin{equation}
    \tensor{F}{^\mu}=q\tensor{F}{^\mu^\nu}\tensor{u}{_\nu}
\end{equation}
Where $\tensor{u}{^\mu}=(1,\vec{v})\transpose$ is the four velocity. As
annother example we can take a look at plane waves. The vector potential of a plane wave is
\begin{equation}
    \tensor{A}{^\mu}=\tensor{\hat{A}}{^\mu}
    \exp\left[\imI(\vec{k}\cdot\vec{r}-\omega
    t)\right]
    =\tensor{\hat{A}}{^\mu}\exp\left(\imI\tensor{K}{^\mu} \tensor{X}{_\mu}\right)\,
    .
\end{equation}
Where $\tensor{K}{^\mu}=\left(\omega,\vec{k}\right)\transpose$ must be a four
vector, because otherwise we could use plane waves to distinguish between
inertial systems. The dispersion relation reads as
\begin{equation}
    \tensor{K}{^\mu}\tensor{K}{_\mu}=-\omega^2+\vec{k}^2=0\, .
\end{equation}
It is equivalent to the on shell condition \eqref{eq:onshell}, when we identify
the photons energy $E=\omega$ momentum $\vec{p}=\vec{k}$ and mass $m=0$.
\subsection{The Energy Momentum Tensor}
Evtl. aus anderer Quelle, arges Durcheinander mit komplexer Notation\ldots kommt
später auch nochmal
\begin{sidenote}
Above a certain limit a system cannot be stabilized by pressure, because the
gravity cuples to the pressure and therefore increasing the pressure would lead
to a stronger attraction due to gravity and hence inevetiable collapse.
\end{sidenote}
The energy momentum tensor is symmetric
\begin{equation}
    \tensor{T}{^\mu^\nu}=\tensor{T}{^\nu^\mu}\, .
\end{equation}
Its trace is given by
\begin{equation}
    \tensor{T}{^\mu_\mu}=(\varrho-p)+4p=-\varrho+3p\, .
\end{equation}
It is traceless for photons. If an external field is present the divergence is
given by
\begin{equation}
    \tensor{T}{^\mu^\nu_,_\nu}=\tensor{D}{^\mu}\,  .
\end{equation}
Where the external momentum is itselve given as the divergence
\begin{equation}
    \tensor{D}{^\mu} =-\tensor{S}{^\mu^\nu_,_\nu}\, .
\end{equation}
Putting both terms together gives
\begin{equation}
    \partial_\mu(\tensor{T}{^\mu^\nu}+\tensor{S}{^\mu^\nu})=0\,,
\end{equation}
Which can be interpretet as the total energy conservation.
\subsection{Noether's Theorem}
We introduce a tensorial generalisation of the angular momentum (density???)
\begin{equation}
    \tensor{T}{^\lambda^\mu^\nu}:=
    \tensor{x}{^\lambda}\tensor{T}{^\mu^\nu}-\tensor{x}{^\mu}\tensor{T}{^\lambda^\nu}\,
    .
\end{equation}
The angular momentum is given by (etwas viele Komponenten?)
\begin{equation}
    \tensor{L}{^i^j} = \int\dif{}^3 x \tensor{M}{^i^j^0}\, .
\end{equation}
It can be shown that it is a conserved quantity
\begin{equation}
    \tensor{\partial}{_\nu}\left(\tensor{x}{^\lambda}\tensor{T}{^\mu^\nu}-\tensor{x}{^\nu}\tensor{T}{^\lambda^\nu}\right)
    =\tensor{T}{^\mu^\lambda}-\tensor{T}{^\lambda^\mu}+\tensor{x}{^\lambda}
    \tensor{T}{^\mu^\nu_,_\nu}-\tensor{x}{^\nu}\tensor{T}{^\lambda^\nu_,_\nu}=0\, .
\end{equation}
\begin{remark}
The existence of a conserved symmetric tensor $\tensor{T}{^\mu^\nu}$ is
nessecary in order to build a relativistic theory
\end{remark}
\begin{example}[Electrodynamics]
We start by inspecting the Lorentz Force $\tensor{F}{^\mu}$ which is given by
\begin{equation}
    \tensor{F}{^\mu}=e\tensor{F}{^\mu^\nu}\tensor{u}{_\nu}=\tensor{F}{^\mu^\nu}\tensor{J}{_\nu}=\tensor{D}{^\mu}\,
    .
\end{equation}
We now ask whether ther is a potential $\tensor{S}{^\mu^\nu}$ so that
$\tensor{D}{^\mu}=-\tensor{S}{^\mu^\nu_{,\nu}}$. Indeed a potential is given by
\begin{equation}
    \tensor{S}{^\mu^\nu}=\tensor{F}{^\mu_\alpha}\tensor{F}{^\nu^\alpha}-\frac{1}{4}\tensor{\eta}{^\mu^\nu}\tensor{F}{_\alpha_\beta}\tensor{F}{^\alpha^\beta}\,
    .
\end{equation}
We can check
\begin{equation}
    \tensor{S}{^\mu^\nu_{,\nu}}=\tensor{F}{^\mu_\alpha_{,\nu}}\tensor{F}{^\nu^\alpha}+
    \tensor{F}{^\mu_\alpha}\tensor{F}{^\nu^\alpha_{,\nu}}
    -\frac{1}{2}\tensor{\eta}{^\mu^\nu}\tensor{F}{_\alpha_\beta_{,\nu}}\tensor{F}{^\alpha^\beta}\,
    .
\end{equation}
\begin{align}
    \tensor{F}{^\mu_\alpha_{,\nu}}\tensor{F}{^\nu^\alpha}
    &=\tensor{F}{^\mu_\nu_{,\alpha}}\tensor{F}{^\alpha^\nu}
    =\tensor{F}{^\nu_\mu_{,\alpha}}\tensor{F}{^\nu^\alpha}
    \\
    %\tensor{F}{^\mu_\alpha}\tensor{F}{^\nu^\alpha_{,\nu}}&=\\
    \tensor{\eta}{^\mu^\nu}\tensor{F}{_\alpha_\beta_{,\nu}}\tensor{F}{^\alpha^\beta}
    &=\tensor{F}{_\alpha_\nu^{,\mu}}\tensor{F}{^\alpha^\nu}
    = -\tensor{F}{_\alpha_\nu^{,\mu}}\tensor{F}{^\nu^\alpha}
    \,
    .
\end{align}
\begin{equation}
\tensor{S}{^\mu^\nu_{,\nu}}=\tensor{F}{^\nu_\mu_{,\alpha}}\tensor{F}{^\nu^\alpha}
+\frac{1}{2}\tensor{F}{_\alpha_\nu^{,\mu}}\tensor{F}{^\nu^\alpha}
\end{equation}
The Lagrangian of electrodynamics is given by
\begin{equation}
    \mathcal{L}=-\frac{1}{2}\tensor{F}{_\mu_\nu}\tensor{F}{^\mu^\nu}+\tensor{J}{_\mu}\tensor{A}{^\mu}-\frac{1}{2}m^2\tensor{A}{_\mu}\tensor{A}{^\mu}\,
    .
\end{equation}
 The tensor $S$ is known as \emph{electromagnetic stress–energy tensor}.
\begin{equation}
\tensor{S}{^\mu^\nu}=
\begin{pmatrix}
u&\vec{S}\transpose\\
\vec{S}& -\tensor{\sigma}{_i_j}
\end{pmatrix}
\end{equation}
Where
$\tensor{\sigma}{_i_j}=\tensor{E}{_i}\tensor{E}{_j}+\tensor{E}{_i}\tensor{E}{_j}-\frac{1}{2}\tensor{\delta}{_i_j}\left(\vec{E}^2+\vec{B}^2\right)$
is the \emph{Maxwell stress tensor}, $\vec{S}=\vec{E}\times\vec{B}$
\emph{Poynting vector} and energy density $u
=\frac{1}{2}\left(\vec{E}^2+\vec{B}^2\right)$.
\end{example}

\chapter{Gravity and Geometry}
The observable universe is stable. There are two obvious configurations in which this is possible:
\begin{enumerate}
    \item Static universe, masses are arranged in a grid, all nett forces cancel.
    However small fluctuations cause the system to collapse therefore this is
    no possible description for the universe. 
%TODO picture
    \item Expanding universe, all masses move away from each other, overcoming the gravitational attraction.
    Theoretically such a system can be described by using Newtonian Physics introducing additional energy contributions.
    This turns out to be inconsistent.
\end{enumerate}
Since in the second description all particles are accelerated relative to each other, there are no inertial systems.
A theory in which all observers are equal must therefore be local and thus be described by means of differential geometry.
We claim that the laws of physics are the same in every system.
If we assume that the \name{Maxwell}'s equations are right, the
Newtonian theory of gravity must be wrong.
Implications:
All free falling systems are equivalent (i.e. indistinguishable by the observer).
Light must bend, otherwise a beam could be used to deduce whether your system is inertial.
The following example illustrates that Euclidean geometry is no
adequate description of space-time.
\begin{example}[Rotating Sphere]
see Introduction to tensor calculus
%TODO copy or reference page
\end{example}
\section{Coordinate Systems}
We will start by studying coordinate systems in the flat space $\Reals^2$, which
should be familiar.
\subsection*{Cartesian Coordinates}
Cartesian coordinates are described by two coordinates $x,y$ that are measured
in two orthogonal directions from the origin. The distance $s$ between two
arbitrary points $(x_1,y_1)$ and $(x_2,y_2)$ can be calculated using
\name{Pythagoras}' theorem
\begin{equation}
    s^2=(x_1-x_2)^2+(y_1-y_2)^2\,.
\end{equation}
An infinitesimal distance is likewise given by
\begin{equation}
    \dif s^2=\dif x^2+\dif y^2\,.  \label{eq:cartline}
\end{equation}
\subsection*{Polar Coordinates}
If we describe a point in flat space by an angle $\varphi$ and an distance $r$
from the origin, we get polar coordinates. The conversion between the systems
reads
\begin{equation}
    x= r\cos\varphi\quad y= r\sin\varphi\,.
\end{equation}
A infinitival change in the polar coordinates therefore results in 
\begin{align}
    \dif x&= \dpd{x}{r}\dif r+\dpd{x}{\varphi}\dif \varphi = \cos\varphi\dif
    r-r\sin\varphi\dif \varphi\,,\\
    \dif y&= \dpd{x}{r}\dif r+\dpd{y}{\varphi}\dif \varphi = \sin\varphi\dif
    r+r\cos\varphi\dif \varphi\,.
\end{align}
Plugging this into \eqref{eq:cartline} gives the line element in polar coordinates
\begin{equation}
    \dif s^2=\dif r^2+r^2\dif \varphi^2
\end{equation}
\begin{figure}[hbtp!]
\centering
 \includegraphics{cartcoord.pdf}
 \includegraphics{polarcoord.pdf}
\caption{}
%TODO Caption
\end{figure}
\begin{figure}[hbtp!]
\centering
 \includegraphics[scale=0.75]{CoordinateGridCartesian.pdf}\quad
 \includegraphics[scale=0.75]{CoordinateGridPolar.pdf}
\caption{Coordinate grids.}
%TODO Caption
\end{figure}

In matrix form
\begin{equation}
\dif s^2=
\begin{bmatrix}
\dif r& \dif \varphi
\end{bmatrix}
\begin{bmatrix}
1& 0\\
0& r^2\\
\end{bmatrix}
\begin{bmatrix}
\dif r\\ \dif \varphi
\end{bmatrix}\, .
\end{equation}
The matrix
\begin{equation}
g(\vec{r})=
\begin{bmatrix}
1& 0\\
0& r^2\\
\end{bmatrix}\, ,
\end{equation}
is called the \emph{metric}.
In general we have
\begin{equation}
    \dif s^2 = g_{ij}\dif x^i\dif x^j\, .
\end{equation}
The idea is to keep the law of inertia, i.e. particles still move on straight
line. However, we need to generalize the concept of a 'straight' line, in a
curved space.
\section{Variation Principle}
\label{sec:varprinc}
We know that straight lines are curves minimizing the distance between two
points. We generalize this concept to curved space by an variation principle.
Again we take a look at flat space, but with curved coordinates.
The length $S$ of a curve $\gamma$ with $\gamma^i(\lambda) = x^i(\lambda)$ is
given by the integral
\begin{equation}
    S=\int_{\gamma}\sqrt{\dif s^2} =
    \int_{\gamma}\sqrt{\tensor{g}{_i_j}\dif \tensor{x}{^i}\dif
    \tensor{x}{^j}}=\int_{a}^{b}\sqrt{\tensor{g}{_i_j}\dod{\tensor{x}{^i}}{\lambda}
    \dod{\tensor{x}{^j}}{\lambda}}\dif \lambda\, .
\end{equation}
As stated above generalised straight lines satisfy $\delta S = 0$. If we define
$L:=\left(\tensor{g}{_i_j}\od{\tensor{x}{^i}}{\lambda}
\od{\tensor{x}{^j}}{\lambda}\right)^{\nicefrac{1}{2}}$, $S$ takes a form
familiar from classical mechanics:
\begin{equation}
    S=\int_a^b L\dif \lambda\, .
\end{equation}
The extremal condition implies the Euler Lagrange equations
\begin{equation}
    \dod{}{\lambda}\pd{L}{\left(\pd{\tensor{x}{^i}}{\lambda}\right)}
    -\pd{L}{\tensor{x}{^i}}
    =0\, .		\end{equation}
We can calculate the relevant terms to 
\begin{align}
\dpd{L}{\tensor{x}{^i}}&=\frac{1}{2\sqrt{g_{ij}\od{x^i}{\lambda}
\od{x^j}{\lambda}}}\tensor{g}{_j_k_{,i}}\dod{x^j}{\lambda}
\dod{x^k}{\lambda}\,,\\
\dpd{L}{\left(\pd{\tensor{x}{^i}}{\lambda}\right)}
&=\frac{1}{\sqrt{g_{ij}\od{x^i}{\lambda}
\od{x^j}{\lambda}}}\tensor{g}{_j_i}\dod{x^j}{\lambda}\, .
\end{align}
If we choose the parameter $\lambda$ so that we are parametrised by the arc
length\footnote{this is impossible for null i.e. lightlike geodesics, it can be
shown however, that the resulting equation also holds true for null geodesics.}
i.e.
\begin{equation}
\od{}{\lambda}\left(\sqrt{g_{ij}\od{x^i}{\lambda}\od{x^j}{\lambda}}\right)=0\,,
\end{equation}
the Euler Lagrange equations simplify to
\begin{equation}
0=\frac{1}{\sqrt{\tensor{g}{_i_j}\od{\tensor{x}{^j}}{\lambda}
\od{\tensor{x}{^j}}{\lambda}}}\dod{}{\lambda}
\left(\tensor{g}{_j_i}\dod{\tensor{x}{^j}}{\lambda}\right)
-\frac{1}{2\sqrt{\tensor{g}{_i_j}\od{x^i}{\lambda}
\od{\tensor{x}{^j}}{\lambda}}}\tensor{g}{_j_k_{,i}}\dod{\tensor{x}{^j}}{\lambda}
\dod{\tensor{x}{^k}}{\lambda}\,,
\end{equation}
or equivalently
\begin{equation}
\begin{split}
0
&=\dod{}{\lambda}\left(\tensor{g}{_j_i}\dod{\tensor{x}{^j}}{\lambda}\right)
-\frac{1}{2}\tensor{g}{_j_k_{,i}}\dod{x^a}{\lambda}
\dod{x^k}{\lambda}\\
&=\tensor{g}{_j_i_{,k}}\dod{\tensor{x}{^j}}{\lambda}\dod{\tensor{x}{^k}}{\lambda}
+\tensor{g}{_j_i}\dod[2]{\tensor{x}{^j}}{\lambda}
-\frac{1}{2}\tensor{g}{_j_k_{,i}}\dod{\tensor{x}{^j}}{\lambda}\\
&=\tensor{g}{_j_i}\dod[2]{\tensor{x}{^j}}{\lambda}
+\frac{1}{2}\left(\tensor{g}{_j_i_{,k}}+\tensor{g}{_i_j_{,k}}
-\tensor{g}{_j_k_{,i}}\right)\dod{\tensor{x}{^j}}{\lambda}
\dod{\tensor{x}{^k}}{\lambda}\label{eq:PreGeo}\,.
\end{split}
\end{equation}
The term invoking derivatives of the metric defines the \emph{Christoffel
symbols of the first kind}
\begin{equation}
    \csym{j}{k}{i}:=\frac{1}{2}
    \left(\tensor{g}{_j_i_{,k}}+\tensor{g}{_i_j_{,k}} -\tensor{g}{_j_k_{,i}}\right)\, .
\end{equation}
It is convenient to multiply \eqref{eq:PreGeo} by the inverse metric $g^{li}$ so
that we obtain the \emph{geodesic equation}
\begin{equation}
    0 =
    \od[2]{\tensor{x}{^l}}{\lambda}
    +\cSym{l}{j}{k}\od{\tensor{x}{^j}}{\lambda}\od{\tensor{x}{^k}}{\lambda}\,
    .\label{eq:geodeq}
\end{equation}
Where $\cSym{l}{j}{k}$ are the \emph{Christoffel symbols of the second kind}
\begin{equation}
    \cSym{l}{j}{k}:=g^{li}\csym{j}{k}{i}=\frac{1}{2}g^{li}
    \left(\tensor{g}{_j_i_{,k}}+\tensor{g}{_i_j_{,k}} -\tensor{g}{_j_k_{,i}}\right)\, .
\end{equation}
% remark is obsolete as long as bracket notation for christoffel symbols is used
%\begin{remark}
%Although the notation looks as the Christoffel symbols form a tensor, however
% they do not.
%\end{remark}
In flat space we have $\tensor{g}{_i_j}=\tensor{\eta}{_i_j}$ and can easily
check that all Christoffel symbols vanish. We therefore recover the ordinary equation of motion for a free particle
\begin{equation}
    0 = \od[2]{\tensor{x}{^i}}{\lambda}\, .
\end{equation}
\begin{figure}[hbtp!]
\centering
 \includegraphics{sphere_geodesics1.pdf}
\caption{Great circles are geodesics, i.e. shortest connections of points, on
a sphere.}
%TODO Caption
\end{figure}
\begin{figure}[hbtp!]
\centering
 \includegraphics{WorldlineLightcones.pdf}
\caption{}
%TODO Caption
\end{figure}


% \begin{example}
% Suppose a observer follows a free falling body in a homogeneous field.
% Therefore a transformation between the system of the earth and the one of the body are given by
% (for simplicity we only consider the coordinate along it is falling)
% \begin{equation}
%     (t,x)\to\left(t,x-\frac{1}{2}gt^2\right)
% \end{equation}
% Analogous to the Riemannian case discussed before, the line element takes the form
% \begin{equation}
%     \begin{split}
% \dif s^2&=-\dif t^2 +\dif x^2\\
% &=-\dif t'^2+(\dif x'- gt\dif t')(\dif x'- gt\dif t')\\
% &=(g^2t'^2-1)\dif t'^2-2gt\dif x'\dif t'+\dif x'^2
% \end{split}
% \end{equation}
% \end{example}
% \section{Newtonian Limit}

\chapter{Differential Geometry}
As we have noted before general relativity is a inherent local theory. It is convenient to formulate it in terms of differential geometry.
We introduce the notion of a manifold.

\begin{definition}
A $n$ dimensional manifold $M$ is a Hausdorff space with countable basis, that
is locally homeomorphic to $\mathbb{R}^n$.
\end{definition}
\begin{remark}
The requirements Hausdorff and countable basis are of a more technical nature and are satisfied for most of the objects one can imagine 
except some pathological examples (we won't go into the details on this).

Locally homeomorphic to $\mathbb{R}^n$ means there exists a set of \emph{charts} 
$(\varphi,U^\varphi)$ called an \emph{atlas} $\mathcal{A}$ with $\cup_{\varphi\in\mathcal{A}} U^\varphi =M$, 
i.e. the charts cover the whole manifold. The maps $\varphi:U^\varphi\to \varphi(U^\varphi)\subset\mathbb{R}^n $ are homoemorphisms, 
meaning that $U^\varphi$ is open, $\varphi$ is bijective and both $\varphi$ and $\varphi^{-1}$ are continuous.
Further for any two $\varphi,\psi\in \mathcal{A}$, the coordinate changes 
$\varphi\circ\psi^{-1}:\psi(U^\psi\cup U^\varphi)\to \phi(U^\psi\cup U^\varphi)$ be infinitely often differentiable.
(BIlder)
\end{remark}

We can now reduce differentiation on the manifold to the ordinary differentiation in $\mathbb{R}^n$. 
Since physical laws are described in terms of differential equations, we can formulate them on $M$. 
The fact that the coordinate changes are $C^\infty$ ensures that differentiability is well defined (and thus the physical laws are).

\begin{sidenote}[Differentiability depends on the Differential Structure]
There can be different \emph{differential structures} on a manifold, 
which means there are multiple (maximal)alases, which
could not be merged because the coordinate changes would not be $C^\infty$. Those differentiable structures therefore imply different notions of differentiability. 
Remarkably this may even play a role in some physical theories. 
As an example an 11d-supergravity can be described as a product
$\Reals^{3+1}\times \Sphere^7$.
Where $\Sphere^7$ is the 7-sphere and $\Reals^{3+1}$ Mikovski space.
This means on every point in the $\mathbb{R}^{3+1}$ there is a (small) $\Sphere^7$  located that contains additional spatial dimensions. 
The $\Sphere^7$ has 28 different differential structures, so the choice of
such a structure affects the theory for the above reasons.
\end{sidenote}
All simple examples we come of can be embedded in a higher space. It holds true
that every real $n$-dimensional Manifold can be embedded to $\Reals^{2n}$
(This is however not true for complex, i.e. analytic manifolds).
For example the $\Sphere^2$ can be interpreted as submanifold of the $\Reals^3$.
However manifolds are objects that exists independent of such embeddings. 
For example a torus can be thought of as a square with the opposite sides identified (leaving to the left results in re-entering in the left).
\begin{sidenote}[Topology of the Universe]
In addition to the local structure, we may question the global, i.e. the
topological structure of the universe.
On may for example imagine that we live on the surface of a 3-sphere (finite but boundless universe). 
However this might be observable in crosscorelation in the cosmic microwave background from photons reaching us 
from different directions but coming from the same event. There is no evidence of such phenomena so far. 
Most models can be excluded to some certainty. A cylindrical
universe is still possible (finite in one, infinite in the
other directions).
\end{sidenote}
\section{Vectors}
Vectors are important objects describing physics. The naive view as an "arrow pointing frow one point to another" is flawed. 
For example on a sphere an arrow connecting two points does not make much sense.
We want to find a description of vectors as objects that are naturally related to the structure of the manifold independent of the embedding.
There are three equivalent definitions for a vector:
\begin{enumerate}
    \item algebraic (mathematical, suitable for proofs)
    \item physically
    \item geometrically (ugly, but plastic)
\end{enumerate}
\subsection*{Definitions}
\begin{definition}[Vector, algebraic]
A vector is a derivation at the germ of a function at $p$.
\end{definition}
The germ is the set of all functions that are locally equal,i.e. vectors are local objects.
\begin{definition}[Derivation]
A derivation $D$ satisfies the following rules for all $f,g\in C^\infty(\Reals)$ and $\lambda \in \Reals$:
\begin{align}
    D(af+bg) &=Df+Dg\\
    D(\lambda f)&=\lambda f\\
    D(fg)&= (Df)g+f(Dg)
\end{align}
\end{definition}
Given two vectors we can construct a new one, the \emph{Lie Braket}
\begin{equation}
    [X,Y]f:=X(Yf)-Y(Xf)\, .
\end{equation}
The only property that has to be checked is that it satisfies the Leibniz rule.
\begin{equation}
    XY(fg)=X((Yf)g+f(Yg))=(XYf)g+(Yf)(Xg)+(Yg)(Xf)+(XYg)f\\
\end{equation}
Subtracting $YX(fg)$ proves that $[X,Y]$ is indeed a vector. The fact that we
have a natural vectorspace structure on the set of vectors at $p$ motivates the
following
\begin{definition}
The Tangentspace $T_pM$ is the space of all vectors in a point $p\in M$.
\end{definition}
A basis of $T_pM$ is given by the derivation along the
coordinates $\partial_i$, therefore its dimension is equal to that of the manifold $M$.
Proof sketch:
\begin{enumerate}
    \item Show $f(x^i)=f(0)+x^i\tilde{f}(x^i)$
    \item Write $X=a^i\partial_i$
    \item Show $Xf=0\quad \forall f \iff X=0$
\end{enumerate}
Every vector $A$ can be written as $A=A^i\pd{}{{x^i}}$, where $A^i$ are the
components of the vector. We can now look how the components of the vector
transform under a change of coordinates (the vector itself is invariant!). 
We usually denote the elements of the transformed systems with a bar.
By the chain rule we have
\begin{equation}
    A= a^k\pd{}{{x^k}}= a^k\pd{\overline{x}^i}{{x^k}}\pd{}{{\overline{x}^k}}\, .
\end{equation}
We can also express $A$ directly in the new basis
\begin{equation}
    A= \overline{a}^i\pd{}{{\overline{x}^i}}\, .
\end{equation}
Comparing the coefficients gives the vector transformation law
\begin{equation}
    \overline{a}^i=a^k\pd{\overline{x}^i}{{x^k}}\label{eq:coefftrafo}\, .
\end{equation}
Sometimes a vector is defined as a object that transforms according to \ref{eq:coefftrafo} under a change of coordinates, 
this is the physical definition. It is a priori not clear that a vector also corresponds to a geometrical object. 
Consider a curve on $M$, i.e. a map $\gamma:\mathbb{R}\to M$, with
$\gamma(0)=p$, $\dot{\gamma}(0)=X$. Then $D_X f=\od{}{t}(f\circ\gamma)(0)$ is
a derivative, namely the directional derivative along $X$.
For the special curves $\gamma_i(t)=p+te_i$
$D_{\dot{\gamma}_i} f=\partial_if$, so we can identify the derivatives with
the geometrical tangent space.
Since we have a basis we can work in local coordinates, 
e.g. let $A=A^i\pd{}{{x^i}}$, $B=B^i\pd{}{{x^i}}$, then the Lie bracket reads
\begin{equation}
    [A,B]^j=A^i\partial_iB^j-B^i\partial_iA^i\, .
\end{equation}
Since the tangent space is a vector space, we can define its dual space
\begin{equation}
    T_pM^*=\{L:T_pM\to \mathbb{R}\, |\, L \text{ linear}\}\, ,
\end{equation}
which is again a vector space of the same dimension. Its elements are called dual or covariant vectors.
We can define a basis on $	T_pM^*$, which we denote by $\dif x^i$ and  which acts on $T_pM$ via
\begin{equation}
    \dif x^i(\partial_j)=\delta^i_j\, . \label{eq:orthdual}
\end{equation}
It can easily deduced by \eqref{eq:orthdual} that the components of a dual vector transform as
\begin{equation}
    \overline{a}_i=\pd{x^k}{{\overline{x}^i}}a_k\, .
\end{equation}
\begin{remark}[Dual vectors in flat space]
If $\vec{a},\vec{b}\in\mathbb{R}^n$ contain the component of a vector and a dual vector respectively, 
then the transformation can be written in matrix form
\begin{align}
    \vec{a}&\to V\vec{a}\, ,\\
    \vec{b}&\to\left(V^\intercal\right)^{-1}\vec{b}\, ,
\end{align}
with $V_{ij}=\pd{\overline{x}^i}{{x^j}}$. 
In normal calculus we restrict ourself to orthogonal transformations (i.e.
mapping orthonormal bases onto each other) for which $(O^\intercal)^{-1}=O$.
Which is the reason why we do not bother to distinguish between vectors and dual vectors because they transform identically. 
In special relativity we have e.g. $(\Lambda^\intercal)^{-1}\neq\Lambda$ for a
boost, the difference becomes even more important in general relativity where the relation can become arbitrarily complicated.
\end{remark}
\section{Tensors}
From vectors $A$ ,$B$ we can construct new objects with multiple indices that posses well defined transformation behaviour. 
For example consider
\begin{equation}
    \overline{T}^{ij}=a^ib^j\, ,
\end{equation}
which transforms as
\begin{equation}
    T^{ij}=\pd{\overline{x}^i}{{x^k}}\pd{\overline{x}^j}{{x^l}}a^kb^l=\pd{\overline{x}^i}{{x^k}}\pd{\overline{x}^j}{{x^l}}T^{kl}\,
   .\label{eq:tensortrafo}
\end{equation}
We call an object that transforms in this way a \emph{tensor}. 
As with vectors, it is possible to define tensors in a coordinate independent
way.
At this point we will make things easier and only consider the physical
definition, i.e. classify tensors by a transformation according to \eqref{eq:tensortrafo}.

A tensor is said to be symmetric in two indices if it stays invariant when exchanging those indices, e.g.
\begin{equation}
    T_{ab}=T_{ba}\, .
\end{equation}
\begin{remark}
We have not yet established a relation between upper and lower indices, i.e. we have no metric. Expressions of the form
\begin{equation}
    \tensor{T}{^a_b}=\tensor{T}{^b_a}
\end{equation}
make no sense, since they can not be true in every system.
\end{remark}
\section{The Metric}
\begin{definition}[Metric]
The metric $g$ is a non-degenerate ($\det(g)\neq 0$), symmetric covariant two
tensor.
\end{definition}
We have already seen examples of metrics for the flat space, e.g. in spherical coordinates $g$ was given as
\begin{equation}
    g=
    \begin{pmatrix}
        1 & 0\\
        0 & r^2\\
    \end{pmatrix}
\end{equation}
Given a metric we relate vectors and dual vectors to each other by
\begin{equation}
    a_i=g_{ij}a^j
\end{equation}
\section{Parallel Transport}
Idea: Generalize parralel transport from flat space.
(PICTURES) 
If we express a vector in non-cartesian coordinates and shift it it's
coordinates do not change.
We take a look on two operations:
\begin{enumerate}
\item the change of the vector itself
\item the change of its coordinates
\end{enumerate}
Let $A_i$ be the coordinates of a vector in a system $x^i$ and $B_i$ the in a
system associated with coordinates $y^i$ respectively.
\begin{equation}
A_i\pd{{y^j}}{{x^i}}B_j\, ,\quad B_i\pd{{x^j}}{{y^i}}A_j\, .
\end{equation}
We look at vectors whose coodinates in the system $y^i$ do not change i.e.
$\delta B_i=0$ The variation of $A_i$ is given by
\begin{equation}
\delta
A_i=\delta\left(\pd{{y^j}}{{x^i}}\right)B_j
=\md{{y^j}}{2}{{x^i}}{}{{x^k}}{}\delta
x^k B_j\, .
\end{equation}
Expressing $B_i$ in terms of $A_i$ yields
\begin{equation}
\delta A_i = \md{{y^j}}{2}{{x^i}}{}{{x^k}}{}\pd{{x^l}}{{y^j}}A_l\delta x^k
=:\affin{l}{i}{k}A_l\delta x^k
\end{equation}
$\affin{l}{i}{k}$ is called \emph{affine connection} or short affinity.
\begin{remark}
We can always find a coordinate system in wich $\affin{l}{i}{k}\equiv 0$.  
\end{remark}
We notice that if 
\begin{equation}
\left(\pd{{A_i}}{{x^j}}-\affin{l}{i}{k}A_l\right)\delta x^k = 0\, ,
\end{equation}
The vector $A$ does not change its cordinates. We define a \emph{covariant
derivative} 
\begin{equation}
\tensor{A}{_i_;_j}:=\pd{{A_i}}{{x^j}}-\affin{l}{i}{k}A_l\, .
\end{equation}
It can easyly be seen that the covariant derivative of a tensor transforms as a
tensor. 
\begin{remark}[The
Covariant Derivative in Electrodynamics] Example from Electrodynamics concerning the covariant derivative. 
The theory is invariant under transformations $\phi\to e^{\imI \alpha}\phi$, 
because $\phi^*\phi$ and $\phi^*\nabla\phi-\phi\nabla\phi^*$ do not change.
Vervollständigen\ldots
\end{remark}
Since we have now established a relation between vectors and dual vectors, we
can also determine the covariant derivative of a dual vector. Therefore we
consider the scalar $A_iB^i$. Since the covariant derivative satisfies the
Leibniz rule we get
\begin{align}
(A_iB^i);j = \tensor{A}{_i_;_j}\tensor{B}{^i}+\tensor{A}{_i}\tensor{B}{^i_;_j}
\end{align}
But for scalars the covariant derivative is identical to the normal derivative
so that 
\begin{align}
(A_iB^i)_{;j} =(A_iB^i)_{,j}=
\tensor{A}{_i_,_j}\tensor{B}{^i}+\tensor{A}{_i}\tensor{B}{^i_,_j}
\end{align}
If we put in the covariant derivative of a we get 
\begin{equation}
\tensor{A}{_i}\tensor{B}{^i_;_j}=\tensor{A}{_i}\left(\pd{}{{x^j}}B^i+\Gamma^i_{kj}\tensor{B}{^k}\right)
\end{equation}
Since $A$ was arbitary, we can deduce that
\begin{equation}
\tensor{B}{^i_;_j}=\left(\pd{}{{x^j}}B^i+\Gamma^i_{kj}\tensor{B}{^k}\right)
\end{equation}
for a (1,1)-tensor we get:
\begin{equation}
\tensor{A}{^i_j_;_k}=\pd{}{{x^k}}\tensor{A}{^i_j}-\Gamma^a_{jk}\tensor{A}{^i_a}+\Gamma^i_{ak}\tensor{A}{^a_j}\,
.\end{equation}
Similar expressions hold for tensors of arbitary rank where each index gives an
aditional term containing a contraction with the affinity $\Gamma^i_{jk}$. 
We now want to consider curved spaces. This can not immediatly be determined by
the metric, for example the polar coordinates do not look flat even though they
describe the ordinary $\Reals^2$. (PICTURES)

\chapter{Einstein's Field Equations}
We will derive Einsteins Equations by physical considerations. The poisson
equation reads as:
\begin{equation}
\Delta\Phi=4\pi\rho
\end{equation}
\begin{equation}
S\textsubscript{EH}=\int\dif x^4 \sqrt{-g}(R-2\Lambda)
\end{equation}
In SR, the energy impuls tensor $\tensor{T}{_\mu_\nu}$ is conserved i.e.
$\tensor{T}{_\mu_\nu^{,\nu}}=0$.As a natural extension we demand that the energy
impuls tensor of general relativity is \emph{covariantly} conserved
\begin{equation}
\tensor{\nabla}{^\nu}\tensor{T}{_\mu_\nu}=\tensor{T}{_\mu_\nu^{;\nu}}=0
\end{equation}
We now ask for the most general tensor satisfying this equation.
\begin{theorem}[Lovelock]
For a fourdimensional space the most general divergence free tensor
$\tensor{A}{_\mu_\nu}$ is given by
\begin{equation}
\tensor{A}{_\mu_\nu}= c_1\tensor{G}{_\mu_\nu}+c_2\tensor{g}{_\mu_\nu}\, .
\end{equation}
Where $\tensor{G}{_\mu_\nu}$ is the \emph{Einstein tensor}
$\tensor{G}{_\mu_\nu}:=\tensor{R}{_\mu_\nu}-\frac{1}{2}\tensor{g}{_\mu_\nu}R$.
\end{theorem}
The Theorem imediatly implies that
\begin{equation}
\tensor{R}{_\mu_\nu}-\frac{1}{2}R\tensor{g}{_\mu_\nu}-\Lambda\tensor{g}{_\mu_\nu}=\kappa\tensor{T}{_\mu_\nu}
\end{equation}
for some constants $\kappa$, $\Lambda$.
Of course we identify $\kappa=\frac{8\pi}{c^2}$ and $\Lambda$ is the
cosmological constant. We can rewrite equation (??) as
\begin{equation}
\tensor{R}{_\mu_\nu}-\frac{1}{2}R\tensor{g}{_\mu_\nu}
=\kappa\left(\tensor{T}{_\mu_\nu}-\frac{\Lambda}{\kappa}\tensor{g}{_\mu_\nu}\right)
\end{equation}
so that the left hand side represents the geometrical part and the right hand
side the matter content.
Wheeler: ``Geometry tells matter how to move, matter tells geometry how to
curve.''
\begin{sidenote}
In the time of the inflation the cosmological constant must have been large.
Since it is small today it has do decay with time
\end{sidenote}
We can identfy $\Lambda$ with an vacuum energy so that. Can we get the Einstein
equations from a variation principle?
\begin{equation}
S\textsubscript{g}=\int\dif x^4 \sqrt{-g}\tilde{\mathcal{L}}
\end{equation}
$\tilde{\mathcal{L}}$ must transform a (scalar) density, therefore we define a
scalar $\sqrt{-g}\mathcal{L}=\mathcal{L}$
\begin{equation}
S\textsubscript{g}=\int\dif x^4 \sqrt{-g}\mathcal{L}
\end{equation}
One can think of various contributions to $\mathcal{L}$
\begin{equation*}
R,\, \square
R,\,\tensor{\nabla}{^\mu}\tensor{\nabla}{^\mu}\tensor{R}{_\mu_\nu},\,
\tensor{R}{_\mu_\nu}\tensor{R}{^\mu^\nu}
,\,\tensor{R}{_\mu_\nu_\sigma_\rho}\tensor{R}{^\mu^\nu^\sigma^\rho}\dots
\end{equation*}
Which have to be contracted so that the resulting quantity becomes a scalar.
We have no contributions of the metric because
$\tensor{g}{_\mu_\nu_{;\sigma}}=0$. From Yang-Mills theory one would expect a
structure
\begin{equation}
\mathcal{L}\sim\tensor{F}{_\mu_\nu}\tensor{F}{^\mu^\nu}
\end{equation}
but $\Gamma$ is not the fundamental field but $g$ is. If we demand that we only
have up to second derivatives of $g$ the only allowed term in the Lagrangian is
$R$.
\begin{sidenote}[On higher derivatives]
If we include higher order derivatives of $g$ in the right way we can make the
resulting theory renormalisable. However we violate unitarity and introduce so
called ghost fields which are associated with the additional degrees of freedom
we get.
\end{sidenote}
\begin{remark}[Dimensions]
In natural units the line element $\dif s^2$ has dimension ${[\dif
s^2]=\textrm{M}^{-2}}$, Since further
${\left[\tensor{x}{^\mu}\right]=\textrm{M}^{-1}}$ the Lagrange density must have Dimension ${\left[\tilde{\mathcal{L}}\right]=\textrm{M}^{4}}$
\end{remark}
This constraint leads to the \emph{Einstein-Hilbert-action}
\begin{equation}
S\textsubscript{EH}=\frac{1}{2\kappa}\int\dif x^4 \sqrt{-g}(R-2\Lambda)
\end{equation}
We now check that its variation indeed reproduces Einsteins equations. To do
so we introduce the formalism of \emph{functional derivative}. Let therefore
$\Phi=\{\varphi,\tensor{A}{^\mu},\Psi,\dots\}$ be a collection of fields.
$F[\Phi]$ a functional.
We define the variation of $F$ as
\begin{equation}
\delta F:=\int \dif x \frac{\delta F}{\delta\Phi^i}\delta\Phi^i\,.
\end{equation}
Typically the functionals are givenn in the form
\begin{equation}
S[\Phi]=\int \dif x L(x,\Phi)\, ,
\end{equation}
where $L$ is some local function.
\begin{equation}
\frac{\delta\tensor{g}{_\rho_\sigma}(x)}{\delta\delta\tensor{g}{_\mu_\nu}(x')}=\tensor*{\delta}{*^\mu*^\nu*_\rho*_\sigma}\delta(x,x')\
\end{equation}
Where
$\tensor*{\delta}{*^\mu*^\nu*_\rho*_\sigma}=\frac{1}{2}\left(\tensor*{\delta}{^\nu_\rho}\tensor*{\delta}{^\mu_\sigma}+\tensor*{\delta}{^\mu_\rho}\tensor*{\delta}{^\nu_\sigma}\right)$
is the unity of the space of symmetric rank two tensors
\begin{remark}
In general $\delta(x,x')\neq \delta(x-x')$
\end{remark}
\section{Introduction of Matter}
In the gravitational kontext we mean be \emph{matter} any non gravitational
fields this include scalar fields $\varphi$, spinor fields $\Psi$, gauge fields
$\tensor{A}{^\mu}$,\dots. We collect all of them in a multivariable $\Phi$
A local action can be written as
\begin{equation}
S\textsubscript{m}[\Phi,g]=\int \dif x^4\sqrt{-g}
L\textsubscript{m}\left(\Phi,\tensor{\nabla}{_\mu}\Phi,g\right)
\end{equation}
$\tensor{g}{^\mu^\nu}$ appears in $L\textsubscript{m}$ because the derivatives
$\tensor{\nabla}{_\mu}, \tensor{\partial}{_\mu}$ must be contracted.
Additionally it enters via $\sqrt{-g}$
\begin{example}[free scalar field in Minkowski space]
\begin{equation}
S\textsubscript{m}=\int \dif x^4 \left(-\frac{1}{2}\tensor{\eta}{^\mu^\nu}
\tensor{\partial}{_\mu}\varphi\tensor{\partial}{_\nu}\varphi-\frac{1}{2}m^2\varphi^2\right)
\end{equation}
The minus sign in front of the partial derivative should come as no surprise
since we have $\tensor{\eta}{^0^0}=-1\, \dot{\varphi}^2>0$. In a non inertial
frame we have to make the usual replacements
\begin{equation}
\tensor{\eta}{_\mu_\nu}\to \tensor{g}{_\mu_\nu}\, , \quad
\tensor{\partial}{_\mu}\to
\tensor{\nabla}{_\mu}\, \quad \dif x^4\to \dif x^4\sqrt{-g}\, .
\end{equation}
\end{example}
\emph{minimal cuppling description} (Kontext???) The action reads as
\begin{equation}
S\textsubscript{m}=\int \dif x^4 \sqrt{-g}\left(-\frac{1}{2}\tensor{g}{^\mu^\nu}
\tensor{\nabla}{_\mu}\varphi\tensor{\nabla}{_\nu}\varphi-\frac{1}{2}m^2\varphi^2\right)
\end{equation}
which is the action for a scalar $\varphi$ in the presence of gravity, i.e. a
dynamical $\tensor{g}{_\mu_\nu}(x)$. The combined action of scalar field and
gravity is given as
\begin{equation}
S[g,\varphi]=S\textsubscript{g}[g]+S\textsubscript{m}[g,\varphi]
\end{equation}
The Euler Lagange Equations in terms of the scalar field read as
\begin{equation}
\frac{\delta S[g,\varphi]}{\delta
\varphi\left(x'\right)}=\frac{\delta S\textsubscript{m}[g,\varphi]}{\delta \varphi\left(x'\right)}
\end{equation}

\begin{equation}
\frac{\delta S\textsubscript{m}[g,\varphi]}{\delta \varphi\left(x'\right)}=\int
\dif x^4\sqrt{-g}\left[-\tensor{g}{^\mu^\nu}
\tensor{\nabla}{_\mu}\varphi\tensor{\nabla}{_\nu}\left(\frac{\delta
\varphi(x)}{\delta \varphi\left(x'\right)}\right)-m^2\varphi\frac{\delta
\varphi(x)}{\delta \varphi\left(x'\right)}\right]
\end{equation}
where we used that the $\delta$ and $\tensor{\nabla}{_\mu}$ commute. Partial
integration yields
\begin{equation}
\begin{split}
\frac{\delta S\textsubscript{m}[g,\varphi]}{\delta \varphi\left(x'\right)}&=\int
\dif
x^4\sqrt{-g}\left(\square_{g}-m^2\right)\varphi\delta(x,x')\\
&=\sqrt{-g}\left(\square_{g}-m^2\right)\varphi
\end{split}
\end{equation}
where $\square_{g}:=\tensor{g}{^\mu^\nu}
\tensor{\nabla}{_\mu}\tensor{\nabla}{_\nu} $ is the
\emph{Laplace–Beltrami operator}, a generalisation of the ordinary laplacian to
curved space.
Demanding that the variation with respect to $\varphi$ vanishes implies the
\emph{Klein-Gordon equation}
\begin{equation}
\left(\square_g-m^2\right)\varphi=0
\end{equation}
Variing with respect to the metric $\tensor{g}{_\mu_\nu}$
\begin{equation}
\frac{\delta S[g,\varphi]}{\delta
\tensor{g}{_\mu_\nu}\left(x'\right)}=
\frac{\delta S\textsubscript{g}[g]}{\delta
\tensor{g}{_\mu_\nu}\left(x'\right)}
+\frac{\delta S\textsubscript{m}[g,\varphi]}{\delta
\tensor{g}{_\mu_\nu}\left(x'\right)}
=\frac{\sqrt{-g}}{2\kappa}\left(\tensor{G}{^\mu^\nu}+\Lambda\tensor{g}{^\mu^\nu}
\right)+\frac{\delta S\textsubscript{m}[g,\varphi]}{\delta
\tensor{g}{_\mu_\nu}\left(x'\right)}
\end{equation}
This makes it convenient to define the energy-momentum tensor (Vorzeichen????)
\begin{equation}
\tensor{T}{^\mu^\nu}:=\frac{2}{\sqrt{-g}}\frac{\delta
S\textsubscript{m}[g,\varphi]}{\delta \tensor{g}{_\mu_\nu}\left(x'\right)}\,.
\end{equation}
\begin{remark}
Attention $\delta_g
\tensor{g}{^\mu^\nu}=-\tensor{g}{^\rho^{(\mu}}\tensor{g}{^{\nu)}^\sigma}\delta_g
\tensor{g}{_\mu_\nu}$
\end{remark}
We can now proceed in calculating the quantity we have ust introduced for a
scalar field
\begin{equation}
\begin{split}
\frac{\delta
S\textsubscript{m}[g,\varphi]}{\delta \tensor{g}{_\mu_\nu}\left(x'\right)}
&=\int \dif x^4 \frac{\delta\sqrt{-g}}{\delta
\tensor{g}{_\mu_\nu}}\left(-\frac{1}{2}\tensor{g}{^\mu^\nu}
\tensor{\nabla}{_\mu}\varphi\tensor{\nabla}{_\nu}\varphi-\frac{1}{2}m^2\varphi^2\right)\\
&\phantom{=}+
\sqrt{-g}\left(-\frac{1}{2}\tensor{g}{^\alpha^\rho}\tensor{g}{^\beta^\sigma} \tensor{\nabla}{_\alpha}\varphi\tensor{\nabla}{_\beta}\varphi\frac{\delta
\tensor{g}{_\rho_\sigma}}{\delta \tensor{g}{_\mu_\nu}}\right)\\
&=\frac{1}{2}\int \dif x^4
\sqrt{-g}\left(-\frac{1}{2}\tensor{g}{^\mu^\nu}\tensor{\nabla}{_\rho}\varphi\tensor{\nabla}{^\rho}\varphi-\frac{1}{2}\tensor{g}{^\mu^\nu}m^2\varphi^2
+\tensor{\nabla}{^\mu}\varphi\tensor{\nabla}{^\nu}\varphi\right)\delta(x,x')\\
&=\frac{1}{2}\sqrt{-g}\left(-\frac{1}{2}\tensor{g}{^\mu^\nu}\tensor{\nabla}{_\rho}\varphi\tensor{\nabla}{^\rho}\varphi-\frac{1}{2}\tensor{g}{^\mu^\nu}m^2\varphi^2
+\tensor{\nabla}{^\mu}\varphi\tensor{\nabla}{^\nu}\varphi\right)
\end{split}
\end{equation}
So that
\begin{equation}
\tensor{T}{^\mu^\nu}(\varphi)
=-\frac{1}{2}\tensor{g}{^\mu^\nu}\tensor{\nabla}{_\rho}\varphi\tensor{\nabla}{^\rho}\varphi
+\tensor{\nabla}{^\mu}\varphi\tensor{\nabla}{^\nu}\varphi
-\frac{1}{2}\tensor{g}{^\mu^\nu}m^2\varphi^2
\end{equation}
As we have noticed, the Einstein Tensor is covariantly conserved (contracted
Bianci identities). The Einstein equation then implies that also
$\tensor{T}{^\mu^\nu_{;\nu}}=0$ this can be checked for the given Tensor
\begin{equation}
\begin{split}
\tensor{\nabla}{_\mu}\tensor{T}{^\mu^\nu}
&=\tensor{g}{^\mu^\nu}\tensor{\nabla}{_\mu}\tensor{\nabla}{_\rho}\varphi\tensor{\nabla}{^\rho}\varphi+\square\varphi\tensor{\nabla}{^\nu}\varphi+\tensor{\nabla}{^\mu}\varphi\tensor{\nabla}{_\mu}\tensor{\nabla}{^\nu}\varphi
-\tensor{g}{^\mu^\nu}m^2\varphi\tensor{\nabla}{_\mu}\varphi\\
&=\tensor{\nabla}{^\nu}\varphi\left(\square-m^2\right)\varphi\\
&=0
\end{split}
\end{equation}
Where the last equality holds because $\varphi$ satisfies the Klein-Gordon
equation. The Einstein equations are 10 quasi linear, i.e. the highest order
derivative apears only linear, differential equations for the metric field
$\tensor{g}{_\mu_\nu}$. Strictly speaking the Einstein equations are
\emph{nonlinear}.
\begin{sidenote}
If you substract the constrains imposed by the Bianci identities you end with 2
DOFs, which are associated with the polarisation states of the graviton.
\end{sidenote}
How do we find a solution to this equations?
\begin{enumerate}
  \item Prescribe $\tensor{T}{_\mu_\nu}$. This is only possible for high
  symmetry problems, e.g. the Schwazschild solution and the cosmological
  solutions (Friedmans equations)
  \item Assume $\tensor{g}{_\mu_\nu}$, then compute $\tensor{T}{_\mu_\nu}$ and
  (try!) to interprete this.
\end{enumerate}
Intrinsic vs extrinsic curvature ????
\subsection{ADM-Decomposition}
(Bild)
Erklärung einschieben was passiert!!!
Assume we know (??)
\begin{itemize}
  \item $\tensor{g}{_\mu_\nu}$ on $\Sigma_{t_0}$
  \item $\tensor{g}{_\mu_\nu_{;j}}$, $\tensor{g}{_\mu_\nu_{;0}}$ on
  $\Sigma_{t_0}$ also allowed
\end{itemize}
In vakuum the field equation are
\begin{equation}
0=G=R-2R\implies\tensor{R}{_\mu_\nu}=0
\end{equation}
The resulting set of equations is (prüfen!!!)
\begin{align}
0&=\tensor{R}{_0_0}=-\frac{1}{2}\tensor{g}{^i^j}\tensor{g}{_i_j_{,00}}+\tensor{M}{_0_0}\\
0&=\tensor{R}{_0_i}=-\frac{1}{2}\tensor{g}{^0^j}\tensor{g}{_i_j_{,00}}+\tensor{M}{_0_i}\\
0&=\tensor{R}{_i_j}=-\frac{1}{2}\tensor{g}{^0^0}\tensor{g}{_i_j_{,00}}+\tensor{M}{_i_j}
\end{align}
Where $\tensor{M}{_\mu_\nu}$ is a rest term containing lower order derivaties.
This shows that ther are no second order time derivatives of
$\tensor{g}{_0_\mu}$. We have 10 euations and 6 undetetrmined functions. The
DOFs can be used for a coordinate transformation, so that
$\tensor{g}{_0_\mu_{,00}}=0$ on $\Sigma_{t_0}$. This is allways possible but we
will not proof this. It can be further shown, by means of the contracted Bianci
identities, that this implies $\tensor{g}{_0_\mu_{,00}}=0$ on \emph{all}
hypersurfaces $\Sigma_{t}$.
\begin{equation}
\tensor{\partial}{_0}\tensor{G}{^0^\nu}=
\tensor{\partial}{_i}\tensor{G}{^i^\nu}
-\cSym{\nu}{0}{\lambda}\tensor{G}{^\lambda^\nu}
-\cSym{0}{\nu}{\lambda}\tensor{G}{^\mu^\lambda}
\end{equation}
Hier muss man nochmal schauen das macht noch nicht viel sinn\ldots\ldots.
We have the freedom to choose four coodinates
\begin{equation}
\tensor{x}{^{\mu^\prime}}=\tensor{f}{^{\mu^\prime}}\left(\tensor{x}{^\mu}\right)
\end{equation}
One typical choice is the \emph{harmonic\footnote{a function $f$ satisfying
$\square f = 0$ is called harmonic.} gauge}
\begin{equation}
\square\tensor{x}{^\mu}=0\,.
\end{equation}
\begin{equation}
\begin{split}
\square\tensor{x}{^\mu}&=g^{-\nicefrac{1}{2}}\tensor{\partial}{_\rho}\left(g^{\nicefrac{1}{2}}\tensor{g}{^\rho^\sigma}\tensor{\partial}{_\sigma}\tensor{x}{^\mu}\right)\\
&=g^{-\nicefrac{1}{2}}\tensor{\partial}{_\rho}\left(g^{\nicefrac{1}{2}}\tensor{g}{^\rho^\sigma}\tensor{\delta}{_\sigma^\mu}\right)\\
&=g^{-\nicefrac{1}{2}}\tensor{\partial}{_\rho}\left(g^{\nicefrac{1}{2}}\tensor{g}{^\rho^\mu}\right)\\
\end{split}
\end{equation}
(Formel für $\square$ die hier genutzt wurde referenzieren)
The harmonic gauge is therefore equivalent to
\begin{equation}
\tensor{\partial}{_\rho}\left(g^{\nicefrac{1}{2}}\tensor{g}{^\rho^\mu}\right)=0\,
.\\
\end{equation}
The equation can be sorted in spatial and time components and derive by the
zero component, so that
\begin{equation}
\tensor*{\partial}{*_0^2}\left(g^{\nicefrac{1}{2}}\tensor{g}{^0^\mu}\right)
=
-\tensor{\partial}{_i}\left[\tensor{\partial}{_0}\left(g^{\nicefrac{1}{2}}\tensor{g}{^0^\mu}\right)\right]\,
,\end{equation}
which fixes the second order time derivatives of the relevant components
$\tensor{g}{^0^\mu}$. Therefore now the time evolution can solved. (Fehlt hier
was???)
\subsubsection{Degrees of freedom}
\begin{itemize}
  \item[\color{section_color}\textsf{\textbf{10}}]
  componnents for every spacetime point from the symmetric $\tensor{g}{_\mu_\nu}(x)$
  \item[\color{section_color}\textsf{\textbf{-4}}] from the
  constraint equation $\tensor{G}{_\mu_\nu^{;\nu}}=0$
  \begin{itemize}
    \item
    $\tensor{G}{^0^0}=\kappa \tensor{T}{^0^0}$ ensures that the
    evolution is independent of the choice of spatial coordinates on
    $\Sigma_{t_0}$.
    \item
    $\tensor{G}{^i^0}=\kappa \tensor{T}{^i^0}$ ensures that the time
    evolution is independent of the way we foliated sacetime into spacial
    hypersurfaces $\Sigma_{t}.    $
  \end{itemize}
  \item[\color{section_color}\textsf{\textbf{-4}}] due to the freedom
  to choose coordinates (i.e. a gauge).
\end{itemize}
We are left with 2 physical degrees of freedom which may be interpreted as the
polarisation states of the graviton field.
\subsubsection{Comparison with electrodynamics in flat spacetime}
In electrodynamics instead of the einstein equations we have the field equations
for the four potential $\tensor{A}{_\mu}$:
\begin{equation}
\square\tensor{A}{_\mu}-\partial_\mu\left(\partial_\nu\tensor{A}{^\nu}
\right)=0\,.
\end{equation}
As we did for the gravitational field we take a look
at the 0 component. We find
\begin{equation}
\begin{split}
0&=-\partial_0^2\tensor{A}{_0}+\partial_i\partial^i\tensor{A}{_0}
-\partial_0\left(-\partial_0\tensor{A}{_0}+\partial_iA^i\right)\\
&= \partial_i\partial^i\tensor{A}{_0}-\partial_0\partial_iA^i
\end{split}
\end{equation}
This equation is equivalent to $\nabla\vec{E}=0$ and the bianci identities.
So once again $\tensor{A}{_0}$ is \emph{not} determined by the dynamical
evolution equation because there is no second order time derivative analogous to
$\tensor{g}{_0_0}$. Since $\tensor{A}{_0}$ is not determinend and cannot be
specified on initial time slice. This reflects some internal redundancy namely
gauge invariance of the theory. For any scalar function $\Lambda$
\begin{equation}
\tensor{A}{_\mu}\to\tensor*{A}{*_\mu^\prime}=
\tensor{A}{_\mu}+\partial_\mu\Lambda
\end{equation}
leaves the physics invariant. It is trivial to check that
the field strength tensor $\tensor{F}{_\mu_\nu}=\partial_\mu
\tensor{A}{_\nu}-\partial_\nu\tensor{A}{_\mu}$ stays invariant. Perhaps more
interesting the field equation is also gauge invariant:
\begin{equation}
\begin{split}
\square\tensor*{A}{*_\mu^\prime}-\partial_\mu\left(\partial_\nu\tensor*{A}{*^\nu^\prime}
\right)
&=
\square\tensor{A}{_\mu}+\square\partial_\mu\Lambda-\partial_\mu\left(\partial_\nu\tensor{A}{^\nu}
\right)-\partial_\mu\square\Lambda\\
&=
\square\tensor{A}{_\mu}-\partial_\mu\left(\partial_\nu\tensor{A}{^\nu}
\right)\,.
\end{split}
\end{equation}
Thus if $\tensor{A}{_\mu}$ is a solution to the field
equation $\tensor*{A}{*_\mu^\prime}$ is and therefore both are physically
undistinguishable. We can also fix a gauge for example the \emph{Lorentz gauge}:
\begin{equation}
\partial_\mu\tensor{A}{^\mu}=0
\end{equation}
If we derive this by the 0 component we get
\begin{equation}
\partial_{0}^2\tensor{A}{^0}=-\partial_i\partial_0\tensor{A}{^i}\, ,
\end{equation}
so as with $\tensor{g}{_0_0}$ the evolution of the 0 component is now related to
the other components. There is still one residual gauge condition, namely we can
still transform
\begin{equation}
\tensor{A}{_\mu}\to\tensor*{A}{*_\mu^\prime}=
\tensor{A}{_\mu}+\partial_\mu\Gamma\, ,
\end{equation}
but to keep the gauge we have also to demand that $\square\Gamma=0$.
Again we count the DOFs:
\begin{itemize}
  \item[\color{section_color}\textsf{\textbf{4}}] components of the potential
  $\tensor{A}{_\mu}$.
  \item[\color{section_color}\textsf{\textbf{-1}}] from constraint
  $\nabla\vec{E}=0$.
  \item[\color{section_color}\textsf{\textbf{-1}}] from gauge freedom
  $\Lambda$.
\end{itemize}
This leaves two physical degrees of freedom, the polarisation states of a
photon.
\begin{remark}
As we have have seen there is a direct correspondence between the gauge freedom
in electrodynamics and the freedom of choice of coordinates of coordinates in
GR.
\end{remark}

\chapter{The Energy Momentum Tensor}
In special relativity we have seen that the energy momentum tensor
$\tensor{T}{^\mu^\nu}$ is conserved or divergencefree respectively, i.e.
\begin{equation}
\tensor{\partial}{_\mu}\tensor{T}{^\mu^\nu}=0\,.\label{eq:EMcons}
\end{equation}
The problem we are faced in general relativity is the very definition of local
energy. Because gravity shurely contributes to the energy, a problem arises
as we can always transform to local flat space.
We start by revisiting the example of dust
\begin{example}[Dust]
\begin{equation}
\tensor{T}{^\mu^\nu}=\rho_0\tensor{u}{^\mu}\tensor{u}{^\nu}\,.
\end{equation}
In special relativity:
${\tensor{u}{^\mu}=\od{\tensor{x}{^\mu}}{\tau}=\gamma(1,\vec{v})\transpose}$,
${\tensor{T}{^0^0}=\rho_0\left(\od{t}{\tau}\right)^2}=\gamma^2\rho_0:=\rho$.
Where $\rho$ is the density with respect to an observer at rest. For the Volume
we have $V=\gamma^{-1}V_0$. For the Energy $E=\gamma\omega_0$. Then the density
is given by $\rho=\frac{E}{V}=\gamma^2\rho_0$. The conservation of
$\tensor{T}{^0^\nu}$ implies
\begin{equation}
\begin{split}
0&=\tensor{T}{^0^\nu_{,\nu}}\\
&=\tensor{T}{^0^0_{,0}}+\tensor{T}{^0^i_{,i}}\\
&=\tensor{\partial}{_t}\left(\rho_0\gamma^2\right)
+\tensor{\partial}{_i}\left(\rho_0\gamma^2\tensor{v}{^i}\right)\\
&=\tensor{\partial}{_t}\rho
+\tensor{\partial}{_i}\left(\rho \tensor{v}{^i}\right)\\
&=\dot{\rho}
+\boldsymbol{\nabla}\left(\rho \vec{v}\right)\,,
\end{split}
\end{equation}
the \emph{continuity equation}. For the remaining spatial components we get
\begin{equation}
\begin{split}
0&=\tensor{T}{^i^\nu_{,\nu}}\\
&=\tensor{T}{^i^0_{,0}}+\tensor{T}{^j^i_{,i}}\\
&=\dot{\rho}\tensor{v}{^i}
+\rho\tensor{\dot{v}}{^i}
+\tensor{v}{^i}\tensor{\partial}{_j}\left(\rho\tensor{v}{^j}\right)
+\tensor{\dot{v}}{^j}\rho\tensor{\partial}{_j}\tensor{v}{^i}\\
&=\tensor{v}{^i}\left[\dot{\rho}+\tensor{\partial}{_j}\left(\rho\tensor{\dot{v}}{^j}\right)\right]
+\rho\left(\tensor{\dot{v}}{^i}+\tensor{\dot{v}}{^j}\tensor{\partial}{_j}\tensor{v}{^i}\right)\\
&=\rho\left(\tensor{\dot{v}}{^i}+\tensor{\dot{v}}{^j}\tensor{\partial}{_j}\tensor{v}{^i}\right)\,,
\end{split}
\end{equation}
the \emph{Euler equation} for vanishing pressure (which was the key assumption
for dust). It is natural to generalize equation \eqref{eq:EMcons} to curved
space
\begin{equation}
\tensor{T}{^\mu^\nu_{;\nu}}=0\,.
\end{equation}
In expanded form
\begin{equation}
\begin{split}
0
&=\tensor{\nabla}{_\nu}\left(\rho_0\tensor{u}{^\mu}\tensor{u}{^\nu}\right)\\
&=\tensor{{\rho_0}}{_{;\nu}}\tensor{u}{^\mu}\tensor{u}{^\nu}
+\rho_0\tensor{u}{^\mu_{;\nu}}\tensor{u}{^\nu}
+\rho_0\tensor{u}{^\mu}\tensor{u}{^\nu_{;\nu}}\\
&=\tensor{u}{^\mu}\left(\rho_0\tensor{u}{^\nu_{;\nu}}+\tensor{{\rho_0}}{_{;\nu}}\tensor{u}{^\nu}\right)
+\rho_0\tensor{u}{^\mu_{;\nu}}\tensor{u}{^\nu} \label{eq:DustEnCons}
\end{split}
\end{equation}
We multiply both sides with $\tensor{u}{_\mu}$
\begin{equation}
\begin{split}
0
&=-\left(\rho_0\tensor{u}{^\nu_{;\nu}}+\tensor{{\rho_0}}{_{;\nu}}\tensor{u}{^\nu}\right)
+\rho_0\tensor{u}{^\nu}\tensor{u}{_\mu}\tensor{u}{^\mu_{;\nu}}\\
&=-\left(\rho_0\tensor{u}{^\nu_{;\nu}}
+\tensor{{\rho_0}}{_{;\nu}}\tensor{u}{^\nu}\right)
\end{split}
\end{equation}
If we plugg this back into equation \eqref{eq:DustEnCons} we get
\begin{equation}
\begin{split}
0&=\tensor{u}{^\nu}\tensor{\nabla}{_\nu}\tensor{u}{^\mu}\\
&=\tensor{u}{^\nu}\tensor{\partial}{_\nu}\tensor{u}{^\mu}
+\tensor{u}{^\nu}\cSym{\mu}{\nu}{\rho}\tensor{u}{^\rho}\\
&=\dod{\tensor{x}{^\nu}}{\tau}\dpd{\tensor{u}{^\mu}}{\tensor{x}{^\nu}}
+\cSym{\mu}{\nu}{\rho}\dod{\tensor{x}{^\nu}}{\tau}\dod{\tensor{x}{^\nu}}{\tau}\\
&=\dpd[2]{\tensor{x}{^\mu}}{\tau}
+\cSym{\mu}{\nu}{\rho}\dod{\tensor{x}{^\nu}}{\nu}\dod{\tensor{x}{^\nu}}{\tau}\\
\end{split}
\end{equation}
The geodesic equation \eqref{eq:geodeq}
\end{example}
\begin{remark}
This is a diference between electrodynamics and general relativity; dust moves
on geodesics, i.e. the path is determined by the field equations alone. In contrast
in electrodynamics an additional Force (Lorentz force) has to be
\emph{postulated} to describe the motion of test particles. The case is not
settled however e.g. it is unclear wheater the paths of spin particles is also
determined by the field equations.
\end{remark}
\chapter{Linearized theory and Newtonian limit}
\section{Linearized theory}
Consider a weak gravitational field. Then we can split the full spacetime metric $\tensor{g}{_\mu_\nu}(x)$ into two parts.
\begin{definition}[Linearization of the metric field.]
    \begin{equation}
        \tensor{g}{_\mu_\nu}(x) = \tensor{\eta}{_\mu_\nu} +
        \tensor{h}{_\mu_\nu}(x) + \landauO(h^2) \, .
    \end{equation}
\end{definition}
Thereby $\tensor{h}{_\mu_\nu}$ is the flat, constant ``background'' metric of Minkowski
space, i.e.\ there is no gravitational field present.
The field $h_{\mu\nu}(x)$ can be interpreted as a perturbation on the fixed background $\eta_{\mu\nu}$.
One can identify a spin-2 particle, the so-called \emph{graviton}, with the excitations (quantized fluctuations) of this field.
Because only the linear order of $h$ is considered, the nonlinearity of Einstein's equations is lost.
We can raise and lower indices with $\tensor{h}{_\mu_\nu}$ and
$\tensor{h}{^\mu^\nu}$.

\begin{remark}
This works only for a weak gravitational field, since a strong gravitational field produces a strong back reaction of ``matter''
on the geometry, which follows from the nonlinearity of Einstein's equations.
Exactly this back reaction is neglected in the linearized theory.
\end{remark}

\subsection{Derivation of the linearized Einstein's equations}
In the following we neglect all terms with $\landauO(h^2)$.
Our goal is to express Einstein's field equations in the linearized approximation.
For this we need to calculate the Christoffel symbols, the Riemann tensor, the Ricci tensor and the Ricci scalar.

\subsubsection*{Christoffel symbols}
\begin{equation}
    \csym{\mu}{\nu}{\varrho} = \frac{1}{2} \left( \tensor{h}{_\mu_\varrho_,_\nu} + \tensor{h}{_\nu_\varrho_,_\mu}
    - \tensor{h}{_\mu_\nu_,_\varrho} \right) + \landauO(h^2) \, .
\end{equation}
Christoffel symbols of the second kind:
\begin{equation}
    \cSym{\varrho}{\mu}{\nu} = g^{\varrho\sigma} \csym{\mu}{\nu}{\sigma} = \eta^{\varrho\sigma} \csym{\mu}{\nu}{\sigma} + \landauO(h^2)
    = \frac{1}{2} \left( \tensor{h}{_\mu^\varrho_,_\nu} + \tensor{h}{_\nu^\varrho_,_\mu} - \tensor{h}{_\mu_\nu^,^\varrho} \right) + \landauO(h^2) \, .
\end{equation}
\subsubsection*{Riemann tensor}
The Riemann tensor can be calculated to
\begin{equation}
    \begin{split}
        \tensor{R}{^\varrho_\sigma_\mu_\nu}
        &= \partial_\mu \cSym{\varrho}{\nu}{\sigma} - \partial_\nu \cSym{\varrho}{\mu}{\sigma}
        + \underbrace{\cSym{\varrho}{\mu}{\lambda} \cSym{\lambda}{\nu}{\sigma} - \cSym{\varrho}{\nu}{\lambda} \cSym{\lambda}{\mu}{\sigma}}_{\landauO(h^2)} \\
        &= \frac{1}{2} \left( \tensor{h}{_\nu^\varrho_,_\sigma_\mu} +
        {\tensor{h}{_\sigma^\varrho_,_\nu_\mu}} - \tensor{h}{_\nu_\sigma^,^\varrho_\mu} - \tensor{h}{^\varrho_\mu_,_\sigma_\nu} -
        {\tensor{h}{_\sigma^\varrho_,_\mu_\nu}} +
        \tensor{h}{_\mu_\sigma^,^\varrho_\nu} \right) + \landauO(h^2) \\
        &= \frac{1}{2} \left( \tensor{h}{_\nu^\varrho_,_\sigma_\mu} - \tensor{h}{_\nu_\sigma^,^\varrho_\mu}
        - \tensor{h}{^\varrho_\mu_,_\sigma_\nu} +
        \tensor{h}{_\mu_\sigma^,^\varrho_\nu} \right) + \landauO(h^2) \, .
    \end{split}
\end{equation}
By contracting we get the Ricci tensor
\begin{equation}
    \tensor{R}{_\sigma_\nu} = \tensor{R}{^\varrho_\sigma_\varrho_\nu}
    = \frac{1}{2} \left( \tensor{h}{_\nu^\varrho_,_\sigma_\varrho} - \tensor{h}{_\nu_\sigma^,^\varrho_\varrho}
    - \tensor{h}{_,_\sigma_\nu} + \tensor{h}{_\varrho_\sigma^,^\varrho_\nu} \right) + \landauO(h^2)\, ,
\end{equation}
where for convenience the trace of $h$ is denoted with $h\coloneqq
h_{\mu\nu}\eta^{\mu\nu}$. Lastly the Ricci scalar is given by
\begin{equation}
    R = g^{\sigma\nu}R_{\sigma\nu} = \eta^{\sigma\nu}R_{\sigma\nu} + \landauO(h^2)
    = \tensor{h}{^\sigma^\nu_,_\sigma_\nu} - \tensor{h}{_,_\sigma^\sigma} + \landauO(h^2)
\end{equation}
We define
\begin{equation}
    \overline{h}_{\mu\nu} \coloneqq h_{\mu\nu} - \frac{1}{2} \eta_{\mu\nu}h\,.
\end{equation}
The trace is given by
\begin{equation}
    \overline{h} \coloneqq  \overline{h}_{\mu\nu}\eta^{\mu\nu} = h -   
    \frac{h}{2} = \frac{h}{2}\,.
\end{equation}
If we repeat the procedure we arrive at the initial metric:
\begin{equation}
    \overline{\overline{h}}_{\mu\nu} = \overline{h}_{\mu\nu} - \frac{1}{2}
    \eta_{\mu\nu}\overline{h} = \overline{h}_{\mu\nu} + \frac{1}{2}
    \eta_{\mu\nu}h= h_{\mu\nu}\,.
\end{equation}
\subsubsection{Linearized Einstein tensor \texorpdfstring{$G_{\mu\nu}$}{Gmunu} in terms of \texorpdfstring{$\overline{h}_{\mu\nu}$}{hbarmunu}}
The linearized Einstein Tensor is given as
\begin{equation}
    \begin{split}
        G_{\mu\nu}^{\text{(L)}} =\ & R_{\mu\nu}^{\text{(L)}} - \frac{1}{2} \eta_{\mu\nu} R^{\text{(L)}} \\
        =\ & \frac{1}{2} \partial_\mu \partial_\varrho \tensor{h}{_\nu^\varrho} + \frac{1}{2} \partial_\nu \partial_\varrho \tensor{h}{_\mu^\varrho}
        - \frac{1}{2} \Box h_{\mu\nu} - \frac{1}{2}
        \partial_{\mu}\partial_\nu h-\frac{1}{2}
        \eta_{\mu\nu}\partial_\varrho\partial_\sigma h^{\varrho\sigma} + \frac{1}{2} \eta_{\mu\nu}\Box h \\
        =\ & \frac{1}{2} \partial_\mu\partial_\varrho \tensor{\overline{h}}{_\nu^\varrho}
        - {\frac{1}{4}\partial_\mu\partial_\nu\overline{h}}
        + \frac{1}{2} \partial_\nu\partial_\varrho\tensor{\overline{h}}{_\mu^\varrho}
        - {\frac{1}{4}\partial_\nu\partial_\mu\overline{h}} - \frac{1}{2}\Box\overline{h}_{\mu\nu} \\
        & + {\frac{1}{2}\eta_{\mu\nu}\Box\overline{h}} + {\frac{1}{2}\partial_\mu\partial_\nu\overline{h}}
        - \frac{1}{2}\eta_{\mu\nu}\partial_\varrho\partial_\sigma\overline{h}^{\varrho\sigma}
        + {\frac{1}{4}\eta_{\mu\nu}\Box\overline{h}} - {\frac{1}{2}\eta_{\mu\nu}\Box\overline{h}} \\
        =\ & -\frac{1}{2} \Box \overline{h}_{\mu\nu} + \partial_\varrho \tensor{\partial}{_(_\mu}\tensor{\overline{h}}{_\nu_)^\varrho}
        - \frac{1}{2} \eta_{\mu\nu}\partial_\varrho\partial_\sigma
        \overline{h}^{\varrho\sigma}\,,
    \end{split}
\end{equation} 
% \begin{equation}
%     \begin{split}
%         G_{\mu\nu}^{\text{(L)}} =\ & R_{\mu\nu}^{\text{(L)}} - \frac{1}{2} \eta_{\mu\nu} R^{\text{(L)}} \\
%         =\ & \frac{1}{2} \partial_\mu \partial_\varrho \tensor{h}{_\nu^\varrho} + \frac{1}{2} \partial_\nu \partial_\varrho \tensor{h}{_\mu^\varrho}
%         - \frac{1}{2} \Box h_{\mu\nu} - \frac{1}{2}
%         \partial_{\mu}\partial_\nu h-\frac{1}{2}
%         \eta_{\mu\nu}\partial_\varrho\partial_\sigma h^{\varrho\sigma} + \frac{1}{2} \eta_{\mu\nu}\Box h \\
%         =\ & \frac{1}{2} \partial_\mu\partial_\varrho \tensor{\overline{h}}{_\nu^\varrho}
%         - \mathunderline{blue}{\frac{1}{4}\partial_\mu\partial_\nu\overline{h}}
%         + \frac{1}{2} \partial_\nu\partial_\varrho\tensor{\overline{h}}{_\mu^\varrho}
%         - \mathunderline{blue}{\frac{1}{4}\partial_\nu\partial_\mu\overline{h}} - \frac{1}{2}\Box\overline{h}_{\mu\nu} \\
%         & + \mathunderline{green}{\frac{1}{2}\eta_{\mu\nu}\Box\overline{h}} + \mathunderline{blue}{\frac{1}{2}\partial_\mu\partial_\nu\overline{h}}
%         - \frac{1}{2}\eta_{\mu\nu}\partial_\varrho\partial_\sigma\overline{h}^{\varrho\sigma}
%         + \mathunderline{green}{\frac{1}{4}\eta_{\mu\nu}\Box\overline{h}} - \mathunderline{green}{\frac{1}{2}\eta_{\mu\nu}\Box\overline{h}} \\
%         =\ & -\frac{1}{2} \Box \overline{h}_{\mu\nu} + \partial_\varrho \tensor{\partial}{_(_\mu}\tensor{\overline{h}}{_\nu_)^\varrho}
%         - \frac{1}{2} \eta_{\mu\nu}\partial_\varrho\partial_\sigma \overline{h}^{\varrho\sigma} \\
%         \overset{!}{=}\ & \kappa T_{\mu\nu}
%     \end{split}
% \end{equation}


%TODO introduce symmetration brackets somewhere
with the (linearized) d'Alembert operator
\begin{equation}
    \Box^{\text{(L)}} = \Box = \partial_\mu\partial_\nu
    \eta^{\mu\nu}=\partial_\mu\partial^\mu\,.
\end{equation}
\begin{definition}[Linearized Einstein equations]
    \begin{equation}
        \label{eq:lineinsteineqs}
        -\frac{1}{2} \Box \overline{h}_{\mu\nu} + \partial_\varrho \tensor{\partial}{_(_\mu}\tensor{\overline{h}}{_\nu_)^\varrho}
        - \frac{1}{2} \eta_{\mu\nu}\partial_\varrho\partial_\sigma \overline{h}^{\varrho\sigma} = \kappa T_{\mu\nu}
    \end{equation}
\end{definition}

\subsubsection{Gauge transformations}
Usually field equations are in the form of
\begin{equation}
    \Box \text{``field''} = \text{``source''}\,.
\end{equation}
Equation~\eqref{eq:lineinsteineqs} can be written in this form:
\begin{equation}
    \underbrace{\Box \overline{h}_{\mu\nu}}_{\Box\text{``field''}}
    \underbrace{- 2 \partial_\varrho \tensor{\partial}{_(_\mu} \tensor{\overline{h}}{_\nu_)^\varrho}
    + \eta_{\mu\nu}\partial_\varrho \overline{h}^{\varrho\sigma}}_{\text{ensures gauge invariance of equation}}
    = \underbrace{-2\kappa T_{\mu\nu}}_{\text{``source''}}
\end{equation}

We are now considering infinitesimal diffeomorphisms, which are given by affine
transformations
\begin{equation}
    x^\mu = x'^\mu + \xi^\mu(x'^\mu), \qquad \xi^\mu \ll 1
\end{equation}
In the following we neglect terms with $\landauO(\xi^2)$, $\landauO(\xi h)$, and
$\landauO(h^2)$ and higher order terms, which we denote by $\landauO$.
The transformed metric reads
\begin{equation}
    \begin{split}
        \eta_{\mu\nu} + h'_{\mu\nu}(x') &= g'_{\mu\nu} \\
        &= \frac{\partial x^\varrho}{\partial x'^\mu} \frac{\partial x^\sigma}{\partial x'^\nu} g_{\varrho\sigma}(x) \\
        &= \frac{\partial \left( x'^\varrho + \xi^\varrho \right)}{\partial x'^\mu}
        \frac{\partial \left( x'^\sigma + \xi^\sigma \right)}{\partial x'^\nu}
        \left( \eta_{\varrho\sigma} + h_{\varrho\sigma}(x) \right) + \landauO \\
        &= \left( \tensor{\delta}{_\mu^\varrho} + \tensor{\xi}{^\varrho_,_\mu} \right)
        \left( \tensor{\delta}{_\nu^\sigma} + \tensor{\xi}{^\sigma_,_\nu} \right)
        \left( \eta_{\varrho\sigma} + h_{\varrho\sigma}(x) \right) + \landauO \\
        &= \left( \tensor{\delta}{_\mu^\varrho} + \tensor{\xi}{^\varrho_,_\mu} \right)
        \left( \eta_{\varrho\nu} + h_{\varrho\nu} + \tensor{\xi}{_\varrho_,_\nu}
        \right) + \landauO \\
        &= \eta_{\mu\nu} + h_{\mu\nu} + \tensor{\xi}{_\mu_,_\nu} +
        \tensor{\xi}{_\nu_,_\mu} + \landauO\,.
    \end{split}
\end{equation}
The perturbation $h_{\mu\nu}$ therefore transforms  under infinitesimal
diffeomorphisms in the following way
\begin{equation}
    \begin{split}
        h'_{\mu\nu}(x) &= h_{\mu\nu}(x) + \tensor{\xi}{_\mu_,_\nu} + \tensor{\xi}{_\nu_,_\mu} \\
        &= h_{\mu\nu}(x) + \left(\liedif{\xi}{\eta} \right)_{\mu\nu}\,.
    \end{split}
\end{equation}
\begin{definition}[Lie derivative]
    The Lie derivative off a tensor field $T$ with $k$ contravariant and $l$
    covariant indices along the vector $\xi$ is defined as
    \begin{equation}
        \begin{split}
            \left( \liedif{\xi}{T} \right)^{\alpha_1\ldots\alpha_k}_{\beta_1\ldots\beta_l}
            \coloneqq \xi^\mu \partial_\mu T^{\alpha_1\ldots\alpha_k}_{\beta_1\ldots\beta_l}
            & - \left( \partial_\mu \xi^{\alpha_1} \right) T^{\mu\alpha_2\ldots\alpha_k}_{\beta_1\ldots\beta_l} - \ldots
            - \left( \partial_\mu \xi^{\alpha_k} \right) T^{\alpha_1\ldots\alpha_{k-1}\mu}_{\beta_1\ldots\beta_l} \\
            & + \left( \partial_{\beta_1} \xi^\mu \right) T^{\alpha_1\ldots\alpha_k}_{\mu\beta_2\ldots\beta_l} + \ldots
            +  \left( \partial_{\beta_l} \xi^\mu \right)
            T^{\alpha_1\ldots\alpha_k}_{\beta_1\ldots\beta_{l-1}\mu}\,.
        \end{split}
    \end{equation}
\end{definition}
Therefore
\begin{equation}
    \left( \liedif{\xi}{\eta} \right)_{\mu\nu} = \underbrace{\xi^\varrho\partial_\varrho\eta_{\mu\nu}}_{=0}
    + \tensor{\xi}{_\mu_,_\nu} + \tensor{\xi}{_\nu_,_\mu} =
    \tensor{\xi}{_\mu_,_\nu} + \tensor{\xi}{_\nu_,_\mu}\,.
\end{equation}
If the derivative of a metric vanishes for a given $\xi^\mu$, then one obtains
the killing equations 
\begin{equation}
    \tensor{\xi}{_\mu_,_\nu} + \tensor{\xi}{_\nu_,_\mu}=0 
\end{equation}
for $\xi^\mu$ and the solutions are referred to as \emph{killing vector fields}.
In Minkowski-space the ten infinitesimal killing vectors correspond to the Poincaré-generators.
\begin{sidenote}
We can use the Lie-derivative on metric to detect symmetries of the Manifold.
\end{sidenote}
\subsubsection{Invariance of the linearized field equations under infinitesimal
diffeomorphisms} 
We now check that linearized field equations are invariant under infinitesimal
diffeomorphism
\begin{equation}
    h'_{\mu\nu} = h_{\mu\nu} + \tensor{\xi}{_\mu_,_\nu} +
    \tensor{\xi}{_\nu_,_\mu}\,.
\end{equation}
The barred metric transforms as
\begin{equation}
    \begin{split}
        \overline{h'}_{\mu\nu} &= h'_{\mu\nu} - \frac{1}{2} \eta_{\mu\nu}h' \\
        &= h_{\mu\nu} + \tensor{\xi}{_\mu_,_\nu} + \tensor{\xi}{_\nu_,_\mu} - \frac{1}{2} \eta_{\mu\nu}h
        -\frac{1}{2}\eta_{\mu\nu}\partial^\varrho\xi_\varrho - \frac{1}{2} \eta_{\mu\nu}\partial^\varrho\xi_\varrho \\
        &= \overline{h}_{\mu\nu} + \tensor{\xi}{_\mu_,_\nu} +
        \tensor{\xi}{_\nu_,_\mu} -
        \eta_{\mu\nu}\tensor{\xi}{^\varrho_,_\varrho}\,.
    \end{split}
\end{equation}
We proceed by plugging this into Einstein's equations, the relevant terms are
\begin{align}
    -\frac{1}{2}\Box \overline{h'}_{\mu\nu} &= -\frac{1}{2}\Box\overline{h}_{\mu\nu} - \frac{1}{2}\Box\tensor{\xi}{_\mu_,_\nu}
    -\frac{1}{2}\Box\tensor{\xi}{_\nu_,_\mu} + \frac{1}{2}
    \eta_{\mu\nu}\Box\tensor{\xi}{^\varrho_,_\varrho}\,, \\
    -\frac{1}{2} \eta_{\mu\nu}\partial_\varrho\partial_\sigma \overline{h'}^{\varrho\sigma} &=
    -\frac{1}{2} \eta_{\mu\nu}\partial_\varrho\partial_\sigma \left( \tensor{\xi}{^\varrho^,^\sigma} + \tensor{\xi}{^\sigma^,^\varrho}
    - \eta^{\varrho\sigma} \tensor{\xi}{^\alpha_,_\alpha} \right) - \frac{1}{2}
    \eta_{\mu\nu}\partial_\varrho\partial_\sigma \overline{h}^{\varrho\sigma}\,,
    \\
    \partial^\varrho \tensor{\partial}{_(_\mu} \tensor{\overline{h'}}{_\nu_)_\varrho} &=
    \partial^\varrho \tensor{\partial}{_(_\mu} \tensor{\overline{h}}{_\nu_)_\varrho} + \frac{1}{2}\Box\tensor{\xi}{_\nu_,_\mu}
    + \frac{1}{2}\Box\eta_{\mu\nu}\,.
\end{align}
Therefore
\begin{equation}
    \begin{split}
        & -\frac{1}{2} \Box \overline{h'}_{\mu\nu} + \partial_\varrho \tensor{\partial}{_(_\mu}\tensor{\overline{h'}}{_\nu_)^\varrho}
        - \frac{1}{2} \eta_{\mu\nu}\partial_\varrho\partial_\sigma \overline{h'}^{\varrho\sigma} \\
        =\ & -\frac{1}{2}\Box\overline{h}_{\mu\nu} - {\frac{1}{2}\Box\tensor{\xi}{_\mu_,_\nu}}
        -{\frac{1}{2}\Box\tensor{\xi}{_\nu_,_\mu}}
        + {\frac{1}{2} \eta_{\mu\nu}\Box\tensor{\xi}{^\varrho_,_\varrho}}
        + \partial^\varrho \tensor{\partial}{_(_\mu}\tensor{\overline{h}}{_\nu_)_\varrho} \\
        & + {\Box\tensor{\xi}{_(_\mu_,_\nu_)}}
        - {\frac{1}{2}\eta_{\mu\nu}\Box\tensor{\xi}{^\varrho_,_\varrho}}
        - \frac{1}{2} \eta_{\mu\nu}\partial_\varrho\partial_\sigma \overline{h}^{\varrho\sigma} \\
        =\ & -\frac{1}{2} \Box \overline{h}_{\mu\nu} + \partial_\varrho \tensor{\partial}{_(_\mu}\tensor{\overline{h}}{_\nu_)^\varrho}
        - \frac{1}{2} \eta_{\mu\nu}\partial_\varrho\partial_\sigma \overline{h}^{\varrho\sigma}
    \end{split}
\end{equation}
This shows that the Einstein equations are invariant under an infinitesimal diffeomorphisms.
Therefore $\overline{h}_{\mu\nu}$ and $\overline{h'}_{\mu\nu}$ are the same \emph{physical} field.

\subsubsection{Harmonic gauge in linearized gravity}
As described above we want to bring the field equation in the form
$\Box\text{``field''}=\text{``source''}$, i.e. a wave equation.
This can ge done with the gauge condition
\begin{equation}
    \chi_\nu \left[ \overline{h} \right] \coloneqq \partial^\mu \overline{h}_{\mu\nu} = 0.
\end{equation}
In therms of the original field this condition reads
\begin{definition}[de Donder gauge, harmonic gauge]
    \begin{equation}
        \chi_\nu \left[ h \right] = \partial^\mu h_{\mu\nu} - \frac{1}{2} h_{\mu\nu} \partial^\mu h = 0
    \end{equation}
\end{definition}
Proof:
\begin{equation}
    \begin{split}
        \partial^\mu \overline{h'}_{\mu\nu} &= \partial^\mu \overline{h}_{\mu\nu} + \Box \xi_\nu + \partial_\nu \partial^\mu \xi_\mu -
        \eta_{\mu\nu} \partial^\mu\partial_\varrho\xi^\varrho \\
        &= \partial^\mu \overline{h}_{\mu\nu} + \Box \xi_\nu = 0
    \end{split}
\end{equation}
Solve for $\Box\xi_\nu$
\begin{equation}
    \implies \Box \overline{h'}_{\mu\nu} = -2\kappa T_{\mu\nu}
\end{equation}
Since $\overline{h}_{\mu\nu}$ and $\overline{h'}_{\mu\nu}$ correspond to the same physical field configuration, we can drop the prime.
\begin{definition}{Linearized field equations in de Donder gauge.}
    \begin{equation}
        \Box \overline{h}_{\mu\nu} = - 2 \kappa T_{\mu\nu}
\end{equation}
\end{definition}
\afterpage{
\clearpage
\thispagestyle{empty}
\begin{landscape}
    \begin{table}[h]
        \caption{Comparison between linearized gravity and electrodynamics.}
        \centering
        \begin{tabulars}{lll}
            \toprule
            & linearized gravity & electrodynamics \\
            \midrule

            basic field
            & $\overline{h}_{\mu\nu}$ ($h_{\mu\nu}$), spin-2, \emph{graviton}
            & $A_\mu$, spin-1, \emph{gauge boson}, \emph{gauge potential}
            \\%Photon????


            field equations
            & $ \underbrace{\Box \overline{h}_{\mu\nu}}_{\Box\text{``field''}} - \underbrace{2 \partial_\rho \tensor{\partial}{_(_\mu}\tensor{\overline{h}}{_\nu_)^\rho} - \eta_{\mu\nu}\partial_\rho\partial_\sigma \overline{h}^{\rho\sigma}}_{\text{ensures gauge inv.}} = -\underbrace{2 \kappa T_{\mu\nu}}_{\text{``source''}}$
            & $\underbrace{\Box A_\mu}_{\Box\text{``field''}} - \underbrace{\partial_\mu \left( \partial_\nu A^\nu \right)}_{\text{ensures gauge inv.}} = - \underbrace{4 \pi j_\mu}_{\text{source}}$ \\

            transf. under inf. gauge trafos
            & $\overline{h'}_{\mu\nu} = h_{\mu\nu}+\tensor{\xi}{_\mu_,_\nu} + \tensor{\xi}{_\nu_,_\mu} - \eta_{\mu\nu} \tensor{\xi}{^\rho_,_\rho}$
            & $A'_\mu = A_\mu + \partial_\mu\lambda(x)$\\

            & inf. coordinate transformation
            & internal symmetry \\

            inv. of field eqs
            & yes
            & yes \\

            specific gauges
            & de Donder gauge, $\partial_\mu \overline{h}^{\mu\nu}=0$
            & Lorentz gauge, $\partial_\mu A^\mu = 0$ \\

            field eqs. in specific gauges
            & $\Box \overline{h}_{\mu\nu} = - 2 \kappa T_{\mu\nu}$
            & $\Box A_\mu = - 4 \pi j_\mu$ \\

            inv. tensors under gauge trafo
            & $\tensor*{R}{^{\tiny(L)\prime}_\mu_\nu} =
             \tensor*{R}{^{\tiny(L)}_\mu_\nu}$ 
            & $F'_{\mu\nu} =F_{\mu\nu}$\\
            \bottomrule
        \end{tabulars}
    \end{table}
\end{landscape}
}

\begin{remark}[Fierz-Pauli action, 1939]
\begin{equation}
    \lagrangian_{\text{FP}} =
    \frac{1}{2} \left( \partial_\mu h^{\mu\nu} \right) \left( \partial_\nu h \right)
    - \partial_\mu h^{\rho \sigma} \partial_\rho \tensor{h}{^\mu_\sigma}
    + \frac{1}{2} \eta^{\mu\nu} \left( \partial_\mu h^{\rho\sigma} \right) \left( \partial_\nu h_{\rho \sigma} \right)
    - \frac{1}{2} \eta^{\mu\nu} \left( \partial_\mu h \right) \left( \partial_\nu h \right)
\end{equation}
For vacuum this is the Lagrangian of a massless spin-2 field $h_{\mu\nu}(x)$ (``the graviton'') in flat spacetime $h^{\mu\nu}$. \\
Problem: non-linearity (in electrodynamics: linear coupling)
\begin{equation}
    T_{\mu\nu}h^{\mu\nu} \rightarrow h_{\mu\nu}^{(2)} \propto \left( h_{\mu\nu}^{(1)}  \right)^2
\end{equation}
$\rightarrow$ Deser 1970: Iterative procedure \\
$\hookrightarrow$ including gravitational self energy and resuming one recovers the full nonlinear Einstein equations.
\end{remark}

\newpage

\section{Newtonian Limit}
Empirically we know
\begin{enumerate}
    \item Newtonian gravity describes the dynamics in our solar system to a high accuracy
    \item On earth, we can measure the gravitation constant $G_\text{N}$ e.g.\ by Cavendish-type  experiments
\end{enumerate}
If General Relativity is a more fundamental gravitational theory than Newton's theory it should
\begin{enumerate}
    \item recover \name{Newton}'s theory in appropriate limit, i.e.\ in the domain where Newtonian Gravity is a good description
    \item be more accurate than Newton's theory, i.e.\ it should predict small corrections to Newtonian Gravity.
\end{enumerate}
Conditions for the Newtonian limit:
\begin{enumerate}
    \item $v \ll c$ (sources move slowly) \\
    slowly changing geometry $\approx$ static: no $\dif{x}^i\dif t$ terms in
    $\dif{s}^2$ (would violate $t\rightarrow -t$ invariance)
    \item $g_{\mu\nu} = \eta_{\mu\nu} + h_{\mu\nu}$ with $\abs{h_{\mu\nu}} \ll 1$ (weak gravitational field)
    \item $p \ll \rho$ (sources have low internal pressure)
\end{enumerate}
\begin{enumerate}[{ad} 1.]
    \item $v\ll c$ is required as (special) relativistic effects must be small
    \item Consider the solar system as a closed system: Then a particle in the outer region with $v\ll c$ initially,
    will fall into the inner region (center of mass) and it will be accelerated by gravity. It will be then have a kinetic energy
    $E_\text{kin} = \frac{1}{2} m v^2 \sim \abs{m \Phi}$, where $\Phi < 0$ is the gravitational Newtonian potential with boundary condition
    $\displaystyle \lim_{x\to \infty}\Phi(x) = 0$. Small velocities of the
    sources imply weak gravitational fields.
    \item Speed of sound
    \begin{align}
        & c_s \coloneqq \abs{\frac{T_{ij}}{T_{00}}} \quad \text{with} \quad T_{\mu\nu} = \diag \left( \rho, \frac{p}{c^2}, \frac{p}{c^2}, \frac{p}{c^2} \right) \quad \text{(perfect fluid)} \\
        & c_s \sim \left( \frac{p}{\rho} \right)^{1/2}
    \end{align}
    The internal pressure of the sources must be small, otherwise they would also create (fast) motion of sound waves. \\
    $\implies p \ll \rho$ \\
    $\implies$ energy-momentum tensor of dust
    \begin{align}
        T^{\mu\nu} &= \rho_0 t^\mu t^\nu \\
        t^\mu &= \delta^\mu_0 = \left( \frac{\partial}{\partial x^0} \right)^\mu
    \end{align}
    $t^\mu$ is the ``direction'' of an internal coordinate system of time
    \begin{equation}
        \Box \overline{h}_{\mu\nu} \approx \Delta \overline{h}_{\mu\nu}
    \end{equation}
    alternatively $\Box = - \frac{1}{c^2} \frac{\partial^2}{\partial t^2} + \Delta \approx \Delta$ as 
    $\frac{1}{c} \frac{\partial}{\partial t} = \frac{1}{c} \frac{\partial}{\partial x} \frac{\partial x}{\partial t} 
    \sim \frac{v}{c} \frac{\partial}{\partial x} \ll \frac{\partial}{\partial x}$
\end{enumerate}

For the Newtonian limit, we must look for solutions to $\Box \overline{h}_{\mu\nu} = - 2 \kappa T_{\mu\nu}$, where time-derivatives are 
negligible and where the energy-momentum tensor is the one of dust. 
\begin{equation}
    \Delta \overline{h}_{\mu\nu} = 
    \begin{cases}
    	-2 \kappa \rho_0 & \mu=\nu=0 \\
    	0 & \text{else}
    \end{cases}
\end{equation}
Consider first the \name{Poisson} equation with vanishing sources. 
The unique solution is $\overline{h}_{\mu\nu} = \const$. The $\const$ can be always adjusted to zero by a residual gauge transformation:
\begin{equation}
    \overline{h}_{\mu\nu} = 0, \quad \mu \neq \nu = 0
\end{equation}
Residual gauge transformation
\begin{equation}
    \partial^\mu \overline{h'}_{_\mu\nu} = \underbrace{\partial^\mu \overline{h}_{\mu\nu}}_{=0} + \Box \xi_\nu = 0
\end{equation}
This means that all gauge transformations $\xi_\mu$ with $\Box \xi_\mu = 0$ are compatible with the de Donder gauge 
(i.e.\ that doesn't lead out of the de Donder gauge). \\
As far we know:
\begin{align}
    \overline{h}_{\mu\nu} &= 0 \qquad \text{for the 0-0 component } \\
    \Delta \overline{h}_{00} &= - 2\kappa \rho_0 \qquad \mu \neq \nu = 0
\end{align}
We identify the gravitational potential as
\begin{equation}
    \Phi \coloneqq -\frac{1}{4} \overline{h}_{00}
\end{equation}
We obtain the \name{Poisson} equation
\begin{equation}
    \Delta \Phi = \frac{\kappa}{2} \rho_0 = 4 \pi G_\text{N} \rho_0
\end{equation}
Solution in terms of the original field $h_{\mu\nu}$:
\begin{equation}
    h_{\mu\nu} = \overline{h}_{\mu\nu} - \frac{1}{2} \eta_{\mu\nu} \overline{h} 
    = \overline{h}_{00} \left( \tensor{\delta}{_\mu^0} \tensor{\delta}{_\nu^0} + \frac{1}{2} \eta_{\mu\nu} \right) 
    = - 4\Phi \left( \tensor{\delta}{_\mu^0} \tensor{\delta}{_\nu^0} + \frac{1}{2} \eta_{\mu\nu} \right)
\end{equation}
We have used $\overline{h}=\eta^{\rho\sigma}\overline{h}_{\rho\sigma} = -\overline{h}_{00} = 4 \Phi$
\begin{align}
    & \implies h_{00}=-2\Phi \qquad h_{ij}=-2\Phi\delta_{ij} \qquad h_{0\mu}=0 \\
    & \implies \dif{s}^2 = g_{\mu\nu} \dif{x}^\mu \dif{x}^\nu = - (1+2\Phi)\dif{t}^2 + (1-2\Phi)\delta_{ij} \dif{x}^i \dif{x}^j
\end{align}
This is the Newtonian geometry.
%TODO ref to first appearance newt. geo.
\subsection{Motion of test particles in Newtonian Geometry}
Test particles carry clocks that read universal time in Newton Geometry.
\begin{align}
    & \frac{\dif{}^2 x^i}{\dif{} t^2} = \frac{\dif{}^2 x^i}{\dif{} \tau^2} 
    = - \cSym{i}{\alpha}{\beta} \frac{\dif x^\alpha}{\dif \tau} \frac{\dif x^\beta}{\dif \tau}
    = - \cSym{i}{0}{0}
    = - \csym{0}{0}{i}
    = \frac{1}{2} \tensor{h}{_0_0_,_i} - \tensor{h}{_0_i_,_0} 
    = \frac{1}{2} \tensor{h}{_0_0_,_i} = - \tensor{\Phi}{_,_i} \\
    & \implies \frac{\dif{}^2 x^i}{\dif{} t^2} =  - \tensor{\Phi}{_,_i} \qquad \left( \quad \widehat{=} \quad \vec{a} = -\nabla \Phi \right) 
\end{align}
where we used $\frac{\dif \tau}{\dif t} = 1$,$v^i \sim \abs{\frac{\dif x^i}{\dif \tau}} \ll 1$, $g_{\mu\nu}=\eta_{\mu\nu}$, and
$\dif{}_t \sim \frac{v^i}{c} \dif{}_{x_i} \ll 1$.
In Newtonian Gravity, we have two equations
\begin{enumerate}
    \item ``field equations'': $\Delta \Phi = 4 \pi G_\text{N} \rho_0$ (Poisson equation), describes how the gravitational potential
    (geometry in General Relativity) reacts on matter $\rho_0$
    \item ``geodesic equation'': $\frac{\dif{}^2 x^i}{\dif{}t^2} = - \tensor{\Phi}{_,_i}$, describes how matter (test particles, dust)
    moves under the influence of the gravitational potential $\Phi$ (in General Relativity: moves in curved geometry)
\end{enumerate}

\subsection{Geodesic deviation in Newtonian Geometry}
From the last analysis, we know
\begin{equation}
    \cSym{i}{0}{0} = \tensor{\Phi}{^,^i} \qquad \text{(all other components are zero).}
\end{equation} 
Insert this in the Riemannian curvature tensor:
\begin{equation}
    \tensor{R}{^i_0_j_0} = -\tensor{R}{^i_0_0_j} = \tensor{\Phi}{^,^i_j} \qquad \text{(all other components are zero).}
\end{equation}
Ricci-tensor: 
\begin{equation}
    R_{00} = \Delta \Phi = 4 \pi G_\text{N} \rho_0
\end{equation}

\subsubsection{geodesic deviation}
(image)
\begin{equation}
    \frac{\difD^2 \eta^i}{\difD \tau^2} \approx \frac{\dif{}^2 \eta^i}{\dif{} \tau^2} 
    = - \tensor{R}{^i_0_j_0} \eta^j = - \tensor{\Phi}{^,^i_j}\tensor{\eta}{^j}
\end{equation}
where we used that
\begin{equation}
    \frac{\difD{} \eta^i}{\difD{} \tau} = \nabla_j \eta^i \frac{\dif x^j}{\dif \tau} = \partial_j \eta^i x^j = \frac{\dif \eta^i}{\dif t}
\end{equation}
Compare to the deviation equation due to tidal forces in Newtonian Gravity.
\begin{equation}
    \begin{split}
        \frac{\dif{}^2 \eta^i}{\dif \tau^2} &= \frac{\dif{}^2 \left( x^i + \eta^i \right)}{\dif{} t^2} - \frac{\dif{}^2 x^i}{\dif{}t^2} \\
        &= - \left. \frac{\partial \Phi}{\partial x_i} \right|_{x+\eta} + \left. \frac{\partial \Phi}{\partial x_i} \right|_{x} \\
        &= - \left. \frac{\partial \Phi}{\partial x_i} \right|_{x} 
        - \left. \frac{\partial^2 \Phi}{\partial x_i \partial x^j} \right|_{x} \eta^j 
        + \left. \frac{\partial \Phi}{\partial x_i} \right|_{x} \\
        &= - \tensor{\Phi}{^,^i_j} \eta^j
    \end{split}
\end{equation}
This shows again that tidal forces are a genuine gravitational effect that is related to curvature of spacetime (in General Relativity) 
and cannot be transformed away.
\chapter{Gravitational Waves}
\begin{itemize}
    \item vacuum solution of linearized Einstein equations
    \begin{equation}
        \Box \overline{h}_{\mu\nu} = 0 \qquad \text{in de Donder gauge}
    \end{equation}
    \item describes \emph{weak} gravitational waves \emph{only} $\rightarrow$ linearized treatment justified
    \item description breaks down for strong gravitational fields as the theory becomes essentially \emph{non-linear} (e.g.\ two black holes merge)
    \item analysis similar to electrodynamics, but here $h_{\mu\nu}$: spin-2 field, $A_\mu$: spin-1 field
    \item vacuum equations
    \begin{equation}
        \Box \overline{h}_{\mu\nu} = 0 \qquad \overline{h}_{\mu\nu}=h_{\mu\nu} - \frac{1}{2} \eta_{\mu\nu} h \qquad \partial^\mu \overline{h}_{\mu\nu} = 0
    \end{equation}
    \item Gauge freedom not yet completely exhausted by de Donder gauge.
    \begin{equation}
        \partial^\mu \overline{h'}_{\mu\nu} = \underbrace{\partial^\mu \overline{h'}_{\mu\nu}}_{=0} + \Box \xi_\nu = 0
    \end{equation}
    $\implies$ All gauge transformations generated by $\xi_\nu$ that satisfy $\Box \xi_\nu = 0$ do not lead out of the de Donder gauge. \\
    Compare with electromagnetic field:
    \begin{align}
        & A_\mu \to {A'}^\mu = A^\mu + \partial_\mu \lambda(x) \\
        & \partial_\mu A^\mu = 0 \qquad \text{Lorentz gauge} \\
        & \partial_\mu {A'}^\mu = \underbrace{ \partial_\mu A^\mu}_{=0} + \partial_\mu \partial^\mu \lambda(x) = 0 \implies \Box \lambda = 0
    \end{align}
    Exploit this remaining gauge freedom to make perturbations $h_{\mu\nu}$ \emph{transverse} and \emph{traceless}
    \item transversality
    \begin{equation}
        \partial^\mu h'_{\mu\nu} = \partial^\mu h_{\mu\nu} + \Box \xi_\nu + \partial_\nu \partial^\mu \xi_\mu \overset{!}{=} 0
    \end{equation}
    Since only gauge transformations that satisfy $\Box \xi_\nu$ are allowed (do not lead out of de Donder gauge), the equation that shoud
    be solved for $\xi_\nu$ is
    \begin{equation}
        \partial_\nu \partial^\mu \xi_\mu = - \partial^\mu h_{\mu\nu}
    \end{equation}
    \item In order for the perturbations to be traceless, we need to find a solution to
    \begin{equation}
        h' = h + \partial_\mu \xi^\mu = 0 \implies \partial_\mu \xi^\mu =
        -\frac{1}{2}h
    \end{equation}
\end{itemize}
% \begin{figure}
% \centering
% \begin{tikzpicture}[decoration={markings,
%   mark=between positions 0 and 1 step 22.3pt
%   with { \draw [fill] (0,0) circle [radius=2pt];}}]
% \draw[->] (0,1) -- (3,1);
% \draw[->] (0,1) -- (0,4);
% \node[text width=0.25cm] at (3,1.25){$x$};
% \node[text width=0.25cm] at (0.25,4){$y$};
%
% \node[text width=2cm, align=center] (1) at (2,5){$\omega t= 0$};
% \node[text width=2cm, align=center] (1) at (5,5){$\omega t= \frac{\pi}{2}$};
% \node[text width=2cm, align=center] (1) at (8,5){$\omega t= \pi$};
% \node[text width=2cm, align=center] (1) at (11,5){$\omega t= \frac{3\pi}{2}$};
% \node[text width=2cm, align=center] (1) at (14,5){$\omega t= 2\pi$};
% \begin{scope}[]
% \draw[dots along my path]  (2,3) ellipse (1 and 1);
% \end{scope}
% \begin{scope}[shift={(3,-0.5)},transform canvas={yscale=1.2}]
% \draw[dots along my path]  (2,3) ellipse (1 and 1);
% \end{scope}
% \begin{scope}[shift={(6,0)}]
% \draw[dots along my path]  (2,3) ellipse (1 and 1);
% \end{scope}
% \begin{scope}[shift={(7.2,0)},transform canvas={xscale=1.2}]
% \draw[dots along my path]  (2,3) ellipse (1 and 1);
% \end{scope}
% \begin{scope}[shift={(12,0)}]
% \draw[dots along my path]  (2,3) ellipse (1 and 1);
% \end{scope}
% \end{tikzpicture}
% \caption{Gravitational wave.}
% \end{figure}
% \begin{figure}
% \centering
% \begin{tikzpicture}[decoration={markings,
%   mark=between positions 0 and 1 step 22.3pt
%   with { \draw [fill] (0,0) circle [radius=2pt];}}]
% \draw[->] (0,1) -- (3,1);
% \draw[->] (0,1) -- (0,4);
% \node[text width=0.25cm] at (3,1.25){$x$};
% \node[text width=0.25cm] at (0.25,4){$y$};
%
% \node[text width=2cm, align=center] (1) at (2,5){$\omega t= 0$};
% \node[text width=2cm, align=center] (1) at (5,5){$\omega t= \frac{\pi}{2}$};
% \node[text width=2cm, align=center] (1) at (8,5){$\omega t= \pi$};
% \node[text width=2cm, align=center] (1) at (11,5){$\omega t= \frac{3\pi}{2}$};
% \node[text width=2cm, align=center] (1) at (14,5){$\omega t= 2\pi$};
% \begin{scope}[]
% \draw[dots along my path]  (2,3) ellipse (1 and 1);
% \end{scope}
% \begin{scope}[shift={(3,-0.5)},transform canvas={yscale=1.2}]
% \draw[dots along my path]  (2,3) ellipse (1 and 1);
% \end{scope}
% \begin{scope}[shift={(6,0)}]
% \draw[dots along my path]  (2,3) ellipse (1 and 1);
% \end{scope}
% \begin{scope}[shift={(7.2,0)},transform canvas={xscale=1.2},rotate around
% ={45,(9.2,3)}] \draw[dots along my path]  (2,3) ellipse (1 and 1);
% \end{scope}
% \begin{scope}[shift={(12,0)}]
% \draw[dots along my path]  (2,3) ellipse (1 and 1);
% \end{scope}
% \end{tikzpicture}
% \caption{Gravitational wave.}
% \end{figure}


\section{Degrees of Freedom (DoF) and Scalar Vector Tensor (SVT) Decomposition in Space/Time}

Parametrize the line element as
\begin{equation}
    \dif s^2 = - \left(1+2\Phi\right)\dif t^2 + \tensor{v}{_i} \left(\dif t \dif \tensor{x}{^i} + \dif t \dif \tensor{x}{^i}\right) + \left(\tensor{\delta}{_{ij}} + \tensor{h}{_{ij}} \right)\dif \tensor{x}{^i}\dif \tensor{x}{^j}\,.
\end{equation}
Block matrix
\begin{equation}
    \tensor{h}{_{\mu\nu}} =
    \begin{bmatrix}
        \tensor{h}{_{00}} & \tensor{h}{_{0j}} \\
        \tensor{h}{_{i0}} & \tensor{h}{_{ij}}
    \end{bmatrix}
    =
    \begin{bmatrix}
        -2\Phi & \tensor{v}{_{j}} \\
        \tensor{v}{_{i}} & \tensor{h}{_{ij}}
    \end{bmatrix}
    \, , \quad \abs{\Phi},\abs{\tensor{v}{_i}},\abs{\tensor{h}{_{ij}}} \ll 1\,.
\end{equation}
In general algebraic decomposition: symmetric/antisymmetric
\begin{equation}
    \tensor{T}{_{\mu\nu}} = \tensor*{T}{_{\mu\nu}^{\text{s}}} + \tensor*{T}{_{\mu\nu}^{\text{as}}} = \tensor{T}{_{(\mu\nu)}} + \tensor{T}{_{[\mu\nu]}} = \frac{1}{2}\left(\tensor{T}{_{\mu\nu}}+\tensor{T}{_{\nu\mu}}\right) + \frac{1}{2}\left(\tensor{T}{_{\mu\nu}}-\tensor{T}{_{\nu\mu}}\right)\,.
\end{equation}
Can we decompose $\tensor*{T}{_{\mu\nu}^{\text{s}}}$ further?\newline
Wake the trace
\begin{equation}
    \tensor{T}{^{\text{s}}} = \tensor{g}{^{\mu\nu}}\tensor*{T}{_{\mu\nu}^{\text{s}}}\, ,
\end{equation}
so we can write $\tensor*{T}{_{\mu\nu}^{\text{s}}}$ as follows:
\begin{equation}
    \tensor*{T}{_{\mu\nu}^{\text{s}}} = \tensor*{T}{_{\mu\nu}^{\text{tf}}} + \frac{1}{d}\,\tensor{g}{_{\mu\nu}}\tensor{T}{^{\text{s}}}\,,
\end{equation}
where $d$ denotes the dimension of spacetime and $\tensor*{T}{_{\mu\nu}^{\text{tf}}}$ is a tracefree, symmetric tensor.
Can we decompose $\tensor*{T}{_{\mu\nu}^{\text{tf}}}$ any further? Possible for tensor fields $\tensor{T}{_{\mu\nu}}(x)$ as this involves derivatives.
Decompose metric perturbations:
\begin{equation}
    \tensor{h}{_{00}} = -2\Phi
\end{equation}
\begin{equation}
    \tensor{h}{_{0i}} = \tensor{v}{_i}
\end{equation}
\begin{equation}
    \tensor{h}{_{ij}} = 2\tensor{s}{_{ij}} - 2\Psi\tensor{s}{_{ij}}\,\quad
    \begin{cases}
\Psi := -\frac{1}{6}\tensor{\delta}{^{ij}}\tensor{h}{_{ij}}\,\quad
&\text{``trace''} \\
\tensor{s}{_{ij}} := \frac{1}{2} \left(\tensor{h}{_{ij}} - \frac{1}{3}\tensor{\delta}{^{kl}}\tensor{h}{_{kl}}\tensor{\delta}{_{ij}}\right)\,\quad &\text{``strain''}
\end{cases}
\end{equation}
\begin{equation}
    \dif s^2 = -\left(1+2\Phi\right)\dif t^2 + \tensor{v}{_i} \left(\dif t \dif \tensor{x}{^i} + \dif t \dif \tensor{x}{^i}\right) + \left[\left(1-2\Psi\right)\tensor{\delta}{_{ij}} + 2 \tensor{S}{_{ij}} \right]\dif \tensor{x}{^i}\dif \tensor{x}{^j}
\end{equation}
$\tensor{\partial}{_i} = \tensor{k}{_i}$ (momentum space)
Decompose vector $\tensor{\omega}{_i}$ into transverse and longitudinal components.
\begin{align}
    \tensor{v}{_i} = &\tensor{S}{_i} + \tensor{\partial}{_i}B\\
    &3\,\,+\,\,\,\,\,1\,\,\,\,\underbrace{-1}_{\tensor{\partial}{_i}\tensor{S}{^i}=0}\,\quad \text{DoF}
\end{align}
\begin{equation}
    \tensor{S}{_{ij}} = \tensor{\partial}{_{(i}}\tensor{F}{_{j)}} + \left( \tensor{\partial}{_i}\tensor{\partial}{_j} - \frac{1}{3}\,\tensor{\delta}{_{ij}}\Delta\right) E + \tensor*{h}{_{ij}^{\text{\tiny TT}}}
\end{equation}
\begin{center}
    \begin{tabular}{c l}
        $4$ & 4 scalars \\
        $+$ & \\
        $2\,(3-1)$ & 2 transverse vectors \\
        $+$ & \\
        $6-1-3$ & 1 symmetric transverse traceless tensor \\
        \midrule
        $10$ & independent components of $\tensor{h}{_{\mu\nu}}$
    \end{tabular}
\end{center}
valid for (reduce free components):
\begin{equation}
    \tensor{\partial}{_i}\tensor{F}{^i}=0\,,\quad\tensor*{h}{_{ij}^{\text{\tiny TT}}}=\tensor*{h}{_{ji}^{\text{\tiny TT}}}\,,\quad\tensor*{h}{_{ij}^{\text{\tiny TT}}}\tensor{\delta}{^{ij}}=0\,,\quad\tensor{\partial}{^i}\tensor*{h}{_{ij}^{\text{\tiny TT}}}=0
\end{equation}

\subsection{Decomposition for tensor \emph{fields}}
Tensor fields depend on the space-time coordinate $x$.
We can build tensors from derivatives of one-rank tensors.
%irgendwas mit $\nabla_i = \partial_i$
Each derivative $\partial_i$ corresponds to $k_i$ in momentum space.
$k_i$ points in some direction.
We can decompose e.g.\ $v_i$ with respect to the direction $k_i$.
\begin{equation}
    v_i = v_{i,\perp} + v_{i,\parallel},
\end{equation}
where $v_{i,\perp}$ is perpendicular and $v_{i,\parallel}$ parallel to $k_i$.
The transverse vector is divergence free:
\begin{equation}
    \tensor{\partial}{_i} \tensor{v}{_\perp^i} = 0 \qquad \left( \implies \vec{k} \perp \vec{v} \right)
\end{equation}
The longitudinal vector $\vec{v}_\parallel$ is curl free.
\begin{equation}
    \tensor{\epsilon}{^i^j^k} \tensor{\partial}{_j} \tensor{v}{_\perp_k} = 0
\end{equation}
$\implies$ conditions for differential equations, makes only sense when applied to a vector field.
The transverse vector can be represented as curl of some other vector:
\begin{equation}
    \tensor{v}{^i_\perp} = \tensor{\epsilon}{^i^j^k} \tensor{\partial}{_i} \tensor{u}{_k} \,.
\end{equation}
The choice of $u_i$ is not ??? unless we impose additional conditions on $u$:
\begin{equation}
    \tensor{\partial}{_i} \tensor{u}{^i} = 0 \qquad \left( \text{``divergence free''} \right)
\end{equation}
The longitudinal vector is the divergence of a scalar function $\lambda(x)$
\begin{equation}
    \tensor{v}{_\partial_j} = \tensor{\partial}{_j} \lambda(x)
\end{equation}
Similarly, we can decompose the strain vector $\tensor{s}{_i_j}$:
\begin{equation}
    \tensor{s}{^i^j} = \tensor{s}{_{\perp}^i^j} + \tensor{s}{_s^i^j} + \tensor{s}{_{\parallel}^i^j}
    \qquad \left( \text{symmetric, trace free}, \tensor{s}{_i_j} = \tensor{s}{_j_i}, \tensor{\delta}{_i_j} \tensor{s}{^i^j} \right)
\end{equation}
with the transverse part $\tensor{s}{_\perp^i^j}$, the solenoidal part $\tensor{s}{_s^i^j}$,
and the parallel part $\tensor{s}{_\parallel^i^j}$.
\begin{itemize}
    \item The transverse part is divergence free $ \tensor{\delta}{_i} \tensor{s}{_\perp^i^j} = 0$.
    \item The divergence of the solenoidal part is a transverse vector,
    \begin{equation}
        \tensor{\partial}{_i} \tensor{\partial}{_j} \tensor{s}{_s^i^j} = 0
    \end{equation}
    \item The divergence of the longitudinal part is a longitudinal vector
    \begin{equation}
        \tensor{\epsilon}{^j^k^l} \tensor{\partial}{_l} \tensor{\partial}{_i} \tensor{s}{_\parallel^i_j} = 0
    \end{equation}
\end{itemize}
The longitudinal part can be constructed from a scalar field $E$:
\begin{equation}
    \tensor{s}{^\parallel_i_j} = \left( \tensor{\partial}{_i} \tensor{\partial}{_j} - \frac{1}{3} \tensor{\partial}{_i_j} \Delta \right) E
\end{equation}
The part in the braces is the only object with two derivatives of a scalar with two free indices. 
The coefficients are fixed by $ \tensor{\delta}{^i^j} \tensor{s}{^\parallel_i_j} = 0$ (trace free).
The solenoidal part can be constructed from a traversal vector field $F_i$.
\chapter{The Schwarzschild Solution}
\begin{figure}[hbtp!]
\centering
 \includegraphics{gravlens.pdf}
\caption{}
%TODO Caption
%TODO Position
\end{figure}
\begin{figure}[hbtp!]
\centering
 \includegraphics{lightdeflection.pdf}
\caption{}
%TODO Caption
%TODO position
\end{figure}


\chapter{Experimental Tests in the Solar System}
How can we test relativistic effects in our solar system?
\begin{itemize}
  \item The sum of the mass of all planets is much smaller than the solar mass
  $M_{\astrosun}\approx \unit[2\cdot 10^{30}]{kg}$. The heaviest planet is
  Jupiter with a mass of $M\textsubscript{Jup}\approx \unit[2\cdot
  10^{27}]{kg}$, therefore we can assume the planets to be testparticles.
  \item The sun is in good approximation a spherically symmetric object. We can
  therefore use the Schwarzschild metric.
\end{itemize}
% Consider the variation of the energy functional 
% \begin{equation}
% \int\tensor{g}{_\mu_\nu}\tensor{\dot{x}}{^\mu}\tensor{\dot{x}}{^\nu}\dif\lambda\,.
% \end{equation}
We define
$K:=-\tensor{g}{_\mu_\nu}\tensor{\dot{x}}{^\mu}\tensor{\dot{x}}{^\nu}$, which is
conserved along geodesics and it holds true that
\begin{equation} 
K=\begin{cases}
-1& \mathrm{\ timelike\ geodesics}\\
\phantom{-}0& \mathrm{\ lightlike\ geodesics}
\end{cases}\,.
\end{equation}
Using the Schwarzschild metric, we can explicitly write
\begin{equation}
K=-e^{2a(r)}\dot{t}^2+e^{2b(r)}\dot{r}^2+r^2\left(\dot{\vartheta}+\sin^2\vartheta
\dot{\phi}\right)
\end{equation}
A Killing-vector $\tensor{\xi}{^\mu}$ satisfies
\begin{equation}
\tensor{\xi}{_\mu}\tensor{\dot{x}}{^\mu}=\mathrm{\ const.}
\end{equation}
For the Schwarzschild metric, there are four independent Killing-vectors,
corresponding to 3 rotations, and staticity (time independence).
Conservation of angular momentum leads to a motion in a plane, w.l.o.g.\ we can
chose a coordinate system in which $\vartheta=\nicefrac{\pi}{2}$.
The Killing-Vectors are given by
\begin{align}
\tensor*{\xi}{_{(\varphi)}^\mu}&=(\partial_\varphi)^\mu=\tensor*{\delta}{_\varphi^\mu}\,,\\
\tensor*{\xi}{_{(t)}^\mu}&=(\partial_t)^\mu=\tensor*{\delta}{_t^\mu}\,.
\end{align}
The associated conserved quantities are
\begin{align}
E&:=\tensor*{\xi}{_{(\varphi)}^\mu}\tensor{g}{_\mu_\nu}\tensor{\dot{x}}{^\nu}
=\tensor{g}{_t_t}\dot{t}
=e^{2a}\dot{t}
=\left(1-\frac{2M}{r}\right)\dot{t}
\\
L&:=\tensor*{\xi}{_{(t)}^\mu}\tensor{g}{_\mu_\nu}\tensor{\dot{x}}{^\nu}
=\tensor{g}{_\varphi_\varphi}\dot{\varphi}
=r^2\dot{\varphi}\,.
\end{align}
For massless
%TODO stimmt das?
particles, we can think of $E$ and $L$ as conserved energy and angular momentum.
It follows
\begin{equation}
K=-\left(1-\frac{2M}{r}\right)\dot{t}^2+\left(1-\frac{2M}{r}\right)^{-1}\dot{r}^2+r^2\varphi^2
\end{equation}
If we insert the conserved quantities $L,E$ and multiply by
$\frac{1}{2}\left(1-\frac{2M}{r}\right)$, we get
\begin{equation}
\frac{E^2}{2}=\frac{\dot{r}^2}{2}
+\left(1-\frac{2M}{r}\right)\left(\frac{L^2}{2r}-\frac{K}{2}\right)\,.
\end{equation}
This expression can be rearranged to the Form
\begin{equation}
\frac{\dot{r}^2}{2}+V\textsubscript{eff}(r)
=\varepsilon\,,
\end{equation}
with the convenient definitions 
\begin{equation}
V\textsubscript{eff}(r):=\frac{MK}{r}+\frac{L^2}{2r^2}-\frac{ML^2}{r^3}\,,\quad
\varepsilon:=\frac{E^2+K}{2}\,.
\end{equation}
\begin{figure}[hbtp!]
\centering
 \includegraphics{plot2.pdf}
\caption{Effective Potential in Newtonian physics.}
%TODO unit of a!
\end{figure}
\begin{figure}[hbtp!]
\centering
 \includegraphics{plot1.pdf}
\caption{Effective Potential in GR for various
$a=L/(mr\textsubscript{S})$.}
%TODO unit of a!
\end{figure}
\begin{figure}[hbtp!]
\centering
 \includegraphics{plot3.pdf}
\caption{Effective Potential for a photon.}
%TODO unit of a!
\end{figure}

\section{Perihelion Shift of Mercury}
% \begin{figure}[b]
% \centering
% \begin{tikzpicture}[auto,node distance=3cm,thick,main node/.style={circle,draw,font=\sffamily\Large\bfseries}]
% \draw [black,thick,dashed] (0,0) circle (1/1.6);
% \draw [black,thick,dashed] (0,0) circle (1/0.4);
% \draw [thick, color=red, domain=0:6*pi, samples=200, smooth]
%   plot (xy polar cs:angle=\x r, radius={1/(1-0.6*cos(\x r+0.2*\x r))});
% \node [fill=white] at (0,1) {Perihelion};s
% \node at (0,0) {\Huge\textbullet};
% \node (1) at (40:1/1.6) {\textbullet};
% \node (2) at (90:1/1.6) {\textbullet};
% \node (3) at (140:1/1.6) {\textbullet};
% \end{tikzpicture} 
% \caption{Perhelion shift of mercury (exaggerated)}
% \end{figure}
It has been known for a long time that the perihelion of the mercury moves by
roughly $5061''\footnote{Where an arcsec $''$ denotes the 3600 part of
an degree.}/\textrm{century}$, if on subtracts other effects such a
fraction of $43''/\textrm{century}$. 
\begin{figure}[hbtp!]
\centering
 \includegraphics{perishift.pdf}
\caption{Perihelion shift of mercury. (exaggerated)}
\end{figure}

There where several proposed explanations for this including
\begin{itemize}
  \item a new planet called Vulcan between mercury and the sun,
  \item a change to newtons $1/r^2$-law,
  \item effects due the suns quadrupole moment. 
\end{itemize}
We will now look at what is predicted by GR.
The trajectories of planets ($K=-1$) are given by
\begin{equation}
\dot{r}^2-\frac{2M}{r}+\frac{L^2}{r^2}-\frac{2ML^2}{r^3}=E^2-1\label{eq:planeteq}\,.
\end{equation}
If we think of the radius as a function of the angle, we can write
\begin{equation}
\dot{r}=\od{r}{\lambda}=\od{r}{\phi}\od{\phi}{\lambda}\,.
\end{equation}
Multiplying \eqref{eq:planeteq} with
%TODO choose between varphi and phi
$\left(\od{\phi}{\lambda}\right)^{-2}=\frac{1}{\dot{\phi}}=\frac{r^4}{L^2}$
yields
\begin{equation}
\left(\dod{r}{\phi}\right)^2-\frac{2Mr^3}{L^2}+r^2-2Mr
=\left(E^2-1\right)\frac{r^4}{L^2}\,.\label{eq:orbitrphi}
\end{equation}
Similar to the treatment of the Keppler problem in classical mechanics, we
define
\begin{equation}
u(\phi):=\frac{L^2}{Mr(\phi)}\,.
\end{equation}
This implies
\begin{equation}
\dif r=-\frac{L^2}{Mu^2}\dif u\,,\quad
\left(\dod{r}{\phi}\right)^2=\frac{L^4}{M^2u^4}\left(\dod{u}{\phi}\right)^2\,.
\end{equation} 
In terms of the new variables \eqref{eq:orbitrphi} reads
\begin{equation}
\left(\dod{u}{\phi}\right)^2-2u+u^2-\frac{2M^2}{L^2}u^3=(E^2-1)\frac{L^2}{M^2}\,.
\end{equation}
We differentiate with respect to $\phi$ and get 
\footnote{primes denote $\phi$
derivatives}
\begin{equation}
2u^\prime u^{\prime\prime}-2u^\prime+2u
u^{\prime}-\frac{6M^2}{L^2}u^2u^\prime=0\,.
\end{equation}
Dividing by $2u^\prime$ yields
\begin{equation}
u^{\prime\prime}-1+u=\frac{3M^2}{L^2}u^2\,,
\end{equation}
which is an equation similar to the Keppler problem. The difference is in the
right hand side, which is equal zero in the classical case.
We may calculate an approximation by pertubative corrections
\footnote{The quantity relevant for the error of a given order is
$\frac{3M^2}{L^2}$.}
\begin{equation}
u=u_0+u_1+u_2+\dots\,.
\end{equation}
At zeroth order, we neglect the quadratic term, so that we are left with
\begin{equation}
u_0^{\prime\prime}-1+u_0=0\,,
\end{equation}
which is equal to the Newtonian problem, with exact solution
\begin{equation}
u_0(\phi)=1+\varepsilon\cos\phi\,.
\end{equation}
\begin{figure}[hbtp!]
\centering
 \includegraphics{ellipsegeo.pdf}
\caption{An ellipse is the solution to the classical unperturbed Kepler
problem.}
\end{figure}

At first order we have
\begin{equation}
\begin{split}
u_1^{\prime\prime}-1+u_1&=\frac{3M^2}{L^2}u_0^2\\
&=\frac{3M^2}{L^2}\left(1+\varepsilon\cos\phi\right)^2\\\
&=\frac{3M^2}{L^2}\left(1+\frac{\varepsilon^2}{2}+2\varepsilon\cos\phi+\frac{1}{2}\varepsilon^2\cos
2\phi\right)\,,
\end{split}
\end{equation}
Where we made use of trigonometric identities in the last step. 
The first order solution reads
\begin{equation}
u_1(\phi)=1+\varepsilon\cos\phi+\frac{3M^2}{L^2}\phi\sin\phi\,.
\end{equation}
which can be checked by using
\begin{align}
\dod[2]{}{\phi}(\phi\sin\phi)+\phi\sin\phi&=2\cos\phi\,,\\
\dod[2]{}{\phi}(\cos 2\phi)+\cos 2\phi&=-3\cos 2\phi\,.
\end{align}
Assuming that we still have approximately a movement along an ellipse and Taylor
expansion yields
\begin{equation}
u(\phi)=1+\varepsilon\cos[(1-\delta)\phi]
\simeq
1+\varepsilon\cos\phi
+\varepsilon\delta\phi\sin\phi+\landauO(\delta^2)\,.
\end{equation}
Comparing this to the expression for $u_1$ we find
\begin{equation}
\delta=\frac{3M^2}{L^2}
\end{equation}
and the error is of order $\delta^2$.
%\begin{equation}
% r(\phi)=\frac{p}{1+\varepsilon\cos\phi}
% =\frac{a\left(1-\varepsilon^2\right)}{1+\varepsilon\cos\phi}
% \end{equation}
If we assume that there was a perihelion at $\phi_p$, the next one will occure
at $\phi+\Delta\Phi$ with 
\begin{equation}
\Delta\phi=2\pi\delta=\frac{6\pi M^2}{L^2}
\end{equation}
The zeroth order solution gives 
\begin{equation}
r_0(\phi)=\frac{L^2}{M(1+\varepsilon\cos\phi)}\stackrel{!}{=}
\frac{a(1-\varepsilon^2)}{1+\varepsilon\cos\phi}\,.
\end{equation}
So $L^2\approx M(1-\varepsilon^2)a$ with an error of order $\delta$, that can
be neglected.
With this approximation and restoring factors of $c$ and $G$ we arrive at the following
expression for the perihelion shift:
\begin{equation}
\Delta\phi=\frac{6\pi G M_{\astrosun}}{c^2 (1-\varepsilon^2)a}\,.
\end{equation}
As expected the effect scalses with $1/a$ and thus the planet nearest to the sun
recives the strongest effect. For mercury we have
\begin{equation}
\frac{GM_{\astrosun}}{c^2}=\unit[1.48]{km}\,,\quad a =\unit[5.79\cdot
10^7]{km}\,,\quad\varepsilon=0.2056\,,
\end{equation}
Which results in an perihelionshift
\begin{equation}
\Delta\phi\textsubscript{mer}
=\unitfrac[5.03\cdot 10^{-7}]{rad}{orbit}
=\unitfrac[43]{{}^{\prime\prime}}{century}\,,
\end{equation}
which fits the observation in the error margin.
\chapter{Black Holes}
\chapter{Cosmology}

\end{document}
