\chapter{Gravity and Geometry}
The observable universe is stable. There are two obvious configurations in which this is possible:
\begin{enumerate}
    \item Static universe, masses are arranged in a grid, all nett forces cancel. 
    	  However small fluctuations cause the system to collapse therefore this is
    	  no possible description for the universe. (PICTURE)
    \item Expanding universe, all masses move away from each other, overcoming the gravitational attraction. 
    	  Theoretically such a system can be described by using Newtonian Physics introducing additional energy contributions. 
    	  This turns out to be inconsistent.
\end{enumerate}
Since in the second description all particles are accelerated relative to each other, there are no inertial systems. 
A theory in which all observers are equal must therefore be local and thus be described by means of differential geometry.
Claim: The laws of physics are the same in every system.
If we assume that the Maxwell equations are right, the Newtonian theory of
gravity must be wrong.
Implications:
All free falling systems are equivalent (i.e. indistinguishable by the observer). 
Light must bend, otherwise a beam could be used to deduce whether your system is inertial.
The following example illustrates that Euclidean geometry is no adequate description of space-time.
\begin{example}[Rotationg Sphere]
see Introduction to tensor calculus
\end{example}
\section{Coordinate Systems}
We will start by studying coordinate systems in flat space, which should be
familiar.
\subsection*{Cartesian Coordinates}
(PICTURE)
\begin{equation}
    s^2=(x_1-x_2)^2+(y_1-y_2)^2
\end{equation}
Infinitesimal line element
\begin{equation}
    \dif s^2=\dif x^2+\dif y^2  \label{eq:cartline}
\end{equation}
\subsection*{Polar Coordinates}
(PICTURE)
\begin{equation}
    x= r\cos\varphi\quad y= r\sin\varphi
\end{equation}
\begin{align}
    \dif x&= \pd{x}{r}\dif r+\pd{x}{\varphi}\dif \varphi = \cos\varphi\dif r-r\sin\varphi\dif \varphi\\
    \dif y&= \pd{x}{r}\dif r+\pd{y}{\varphi}\dif \varphi = \sin\varphi\dif r+r\cos\varphi\dif \varphi
\end{align}
Plugging this into \eqref{eq:cartline} gives the line element in polar coordinates
\begin{equation}
    \dif s^2=\dif r^2+r^2\dif \varphi^2
\end{equation}
In matrix form
\begin{equation}
    \dif s^2=
    \begin{bmatrix}
        \dif r& \dif \varphi
    \end{bmatrix}
    \begin{bmatrix}
        1& 0\\
        0& r^2\\
    \end{bmatrix}
    \begin{bmatrix}
        \dif r\\ \dif \varphi
    \end{bmatrix}\, .
\end{equation}
The matrix
\begin{equation}
    g(\vec{r})=
    \begin{bmatrix}
        1& 0\\
        0& r^2\\
    \end{bmatrix}\, ,
\end{equation}
is called the \emph{metric}.
In general we have
\begin{equation}
    \dif s^2 = g_{ij}\dif x^i\dif x^j\, .
\end{equation}
The idea is to keep the law of inertia, i.e. particles still move on straight
line. However, we need to generalize the concept of a 'straight' line, in a
curved space.
Straight lines are curves minimizing the distance between two points, we therefore introduce a
\section{Variation Principle}
Again we take a look at flat space, but with curved coordinates. 
The length $S$ of a curve $\gamma$ with $\gamma^i(\lambda) = x^i(\lambda)$ is
given by the integral
\begin{equation}
    S=\int_{\gamma}\sqrt{\dif s^2} = 
    \int_{\gamma}\sqrt{g_{ij}\dif x^i\dif x^j}=\int_{a}^{b}\sqrt{g_{ij}\od{x^i}{\lambda} \od{x^j}{\lambda}}\dif \lambda\, .
\end{equation}
As stated above generalised straight lines satisfy $\delta S = 0$. It is
convinient to parametrise the curve by the arc length $\dif s$. If we define
$L:=\sqrt{g_{ij}\od{x^i}{s} \od{x^j}{s}}$, $S$ takes a form
familiar from classical mechanics:
\begin{equation}
    S=\int_a^b L\dif s\, .
\end{equation}
The extremal condition implies the Euler Lagrange equations
\begin{equation}
    \pd{}{\lambda}\pd{L}{\dot{q}_i}-\pd{L}{\dot{q}_i}
=0\, .		\end{equation}
In the case at hand the equations are
\begin{equation}
    0=\tensor{g}{_i_l}\tensor{\ddot{x}}{^j}+\frac{1}{2} \left(\partial_i g_{jl} + \partial_j g_{il} - 
    \partial_l g_{ij}\right)\dot{x}^j\dot{x}^l\, . \label{eq:PreGeo}
\end{equation}
The term invoking derivatives of the metric defines the \emph{Christoffel
symbols of the first kind}
\begin{equation}
    \csym{i}{j}{l}:=\frac{1}{2} \left(\partial_i g_{jl} + \partial_j g_{il} -
    \partial_l g_{ij}\right)\, .
\end{equation}
It is convenient to multiply \eqref{eq:PreGeo} by the inverse metric $g^{ki}$ so that we obtain the \emph{geodesic equation}
\begin{equation}
    0 = \od[2]{x^i}{s}+\cSym{i}{j}{k}\od{x^j}{s}\od{x^k}{s}\, .
\end{equation}
Where $\cSym{i}{j}{k}$ are the \emph{Christoffel symbols of the second kind}
\begin{equation}
   \cSym{i}{j}{k}:=g^{km}\csym{m}{j}{k}=\frac{1}{2}g^{il}\left(\partial_i
    g_{jl} + \partial_j g_{il} - \partial_l g_{ij}\right)\, .
\end{equation}
% remark is obsolete as long as bracket notation for christoffel symbols is used
%\begin{remark}
%Although the notation looks as the Christoffel symbols form a tensor, however
% they do not.
%\end{remark}
In flat space we have $g_{ij}=\eta_{ij}$ and can easily check that all
Christoffel symbols vanish. We therefore recover the ordinary equation of motion for a free particle
\begin{equation}
    0 = \od[2]{x^i}{s}\, .
\end{equation}
\begin{example}
Suppose a observer follows a free falling body in a homogeneous field. 
Therefore a transformation between the system of the earth and the one of the body are given by 
(for simplicity we only consider the coordinate along it is falling)
\begin{equation}
    (t,x)\to\left(t,x-\frac{1}{2}gt^2\right)
\end{equation}
Analogous to the Riemannian case discussed before, the line element takes the form
\begin{equation}
    \begin{split}
	    \dif s^2&=-\dif t^2 +\dif x^2\\
	    &=-\dif t'^2+(\dif x'- gt\dif t')(\dif x'- gt\dif t')\\
	    &=(g^2t'^2-1)\dif t'^2-2gt\dif x'\dif t'+\dif x'^2
    \end{split}
\end{equation}
\end{example}
\section{Newtonian Limit}