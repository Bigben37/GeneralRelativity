\chapter{Differential Geometry}
As we have noted before general relativity is a inherent local theory. It is convenient to formulate it in terms of differential geometry.
We introduce the notion of a manifold.

\begin{definition}
A $n$ dimensional manifold $M$ is a Hausdorff space with countable basis, that
is locally homeomorphic to $\mathbb{R}^n$.
\end{definition}
\begin{remark}
The requirements Hausdorff and countable basis are of a more technical nature and are satisfied for most of the objects one can imagine 
except some pathological examples (we won't go into the details on this).

Locally homeomorphic to $\mathbb{R}^n$ means there exists a set of \emph{charts} 
$(\varphi,U^\varphi)$ called an \emph{atlas} $\mathcal{A}$ with $\cup_{\varphi\in\mathcal{A}} U^\varphi =M$, 
i.e. the charts cover the whole manifold. The maps $\varphi:U^\varphi\to \varphi(U^\varphi)\subset\mathbb{R}^n $ are homoemorphisms, 
meaning that $U^\varphi$ is open, $\varphi$ is bijective and both $\varphi$ and $\varphi^{-1}$ are continuous.
Further for any two $\varphi,\psi\in \mathcal{A}$, the coordinate changes 
$\varphi\circ\psi^{-1}:\psi(U^\psi\cup U^\varphi)\to \phi(U^\psi\cup U^\varphi)$ be infinitely often differentiable.
(BIlder)
\end{remark}

We can now reduce differentiation on the manifold to the ordinary differentiation in $\mathbb{R}^n$. 
Since physical laws are described in terms of differential equations, we can formulate them on $M$. 
The fact that the coordinate changes are $C^\infty$ ensures that differentiability is well defined (and thus the physical laws are).

\begin{sidenote}[Differentiability depends on the Differential Structure]
There can be different \emph{differential structures} on a manifold, 
which means there are multiple (maximal)alases, which
could not be merged because the coordinate changes would not be $C^\infty$. Those differentiable structures therefore imply different notions of differentiability. 
Remarkably this may even play a role in some physical theories. 
As an example an 11d-supergravity can be described as a product
$\Reals^{3+1}\times \Sphere^7$.
Where $\Sphere^7$ is the 7-sphere and $\Reals^{3+1}$ Mikovski space.
This means on every point in the $\mathbb{R}^{3+1}$ there is a (small) $\Sphere^7$  located that contains additional spatial dimensions. 
The $\Sphere^7$ has 28 different differential structures, so the choice of
such a structure affects the theory for the above reasons.
\end{sidenote}
All simple examples we come of can be embedded in a higher space. It holds true
that every real $n$-dimensional Manifold can be embedded to $\Reals^{2n}$
(This is however not true for complex, i.e. analytic manifolds).
For example the $\Sphere^2$ can be interpreted as submanifold of the $\Reals^3$.
However manifolds are objects that exists independent of such embeddings. 
For example a torus can be thought of as a square with the opposite sides identified (leaving to the left results in re-entering in the left).
\begin{sidenote}[Topology of the Universe]
In addition to the local structure, we may question the global, i.e. the
topological structure of the universe.
On may for example imagine that we live on the surface of a 3-sphere (finite but boundless universe). 
However this might be observable in crosscorelation in the cosmic microwave background from photons reaching us 
from different directions but coming from the same event. There is no evidence of such phenomena so far. 
Most models can be excluded to some certainty. A cylindrical
universe is still possible (finite in one, infinite in the
other directions).
\end{sidenote}
\section{Vectors}
Vectors are important objects describing physics. The naive view as an "arrow pointing frow one point to another" is flawed. 
For example on a sphere an arrow connecting two points does not make much sense.
We want to find a description of vectors as objects that are naturally related to the structure of the manifold independent of the embedding.
There are three equivalent definitions for a vector:
\begin{enumerate}
    \item algebraic (mathematical, suitable for proofs)
    \item physically
    \item geometrically (ugly, but plastic)
\end{enumerate}
\subsection*{Definitions}
\begin{definition}[Algebraic]
A vector is a derivation at the germ of a function at $p$.
\end{definition}
The germ is the set of all functions that are locally equal,i.e. vectors are local objects.
\begin{definition}[Derivation]
A derivation $D$ satisfies the following rules for all $f,g\in C^\infty(\Reals)$ and $\lambda \in \Reals$:
\begin{align}
    D(f+g) &=Df+Dg\\
    D(\lambda)f&=\lambda f\\
    D(fg)&= (Df)g+f(Dg)
\end{align}
\end{definition}
Given two vectors we can construct a new one, the \emph{Lie Braket}
\begin{equation}
    [X,Y]f:=X(Yf)-Y(Xf)\, .
\end{equation}
The only property that has to be checked is that it satisfies the Leibniz rule.
\begin{equation}
    XY(fg)=X((Yf)g+f(Yg))=(XYf)g+(Yf)(Xg)+(Yg)(Xf)+(XYg)f\\
\end{equation}
Subtracting $YX(fg)$ proves that $[X,Y]$ is indeed a vector.
The set of all vectors in a point $p$ is called the tangent space $T_pM$. A basis of $T_pM$ is given by $\partial_i$.
Proof sketch:
\begin{enumerate}
    \item Show $f(x^i)=f(0)+x^i\tilde{f}(x^i)$
    \item Write $X=a^i\partial_i$
    \item Show $Xf=0\quad \forall f \iff X=0$
\end{enumerate}
Every vector $A$ can be written as $A=a^i\pd{}{{x^i}}$. We can now look how the components of the vector transform 
under a change of coordinates (the vector itself is invariant!). We usually denote the elements of the transformed systems with a bar.
\begin{equation}
    A= a^k\pd{}{{x^k}}= a^k\pd{\overline{x}^i}{{x^k}}\pd{}{{\overline{x}^k}}
\end{equation}
also we can express $A$ directly in the new basis
\begin{equation}
    A= \overline{a}^i\pd{}{{\overline{x}^i}}
\end{equation}
Comparing the coefficients gives the vector transformation law
\begin{equation}
    \overline{a}^i=a^k\pd{\overline{x}^i}{{x^k}}\label{eq:coefftrafo}
\end{equation}
Sometimes a vector is defined as a object that transforms according to \ref{eq:coefftrafo} under a change of coordinates, 
this is the physical definition. It is a priori not clear that a vector also corresponds to a geometrical object. 
Consider a curve on $M$, i.e. a map $\gamma:\mathbb{R}\to M$
Then $D_\gamma f=\od{}{t}(f\circ\gamma)(0)$ is a derivative.
For the special curves $\gamma_i(t)=p+te_i$
$D_{\gamma_i} f=\partial_if$, so we can identify the derivatives with the geometrical tangent space.
Since we have a basis we can work in local coordinates, e.g. let $A=a^i\pd{}{{x^i}}$, $B=b^i\pd{}{{x^i}}$ then the lie bracket reads
\begin{equation}
    [A,B]^j=a^i\partial_ib^j-b^i\partial_ia^i
\end{equation}
Since the tangent space is a vector space, we can define its dual space
\begin{equation}
    T_pM^*=\{L:T_pM\to \mathbb{R}\, |\, L \text{ linear}\}
\end{equation}
which is again a vector space of the same dimension. Its elements are called dual or covariant vectors.
We can define a basis on $	T_pM^*$, which we denote by $\dif x^i$ and  which acts on $T_pM$ via
\begin{equation}
    \dif x^i(\partial_j)=\delta^i_j\, . \label{eq:orthdual}
\end{equation}
It can easily deduced by \eqref{eq:orthdual} that the components of a dual vector transform as
\begin{equation}
    \overline{a}_i=\pd{x^k}{{\overline{x}^i}}a_k\, .
\end{equation}
If $\vec{a},\vec{b}\in\mathbb{R}^n$ contain the component of a vector and a dual vector respectively, 
then the transformation can be written in matrix form
\begin{align}
    \vec{a}&\to V\vec{a}\, ,\\
    \vec{b}&\to\left(V^T\right)^{-1}\vec{b}\, .
\end{align}
with $V_{ij}=\pd{\overline{x}^i}{{x^j}}$. 
In normal calculus we restrict ourself to orthogonal transformations (i.e. mapping orthonormal bases onto each other) for which $(O^T)^{-1}=O$. 
Which is the reason why we do not bother to distinguish between vectors and dual vectors because they transform identically. 
In special relativity we have e.g. $(\Lambda^T)^{-1}\neq\Lambda$ for a boost, the difference becomes even more important in general 
relativity where the relation can become arbitrarily complicated.
\section{Tensors}
From vectors $A$ ,$B$ we can construct new objects with multiple indices that posses well defined transformation behaviour. 
For example we can define
\begin{equation}
    \overline{T}^{ij}=a^ib^j\, ,
\end{equation}
which transforms as
\begin{equation}
    T^{ij}=\pd{\overline{x}^i}{{x^k}}\pd{\overline{x}^j}{{x^l}}a^kb^l=\pd{\overline{x}^i}{{x^k}}\pd{\overline{x}^j}{{x^l}}T^{kl}\,
   .\label{eq:tensortrafo}
\end{equation}
Again it is possible to define tensors in a coordinate independent way. 
At this point we will make things easier and only consider the physical definition. 
A tensor is then per definitionem a object that transforms similar to
\eqref{eq:tensortrafo}.
\subsection{Symmetries}
A tensor is said to be symmetric in two indices if it stays invariant when exchanging those indices, e.g.
\begin{equation}
    T_{ab}=T_{ba}\, .
\end{equation}
\begin{remark}
We have not yet established a relation between upper and lower indices, i.e. we have no metric. Expressions of the form
\begin{equation}
    \tensor{T}{^a_b}=\tensor{T}{^b_a}
\end{equation}
make no sense, since they can not be true in every system.
\end{remark}
\section{The Metric}
\begin{definition}[Metric]
The metric $g$ is a non-degenerate ($\det(g)\neq 0$), symmetric covariant two
tensor.
\end{definition}
We have already seen examples of metrics for the flat space, e.g. in spherical coordinates $g$ was given as
\begin{equation}
    g=
    \begin{pmatrix}
        1 & 0\\
        0 & r^2\\
    \end{pmatrix}
\end{equation}
Given a metric we relate vectors and dual vectors to each other by
\begin{equation}
    a_i=g_{ij}a^j
\end{equation}
\section{Parallel Transport}
Idea: Generalize parralel transport from flat space.
(PICTURES) 
If we express a vector in non-cartesian coordinates and shift it it's
coordinates do not change.
We take a look on two operations:
\begin{enumerate}
\item the change of the vector itself
\item the change of its coordinates
\end{enumerate}
Let $A_i$ be the coordinates of a vector in a system $x^i$ and $B_i$ the in a
system associated with coordinates $y^i$ respectively.
\begin{equation}
A_i\pd{{y^j}}{{x^i}}B_j\, ,\quad B_i\pd{{x^j}}{{y^i}}A_j\, .
\end{equation}
We look at vectors whose coodinates in the system $y^i$ do not change i.e.
$\delta B_i=0$ The variation of $A_i$ is given by
\begin{equation}
\delta
A_i=\delta\left(\pd{{y^j}}{{x^i}}\right)B_j
=\md{{y^j}}{2}{{x^i}}{}{{x^k}}{}\delta
x^k B_j\, .
\end{equation}
Expressing $B_i$ in terms of $A_i$ yields
\begin{equation}
\delta A_i = \md{{y^j}}{2}{{x^i}}{}{{x^k}}{}\pd{{x^l}}{{y^j}}A_l\delta x^k
=:\affin{l}{i}{k}A_l\delta x^k
\end{equation}
$\affin{l}{i}{k}$ is called \emph{affine connection} or short affinity.
\begin{remark}
We can always find a coordinate system in wich $\affin{l}{i}{k}\equiv 0$.  
\end{remark}
We notice that if 
\begin{equation}
\left(\pd{{A_i}}{{x^j}}-\affin{l}{i}{k}A_l\right)\delta x^k = 0\, ,
\end{equation}
The vector $A$ does not change its cordinates. We define a \emph{covariant
derivative} 
\begin{equation}
\tensor{A}{_i_;_j}:=\pd{{A_i}}{{x^j}}-\affin{l}{i}{k}A_l\, .
\end{equation}
It can easyly be seen that the covariant derivative of a tensor transforms as a
tensor. 
\begin{remark}[The
Covariant Derivative in Electrodynamics] Example from Electrodynamics concerning the covariant derivative. 
The theory is invariant under transformations $\phi\to e^{\imI \alpha}\phi$, 
because $\phi^*\phi$ and $\phi^*\nabla\phi-\phi\nabla\phi^*$ do not change.
Vervollständigen\ldots
\end{remark}
Since we have now established a relation between vectors and dual vectors, we
can also determine the covariant derivative of a dual vector. Therefore we
consider the scalar $A_iB^i$. Since the covariant derivative satisfies the
Leibniz rule we get
\begin{align}
(A_iB^i);j = \tensor{A}{_i_;_j}\tensor{B}{^i}+\tensor{A}{_i}\tensor{B}{^i_;_j}
\end{align}
But for scalars the covariant derivative is identical to the normal derivative
so that 
\begin{align}
(A_iB^i)_{;j} =(A_iB^i)_{,j}=
\tensor{A}{_i_,_j}\tensor{B}{^i}+\tensor{A}{_i}\tensor{B}{^i_,_j}
\end{align}
If we put in the covariant derivative of a we get 
\begin{equation}
\tensor{A}{_i}\tensor{B}{^i_;_j}=\tensor{A}{_i}\left(\pd{}{{x^j}}B^i+\Gamma^i_{kj}\tensor{B}{^k}\right)
\end{equation}
Since $A$ was arbitary, we can deduce that
\begin{equation}
\tensor{B}{^i_;_j}=\left(\pd{}{{x^j}}B^i+\Gamma^i_{kj}\tensor{B}{^k}\right)
\end{equation}
for a (1,1)-tensor we get:
\begin{equation}
\tensor{A}{^i_j_;_k}=\pd{}{{x^k}}\tensor{A}{^i_j}-\Gamma^a_{jk}\tensor{A}{^i_a}+\Gamma^i_{ak}\tensor{A}{^a_j}\,
.\end{equation}
Similar expressions hold for tensors of arbitary rank where each index gives an
aditional term containing a contraction with the affinity $\Gamma^i_{jk}$. 
We now want to consider curved spaces. This can not immediatly be determined by
the metric, for example the polar coordinates do not look flat even though they
describe the ordinary $\Reals^2$. (PICTURES)
