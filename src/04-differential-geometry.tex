\chapter{Differential Geometry}
As we have noted before, general relativity is a inherent local theory. It is
convenient to formulate it in terms of differential geometry.
\section{Manifolds}
\begin{definition}
A $n$ dimensional manifold $M$ is a Hausdorff space with countable basis, that
is locally homeomorphic to $\mathbb{R}^n$. 
\end{definition}
%TODO picture
We will give a short introduction to
the most important terms.
\begin{remark}
The requirements Hausdorff and countable basis are of a more technical nature and are satisfied for most of the objects one can imagine 
except some pathological examples (we won't go into the details on this).

Locally homeomorphic to $\mathbb{R}^n$ means there exists a set of \emph{charts} 
$(\varphi,U^\varphi)$ called an \emph{atlas} $\mathcal{A}$ with $\cup_{\varphi\in\mathcal{A}} U^\varphi =M$, 
i.e. the charts cover the whole manifold. The maps $\varphi:U^\varphi\to \varphi(U^\varphi)\subset\mathbb{R}^n $ are homeomorphisms, 
meaning that $U^\varphi$ is open, $\varphi$ is onto and both $\varphi$ and
$\varphi^{-1}$ are continuous.
Further for any two $\varphi,\psi\in \mathcal{A}$, the coordinate changes 
$\varphi\circ\psi^{-1}:\psi(U^\psi\cup U^\varphi)\to \phi(U^\psi\cup U^\varphi)$
be smooth\footnotemark .
\end{remark}
\begin{figure}[hbtp!]
\centering
 \includegraphics{coordinateenvironment.pdf}
\caption{Coordinate change}
%TODO Caption
\end{figure}

\footnotetext{Infinitely often differentiable or short $C^\infty$.}
We can now reduce differentiation on the manifold to the ordinary differentiation in $\mathbb{R}^n$. 
Since physical laws are described in terms of differential equations, we can formulate them on $M$. 
The fact that the coordinate changes are smooth ensures that differentiability
is well defined (and thus the physical laws are).

\begin{sidenote}[Differential Structures]
There can be different \emph{differential structures} on a manifold, 
which means there are multiple (maximal)alases, which
could not be merged because the coordinate changes would not be $C^\infty$. Those differentiable structures therefore imply different notions of differentiability. 
Remarkably this may even play a role in some physical theories. 
As an example an 11d-supergravity can be described as a product
$\Reals^{3+1}\times \Sphere^7$.
Where $\Sphere^7$ is the seven-sphere and $\Reals^{3+1}$ Minkovski space.
This means on every point in the $\mathbb{R}^{3+1}$ there is a (small) $\Sphere^7$  located that contains additional spatial dimensions. 
The $\Sphere^7$ has 28 different differential structures, so the choice of
such a structure affects the theory for the above reasons.
\end{sidenote}
All simple examples we come of can be embedded in a higher space. The
\name{Whitney} embedding theorem states that every real $n$-dimensional Manifold
can be embedded to $\Reals^{2n}$ (This is however not true for complex, i.e. analytic manifolds).
For example the $\Sphere^2$ can be interpreted as submanifold of the
$\Reals^3$.
However manifolds are objects that exists independent of such embeddings. 
For example a torus can be thought of as a square with the opposite sides identified (leaving to the left results in re-entering in the left).
\begin{sidenote}[Topology of the Universe]
In addition to the local structure, we may question the global, i.e. the
topological structure of the universe.
On may for example imagine that we live on the surface of a three-sphere (finite
but boundless universe).
However this might be observable in cross-corelation in the cosmic microwave
background from photons reaching us from different directions but coming from the same event. There is no evidence of such phenomena so far. 
Most models can be excluded to some certainty. A cylindrical
universe is still possible (finite in one, infinite in the
other directions).
\end{sidenote}
\section{Vectors}
Vectors are important objects describing physics. The naive view as an 'arrow
pointing frow one point to another' is flawed.
For example on a sphere an arrow connecting two points does not make much sense.
We want to find a description of vectors as objects that are naturally related to the structure of the manifold independent of the embedding.
\subsection{Definitions}
There are three equivalent definitions for a
(contravariant) vector:
\begin{enumerate}
    \item algebraic 
    \item physical
    \item geometrical.
\end{enumerate}
We start by giving the algebraic definition which is the most
abstract and preferred by mathematicians, because it is suitable for proofs.
Vectors are identified with derivatives, which are formally defined by
\begin{definition}[Derivation]
A derivation $D$ satisfies the following rules, for all $f,g\in
C^\infty(M,\Reals)$ and $\lambda \in \Reals$:
\begin{align}
    D(af+bg) &=Df+Dg\,,\\
    D(\lambda f)&=\lambda f\,,\\
    D(fg)&= (Df)g+f(Dg)\,.
\end{align}
\end{definition}
We then define a vector by 
\begin{definition}[Vector,
algebraic] A vector in $p$ is a derivation on the germ at $p$.
\end{definition}
The germ is the set of all functions $f\in C^\infty(M,\Reals)$, where we
identify all functions that are equal in some neighbourhood of $p$, i.e. vectors are local objects.
Given two vectors we can construct a new one, the \emph{Lie bracket}
\begin{equation}
    [X,Y]f:=X(Yf)-Y(Xf)\, .\label{eq:LieBraket}
\end{equation}
The only property that has to be checked is that it satisfies the Leibniz rule.
\begin{equation}
    XY(fg)=X[(Yf)g+f(Yg)]=(XYf)g+(Yf)(Xg)+(Yg)(Xf)+(XYg)f\\
\end{equation}
Subtracting $YX(fg)$ proves that $[X,Y]$ is indeed a vector. The fact that we
have a natural vector space structure on the set of vectors at $p$ motivates the
following
\begin{definition}
The tangent space $T_pM$ is the space of all vectors in $p\in M$.
\end{definition}
A basis of $T_pM$ is given by the derivation along the
coordinates $\partial_i$, therefore its dimension is equal to that of the manifold $M$.
Proof sketch:
\begin{enumerate}
    \item Show $f(x^i)=f(0)+x^i\tilde{f}(x^i)$
    \item Write $X=a^i\partial_i$
    \item Show $Xf=0\quad \forall f \iff X=0$
\end{enumerate}
Every vector $A$ can be written as $A=A^i\pd{}{{x^i}}$, where $A^i$ are the
components of the vector. We can now look how the components of the vector
transform under a change of coordinates \footnote{The vector itself is
invariant!}.
We usually denote the elements of the transformed systems with a bar.
By the chain rule we have
\begin{equation}
    A= a^k\pd{}{{x^k}}= a^k\pd{\overline{x}^i}{{x^k}}\pd{}{{\overline{x}^k}}\, .
\end{equation}
We can also express $A$ directly in the new basis
\begin{equation}
    A= \overline{a}^i\pd{}{{\overline{x}^i}}\, .
\end{equation}
Comparing the coefficients gives the vector transformation law
\begin{equation}
    \overline{a}^i=a^k\pd{\overline{x}^i}{{x^k}}\label{eq:coefftrafo}\, .
\end{equation}
\begin{definition}[Vector,
physical] A vector with components $A^i$ is a object that transforms according
to \ref{eq:coefftrafo} under a change of coordinates.
\end{definition}
Consider a curve on a Manifold $M$, i.e. a
map $\gamma:\mathbb{R}\to M$, with $\gamma(0)=p$, $\dot{\gamma}(0)=X$. Then $D_X f=\od{}{t}(f\circ\gamma)(0)$ is
a derivative, namely the directional derivative along $X$.
Consider the special curves $\gamma_i(t)=\varphi(p+te_i)$, with $\varphi$ a
chart of $M$. Then $D_{\dot{\gamma}_i} f=\partial_if$,  so $D_{\dot{\gamma}_i}$
represents a the directional derivative and we can relate derivatives to the
geometrical tangent space.

Since we have a basis, we can work in (local) coordinates and will do so most of
the time.
\begin{example}[Lie brackets in local coordinates]
Let $A=A^i\partial_i$, $B=B^i\partial_i$ be vectors, then the \name{Lie} bracket
\eqref{eq:LieBraket} in local coordinates is given as
\begin{equation}
    [A,B]^j=A^i\partial_iB^j-B^i\partial_iA^i\, .
\end{equation}
\end{example}
Since the tangent space is a vector space, we can define its dual space
\begin{definition}[Cotangent space] The cotangent space $T_pM^*$ is the set of
linear maps from $T_pM$ to $\mathbb{R}$, i.e.
\begin{equation}
    T_pM^*:=\{L:T_pM\to \mathbb{R}\, |\, L \text{ linear}\}\, .
\end{equation}
\end{definition}
The cotangent space is again a vector space of the same dimension. Its elements
are called \emph{dual} or \emph{covariant} vectors.
We can define a basis on $	T_pM^*$, which we denote by $\dif x^i$ and  which acts on $T_pM$ via
\begin{equation}
    \dif x^i(\partial_j)=\delta^i_j\, . \label{eq:orthdual}
\end{equation}
It can easily deduced by \eqref{eq:orthdual} that the components of a dual vector transform as
\begin{equation}
    \overline{a}_i=\pd{x^k}{{\overline{x}^i}}a_k\, .
\end{equation}
\begin{remark}[Dual vectors in euclidean space]
If $\vec{a},\vec{b}\in\mathbb{R}^n$ contain the components of a vector and a
dual vector respectively, then the transformation can be written in matrix form
\begin{align}
    \vec{a}&\to\overline{\vec{a}}= V\vec{a}\, ,\\
    \vec{b}&\to\overline{\vec{b}}=\left(V\transpose\right)^{-1}\vec{b}\, ,
\end{align}
with $V_{ij}=\dpd{\overline{x}^i}{{x^j}}$ the
Jacobian of the transformation.
In normal calculus we restrict ourselves to orthogonal transformations (i.e.
mapping orthonormal bases onto each other) for which
$\left(O\transpose\right)^{-1}=O$.
Which is the reason why we do not bother to distinguish between vectors and dual vectors because they transform identically. 
In special relativity we have e.g.
$\left(\Lambda\transpose\right)^{-1}\neq\Lambda$ for a boost and the difference
becomes even more important in general relativity where the relation can become arbitrarily complicated.
\end{remark}
\section{Tensors}
From vectors $A$ ,$B$ we can construct new objects with multiple indices that posses well defined transformation behaviour. 
For example consider
\begin{equation}
    \overline{T}^{ij}=a^ib^j\, ,
\end{equation}
which transforms as
\begin{equation}
    T^{ij}=\pd{\overline{x}^i}{{x^k}}\pd{\overline{x}^j}{{x^l}}a^kb^l=\pd{\overline{x}^i}{{x^k}}\pd{\overline{x}^j}{{x^l}}T^{kl}\,
   .\label{eq:tensortrafo}
\end{equation}
We call an object that transforms in this way a \emph{tensor}. 
As with vectors, it is possible to define tensors in a coordinate independent
way.
At this point we will make things easier and only consider the physical
definition, i.e. classify tensors by a transformation according to \eqref{eq:tensortrafo}.

A tensor is said to be symmetric in two indices if it stays invariant when exchanging those indices, e.g.
\begin{equation}
    T_{ab}=T_{ba}\, .
\end{equation}
\begin{remark}
We have not yet established a relation between upper and lower indices, i.e. we have no metric. Expressions of the form
\begin{equation}
    \tensor{T}{^a_b}=\tensor{T}{^b_a}
\end{equation}
therefore make no sense.
\end{remark}
\section{The Metric}
So far we have not defined a length scale on manifolds yet. We will do so now by
introducing a \emph{metric} 
\begin{definition}[Metric]
A metric $g$ on a manifold $M$, is a non-degenerate ($\det(g)\neq 0$), symmetric
covariant two tensor.
\end{definition}
We have already seen examples of metrics for the flat space, e.g. in spherical coordinates $g$ was given as
\begin{equation}
    g=
    \begin{bmatrix}
        1 & 0\\
        0 & r^2\\
    \end{bmatrix}\,.
\end{equation}
A metric that is positive definite is called \emph{\name{Riemann}ian metric}. In
relativity we deal with \emph{\name{Lorentz}ian metrics}, for which there are
vectors beside the zero vector which have zero 'length'. In flat space we have
$\tensor{g}{_i_j}=\tensor{\eta}{_i_j}$.
A metric gives two natural notions on the tangent space of the manifold.
Inner product 
\begin{equation}
g(A,B)=\tensor{g}{_i_j}\tensor{A}{^i}\tensor{B}{^j}
\end{equation}
Pseudo\footnote{Not positive definite.} norm
\begin{equation}
g(A,A)=\tensor{g}{_i_j}\tensor{A}{^i}\tensor{A}{^j}
\end{equation}
\begin{remark}[About
raising and lowering indices] Suppose we have given a vector $A=A^i\partial_i$
with coordinates $A^i$, then we can relate it 
in a natural way to a linear form $A^\flat\footnote{The symbols
$\flat$ and $\sharp$ are borrowed from musical notation.}:=g(A,\cdot)$,
\begin{equation}
\begin{split}
 A^\flat\,\colon T_pM &\to \Reals\\
 X &\mapsto g(A,X)\, ,
\end{split}
\end{equation}
i.e. a dual vector. The components of this dual vector are given by its action
on the basis elements of the tangent space
\begin{equation}
A_i:=(A^\flat)_i=g(A,\partial_i)=A^jg(\partial_j,\partial_i)=g_{ji}A^j\,,
\end{equation}
which is exactly the law for lowering indices. Given the inverse metric we can
multiply this equation by it to obtain $A^i$ in terms of $A_i$. 
\end{remark}
\section{Affine Connections}
To derive a vector field, we have to relate different tangent spaces. The idea
is again to generalize from flat space. The connection is established by
introducing a affine connection.
\subsection{Parallel Transport}
We consider a the parallel transport of a
vector.
%TODO pictures
If we express a vector in non-Cartesian coordinates and shift it it's
coordinates do not change.
We take a look on two operations:
\begin{enumerate}
\item the change of the vector itself
\item the change of its coordinates.
\end{enumerate}
Let $A_i$ be the coordinates of a vector in a system $x^i$ and $B_i$ in a
system associated with coordinates $y^i$ respectively. They are therefore
related by
\begin{equation}
A_i=\dpd{{y^j}}{{x^i}}B_j\, ,\quad B_i=\pd{{x^j}}{{y^i}}A_j\, .
\end{equation}
We look at vectors whose coordinates in the system $y^i$ do not change i.e.\ 
$\delta B_i=0$. The variation of $A_i$ is given by
\begin{equation}
\delta
A_i=\delta\left(\dpd{{y^j}}{{x^i}}\right)B_j
=\md{{y^j}}{2}{{x^i}}{}{{x^k}}{}\delta
x^k B_j\, .
\end{equation}
Expressing $B_i$ in terms of $A_i$ yields
\begin{equation}
\delta A_i = \md{{y^j}}{2}{{x^i}}{}{{x^k}}{}\pd{{x^l}}{{y^j}}A_l\delta x^k
=:\affin{l}{i}{k}A_l\delta x^k
\end{equation}
$\affin{l}{i}{k}$ is called \emph{affine connection} or short affinity.
\begin{remark}
We can always find a coordinate system in which $\affin{l}{i}{k}\equiv 0$, this
system is called \emph{\name{Riemann}ian normal coordinate system} (RNCS).
\end{remark}
%TODO construction i.e. geodesic coordinates
We notice that if 
\begin{equation}
\left(\dpd{{A_i}}{{x^j}}-\affin{l}{i}{k}A_l\right)\delta x^k = 0\,
,\label{eq:covdev}
\end{equation}
The vector $A$ does not change its coordinates. We define a \emph{covariant
derivative} 
\begin{equation}
\tensor{A}{_i_;_j}:=\tensor{\nabla}{_j}\tensor{A}{_i}:=\pd{{A_i}}{{x^j}}-\affin{l}{i}{k}A_l\,
.
\end{equation}
It can easily be seen that the covariant derivative of a tensor transforms as a
tensor, by inspecting \eqref{eq:covdev} and applying the quotient theorem.
\begin{remark}[The
Covariant Derivative in Electrodynamics] Example from Electrodynamics concerning the covariant derivative. 
The theory is invariant under transformations $\phi\to e^{\imI \alpha}\phi$, 
because $\phi^*\phi$ and $\phi^*\nabla\phi-\phi\nabla\phi^*$ do not change.
In order to make the Lagrangian gauge invariant we exchange
\begin{equation}
\tensor{\partial}{_\mu}\to\tensor{D}{_\mu}+\imI\tensor{A}{_\mu}\, ,
\end{equation}
which effectively produces additional terms in the Lagrangian, namely
\begin{equation}
\tensor{A}{^\mu}\left(\phi^*\tensor{\partial}{_\mu}\phi
-\phi\tensor{\partial}{_\mu}\phi^*\right)=\tensor{A}{^\mu}\tensor{J}{_\mu}\,,
\quad\tensor{A}{_\mu}\tensor{A}{^\mu}\phi^2\,.
\end{equation}
The commutator between the covariant derivatives calculates to  
\begin{equation}
\left[\tensor{D}{_\mu},\tensor{D}{_\nu}\right]=\imI \tensor{F}{_\mu_\nu}\,.
\end{equation}
So the noncommutativity is associated with the presence of a field. This is
similar to GR where it was related with curvature. 
\end{remark}
Since we have now established a relation between vectors and dual vectors, we
can also determine the covariant derivative of a dual vector. Therefore we
consider the scalar $A_iB^i$. Since the covariant derivative satisfies the
Leibniz rule we get
\begin{align}
\tensor{(A_iB^i)}{_{;j}} =
\tensor{A}{_i_;_j}\tensor{B}{^i}+\tensor{A}{_i}\tensor{B}{^i_;_j}\,.
\end{align}
But for scalars the covariant derivative is identical to the normal derivative
so that 
\begin{align}
(A_iB^i)_{;j} =(A_iB^i)_{,j}=
\tensor{A}{_i_,_j}\tensor{B}{^i}+\tensor{A}{_i}\tensor{B}{^i_,_j}
\end{align}
If we put in the covariant derivative of a we get 
\begin{equation}
\tensor{A}{_i}\tensor{B}{^i_;_j}=\tensor{A}{_i}\left(\tensor{B}{^i_{,j}}+\Gamma^i_{kj}\tensor{B}{^k}\right)
\end{equation}
Since $A$ was arbitrary, we can deduce that
\begin{equation}
\tensor{B}{^i_;_j}=\left(\tensor{B}{^i_{,j}}+\Gamma^i_{kj}\tensor{B}{^k}\right)
\end{equation}
for a (1,1)-tensor we get:
\begin{equation}
\tensor{A}{^i_j_;_k}=\tensor{A}{^i_j_{,k}}-\Gamma^a_{jk}\tensor{A}{^i_a}+\Gamma^i_{ak}\tensor{A}{^a_j}\,
.\end{equation}
Similar expressions hold for tensors of arbitrary rank, where each index gives
an additional term containing a contraction with the affinity $\Gamma^i_{jk}$. 
We now want to consider curved spaces. This can not immediately be determined by
the metric, for example the polar coordinates do not look flat even though they
describe the ordinary $\Reals^2$.
\subsection{Geodesics}
A curve is a map $\gamma:\Reals\to M$. The parametrisation is arbitrary e.g.
\begin{equation}
\begin{split}
\gamma:\, &\Reals\to \Reals^2\\
& t\mapsto
\frac{1}{2} t^2
\begin{bmatrix}
1 \\
1
\end{bmatrix}\, ,
\end{split}
\end{equation}
clearly describes a straight line with $y=x$.
However, we notice that $\od{\tensor{\gamma}{^i}}{t}$ is parallel to
$\od[2]{\tensor{\gamma}{^i}}{t}$. This gives rise to another possible
generalisation of a straight line.
In curved coordinates we have 
\begin{equation}
\nabla \left( \dod{\tensor{x}{^j}}{t} \right)  = 
\od[2]{\tensor{x}{^i}}{t}+\affin{i}{j}{k}\od{\tensor{x}{^j}}{t}\od{\tensor{x}{^k}}{t}
=\lambda(t)\od{\tensor{x}{^j}}{t}\,.
\end{equation}
Suppose we have a different parametrisation $s(t)$
\begin{equation}
\od{\tensor{x}{^i}}{t}=\od{\tensor{x}{^i}}{s}\od{s}{t}\,,\quad
\od[2]{\tensor{x}{^i}}{t}=\od[2]{\tensor{x}{^i}}{s}\left(\dod{s}{t}\right)^2
+\od{\tensor{x}{^i}}{s}\dod[2]{s}{t}\, ,
\end{equation} 
then the equation for a straight line reads as
\begin{equation}
\left(\dod[2]{\tensor{x}{^i}}{s}+\affin{i}{j}{k}\dod{\tensor{x}{^j}}{s}\dod{\tensor{x}{^k}}{s}
\right)\left(\dod{s}{t}\right)^2+\dod[2]{s}{t}\dod{\tensor{x}{^i}}{s}
=\lambda(t)\dod{\tensor{x}{^j}}{t}\dod{s}{t}\,.
\end{equation}
To get to the usual form of the geodesic equation we choose $s$ so that
\begin{equation}
\dod[2]{s}{t}=\lambda(t)\dod{s}{t}\label{eq:bedaffpara}
\end{equation}
Which is an ordinary differential equation of type $\ddot{s}=\lambda\dot{s}$
that should posses a solution. 
For this special choice of parametrisation we have
\begin{equation}
\dod[2]{\tensor{x}{^i}}{s}+\affin{i}{j}{k}\dod{\tensor{x}{^j}}{s}\od{\tensor{x}{^k}}{s}
=0\,.\label{eq:affingeod}
\end{equation}
We therefore have a preferred set of parameters
called affine parameters that satisfy \eqref{eq:bedaffpara}. The character
of the differential equation implies that the affine parameter is only
defined modulo affine transformations $s\to as+b$. This freedom once more
reflects some kind of gauge invariance.
\begin{remark}
Only the symmetric part of the affinity does contribute to
the geodesic equation \eqref{eq:affingeod}. As long as you are on a geodesic you
can always compare lengths without needing a metric, by the affine parameter.
The 'length' defined this way does not have to concede with the length given by
the metric and is also defined for example for lightlike curves (which cannot
be parametrised by the arc length).
\end{remark}
If we demand that the two definitions of a geodesic \eqref{eq:affingeod} is
identical to \eqref{eq:geodeq} coincide, we get a preferred
connection.
This corresponds to the choice $\affin{k}{i}{j}=\cSym{k}{i}{j}$.
This is called the \emph{metric}, \emph{Levi-Civita} or \emph{Christoffel
connection}, and we will always choose it in the following.
A general metric compatible
connection\footnote{$\tensor{\nabla}{_k}\tensor{g}{_i_j}=0$} satisfies
\begin{equation}
\affin{k}{i}{j}=\cSym{k}{i}{j}+\tensor{T}{^k_i_j}
+\tensor{g}{^i^r}\left(\tensor{T}{^s_j_r}\tensor{g}{_s_i}+\tensor{T}{^s_i_r}\tensor{g}{_s_j}\right)\,,
\end{equation}
with the torsion tensor
\begin{equation}
\tensor{T}{^k_i_j}=\frac{1}{2}\left(\affin{k}{i}{j}-\affin{k}{j}{i}\right)\,.
\end{equation}
Therefore the Christoffel connection is the unique metric compatible
symmetric connection.
\begin{remark}
Non-Christoffel connections play a role, for example when dealing with spinors.
\end{remark}
\begin{figure}[hbtp!]
\centering
 \includegraphics{fold1.pdf}\qquad\qquad
 \includegraphics{fold2.pdf}\qquad\qquad
 \includegraphics{fold3.pdf}
\caption{The geodesics on a cylinder can be obtained by ``folding'' flat space.}
\end{figure}
%TODO pictures
\section{The Riemannian Curvature Tensor}
%TODO Missing definition of R
We can contract indices on the Riemann Tensor
\begin{equation}
\tensor{R}{^i_i_k_l}=\pd{{\Gamma^i_{il}}}{{x^k}}-\pd{{\Gamma^i_{ik}}}{{x^l}}\,,
\end{equation}
which is zero for a metric affinity. The Ricci tensor is given by
\begin{equation}
\tensor{R}{_j_k}=\tensor{R}{^i_j_k_i}\, .
\end{equation}
Because of the symmetry this are all independent
contractions.
Notice that at this point we cannot raise or lower indices to contract different indices.
Given a metric the curvature or Ricci skalar is defined as 
\begin{equation}
\tensor{R}{^i_i}=\tensor{g}{^i^j}\tensor{R}{_j_i}\, .
\end{equation}
%TODO here we assume we have not metric..
Since we can express the \name{Riemann} tensor in terms of commutators there is
a also a symmetry that can be derived from the Jacobi identity:
\begin{equation}
\left[A\left[B,C\right]\right]
+\left[C\left[A,B\right]\right]
+\left[B\left[C,A\right]\right]=0\,.
\end{equation}
Physics can be described in terms of differential equations. We for example
would like to have a object similar to the Laplacian in curved coordinates.
However $\partial_i\partial_i$ is not coordinate invariant. We therefore
introduce a metric. The equivalence principle implies that space is locally
Minkovski.
\begin{sidenote}
There are certain theories that can be formulated without a metric. An example
being three dimensional gravity, which is non-dynamic. 
\end{sidenote}
Since the connection and the metric are not related, we still don't have a length
scale on the manifold
% \begin{equation}
% \tensor{R}{_i_j_k_l}:=\tensor{g}{_i_a}\tensor{R}{^a_j_k_l}
% \end{equation}
% Only useful if $R$ and $g$ are related i.e. metric connection.
% (Bilder)
% TODO the following is fits better into parallel transport section, we can
% possibly leave out what is written there

\subsection{Symmetries of the Riemann Tensor}
\begin{equation}
\tensor{R}{^i_j_k_l}=-\tensor{R}{^i_j_l_k}\,.
\end{equation}
Bianci identities for the Riemann tensor
\begin{equation}
\tensor{R}{^i_j_k_l_{;m}}+\tensor{R}{^i_j_m_k_{;l}}+\tensor{R}{^i_j_l_m_{;k}}=0
\end{equation}
Proof: If we would write it all out we wold have to write 22 terms. Instead we
use a RNCS so that the Riemann tensor simplifies to 
\begin{equation}
\tensor{R}{^i_j_k_l}
=\tensor{\partial}{_k}\affin{i}{j}{l}
-\tensor{\partial}{_l}\affin{i}{j}{k}
\end{equation}
And with $\tensor{\nabla}{_k}=\tensor{\partial}{_k}$ we get
\begin{equation}
\tensor{R}{^i_j_k_l_{;m}}
=\tensor{\partial}{_m}\tensor{R}{^i_j_k_l}
=\tensor{\partial}{_k}\tensor{\partial}{_m}\affin{i}{j}{l}
-\tensor{\partial}{_l}\tensor{\partial}{_m}\affin{i}{j}{k}\,.
\end{equation}
Plugging in gives the result in the RNCS system, but since the equation is
tensorial, it holds in all systems. Notice the similarity to 
\begin{equation}
\tensor{\partial}{_m}\tensor{F}{_a_b}
=\tensor{\partial}{_b}\tensor{\partial}{_m}\tensor{A}{^a}
-\tensor{\partial}{_a}\tensor{\partial}{_m}\tensor{A}{^b}
\end{equation}
Therefore the second set of Maxwell's
equations \eqref{eq:maxwell_eqs} is in fact a the Bianchi identity.
If the affinity is symmetrical $\affin{k}{i}{j}=\affin{k}{j}{i}$ we further have
the identity
\begin{equation}
\tensor{R}{^i_j_k_l}+\tensor{R}{^i_l_j_k}+\tensor{R}{^i_k_l_j}=0
\end{equation}
Given a metric $\tensor{g}{_i_j}$, we can raise and lower
indices
\begin{equation}
\begin{split}
\tensor{R}{_i_j_k_l}
&=\tensor{g}{_i_a}\tensor{R}{^a_j_k_l}\\
&=\tensor{g}{_i_a}\left(\tensor{\partial}{_k}\affin{i}{j}{l}
-\tensor{\partial}{_l}\affin{i}{j}{k}\right)\\
&=\tensor{g}{_i_a}\left(\tensor{\partial}{_k}\tensor{g}{^a^s}\csym{j}{l}{s}
-\tensor{\partial}{_l}\tensor{g}{^a^s}\csym{j}{k}{s}\right)\\
&=\tensor{\partial}{_k}\csym{j}{l}{i}
-\tensor{\partial}{_l}\csym{j}{k}{i}\\
&=\frac{1}{2}\left(
\dmd{{\tensor{g}{_i_l}}}{2}{\tensor{x}{^j}}{}{\tensor{x}{^k}}{}
+\dmd{{\tensor{g}{_j_k}}}{2}{\tensor{x}{^i}}{}{\tensor{x}{^l}}{}
-\dmd{{\tensor{g}{_i_k}}}{2}{\tensor{x}{^j}}{}{\tensor{x}{^l}}{}
-\dmd{{\tensor{g}{_j_l}}}{2}{\tensor{x}{^i}}{}{\tensor{x}{^j}}{}\right)
\end{split}
\end{equation}
Immediately we can extract symmetries
\begin{align}
\tensor{R}{_i_j_k_l}&=-\tensor{R}{_j_i_k_l}\\
\tensor{R}{_i_j_k_l}&=-\tensor{R}{_i_j_l_k}
\end{align} 
If we further introduce multi-indices $A=(i,j)$, $B=(k,l)$ (by antisymmetry
there are six independent components for $A,B$ each)
\begin{equation}
\tensor{R}{_A_B} = \tensor{R}{_B_A}
\end{equation}
So $R$ can be thought as a $6\times 6$ matrix.
\begin{table}
    \centering
        \caption{Number of independent components of Riemann
    and \name{Ricci} tensor.\label{tab:Nindcomp}}
    \begin{tabulars}{rrr}
      	\toprule
		dimension&Riemann tensor &Ricci tensor \\
		\midrule
		1&0&0\\
		2&1&1\\
		3&6&6\\
		4&20&10\\
		$n$&$\frac{1}{12}n^2(n^2-1)$&$\frac{1}{2}n(n+1)$\\
		\bottomrule
    \end{tabulars}
\end{table}
Table~\ref{tab:Nindcomp} lists the number of independent components of the
curvature tensor and the \name{Ricci} tensor in various dimensions. Thereby a
one dimensional space is always flat, a two dimensional is characterised by the
curvature scalar $R$ alone and in three dimensions the Ricci tensor is
sufficient to know the Riemann tensor. Therefore four is the lowest dimension in
which the Riemann tensor contains additional information of the curvature of
space.
