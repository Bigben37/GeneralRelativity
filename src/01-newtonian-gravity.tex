\chapter{Newtonian Gravity}
In \name{Newton}ian physics we assume that we have absolute space and time
that can be described by the a set of numbers $x^1,x^2,x^3,t$.
We express the coordinates as functions of time.
\section{Forces}
The force $\vec{F}_{AB}$, which a massive body $A$ with mass $m_A$ exerts on another massive body $B$ with mass $m_B$, is given by
\begin{equation}
    \vec{F}_{AB}=-m_B\frac{G\textsubscript{N}m_A}{r^2}\vec{e}_r\, ,
\end{equation}
%TODO remove subscript of G
where $G\textsubscript{N}$ denotes \emph{\name{Newton}'s constant},
numerically equal to $G\textsubscript{N}\approx \unitfrac[6.673\cdot 10^{-11}]{m^3}{kg\,s}$.
Although there is no need for $G\textsubscript{N}$ to be constant over time,
there is evidence that the relative variation is less than $10^{-12}$ per year.
The force can be expressed in terms of \emph{gravitational potential} $\Phi$:
\begin{equation}
    \vec{F}_{AB}=-m_B\nabla\left(-\frac{G\textsubscript{N}m_A}{r}\right)=:
    -m_B\nabla\Phi(\vec{r}_B)\, .
\end{equation}
Given $N$ particles labeled by $n$, the total force $B$ experiences is
\begin{equation}
    \vec{F}_{B}=-\sum_n \vec{F}_{nB}\, .
\end{equation}
The potential at $\vec{r}$ is then easily found to be
\begin{equation}
    \Phi(\vec{r})=-G\textsubscript{N}\sum_n\frac{m_n}{|\vec{r}-\vec{r}_n|}\, .
\end{equation}
In general, we assume a mass distribution $\varrho(\vec{r})$ and the sum is
replaced by an integral:
\begin{equation}
    \Phi(\vec{r}) = -G\textsubscript{N}\int_{\Reals^3}\dif{\vec{r}^{\prime}}
    \frac{\varrho(\vec{r}^{\prime})}{|\vec{r}-\vec{r}^{\prime}|}\,.
\end{equation}
\section{Comparison with electrostatics}
The classical theory of gravity bears a striking similarity to electrostatics.
To make this clearer, we introduce the gravitational field
$\vec{g}(\vec{r}):= -\nabla\Phi(\vec{r})$.
\begin{table}[htb]
    \caption{Comparison of electrostatics and \name{Newton}ian gravity.}
    \begin{center}
        \begin{tabulars}{lll}
            \toprule
            &\name{Newton}ian Gravity&Electrostatics\\
            \midrule
            Force&$\displaystyle\vec{F}=q\frac{kQ}{r^2}\vec{e}_r$&$\vec{F}=m\frac{G\textsubscript{N}M}{r^2}\vec{e}_r$\\
            Potential
            &$\Phi\textsubscript{el}(\vec{r})=q\frac{kQ}{r}$
            &$\Phi\textsubscript{g}(\vec{r})=m\frac{G\textsubscript{N}M}{r}$\\
            Field
            &$\vec{E}(\vec{r})=-\nabla\Phi\textsubscript{el}(\vec{r})$
            &$\vec{g}(\vec{r})=-\nabla\Phi\textsubscript{g}(\vec{r})$\\
            \name{Laplace} equation
            &$\Delta\Phi\textsubscript{el}=-4\pi k\varrho\textsubscript{el}(\vec{r})$&
            $\Delta\Phi\textsubscript{g}=4\pi
            G\textsubscript{N}\varrho\textsubscript{g}(\vec{r})$
            \\
            \bottomrule
        \end{tabulars}
    \end{center}
\end{table}
\begin{example}[Field of a spherical mass distribution]
Assume we have a spherical mass distribution, i.e.\ $\varrho(\vec{r})=\varrho(r)$.
By symmetry considerations it follows, that the gravitational field can be
expressed by
\begin{equation}
    \vec{g}(\vec{r})=g(r)\vec{e}_r\, .
\end{equation}
We integrate the divergence of the field over a ball $B$ of radius $r$
\begin{equation}
    \int_B\dif{\vec{r}}\,\nabla\vec{g}=-\int_B\dif{\vec{r}}\, \Delta\Phi
    = -4\pi G\textsubscript{N}\int_B\dif{\vec{r}}\,\varrho(r)= -4\pi
    G\textsubscript{N} M\, ,
\end{equation}
where $M$ is the mass enclosed in $B$. On the other hand we can use Gauss's theorem to deduce
\begin{equation}
    \int_B\dif{\vec{r}}\,\nabla\vec{g}=\oint_{\partial B}\dif{\vec{A}}\cdot
    \vec{g} = \oint_{\Omega}\dif{\Omega}\, g(r)r^2=4\pi r^2g(r)\, .
\end{equation}
Together the gravitational field is given by
\begin{equation}
    \vec{g}(r)=-\frac{G\textsubscript{N}M}{r^2}\vec{e}_{r}\, .
\end{equation}
\end{example}
% \subsection{Inertial systems}
% \begin{definition}
% An inertial system is a system in which force-free particles move with constant uniform velocity on straight lines.
% \end{definition}
\subsection*{Weak Equivalence Principle (WEP)}
\name{Newton}'s first law reads
\begin{equation}
    \vec{F}=m\textsubscript{I} \vec{\ddot{x}} \, ,
\end{equation}
where $m\textsubscript{I}$ is the inertial mass that works against the acceleration of the body.
The force which a body with ``active'' mass $m\textsubscript{g,a}$ exerts on
another body with mass $m\textsubscript{g,p}$ is given by 
\begin{equation}
    \vec{F}=m\textsubscript{g,p}\frac{G\textsubscript{N}m\textsubscript{g,a}}{r^2}\vec{e}_r \, .
\end{equation}
A priori, there is no reason to assume any relation between this masses.
The first question one might ask is whether the active and the passive mass are equal.
Suppose we have two masses $A$ and $B$. Using \name{Newton}'s first law, we
can explicitly write
\begin{align}
    m^{B}_{\text{I}}\vec{\ddot{x}}&=\vec{F}_{AB}=-
    m^{B}_{\text{g,p}}\frac{G\textsubscript{N}m^{A}_{\text{g,a}}}{r^2}\vec{e}_r
    \, ,\\
    m^{A}_{\text{I}}\vec{\ddot{x}}&=\vec{F}_{BA}=-
    m^{A}_{\text{g,p}}\frac{G\textsubscript{N}m^{B}_{\text{g,a}}}{r^2}\vec{e}_r
    \, .
\end{align}
By the third law $\vec{F}_{AB}=-\vec{F}_{BA}$ we have
\begin{equation}
\frac{m^{B}_{\text{g,p}}}{m^{B}_{\text{g,a}}}=\frac{m^{B}_{\text{g,p}}}{m^{B}_{\text{g,a}}}\,.
\end{equation}
By proper choice of mass units we can set this quotient to one so that 
\begin{equation}
m\textsubscript{g,a}=m\textsubscript{g,p}=:m\textsubscript{g}
\end{equation}
The next question is wheater the inertial mass equivalent to the gravitional
mass.
By \name{Newton}'s first law we have
\begin{equation}
m\textsubscript{I}\vec{\ddot{x}}
=-m_{\text{g}}\frac{G\textsubscript{N}M_{\text{g}}}{r^2}\vec{e}_r 
=-m_{\text{g}}\vec{g}\,.
\end{equation}
As a experimental result that has been measured up to a high accuracy (compare
tabular~\ref{tab:WEPExp}) all bodys recive the same acceleration due to gravity
$\ddot{\vec{x}}\sim \vec{g}$. By a 'proper' choice of units of the flight-time $t=\sqrt{\frac{m\textsubscript{I}}{m\textsubscript{g}}}\sqrt{\frac{2h}{g}}$,
we get
\begin{equation}
m\textsubscript{I}=m\textsubscript{g}=:m\,.
\end{equation}
\begin{table}
\centering
\begin{tabulars}{rllr}
\toprule
Year&Experimenter&Experiment&Accuracy\\
\midrule
1636&\name{Galilei}&inclined planes&$10^{-2}$\\
1689&\name{Newton}&pendulum&$10^{-3}$\\
1832&\name{Bessel}&pendulum&$10^{-5}$\\
1922&\name{Eötvös}&pendulum&$10^{-9}$\\
1922&\name{Shapro} et al.&pendulum&$10^{-12}$\\
1999&\name{Baesler}&torsion balance&$10^{-14}$\\
\bottomrule
\end{tabulars}
\caption{Experiments measuring the ratio
$\frac{m\textsubscript{I}}{m\textsubscript{g}}$.\label{tab:WEPExp}}
\end{table}
\subsection{Tidal Forces}
Assume we have a body of finite extention in a gravitationational
potential $\Phi$, an example beeing the earth in the potential of the moon.
On the center of the body we have 
%TODO masses???
\begin{equation}
\dod[2]{\tensor{x}{^i}}{t}=-\dpd{\Phi}{\tensor{x}{_i}}\,.
\end{equation}
If we consider a point shifted by $\tensor{\chi}{^i}$ from the center then the
acceleration is given as
\begin{equation}
\begin{split}
\dod[2]{}{t}\left(\tensor{x}{^i}+\tensor{\chi}{^i}\right)
&=-\dpd{\Phi\left(\tensor{x}{^i}+\tensor{\chi}{^i}\right)}{\tensor{x}{_i}}\\
&\simeq-\dpd{\Phi\left(\tensor{x}{^i}\right)}{\tensor{x}{_i}}
-\dmd{\Phi\left(\tensor{x}{^i}\right)}{2}{\tensor{x}{_i}}{}{\tensor{x}{_j}}{}\tensor{\chi}{^j}\,.
\end{split}
\end{equation}
Subtracting the previous equations yields the tidal force
\begin{equation}
\dod[2]{\tensor{\chi}{^i}}{t}=-\dmd{\Phi\left(\tensor{x}{^i}\right)}{2}{\tensor{x}{_i}}{}{\tensor{x}{_j}}{}\tensor{\chi}{^j}\,.
\end{equation}
The tidal force tensor $\dmd{\Phi}{2}{\tensor{x}{_i}}{}{\tensor{x}{_j}}{}$ is of
the form
\begin{equation}
\dmd{\Phi}{2}{\tensor{x}{_i}}{}{\tensor{x}{_j}}{}
=\frac{G\textsubscript{N}M}{r^3}\left(\tensor{\delta}{_i_j}-3\frac{\tensor{x}{_i}\tensor{x}{_j}}{r^2}\right)\,.
\end{equation}
%TODO newtons laws?
%TODO picture
%TODO finish chapter 1