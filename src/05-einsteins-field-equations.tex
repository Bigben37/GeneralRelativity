\chapter{Einstein's Field Equations}
We will derive Einsteins Equations by physical considerations. The poisson
equation reads as:
\begin{equation}
\Delta\Phi=4\pi\rho
\end{equation}
\begin{equation}
S\textsubscript{EH}=\int\dif x^4 \sqrt{-g}(R-2\Lambda)
\end{equation}
In SR, the energy impuls tensor $\tensor{T}{_\mu_\nu}$ is conserved i.e.
$\tensor{T}{_\mu_\nu^{,\nu}}=0$.As a natural extension we demand that the energy
impuls tensor of general relativity is \emph{covariantly} conserved 
\begin{equation}
\tensor{\nabla}{^\nu}\tensor{T}{_\mu_\nu}=\tensor{T}{_\mu_\nu^{;\nu}}=0
\end{equation}
We now ask for the most general tensor satisfying this equation. 
\begin{theorem}[Lovelock]
For a fourdimensional space the most general divergence free tensor
$\tensor{A}{_\mu_\nu}$ is given by
\begin{equation}
\tensor{A}{_\mu_\nu}= c_1\tensor{G}{_\mu_\nu}+c_2\tensor{g}{_\mu_\nu}\, .
\end{equation}
Where $\tensor{G}{_\mu_\nu}$ is the \emph{Einstein tensor}
$\tensor{G}{_\mu_\nu}:=\tensor{R}{_\mu_\nu}-\frac{1}{2}\tensor{g}{_\mu_\nu}R$.
\end{theorem}
The Theorem imediatly implies that 
\begin{equation}
\tensor{R}{_\mu_\nu}-\frac{1}{2}R\tensor{g}{_\mu_\nu}-\Lambda\tensor{g}{_\mu_\nu}=\kappa\tensor{T}{_\mu_\nu}
\end{equation}
for some constants $\kappa$, $\Lambda$.
Of course we identify $\kappa=\frac{8\pi}{c^2}$ and $\Lambda$ is the
cosmological constant. We can rewrite equation (??) as
\begin{equation}
\tensor{R}{_\mu_\nu}-\frac{1}{2}R\tensor{g}{_\mu_\nu}
=\kappa\left(\tensor{T}{_\mu_\nu}-\frac{\Lambda}{\kappa}\tensor{g}{_\mu_\nu}\right)
\end{equation}
so that the left hand side represents the geometrical part and the right hand
side the matter content. 
Wheeler: ``Geometry tells matter how to move, matter tells geometry how to
curve.''
\begin{sidenote}
In the time of the inflation the cosmological constant must have been large.
Since it is small today it has do decay with time
\end{sidenote}
We can identfy $\Lambda$ with an vacuum energy so that. Can we get the Einstein
equations from a variation principle?
\begin{equation}
S\textsubscript{g}=\int\dif x^4 \sqrt{-g}\tilde{\mathcal{L}}
\end{equation}
$\tilde{\mathcal{L}}$ must transform a (scalar) density, therefore we define a
scalar $\sqrt{-g}\mathcal{L}=\mathcal{L}$
\begin{equation}
S\textsubscript{g}=\int\dif x^4 \sqrt{-g}\mathcal{L}
\end{equation}
One can think of various contributions to $\mathcal{L}$
\begin{equation*}
R,\, \square
R,\,\tensor{\nabla}{^\mu}\tensor{\nabla}{^\mu}\tensor{R}{_\mu_\nu},\,
\tensor{R}{_\mu_\nu}\tensor{R}{^\mu^\nu}
,\,\tensor{R}{_\mu_\nu_\sigma_\rho}\tensor{R}{^\mu^\nu^\sigma^\rho}\dots
\end{equation*}
Which have to be contracted so that the resulting quantity becomes a scalar.
We have no contributions of the metric because
$\tensor{g}{_\mu_\nu_{;\sigma}}=0$. From Yang-Mills theory one would expect a
structure 
\begin{equation}
\mathcal{L}\sim\tensor{F}{_\mu_\nu}\tensor{F}{^\mu^\nu}
\end{equation}
but $\Gamma$ is not the fundamental field but $g$ is. If we demand that we only
have up to second derivatives of $g$ the only allowed term in the Lagrangian is
$R$.
\begin{sidenote}[On higher derivatives]
If we include higher order derivatives of $g$ in the right way we can make the
resulting theory renormalisable. However we violate unitarity and introduce so
called ghost fields which are associated with the additional degrees of freedom
we get.
\end{sidenote}
\begin{remark}[Dimensions]
In natural units the line element $\dif s^2$ has dimension ${[\dif
s^2]=\textrm{M}^{-2}}$, Since further
${\left[\tensor{x}{^\mu}\right]=\textrm{M}^{-1}}$ the Lagrange density must have Dimension ${\left[\tilde{\mathcal{L}}\right]=\textrm{M}^{4}}$
\end{remark} 
This constraint leads to the \emph{Einstein-Hilbert-action}
\begin{equation}
S\textsubscript{EH}=\frac{1}{2\kappa}\int\dif x^4 \sqrt{-g}(R-2\Lambda)
\end{equation}
We now check that its variation indeed reproduces Einsteins equations. To do
so we introduce the formalism of \emph{functional derivative}. Let therefore
$\Phi=\{\}$ be a collection of fields. $F[\Phi]$ a functional. We define the
variation of $F$ as
\begin{equation}
\delta F:=\int \dif x \frac{\delta F}{\delta\Phi^i}\delta\Phi^i\,.
\end{equation} 
Typically the functionals are givenn in the form 
\begin{equation}
S[\Phi]=\int \dif x L(x,\Phi)\, ,
\end{equation}
where $L$ is some local function.
\begin{equation}
\frac{\delta\tensor{g}{_\rho_\sigma}(x)}{\delta\delta\tensor{g}{_\mu_\nu}(x')}=\tensor*{\delta}{*^\mu*^\nu*_\rho*_\sigma}\delta(x,x')\
\end{equation}
Where
$\tensor*{\delta}{*^\mu*^\nu*_\rho*_\sigma}=\frac{1}{2}\left(\tensor*{\delta}{^\nu_\rho}\tensor*{\delta}{^\mu_\sigma}+\tensor*{\delta}{^\mu_\rho}\tensor*{\delta}{^\nu_\sigma}\right)$
is the unity of the space of symmetric rank two tensors
\begin{remark}
In general $\delta(x,x')\neq \delta(x-x')$
\end{remark}
\section{Introduction of Matter}
In the gravitational kontext we mean be \emph{matter} any non gravitational
fields this include scalar fields $\varphi$, spinor fields $\Psi$, gauge fields
$\tensor{A}{^\mu}$,\dots. We collect all of them in a multivariable $\Phi$
A local action can be written as 
\begin{equation}
S\textsubscript{m}[\Phi,g]=\int \dif x^4\sqrt{-g}
L\textsubscript{m}\left(\Phi,\tensor{\nabla}{_\mu}\Phi,g\right)
\end{equation}
$\tensor{g}{^\mu^\nu}$ appears in $L\textsubscript{m}$ because the derivatives
$\tensor{\nabla}{_\mu}, \tensor{\partial}{_\mu}$ must be contracted.
Additionally it enters via $\sqrt{-g}$
\begin{example}[free scalar field in Minkowski space]
\begin{equation}
S\textsubscript{m}=\int \dif x^4 \left(-\frac{1}{2}\tensor{\eta}{^\mu^\nu}
\tensor{\partial}{_\mu}\varphi\tensor{\partial}{_\nu}\varphi-\frac{1}{2}m^2\varphi^2\right)
\end{equation}
The minus sign in front of the partial derivative should come as no surprise
since we have $\tensor{\eta}{^0^0}=-1\, \dot{\varphi}^2>0$. In a non inertial
frame we have to make the usual replacements 
\begin{equation}
\tensor{\eta}{_\mu_\nu}\to \tensor{g}{_\mu_\nu}\, , \quad
\tensor{\partial}{_\mu}\to
\tensor{\nabla}{_\mu}\, \quad \dif x^4\to \dif x^4\sqrt{-g}\, .
\end{equation}
\end{example}
\emph{minimal cuppling description} (Kontext???) The action reads as 
\begin{equation}
S\textsubscript{m}=\int \dif x^4 \sqrt{-g}\left(-\frac{1}{2}\tensor{g}{^\mu^\nu}
\tensor{\nabla}{_\mu}\varphi\tensor{\nabla}{_\nu}\varphi-\frac{1}{2}m^2\varphi^2\right)
\end{equation}
which is the action for a scalar $\varphi$ in the presence of gravity, i.e. a
dynamical $\tensor{g}{_\mu_\nu}(x)$. The combined action of scalar field and
gravity is given as 
\begin{equation}
S[g,\varphi]=S\textsubscript{g}[g]+S\textsubscript{m}[g,\varphi]
\end{equation}
The Euler Lagange Equations in terms of the scalar field read as 
\begin{equation}
\frac{\delta S[g,\varphi]}{\delta
\varphi\left(x'\right)}=\frac{\delta S\textsubscript{m}[g,\varphi]}{\delta \varphi\left(x'\right)}
\end{equation}

\begin{equation}
\frac{\delta S\textsubscript{m}[g,\varphi]}{\delta \varphi\left(x'\right)}=\int
\dif x^4\sqrt{-g}\left[-\tensor{g}{^\mu^\nu}
\tensor{\nabla}{_\mu}\varphi\tensor{\nabla}{_\nu}\left(\frac{\delta
\varphi(x)}{\delta \varphi\left(x'\right)}\right)-m^2\varphi\frac{\delta
\varphi(x)}{\delta \varphi\left(x'\right)}\right]
\end{equation}
where we used that the $\delta$ and $\tensor{\nabla}{_\mu}$ commute. Partial
integration yields 
\begin{equation}
\begin{split}
\frac{\delta S\textsubscript{m}[g,\varphi]}{\delta \varphi\left(x'\right)}&=\int
\dif
x^4\sqrt{-g}\left[\square_{g}-m^2\right]\varphi\delta(x,x')\\
&=\sqrt{-g}\left[\square_{g}-m^2\right]\varphi
\end{split}
\end{equation}
where $\square_{g}:=\tensor{g}{^\mu^\nu}
\tensor{\nabla}{_\mu}\tensor{\nabla}{_\nu} $ is the
\emph{Laplace–Beltrami operator}, a generalisation of the ordinary laplacian to
curved space.
Demanding that the variation with respect to $\varphi$ vanishes implies the
\emph{Klein-Gordon equation}
\begin{equation}
\left(\square_g-m^2\right)\varphi=0
\end{equation}
Variing with respect to the metric $\tensor{g}{_\mu_\nu}$
\begin{equation}
\frac{\delta S[g,\varphi]}{\delta
\tensor{g}{_\mu_\nu}\left(x'\right)}=
\frac{\delta S\textsubscript{g}[g]}{\delta
\tensor{g}{_\mu_\nu}\left(x'\right)}
+\frac{\delta S\textsubscript{m}[g,\varphi]}{\delta
\tensor{g}{_\mu_\nu}\left(x'\right)}
=\frac{\sqrt{-g}}{2\kappa}\left(\tensor{G}{^\mu^\nu}+\Lambda\tensor{g}{^\mu^\nu}
\right)+\frac{\delta S\textsubscript{m}[g,\varphi]}{\delta
\tensor{g}{_\mu_\nu}\left(x'\right)}
\end{equation}
This makes it convenient to define the energy-momentum tensor (Vorzeichen????)
\begin{equation}
\tensor{T}{^\mu^\nu}:=\frac{2}{\sqrt{-g}}\frac{\delta
S\textsubscript{m}[g,\varphi]}{\delta \tensor{g}{_\mu_\nu}\left(x'\right)}\,.
\end{equation}
\begin{remark}
Attention $\delta_g
\tensor{g}{^\mu^\nu}=-\tensor{g}{^\rho^{(\mu}}\tensor{g}{^{\nu)}^\sigma}\delta_g
\tensor{g}{_\mu_\nu}$
\end{remark}
We can now proceed in calculating the quantity we have ust introduced for a
scalar field
\begin{equation}
\begin{split}
\frac{\delta
S\textsubscript{m}[g,\varphi]}{\delta \tensor{g}{_\mu_\nu}\left(x'\right)}\\
&=\int \dif x^4 \frac{\delta\sqrt{-g}}{\delta
\tensor{g}{_\mu_\nu}}\left(-\frac{1}{2}\tensor{g}{^\mu^\nu}
\tensor{\nabla}{_\mu}\varphi\tensor{\nabla}{_\nu}\varphi-\frac{1}{2}m^2\varphi^2\right)
+ \sqrt{-g}\left(-\frac{1}{2}\tensor{g}{^\alpha^\rho}\tensor{g}{^\beta^\sigma}
\tensor{\nabla}{_\alpha}\varphi\tensor{\nabla}{_\beta}\varphi\frac{\delta
\tensor{g}{_\rho_\sigma}}{\delta \tensor{g}{_\mu_\nu}}\right)\\
&=\frac{1}{2}\int \dif x^4
\sqrt{-g}\left(-\frac{1}{2}\tensor{g}{^\mu^\nu}\tensor{\nabla}{_\rho}\varphi\tensor{\nabla}{^\rho}\varphi-\frac{1}{2}\tensor{g}{^\mu^\nu}m^2\varphi^2
+\tensor{\nabla}{^\mu}\varphi\tensor{\nabla}{^\nu}\varphi\right)\delta(x,x')\\
&=\frac{1}{2}\sqrt{-g}\left(-\frac{1}{2}\tensor{g}{^\mu^\nu}\tensor{\nabla}{_\rho}\varphi\tensor{\nabla}{^\rho}\varphi-\frac{1}{2}\tensor{g}{^\mu^\nu}m^2\varphi^2
+\tensor{\nabla}{^\mu}\varphi\tensor{\nabla}{^\nu}\varphi\right)
\end{split}
\end{equation}
So that 
\begin{equation}
\tensor{T}{^\mu^\nu}(\varphi)=
\left(-\frac{1}{2}\tensor{g}{^\mu^\nu}\tensor{\nabla}{_\rho}\varphi\tensor{\nabla}{^\rho}\varphi-\frac{1}{2}\tensor{g}{^\mu^\nu}m^2\varphi^2
+\tensor{\nabla}{^\mu}\varphi\tensor{\nabla}{^\nu}\varphi\right)
\end{equation}
As we have noticed, the Einstein Tensor is covariantly conserved (contracted
Bianci identities). The Einstein equation then implies that also
$\tensor{T}{_\mu_\nu^{;\nu}}=0$
