\chapter{Einstein's Field Equations}
We will derive Einsteins Equations by physical considerations. The poisson
equation reads as:
\begin{equation}
\Delta\Phi=4\pi\varrho\, ,
\end{equation}
So matter (energy) is the source of the gravitational field $\Phi$. 
We now ask for the most general tensor $\tensor{S}{_\mu_\nu}$, related to the
geometry, so that we can express
\begin{equation}
\tensor{T}{_\mu_\nu}=\tensor{S}{_\mu_\nu}
\,.
\end{equation}
In SR, the energy impuls tensor $\tensor{T}{_\mu_\nu}$ is conserved i.e.
$\tensor{T}{_\mu_\nu^{,\nu}}=0$ .As a natural extension, we demand that the
energy impuls tensor of general relativity is \emph{covariantly} conserved
\begin{equation}
\tensor{\nabla}{^\nu}\tensor{T}{_\mu_\nu}=\tensor{T}{_\mu_\nu^{;\nu}}=0
\end{equation}
\begin{theorem}[Lovelock]
For a fourdimensional space the most general divergence free tensor
$\tensor{A}{_\mu_\nu}$ is given by
\begin{equation}
\tensor{A}{_\mu_\nu}= c_1\tensor{G}{_\mu_\nu}+c_2\tensor{g}{_\mu_\nu}\, .
\end{equation}
Where $\tensor{G}{_\mu_\nu}$ is the \emph{Einstein tensor}
$\tensor{G}{_\mu_\nu}:=\tensor{R}{_\mu_\nu}-\frac{1}{2}\tensor{g}{_\mu_\nu}R$.
\end{theorem}
The Theorem imediatly implies \emph{Einstein's field equations}
\begin{equation}
\tensor{R}{_\mu_\nu}-\frac{1}{2}R\tensor{g}{_\mu_\nu}-\Lambda\tensor{g}{_\mu_\nu}
=\kappa\tensor{T}{_\mu_\nu}\, ,\label{eq:EinstFG}
\end{equation}
with some constants $\kappa$, $\Lambda$.
Of course we identify Einsteins constant $\kappa=\frac{8\pi}{c^2}$ and the
cosmological constant $\Lambda$. As a slight variation, we can rewrite equation
\eqref{eq:EinstFG} as
\begin{equation}
\tensor{R}{_\mu_\nu}-\frac{1}{2}R\tensor{g}{_\mu_\nu}
=\kappa\left(\tensor{T}{_\mu_\nu}-\frac{\Lambda}{\kappa}\tensor{g}{_\mu_\nu}\right)
\end{equation}
so that the left hand side represents the geometrical part and the right hand
side the matter content and we identfy $\Lambda$ with an vacuum energy.
Wheeler condenses this in the statement:
\begin{quote}
Geometry tells matter how to move, matter tells geometry how to
curve.
\end{quote}
\begin{sidenote}
In the time of the inflation the cosmological constant must have been large.
Since it is small today it has do decay with time.
\end{sidenote}
There is also an variational derivation dating back to D.Hilbert that is simpler
than Einsteins initial derivation. We start by considereing a general action
\begin{equation}
S\textsubscript{g}=\int\dif x^4 \tilde{\mathcal{L}}\, 
\end{equation}
where $\tilde{\mathcal{L}}$ must transform as a (scalar) density. Therefore we
define a scalar $\mathcal{L}=\frac{\tilde{\mathcal{L}}}{\sqrt{-g}}$
\begin{equation}
S\textsubscript{g}=\int\dif x^4 \sqrt{-g}\mathcal{L}
\end{equation}
One can think of various contributions to $\mathcal{L}$
\begin{equation*}
R,\, \square
R,\,\tensor{\nabla}{^\mu}\tensor{\nabla}{^\mu}\tensor{R}{_\mu_\nu},\,
\tensor{R}{_\mu_\nu}\tensor{R}{^\mu^\nu}
,\,\tensor{R}{_\mu_\nu_\sigma_\varrho}\tensor{R}{^\mu^\nu^\sigma^\varrho}\dots\,,
\end{equation*}
which have to be contracted so that the resulting quantity becomes a scalar.
We have no contributions of the metric because
$\tensor{g}{_\mu_\nu_{;\sigma}}=0$. From Yang-Mills theory one would expect a
structure
\begin{equation}
\mathcal{L}\sim\tensor{F}{_\mu_\nu}\tensor{F}{^\mu^\nu}
\end{equation}
but $\Gamma$ is not the fundamental field but $g$ is. If we demand that we only
have up to second derivatives of $g$ the only allowed term in the Lagrangian is
$R$.
\begin{sidenote}[On higher derivatives]
If we include higher order derivatives of $g$ in the right way we can make the
resulting theory renormalisable. However we violate unitarity and introduce so
called ghost fields which are associated with the additional degrees of freedom
we get.
\end{sidenote}
\begin{remark}[Dimensions]
In natural units the line element $\dif s^2$ has dimension ${[\dif
s^2]=\textrm{M}^{-2}}$, Since further
${\left[\tensor{x}{^\mu}\right]=\textrm{M}^{-1}}$ the Lagrange density must have Dimension ${\left[\tilde{\mathcal{L}}\right]=\textrm{M}^{4}}$
\end{remark}
This constraint leads to the \emph{Einstein-Hilbert-action}
\begin{equation}
S\textsubscript{EH}=\frac{1}{2\kappa}\int\dif x^4 \sqrt{-g}(R-2\Lambda)
\end{equation}
We now check that its variation indeed reproduces Einsteins equations. To do
so we introduce the formalism of \emph{functional derivative}. Let therefore
$\Phi=\{\varphi,\tensor{A}{^\mu},\Psi,\dots\}$ be a collection of fields.
$F[\Phi]$ a functional.
We define the variation of $F$ as
\begin{equation}
\delta F:=\int \dif x \frac{\delta F}{\delta\Phi^i}\delta\Phi^i\,.
\end{equation}
Typically the functionals are givenn in the form
\begin{equation}
S[\Phi]=\int \dif x L(x,\Phi)\, ,
\end{equation}
where $L$ is some local function.
\begin{equation}
\frac{\delta\tensor{g}{_\varrho_\sigma}(x)}{\delta\tensor{g}{_\mu_\nu}(x')}=\tensor*{\delta}{*^\mu*^\nu*_\varrho*_\sigma}\delta(x,x')\
\end{equation}
Where
$\tensor*{\delta}{*^\mu*^\nu*_\varrho*_\sigma}=\frac{1}{2}\left(\tensor*{\delta}{^\nu_\varrho}\tensor*{\delta}{^\mu_\sigma}+\tensor*{\delta}{^\mu_\varrho}\tensor*{\delta}{^\nu_\sigma}\right)$
is the unity of the space of symmetric rank two tensors
\begin{remark}
In general $\delta(x,x')\neq \delta(x-x')$
\end{remark}
We transform to the origin of an RNKS, so that the christoffel symbols vanish.
In that coordinate system the kovariant and the partial derivative coincide:
$\tensor{\partial}{_\mu}=\tensor{\nabla}{_\mu}$.
\begin{equation}
\begin{split}
\delta \tensor{R}{^\rho_\mu_\nu_\sigma}
&=\delta \tensor{\partial}{_\nu}\cSym{\rho}{\mu}{\sigma}
-\delta \tensor{\partial}{_\mu}\cSym{\rho}{\nu}{\sigma}\\
&=\tensor{\partial}{_\nu}\delta \cSym{\rho}{\mu}{\sigma}
-\tensor{\partial}{_\mu}\delta \cSym{\rho}{\nu}{\sigma}\\
\end{split}
\end{equation}
Attention: i.A. $\delta\tensor{\partial}{_\mu}\neq\tensor{\partial}{_\mu}\delta$
\begin{equation}
\begin{split}
\delta \tensor{R}{_\mu_\nu}
&=\delta \tensor{R}{^\rho_\mu_\rho_\nu}\\
&=\tensor{\partial}{_\rho}\delta \cSym{\rho}{\mu}{\nu}
-\tensor{\partial}{_\mu}\delta \cSym{\rho}{\rho}{\nu}\\
&=\tensor{\nabla}{_\rho}\delta \cSym{\rho}{\mu}{\nu}
-\tensor{\nabla}{_\mu}\delta \cSym{\rho}{\rho}{\nu}\\
\end{split}
\end{equation}
This holds in a general frame since it is a tensor equation.
\begin{equation}
\begin{split}
\delta R &=\delta \left(\tensor{g}{^\mu^\nu}\tensor{R}{_\mu_\nu}\right)\\
&=\tensor{R}{_\mu_\nu}\delta\tensor{g}{^\mu^\nu}
+\tensor{g}{^\mu^\nu}\delta\tensor{R}{_\mu_\nu}\\
\end{split}
\end{equation}
Use 
\begin{align}
\delta\tensor{g}{^\mu^\nu}
&=-\tensor{g}{^\varrho^{(\mu}}\tensor{g}{^{\nu)}^\sigma}\delta\tensor{g}{_\mu_\nu}\\
\delta\det g&=\det g \tensor{g}{^\mu^\nu}\delta
\tensor{g}{_\mu_\nu}
\end{align}
\begin{equation}
\begin{split}
2\kappa\delta S\textsubscript{EH}
&=\int\dif x^4\left[
(R-2\Lambda)\delta\sqrt{-g}+\sqrt{-g}\delta R\right]\\
&=\int\dif x^4\left[
\frac{1}{2}\sqrt{-g}\tensor{g}{^\mu^\nu}\delta\tensor{g}{_\mu_\nu}
(R-2\Lambda)+\sqrt{-g}\left(\tensor{R}{_\mu_\nu}\delta\tensor{g}{^\mu^\nu}
+\tensor{g}{^\mu^\nu}\delta\tensor{R}{_\mu_\nu}\right)\right]\\
&=\int\dif x^4\sqrt{-g}\left[
\frac{1}{2}\tensor{g}{^\mu^\nu}
(R-2\Lambda)+\tensor{R}{^\mu^\nu}\right]\delta\tensor{g}{_\mu_\nu}
+\int\dif x^4\sqrt{-g}\tensor{g}{^\mu^\nu}\delta\tensor{R}{_\mu_\nu}
\\
\end{split}
\end{equation}
We treat both occuring terms sepertately
\begin{equation}
\begin{split}
\int\dif x^4\sqrt{-g}\tensor{g}{^\mu^\nu}\delta\tensor{R}{_\mu_\nu}
&=\int\dif x^4\sqrt{-g}\tensor{g}{^\mu^\nu}\left(\tensor{\nabla}{_\rho}\delta
\cSym{\rho}{\mu}{\nu} -\tensor{\nabla}{_\mu}\delta \cSym{\rho}{\rho}{\nu}\right)
\\
&=\int\dif 
x^4\tensor{\nabla}{_\rho}\left(\sqrt{-g}\tensor{g}{^\mu^\nu}\delta
\cSym{\rho}{\mu}{\nu}\right)\\
&\phantom{=}-\int\dif
x^4\tensor{\nabla}{_\mu}\left(\sqrt{-g}\tensor{g}{^\mu^\nu}\delta
\cSym{\rho}{\rho}{\nu}\right)
\end{split}
\end{equation}
The Integrals vanish by Gauß law (neglegting surface Terms). We are left with
\begin{equation}
\begin{split}
2\kappa\delta S\textsubscript{EH}
&=\int\dif x^4\sqrt{-g}\left[
\frac{1}{2}\tensor{g}{^\mu^\nu}
(R-2\Lambda)+\tensor{R}{^\mu^\nu}\right]\delta\tensor{g}{_\mu_\nu}
\end{split}
\end{equation}
So that we can now finally calculate the variation with respect to the metric
field
\begin{equation}
\frac{\delta
S\textsubscript{EH}[\tensor{g}{_\mu_\nu}(x)]}{\delta\tensor{g}{_\mu_\nu}(x')}
=\sqrt{-g}\left[\frac{1}{2}\tensor{g}{^\mu^\nu}
(R-2\Lambda)+\tensor{R}{^\mu^\nu}\right]
\end{equation}
It vanishes if 
\begin{equation}
\tensor{R}{^\mu^\nu}+\frac{R}{2}\tensor{g}{^\mu^\nu}
-\Lambda\tensor{g}{^\mu^\nu}=0\, .
\end{equation}
so we have finally derived Einsteins field equations from an variational
principle.
\section{Introduction of Matter}
In the gravitational kontext we mean be \emph{matter} any non gravitational
fields this include scalar fields $\varphi$, spinor fields $\Psi$, gauge fields
$\tensor{A}{^\mu}$,\dots. We collect all of them in a multivariable $\Phi$
A local action can be written as
\begin{equation}
S\textsubscript{m}[\Phi,g]=\int \dif x^4\sqrt{-g}
L\textsubscript{m}\left(\Phi,\tensor{\nabla}{_\mu}\Phi,g\right)
\end{equation}
$\tensor{g}{^\mu^\nu}$ appears in $L\textsubscript{m}$ because the derivatives
$\tensor{\nabla}{_\mu}, \tensor{\partial}{_\mu}$ must be contracted.
Additionally it enters via $\sqrt{-g}$
\begin{example}[free scalar field in Minkowski space]
\begin{equation}
S\textsubscript{m}=\int \dif x^4 \left(-\frac{1}{2}\tensor{\eta}{^\mu^\nu}
\tensor{\partial}{_\mu}\varphi\tensor{\partial}{_\nu}\varphi-\frac{1}{2}m^2\varphi^2\right)
\end{equation}
The minus sign in front of the partial derivative should come as no surprise
since we have $\tensor{\eta}{^0^0}=-1\, \dot{\varphi}^2>0$. In a non inertial
frame we have to make the usual replacements
\begin{equation}
\tensor{\eta}{_\mu_\nu}\to \tensor{g}{_\mu_\nu}\, , \quad
\tensor{\partial}{_\mu}\to
\tensor{\nabla}{_\mu}\, \quad \dif x^4\to \dif x^4\sqrt{-g}\, .
\end{equation}
\end{example}
\emph{minimal cuppling description} (Kontext???) The action reads as
\begin{equation}
S\textsubscript{m}=\int \dif x^4 \sqrt{-g}\left(-\frac{1}{2}\tensor{g}{^\mu^\nu}
\tensor{\nabla}{_\mu}\varphi\tensor{\nabla}{_\nu}\varphi-\frac{1}{2}m^2\varphi^2\right)
\end{equation}
which is the action for a scalar $\varphi$ in the presence of gravity, i.e. a
dynamical $\tensor{g}{_\mu_\nu}(x)$. The combined action of scalar field and
gravity is given as
\begin{equation}
S[g,\varphi]=S\textsubscript{g}[g]+S\textsubscript{m}[g,\varphi]
\end{equation}
The Euler Lagange Equations in terms of the scalar field read as
\begin{equation}
\frac{\delta S[g,\varphi]}{\delta
\varphi\left(x'\right)}=\frac{\delta S\textsubscript{m}[g,\varphi]}{\delta \varphi\left(x'\right)}
\end{equation}

\begin{equation}
\frac{\delta S\textsubscript{m}[g,\varphi]}{\delta \varphi\left(x'\right)}=\int
\dif x^4\sqrt{-g}\left[-\tensor{g}{^\mu^\nu}
\tensor{\nabla}{_\mu}\varphi\tensor{\nabla}{_\nu}\left(\frac{\delta
\varphi(x)}{\delta \varphi\left(x'\right)}\right)-m^2\varphi\frac{\delta
\varphi(x)}{\delta \varphi\left(x'\right)}\right]
\end{equation}
where we used that the $\delta$ and $\tensor{\nabla}{_\mu}$ commute. Partial
integration yields
\begin{equation}
\begin{split}
\frac{\delta S\textsubscript{m}[g,\varphi]}{\delta \varphi\left(x'\right)}&=\int
\dif
x^4\sqrt{-g}\left(\square_{g}-m^2\right)\varphi\delta(x,x')\\
&=\sqrt{-g}\left(\square_{g}-m^2\right)\varphi
\end{split}
\end{equation}
where $\square_{g}:=\tensor{g}{^\mu^\nu}
\tensor{\nabla}{_\mu}\tensor{\nabla}{_\nu} $ is the
\emph{Laplace–Beltrami operator}, a generalisation of the ordinary laplacian to
curved space.
Demanding that the variation with respect to $\varphi$ vanishes implies the
\emph{Klein-Gordon equation}
\begin{equation}
\left(\square_g-m^2\right)\varphi=0
\end{equation}
Variing with respect to the metric $\tensor{g}{_\mu_\nu}$
\begin{equation}
\frac{\delta S[g,\varphi]}{\delta
\tensor{g}{_\mu_\nu}\left(x'\right)}=
\frac{\delta S\textsubscript{g}[g]}{\delta
\tensor{g}{_\mu_\nu}\left(x'\right)}
+\frac{\delta S\textsubscript{m}[g,\varphi]}{\delta
\tensor{g}{_\mu_\nu}\left(x'\right)}
=\frac{\sqrt{-g}}{2\kappa}\left(\tensor{G}{^\mu^\nu}+\Lambda\tensor{g}{^\mu^\nu}
\right)+\frac{\delta S\textsubscript{m}[g,\varphi]}{\delta
\tensor{g}{_\mu_\nu}\left(x'\right)}
\end{equation}
This makes it convenient to define the energy-momentum tensor (Vorzeichen????)
\begin{equation}
\tensor{T}{^\mu^\nu}:=\frac{2}{\sqrt{-g}}\frac{\delta
S\textsubscript{m}[g,\varphi]}{\delta \tensor{g}{_\mu_\nu}\left(x'\right)}\,.
\end{equation}
\begin{remark}
Attention $\delta_g
\tensor{g}{^\mu^\nu}=-\tensor{g}{^\varrho^{(\mu}}\tensor{g}{^{\nu)}^\sigma}\delta_g
\tensor{g}{_\mu_\nu}$
\end{remark}
We can now proceed in calculating the quantity we have ust introduced for a
scalar field
\begin{equation}
\begin{split}
\frac{\delta
S\textsubscript{m}[g,\varphi]}{\delta \tensor{g}{_\mu_\nu}\left(x'\right)}
&=\int \dif x^4 \frac{\delta\sqrt{-g}}{\delta
\tensor{g}{_\mu_\nu}}\left(-\frac{1}{2}\tensor{g}{^\mu^\nu}
\tensor{\nabla}{_\mu}\varphi\tensor{\nabla}{_\nu}\varphi-\frac{1}{2}m^2\varphi^2\right)\\
&\phantom{=}+
\sqrt{-g}\left(-\frac{1}{2}\tensor{g}{^\alpha^\varrho}\tensor{g}{^\beta^\sigma} \tensor{\nabla}{_\alpha}\varphi\tensor{\nabla}{_\beta}\varphi\frac{\delta
\tensor{g}{_\varrho_\sigma}}{\delta \tensor{g}{_\mu_\nu}}\right)\\
&=\frac{1}{2}\int \dif x^4
\sqrt{-g}\left(-\frac{1}{2}\tensor{g}{^\mu^\nu}\tensor{\nabla}{_\varrho}\varphi\tensor{\nabla}{^\varrho}\varphi-\frac{1}{2}\tensor{g}{^\mu^\nu}m^2\varphi^2
+\tensor{\nabla}{^\mu}\varphi\tensor{\nabla}{^\nu}\varphi\right)\delta(x,x')\\
&=\frac{1}{2}\sqrt{-g}\left(-\frac{1}{2}\tensor{g}{^\mu^\nu}\tensor{\nabla}{_\varrho}\varphi\tensor{\nabla}{^\varrho}\varphi-\frac{1}{2}\tensor{g}{^\mu^\nu}m^2\varphi^2
+\tensor{\nabla}{^\mu}\varphi\tensor{\nabla}{^\nu}\varphi\right)
\end{split}
\end{equation}
So that
\begin{equation}
\tensor{T}{^\mu^\nu}(\varphi)
=-\frac{1}{2}\tensor{g}{^\mu^\nu}\tensor{\nabla}{_\varrho}\varphi\tensor{\nabla}{^\varrho}\varphi
+\tensor{\nabla}{^\mu}\varphi\tensor{\nabla}{^\nu}\varphi
-\frac{1}{2}\tensor{g}{^\mu^\nu}m^2\varphi^2
\end{equation}
As we have noticed, the Einstein Tensor is covariantly conserved (contracted
Bianci identities). The Einstein equation then implies that also
$\tensor{T}{^\mu^\nu_{;\nu}}=0$ this can be checked for the given Tensor
\begin{equation}
\begin{split}
\tensor{\nabla}{_\mu}\tensor{T}{^\mu^\nu}
&=\tensor{g}{^\mu^\nu}\tensor{\nabla}{_\mu}\tensor{\nabla}{_\varrho}\varphi\tensor{\nabla}{^\varrho}\varphi+\square\varphi\tensor{\nabla}{^\nu}\varphi+\tensor{\nabla}{^\mu}\varphi\tensor{\nabla}{_\mu}\tensor{\nabla}{^\nu}\varphi
-\tensor{g}{^\mu^\nu}m^2\varphi\tensor{\nabla}{_\mu}\varphi\\
&=\tensor{\nabla}{^\nu}\varphi\left(\square-m^2\right)\varphi\\
&=0
\end{split}
\end{equation}
Where the last equality holds because $\varphi$ satisfies the Klein-Gordon
equation. The Einstein equations are 10 quasi linear, i.e. the highest order
derivative apears only linear, differential equations for the metric field
$\tensor{g}{_\mu_\nu}$. Strictly speaking the Einstein equations are
\emph{nonlinear}.
\begin{sidenote}
If you substract the constrains imposed by the Bianci identities you end with 2
DOFs, which are associated with the polarisation states of the graviton.
\end{sidenote}
How do we find a solution to this equations?
\begin{enumerate}
  \item Prescribe $\tensor{T}{_\mu_\nu}$. This is only possible for high
  symmetry problems, e.g. the Schwazschild solution and the cosmological
  solutions (Friedmans equations)
  \item Assume $\tensor{g}{_\mu_\nu}$, then compute $\tensor{T}{_\mu_\nu}$ and
  (try!) to interprete this.
\end{enumerate}
Intrinsic vs extrinsic curvature ????
\subsection{ADM-Decomposition}
\begin{figure}
\centering
\begin{tikzpicture}[x={(170:1cm)},y={(55:.7cm)},z={(90:1cm)}]
\node at  (0,-1,-3) (a) {} ;
\node[inner sep=1pt,circle,fill=black] at  (0.5,0.5,-1) (b) {} ;
\node[inner sep=1pt,circle,fill=black] at  (0,0,2) (c) {} ;
\node at  (0,1,3) (d) {} ;
%\draw (a) edge[in=-70,out=110,->-,-] (b);
\filldraw[fill=gray, draw=black, looseness=0.4,opacity=0.5,text opacity=1, draw opacity=1] (2.5,-2.5,-1) node[above right] {$\hspace{0.2cm}\Sigma_{t_0}$}
to[bend left] (2.5,2.5,-1)
to[bend left] coordinate (mp) (-2.5,2.5,-1)
to[bend right] (-2.5,-2.5,-1)
to[bend right] coordinate (mm) (2.5,-2.5,-1)
-- cycle;
\draw (b) edge[in=-70,out=100,-,->-](c);
\filldraw[fill=gray, draw=black,opacity=0.5,text opacity=1, draw opacity=1] (2.5,-2.5,2) node[above right] {$\hspace{0.4cm}\Sigma_{t}$} -- (2.5,2.5,2) -- (-2.5,2.5,2) -- (-2.5,-2.5,2) -- cycle;
\node[inner sep=1pt,circle,fill=black] at   (0.5,0.5,-1)   {} ;
\node[inner sep=1pt,circle,fill=black] at  (0,0,2)  {} ;
\draw (c) edge[in=-110,out=110,->-,-] node[pos=0.2,right]{$t$}(d);
\end{tikzpicture}
\caption{Foliation of spacetime into spaceial hypersurfaces $\Sigma_t$.}
\end{figure}
Erklärung einschieben was passiert!!!
Assume we know (??)
\begin{itemize}
  \item $\tensor{g}{_\mu_\nu}$ on $\Sigma_{t_0}$
  \item $\tensor{g}{_\mu_\nu_{;j}}$, $\tensor{g}{_\mu_\nu_{;0}}$ on
  $\Sigma_{t_0}$ also allowed
\end{itemize}
$\Sigma_t=\left\{\tensor{x}{^0}=t\right\}$
In vakuum the field equation are
\begin{equation}
0=G=R-2R\implies\tensor{R}{_\mu_\nu}=0
\end{equation}
The resulting set of equations is (prüfen!!!)
\begin{align}
0&=\tensor{R}{_0_0}=-\frac{1}{2}\tensor{g}{^i^j}\tensor{g}{_i_j_{,00}}+\tensor{M}{_0_0}\\
0&=\tensor{R}{_0_i}=-\frac{1}{2}\tensor{g}{^0^j}\tensor{g}{_i_j_{,00}}+\tensor{M}{_0_i}\\
0&=\tensor{R}{_i_j}=-\frac{1}{2}\tensor{g}{^0^0}\tensor{g}{_i_j_{,00}}+\tensor{M}{_i_j}
\end{align}
Where $\tensor{M}{_\mu_\nu}$ is a rest term containing lower order derivaties.
This shows that ther are no second order time derivatives of
$\tensor{g}{_0_\mu}$. We have 10 euations and 6 undetetrmined functions. The
DOFs can be used for a coordinate transformation, so that
$\tensor{g}{_0_\mu_{,00}}=0$ on $\Sigma_{t_0}$. This is allways possible but we
will not proof this. It can be further shown, by means of the contracted Bianci
identities, that this implies $\tensor{g}{_0_\mu_{,00}}=0$ on \emph{all}
hypersurfaces $\Sigma_{t}$.
\begin{equation}
\tensor{\partial}{_0}\tensor{G}{^0^\nu}=
\tensor{\partial}{_i}\tensor{G}{^i^\nu}
-\cSym{\nu}{0}{\lambda}\tensor{G}{^\lambda^\nu}
-\cSym{0}{\nu}{\lambda}\tensor{G}{^\mu^\lambda}
\end{equation}
Hier muss man nochmal schauen das macht noch nicht viel sinn\ldots\ldots.
We have the freedom to choose four coodinates
\begin{equation}
\tensor{x}{^{\mu^\prime}}=\tensor{f}{^{\mu^\prime}}\left(\tensor{x}{^\mu}\right)
\end{equation}
One typical choice is the \emph{harmonic\footnote{a function $f$ satisfying
$\square f = 0$ is called harmonic.} gauge}
\begin{equation}
\square\tensor{x}{^\mu}=0\,.
\end{equation}
\begin{equation}
\begin{split}
\square\tensor{x}{^\mu}&=g^{-\nicefrac{1}{2}}\tensor{\partial}{_\varrho}\left(g^{\nicefrac{1}{2}}\tensor{g}{^\varrho^\sigma}\tensor{\partial}{_\sigma}\tensor{x}{^\mu}\right)\\
&=g^{-\nicefrac{1}{2}}\tensor{\partial}{_\varrho}\left(g^{\nicefrac{1}{2}}\tensor{g}{^\varrho^\sigma}\tensor{\delta}{_\sigma^\mu}\right)\\
&=g^{-\nicefrac{1}{2}}\tensor{\partial}{_\varrho}\left(g^{\nicefrac{1}{2}}\tensor{g}{^\varrho^\mu}\right)\\
\end{split}
\end{equation}
(Formel für $\square$ die hier genutzt wurde referenzieren)
The harmonic gauge is therefore equivalent to
\begin{equation}
\tensor{\partial}{_\varrho}\left(g^{\nicefrac{1}{2}}\tensor{g}{^\varrho^\mu}\right)=0\,
.\\
\end{equation}
The equation can be sorted in spatial and time components and derive by the
zero component, so that
\begin{equation}
\tensor*{\partial}{*_0^2}\left(g^{\nicefrac{1}{2}}\tensor{g}{^0^\mu}\right)
=
-\tensor{\partial}{_i}\left[\tensor{\partial}{_0}\left(g^{\nicefrac{1}{2}}\tensor{g}{^0^\mu}\right)\right]\,
,\end{equation}
which fixes the second order time derivatives of the relevant components
$\tensor{g}{^0^\mu}$. Therefore now the time evolution can be solved. (Fehlt
hier was???)
\subsubsection{Degrees of freedom}
\begin{itemize}
  \item[\color{section_color}\textsf{\textbf{10}}]
  componnents for every spacetime point from the symmetric $\tensor{g}{_\mu_\nu}(x)$
  \item[\color{section_color}\textsf{\textbf{-4}}] from the
  constraint equation $\tensor{G}{_\mu_\nu^{;\nu}}=0$
  \begin{itemize}
    \item
    $\tensor{G}{^0^0}=\kappa \tensor{T}{^0^0}$ ensures that the
    evolution is independent of the choice of spatial coordinates on
    $\Sigma_{t_0}$.
    \item
    $\tensor{G}{^i^0}=\kappa \tensor{T}{^i^0}$ ensures that the time
    evolution is independent of the way we foliated sacetime into spacial
    hypersurfaces $\Sigma_{t}.    $
  \end{itemize}
  \item[\color{section_color}\textsf{\textbf{-4}}] due to the freedom
  to choose coordinates (i.e. a gauge).
\end{itemize}
We are left with 2 physical degrees of freedom which may be interpreted as the
polarisation states of the graviton field.
\subsubsection{Comparison with electrodynamics in flat spacetime}
In electrodynamics instead of the einstein equations we have the field equations
for the four potential $\tensor{A}{_\mu}$:
\begin{equation}
\square\tensor{A}{_\mu}-\partial_\mu\left(\partial_\nu\tensor{A}{^\nu}
\right)=0\,.
\end{equation}
As we did for the gravitational field we take a look
at the 0 component. We find
\begin{equation}
\begin{split}
0&=-\partial_0^2\tensor{A}{_0}+\partial_i\partial^i\tensor{A}{_0}
-\partial_0\left(-\partial_0\tensor{A}{_0}+\partial_iA^i\right)\\
&= \partial_i\partial^i\tensor{A}{_0}-\partial_0\partial_iA^i
\end{split}
\end{equation}
This equation is equivalent to $\nabla\vec{E}=0$ and the bianci identities.
So once again $\tensor{A}{_0}$ is \emph{not} determined by the dynamical
evolution equation because there is no second order time derivative analogous to
$\tensor{g}{_0_0}$. Since $\tensor{A}{_0}$ is not determinend and cannot be
specified on initial time slice. This reflects some internal redundancy namely
gauge invariance of the theory. For any scalar function $\Lambda$
\begin{equation}
\tensor{A}{_\mu}\to\tensor*{A}{*_\mu^\prime}=
\tensor{A}{_\mu}+\partial_\mu\Lambda
\end{equation}
leaves the physics invariant. It is trivial to check that
the field strength tensor $\tensor{F}{_\mu_\nu}=\partial_\mu
\tensor{A}{_\nu}-\partial_\nu\tensor{A}{_\mu}$ stays invariant. Perhaps more
interesting the field equation is also gauge invariant:
\begin{equation}
\begin{split}
\square\tensor*{A}{*_\mu^\prime}-\partial_\mu\left(\partial_\nu\tensor*{A}{*^\nu^\prime}
\right)
&=
\square\tensor{A}{_\mu}+\square\partial_\mu\Lambda-\partial_\mu\left(\partial_\nu\tensor{A}{^\nu}
\right)-\partial_\mu\square\Lambda\\
&=
\square\tensor{A}{_\mu}-\partial_\mu\left(\partial_\nu\tensor{A}{^\nu}
\right)\,.
\end{split}
\end{equation}
Thus if $\tensor{A}{_\mu}$ is a solution to the field
equation $\tensor*{A}{*_\mu^\prime}$ is and therefore both are physically
undistinguishable. We can also fix a gauge for example the \emph{Lorentz gauge}:
\begin{equation}
\partial_\mu\tensor{A}{^\mu}=0\, .
\end{equation}
If we derive this by the 0 component we get
\begin{equation}
\partial_{0}^2\tensor{A}{^0}=-\partial_i\partial_0\tensor{A}{^i}\, ,
\end{equation}
so as with $\tensor{g}{_0_0}$ the evolution of the 0 component is now related to
the other components. There is still one residual gauge condition, namely we can
still transform
\begin{equation}
\tensor{A}{_\mu}\to\tensor*{A}{*_\mu^\prime}=
\tensor{A}{_\mu}+\partial_\mu\Lambda\, ,
\end{equation}
but to keep the gauge, we have to demand that $\square\Lambda=0$.
Again we count the DOFs:
\begin{itemize}
  \item[\color{section_color}\textsf{\textbf{4}}] components of the potential
  $\tensor{A}{_\mu}$.
  \item[\color{section_color}\textsf{\textbf{-1}}] from constraint
  $\nabla\vec{E}=0$.
  \item[\color{section_color}\textsf{\textbf{-1}}] from gauge freedom
  $\Lambda$.
\end{itemize}
This leaves two physical degrees of freedom, the polarisation states of a
photon.
\begin{remark}
As we have have seen there is a direct correspondence between the gauge freedom
in electrodynamics and the freedom of choice of coordinates of coordinates in
GR.
\end{remark}
